\chapter{La Normalizzazione}

\section{Teoria Relazionale}

Due metodi per produrre uno schema relazionale:

\begin{itemize}
	\item partire da un buon schema a oggetti e tradurlo,
	\item partire da uno schema relazionale fatto da altri e modificarlo o completarlo.
\end{itemize}

\noindent Teoria della progettazione relazionale:\ studia cosa sono le ``anomalie'' e come eliminarle (\textbf{normalizzazione}).\
È particolarmente utile se si usa il secondo metodo.\
È moderatamente utile anche quando si usa il primo.

\textbf{Attenzione alle anomalie}:\ ridondanze, potenziali inconsistenze e anomalie nelle inserzioni/eliminazioni.

\subsubsection{Obiettivi}

Obiettivi della teoria:

\begin{itemize}
	\item \textbf{Equivalenza} di schemi:\ in che misura si può dire che uno schema ne rappresenta un altro.
	\item \textbf{Qualità} degli schemi (forme normali).
	\item \textbf{Trasformazione} degli schemi (normalizzazione di schemi).
\end{itemize}

\noindent Ipotesi dello \textbf{schema di relazione universale}:\ tutti i fatti sono descritti da attributi di un'unica relazione (\textbf{relazione universale}), cioè gli attributi hanno un significato globale.\
Definizione:\ lo schema di \textbf{relazione universale} U di una base di dati relazionale ha come attributi l'unione degli attributi di tutte le relazioni della base di dati.

Una \textbf{forma normale} è una proprietà di una base di dati relazionale che ne garantisce la ``qualità'', cioè l'assenza di determinati difetti.\
Quando una relazione \textit{non è normalizzata}:\ presenta ridondanze e si presta a comportamenti poco desiderabili durante gli aggiornamenti.\
La normalizzazione è una procedura che permette di trasformare schemi non normalizzati in schemi che soddisfano una forma normale.

\subsubsection{Linee guida per una corretta progettazione}

\begin{itemize}
	\item Semantica degli attributi:\ si progetti ogni schema relazionale in modo tale che \textbf{sia semplice spiegarne il significato}.\ Non si uniscano attributi provenienti da più tipi di classi e tipi di associazione in una unica relazione.
	\item Ridondanza:\ si progettino gli schemi relazionale in modo che nelle relazioni \textbf{non siano presenti anomalie di inserimento, cancellazione o modifica}.\ Se sono presenti delle anomalie (che si vuole mantenere), le si rilevi chiaramente e \textit{ci si assicuri} che i programmi che aggiornano la base di dati operino correttamente.
	\item Valori nulli:\ per quanto possibile, \textbf{si eviti di porre in relazione di base attributi i cui valori possono essere (frequentemente) nulli}.\ Se è inevitabile, ci si assicuri che essi si presentino solo in casi eccezionali e che non riguardino una maggioranza di tuple nella relazione.\
	\item Tuple spurie:\ si progettino schemi di relazione in modo tale che essi possano essere riuniti tramite \texttt{JOIN} con condizioni di uguaglianza su attributi che sono o \textbf{chiavi primarie} o \textbf{chiavi esterne} in modo da garantire che non vengano generate tuple spurie.\ \textit{Non si abbiano relazioni che contengono attributi di ``accoppiamento'' diversi dalle combinazioni chiave esterna-chiave primaria}.
\end{itemize}

\subsection{Dipendenze funzionali}

Per formalizzare la nozione di schema senza anomalie, occorre una descrizione formale della semantica dei fatti rappresentati in uno schema relazionale.\
\textbf{Istanza valida di R}:\ è una \textbf{nozione semantica} che dipende da ciò che sappiamo del dominio del discorso (non estensionale, non deducibile da alcune istanze dello schema).\

\noindent Istanza valida \textit{r} su $R(T)$.\
Siano $X$ e $Y$ due sottoinsiemi non vuoti di $T$.\
Esiste in \textit{r} una \textbf{dipendenza funzionale} da $X$ a $Y$ se, per ogni coppia di ennuple $t_1$ e $t_2$ di \textit{r} con gli stessi valori su $X$, risulta che $t_1$ e $t_2$ hanno gli stessi valori anche su $Y$.\
La dipendenza funzionale da $X$ a $Y$ si denota con $X \rightarrow Y$.

\begin{center}
	Formalmente
\end{center}

\noindent Dato uno schema $R(T)$ e $X, Y \subseteq T$, una \textbf{dipendenza funzionale} (DF) fra gli attributi $X$ e $Y$, è un vincolo su R sulle istanze della relazione, espresso nella forma $X \rightarrow Y$, cioè $X$ \textit{determina funzionalmente} $Y$ o $Y$ \textit{è determinato da} $X$, se per ogni istanza valida \textit{r} di $R$ un valore di $X$ determina in modo univoco un valore di $Y$:\

\begin{center}
	$\forall$ \textit{r} istanza valida di R, $\forall$ t\textsubscript{1}, t\textsubscript{2} $\in$ \textit{r}. se t\textsubscript{1}[X] = t\textsubscript{2}[X] allora t\textsubscript{1}[Y] = t\textsubscript{2}[Y].\
\end{center}

\noindent In altre parole:\ un'istanza \textit{r} di R(T) soddisfa la dipendenza X $\rightarrow$ Y (o X $\rightarrow$ Y vale in \textit{r}), se per ogni coppia di ennuple t\textsubscript{1} e t\textsubscript{2} di \textit{r}, se t\textsubscript{1}[X] = t\textsubscript{2}[X] allora t\textsubscript{1}[Y] = t\textsubscript{2}[Y].

Si dice che un'istanza \textit{r}\textsubscript{0} di R \textbf{\textit{soddisfa}} la DF X$\rightarrow$Y (r\textsubscript{0} $\models$ X $\rightarrow$ Y) se la proprietà vale per \textit{r}\textsubscript{0}:\ $\forall$ t\textsubscript{1}, t\textsubscript{2} $\in$ \textit{r}. se t\textsubscript{1}[X] = t\textsubscript{2}[X] allora t\textsubscript{1}[Y] = t\textsubscript{2}[Y] e che un'istanza \textit{r}\textsubscript{0} di R \textbf{\textit{soddisfa}} un insieme F di DF se per ogni X $\rightarrow$ Y $\in$ F, vale \textit{r}\textsubscript{0} $\models$ X $\rightarrow$ Y:
\begin{center}
	\textit{r}\textsubscript{0} $\models$ X $\rightarrow$ Y \textit{sse} $\forall$ t\textsubscript{1}, t\textsubscript{2} $\in$ \textit{r}\textsubscript{0}. se t\textsubscript{1}[X] = t\textsubscript{2}[X] allora t\textsubscript{1}[Y] = t\textsubscript{2}[Y].
\end{center}

\subsubsection{Dipendenze funzionali atomiche}

Ogni dipendenza funzionale X $\rightarrow$ A\textsubscript{1}, A\textsubscript{2},\ \dots, A\textsubscript{n}, si può scomporre nelle dipendenze funzionali X $\rightarrow$ A\textsubscript{1}, X $\rightarrow$ A\textsubscript{2},\ \dots, X $\rightarrow$ A\textsubscript{n}.\
Le dipendenze funzionali del tipo X $\rightarrow$ A si chiamano \textbf{dipendenze funzionali atomiche}.

\subsubsection{Dipendenze funzionali banali}

Y $\rightarrow$ A è \textbf{non banale} se A non è contenuta in Y.\
Siamo interessati alle dipendenze funzionali non banali.

\subsubsection{Proprietà delle dipendenze funzionali}

Notazione:\ $\mathrm{R\langle T,F \rangle}$ denota uno \textit{schema} con \textbf{attributi} \texttt{T} e \textbf{dipendenze funzionali} \texttt{F}.\

\vspace{12pt}
\noindent \textit{Le DF sono una proprietà semantica, cioè dipendono dai fatti rappresentati e non da come gli attributi sono combinati in schemi di relazione}.\
\vspace{12pt}

\noindent Si parla di DF \textbf{completa} quando X $\rightarrow$ Y e per ogni W $\subset$ X, non vale W $\rightarrow$ Y.\
Se X è una \textbf{superchiave}, allora X determina ogni altro attributo della relazione, cioè X $\rightarrow$ T.\
Se X è una \textbf{chiave}, allora X $\rightarrow$ T è una DF completa.\
Da un insieme F di DF, in generale altre DF sono ``implicate'' da F.

\begin{itemize}
	\item\textbf{Dipendenze implicate}:\ sia F un insieme di DF sullo schema R, diremo che
	      \begin{center}
		      F \textbf{implica logicamente}  X $\rightarrow$ Y ($F \models X \rightarrow Y$),
	      \end{center}
	      se ogni istanza r di R che soddisfa F soddisfa anche X $\rightarrow Y$.\
	\item\textbf{Dipendenze banali}:\
	      \begin{center}
		      implicate dal vuoto, es. $\{\} \models X \rightarrow X$
	      \end{center}
\end{itemize}

\subsection{Regole di inferenza}

Come derivare DF implicate logicamente da F?\
Usando un insieme di regole di inferenza.

\begin{table}[H]
	\centering
	\begin{tabular}{l l}
		\multicolumn{2}{c}{``\textbf{Assiomi}'' (sono in realtà regole di inferenza) di \textbf{Armstrong}:} \\
		Se Y $\subseteq$ X, allora X $\rightarrow$ Y                       & \textbf{Riflessività}           \\
		Se X $\rightarrow$ Y, Z $\subseteq$ T, allora XZ $\rightarrow$ YZ  & \textbf{Arricchimento}          \\
		Se X $\rightarrow$ Y, Y $\rightarrow$ Z, allora  X $\rightarrow$ Z & \textbf{Transitività}           \\
	\end{tabular}
\end{table}

\subsubsection{Derivazione}

Definizione:\ sia F un insieme di DF, diremo che X $\rightarrow$ Y sia \textbf{\textit{derivabile}} da F ($F \vdash X \rightarrow Y$), se X $\rightarrow$ Y può essere inferito da F usando gli assiomi di Armstrong.\

\begin{center}
	Formalmente
\end{center}

\noindent Una \textbf{derivazione} di f da F è una sequenza finita f\textsubscript{1}, \dots, f\textsubscript{m} di dipendenze, dove f\textsubscript{m} = f e ogni f\textsubscript{i} è un elemento di F oppure è ottenuta dalla precedenti dipendenze f\textsubscript{1}, \dots, f\textsubscript{i-1} della derivazione usando una regola di inferenza.\
Si dimostra che valgono anche le regole

\begin{table}[H]
	\centering
	\begin{tabular}{l c l}
		$\{X \rightarrow Y, X \rightarrow Z\}$ & $\vdash$ & $X \rightarrow YZ$ (\textbf{unione U})        \\
		$Z \subseteq Y,\ \{X \rightarrow Y\}$  & $\vdash$ & $X \rightarrow Z$ (\textbf{decomposizione D}) \\
	\end{tabular}
\end{table}

\noindent Da \textbf{Unione} e \textbf{Decomposizione} si ricava che se Y = A\textsubscript{1}A\textsubscript{2}\dots A\textsubscript{n} allora
\[
	X \rightarrow Y \Leftrightarrow \{X \rightarrow A_1, X \rightarrow A_2, \dots, X \rightarrow A_n\}.
\]

\subsubsection{Correttezza e completezza degli assiomi di Armstrong}

\textbf{\textit{Teorema}}\quad Gli assiomi di Armstrong sono corretti e completi.\
Attraverso gli assiomi di Armstrong, si può mostrare l'equivalenza della nozione di \textbf{implicazione logica} ($\models$) e di quella di derivazione ($\vdash$):\ \textit{se una dipendenza è derivabile con gli assiomi di Armstrong allora è anche implicata logicamente} (\textbf{correttezza} degli assiomi) e viceversa \textit{se una dipendenza è implicata logicamente allora è anche derivabile dagli assiomi} (\textbf{completezza} degli assiomi).\

\begin{itemize}
	\item \textbf{Correttezza} degli assiomi:\
	      \[
		      \forall f, F \vdash f \Rightarrow F \models f
	      \]
	\item \textbf{Completezza} degli assiomi:\
	      \[
		      \forall f, F \models f \Rightarrow F \vdash f
	      \]
\end{itemize}

\subsubsection{Chiusura di un insieme F}

\begin{definition}
	Dato un insieme F di DF, la \textbf{chiusura di F}, denotata con \textbf{F\textsuperscript{+}}, è:
	\[
		F^+ = \{ X \rightarrow Y\ |\ F \vdash X \rightarrow Y\}
	\]
\end{definition}

\noindent Un problema che si presenta spesso è quello di decidere se una dipendenza funzionale appartiene a F\textsuperscript{+} (problema dell'implicazione):\ controllare se una DF $V \rightarrow W \in F^+$.\
La sua risoluzione con l'algoritmo banale (di generare F\textsuperscript{+} applicando ad F ripetutamente gli assiomi di Armstrong) ha una complessità esponenziale rispetto al numero di attributi dello schema.\

\begin{definition}
	Dato $\mathrm{R\langle T,F \rangle}$, e X $\subseteq$ T, la \textbf{\textit{chiusura} di X rispetto ad F}, denotata con X\textsubscript{F}\textsuperscript{+} (o X\textsuperscript{+}, se F è chiaro nel contesto) è
	\[
		X_F^+ = \{A_i \in T\ |\ F \vdash X \rightarrow A_i\}
	\]
\end{definition}

\noindent Un algoritmo efficiente per risolvere il problema dell'implicazione senza calcolare la chiusura di F scaturisce dal seguente teorema.\

\begin{theorem}
	F $\vdash$ X $\rightarrow$ Y $\Leftrightarrow$ Y $\subseteq$ X\textsubscript{F}\textsuperscript{+}.
\end{theorem}

\subsubsection{Algoritmo per calcolare X\textsubscript{F}\textsuperscript{+}}

\textbf{Chiusura Lenta}:\ sia X un insieme di attributi e F un insieme di dipendenze.\
Vogliamo calcolare X\textsubscript{F}\textsuperscript{+}.
\begin{enumerate}
	\item Inizializziamo X\textsuperscript{+} con l'insieme X.
	\item Se fra le dipendenze di F c'è una dipendenza Y $\rightarrow$ A con Y $\subseteq$ X\textsuperscript{+} allora si inserisce A in X\textsuperscript{+}, ossia $X^+ = X^+ \cup \{A\}$.
	\item Si ripete il passo 2 fino a quando non ci sono altri attributi da aggiungere a X\textsuperscript{+}.
	\item Si dà in output X\textsubscript{F}\textsuperscript{+} = X\textsuperscript{+}.

\end{enumerate}

\subsubsection{Terminazione dell'algoritmo}

\textbf{L'algoritmo termina} perché ad ogni passo viene aggiunto un nuovo attributo a X\textsuperscript{+}.\
Essendo gli attributi in numero finito, a un certo punto l'algoritmo deve fermarsi.\
Per dimostrare la correttezza, si dimostra che X\textsubscript{F}\textsuperscript{+} = X\textsuperscript{+} (per induzione).\

\subsection{Chiavi e attributi primi}

\textbf{Definizione}:\ dato lo schema $\mathrm{R\langle T,F \rangle}$, diremo che un insieme di attributi W $\subseteq$ T è una \textbf{chiave candidata} di R, se

\begin{table}[H]
	\centering
	\begin{tabular}{l l}
		$W \rightarrow T \in F^+$                           & (W superchiave)                       \\
		$\forall\ V \subset W, V \rightarrow T \not\in F^+$ & (se V $\subset$ W, V non superchiave) \\
	\end{tabular}
\end{table}

\noindent \textbf{\textit{Attributo primo}}:\ attributo che appartiene ad almeno una chiave.

\begin{itemize}
	\item Il problema di trovare tutte le chiavi di una relazione richiede un algoritmo di complessità esponenziale nel caso peggiore.
	\item Il problema di controllare se un attributo è primo è NP-completo.
\end{itemize}

\section{Copertura canonica}

\textbf{Definizione}\quad Due insiemi di DF, F e G, sullo schema R sono \textbf{equivalenti},
\begin{center}
	F $\equiv$ G, sse F\textsuperscript{+} = G\textsuperscript{+}.\
\end{center}
Se F $\equiv$ G, allora F è una \textbf{copertura} di G (e G una copertura di F).

\begin{definition}
	Sia F un insieme di DF:
	\begin{enumerate}
		\item Data una $X \rightarrow Y \in F$, si dice che X contiene un \textbf{attributo \textit{estraneo}} A\textsubscript{i} sse
		      \[
			      (X - \{A_i\}) \rightarrow Y \in F^+,\ ovvero\ F \vdash (X - \{A_i\}) \rightarrow Y
		      \]
		      Per verificare se A è estraneo calcoliamo X\textsuperscript{+} e verifichiamo se include B, ovvero se basta X a determinare B.
		\item X $\rightarrow$ Y è una \textbf{dipendenza \textit{ridondante}} sse
		      \[ (F-\{X \rightarrow Y\})^+ = F^+ \]
		      Equivalentemente
		      \[F-\{X \rightarrow Y\} \vdash X \rightarrow Y\]
		      Per stabilire se una DF del tipo X $\rightarrow$ A è ridondante la eliminiamo da F, calcoliamo X\textsuperscript{+} e verifichiamo se include A, ovvero se con le DF che restano riusciamo ancora a dimostrare che X determina A.\
	\end{enumerate}
\end{definition}

\noindent F è detta una \textbf{copertura \textit{canonica}} \textit{sse}
\begin{itemize}
	\item la parte destra di ogni DF in F è un attributo;
	\item non esistono attributi estranei;
	\item nessuna dipendenza in F è ridondante.
\end{itemize}

\begin{theorem}
	Per ogni insieme di dipendenze F esiste una copertura canonica.\
\end{theorem}

\noindent Algoritmo per calcolare una copertura canonica.\
Trasformare le dipendenze nella forma X $\rightarrow$ A:\ si sostituisce l'insieme dato con quello equivalente che ha tutti i secondi membri costituiti da singoli attributi (dipendenze atomiche); eliminare gli attributi estranei:\ per ogni dipendenza si verifica se esitono attributi eliminabili dal primo membro; eliminare le dipendenze ridondanti.\

\subsubsection{Riepilogo}

Le \textbf{ridondanze} sui dati possono essere di due tipi:\ \textbf{ridondanza concettuale}, non ci sono duplicazioni dello stesso dato, ma sono memorizzate \textit{informazioni che possono essere ricavate da altre già contenute nel DB}; \textbf{ridondanza logica}, esistono \textit{duplicazioni sui dati che possono generare anomalie} nelle operazioni sui dati.\

Le dipendenze funzionali sono definite a \textbf{livello di schema} e non a livello di istanza ed hanno sempre \textbf{un verso}.\

Le dipendenze funzionali sono una \textbf{generalizzazione del vincolo di chiave e di \textit{superchiave}}.\
Data una tabella con schema R(X) con superchiave, esiste un vincolo di dipendenza funzionale tra K e qualsiasi attributo dello schema $r$.\

\[
	K \rightarrow X_1,\quad X_1 \subseteq X
\]

\section{Decomposizione di Schemi}

In generale, per eliminare anomalie da uno schema occorre decomporlo in schemi più piccoli ``equivalenti''.\

\begin{definition}
	Dato uno schema R(T),
	$\rho = \{R_1(T_1), \dots, R_k(T_k)\}$ è una \textbf{decomposizione} di R \textit{sse} $T_1 \cup \dots \cup T_k = T$
\end{definition}

\noindent Due proprietà desiderabili di una decomposizione:\ \textbf{conservazione dei dati} (\textit{notazione semantica}) e \textbf{conservazione delle dipendenze}.\

\begin{definition}
	$\rho = \{R_1(T_1), \dots, R_k(T_k) \}$ è una decomposizione di uno schema R(T) che \textbf{preserva i dati} \textit{sse} per ogni \textbf{\textit{istanza valida}} \textit{r} di R:\
	\[
		r = (\pi_{T_1}r) \lor (\pi_{T_2}r) \lor \dots \lor (\pi_{T_k}r)
	\]
\end{definition}

\noindent Dalla definizione di giunzione naturale scaturisce il seguente risultato.\

\begin{theorem}
	Se $\rho = \{R_1(T_1), \dots, R_k(T_k)\}$ è una decomposizione di R(T), allora per ogni istanza \textit{r} di R:
	\[
		r = (\pi_{T_1}r) \lor (\pi_{T_2}r) \lor \dots \lor (\pi_{T_k}r)
	\]
\end{theorem}

\noindent Uno schema R(X) si \textbf{decompone senza perdita dei dati} negli schemi R\textsubscript{1}(X\textsubscript{1}) ed R\textsubscript{2}(X\textsubscript{2}) se, per ogni possibile istanza di R(X), \textit{il join naturale delle proiezioni di r su X\textsubscript{1} ed X\textsubscript{2} produce la tabella di partenza} (cioè non contiene \textbf{\textit{ennuple spurie}}).

\[
	\pi_{X_1}(r) \Join \pi_{X_2}(r) = r
\]

\noindent La decomposizione senza perdita è garantita se l'insieme degli attributi comuni alle due relazioni ($X_1 \cap X_2$) è \textbf{chiave per almeno una delle due relazioni decomposte}.

Sia \textit{r} una relazione su un insieme di attributi X e siano X\textsubscript{1} e X\textsubscript{2} due sottoinsiemi di X la cui unione sia pari a X stesso.\
Inoltre, sia X\textsubscript{0} l'intersezione di X\textsubscript{1} e X\textsubscript{2}, allora \textit{r} si decompone senza perdita su X\textsubscript{1} e X\textsubscript{2} se soddisfa la dipendenza funzionale X\textsubscript{0} $\rightarrow$ X\textsubscript{1} oppure la dipendenza funzionale X\textsubscript{0} $\rightarrow$ X\textsubscript{2}.

\begin{theorem}[non formale]
	Se l'insieme degli attributi comuni alle due relazioni ($X_1 \cap X_2$) è chiave per almeno una delle due relazioni decomposte allora la decomposizione è senza perdita.\
\end{theorem}

\begin{proof}[Dimostrazione (non formale)]

	Supponiamo \textit{r} sia una relazione sugli attributi ABC e consideriamo la sue proiezioni $r_1$ su AB e $r_2$ su AC.\
	Supponiamo che \textit{r} soddisfi la dipendenza funzionale $A \rightarrow C$.\
	Allora \textbf{A è chiave per \textit{r}\textsubscript{1} su AC} e quindi non ci sono in tale proiezione due tuple diverse sugli stessi valori di A.\
	Il join costruisce tuple a partire dalle tuple nelle due proiezioni.\
	Sia $t = (a,b,c)$ una tupla del \textbf{join di \textit{r}\textsubscript{1} e \textit{r}\textsubscript{2}}.\
	Mostriamo che appartiene ad \textit{r} (cioè non è spuria).\
	\textit{t} è ottenuta mediante il join da $t_1 = (a,b)$ di $r_1$ e $t_2= (a,c)$ su $r_2$.\
	Allora per definizione di proiezione, esistono due tuple in \textit{r}, $t'_1 = (a,b,*)$ e $t'_2= (a,*,c)$ (dove * sta per un valore non noto).\
	Poiché $A \rightarrow C$, allora esiste un solo valore in C associato al valore \textit{a}.\ Dato che $(a,c)$ compare nella proiezione, questo valore è proprio \textit{c}.\
	Ma allora nella tupla $t'_1$ il valore incognito deve essere proprio \textit{c}, quindi $(a,b,c) \in r$.\
\end{proof}

\noindent Si noti che quella enunciata è una condizione sufficiente ma non necessaria.\

\begin{theorem}
	Sia $\mathrm{R\langle T,F \rangle}$ uno schema di relazione, la decomposizione $\rho = \{R_1(T_1), R_2(T_2)\}$ \textbf{preserva i dati} \textit{sse}
	\[
		T_1 \cap T_2 \rightarrow T_1 \in F^+ \quad oppure \quad  T_1 \cap T_2 \rightarrow T_2 \in F^+
	\]
	Esistono criteri anche per decomposizioni in più di due schemi.
\end{theorem}

\noindent Anche se una decomposizione è senza perdite, può comunque presentare dei problemi di conservazione delle dipendenze.\

\subsubsection{Proiezione delle dipendenze}

\begin{definition}
	Dato lo schema $\mathrm{R\langle T,F \rangle}$ e $T_1 \subseteq T$, la \textbf{proiezione di F su T\textsubscript{1}} è
	\[ \pi_{T_1}(F) = \{X \rightarrow Y \in F^+\ |\ XY \subseteq T_1 \} \]
\end{definition}

\noindent Algoritmo banale per il calcolo di $\pi_{T_1}(F)$:

\begin{center}
	for each Y $\subseteq$ T\textsubscript{1} do (Z:=Y\textsuperscript{+}; output Y $\rightarrow$ Z $\cap$ T\textsubscript{1})
\end{center}

\subsubsection{Preservazione delle dipendenze}

\begin{definition}
	Dato lo schema $\mathrm{R\langle T,F \rangle}$, la decomposizione \[\rho = \{R_1, \dots, R_n\}\] \textbf{preserva le dipendenze} se e solo se l'unione delle dipendenze in $\pi_{T_i}(F)$ è una \textbf{\textit{copertura}} di F.
\end{definition}

\noindent \textbf{Proposizione}:\ dato lo schema $\mathrm{R\langle T,F \rangle}$, il problema di stabilire se la decomposizione $\rho = \{R_1,\dots,R_n\}$ preserva le dipendenze ha complessità polinomiale.

\begin{theorem}
	Sia $\rho= \{\mathrm{R\langle T_i,F_i \rangle}\}$ una decomposizione di $\mathrm{R\langle T,F \rangle}$ che \textbf{preservi le dipendenze} e tale che un T\textsubscript{\textit{j}} per qualche \textit{j} \textbf{sia una superchiave per R}.\
	Allora $\rho$ preserva i dati.
\end{theorem}

\section{Forme normali}

Una \textbf{forma normale} è una proprietà di una base di dati relazionale che ne garantisce la ``qualità'', cioè l'assenza di determinati difetti.\
Quando una relazione \textbf{non è normalizzata} presenta ridondanze e si presta a comportamenti poco desiderabili durante gli aggiornamenti.\
Esistono diversi tipi di forma normale
\begin{itemize}
	\item 1FN:\ impone una restrizione sul tipo di relazione, ogni attributo ha un tipo elementare.
	\item 2FN, 3FN e FNBC:\ impongono restrizioni sulle dipendenze.\ FNBC (Boyce-Codd) è la più naturale e la più restrittiva.
\end{itemize}

\subsection{Forma normale di Boyce e Codd}

Una relazione \textit{r} è in \textbf{forma normale di Boyce e Codd} se, per ogni dipendenza funzionale (non banale) X $\rightarrow$ Y definita su di essa, \textbf{X contiene una chiave K di \textit{r}} (\textbf{è una superchiave}).\
La forma normale richiede che i concetti in una relazione siano omogenei (solo proprietà \textit{direttamente associate alla chiave}).\

Intuizione:\ se esiste in R una dipendenza $X \rightarrow A$ non banale ed X \textbf{non è chiave}, allora X modella l'identità di un'entità \textbf{diversa} da quelle modellate dall'intera R.

\begin{theorem}
	$\mathrm{R\langle T,F \rangle}$ è in BCNF $\Leftrightarrow$ per ogni $X \rightarrow A \in F^+$, con $A \not\in X$ (non banale), X è una superchiave.
\end{theorem}

\subsubsection{Algoritmo di analisi}

$\mathrm{R\langle T,F \rangle}$ è decomposta in R\textsubscript{1}(X,Y) e R\textsubscript{2}(X,Y) e su di esse si ripete il procedimento (esponenziale).\

\begin{flushleft}
	Input:\ $\mathrm{R\langle T,F \rangle}$, con F copertura canonica.

	Output:\ \textbf{decomposizione in BCNF che preserva i dati}.
\end{flushleft}

\begin{flushleft}

	$\rho = \{\mathrm{R\langle T,F \rangle}\}$

	while esiste in $\rho$ una $\mathrm{R_i\langle T_i,F_i \rangle}$ non in BCNF per la DF $X \rightarrow A$ do

	\quad $T_a = X^+$

	\quad $F_a = \pi_{T_a}(F_i)$

	\quad $T_b = T_i - X^+ + X$

	\quad $F_b = \pi_{T_b}(F_i)$

	\quad $\rho = \rho - R_i + \{R_a\langle T_a,F_a \rangle,\ R_b \langle T_b,F_b \rangle\}$

	end

\end{flushleft}


\noindent Per ogni dipendenza funzionale che non rispetta la forma normale di Boyce-Codd si prende l'insieme di attributi $T_a$ ottenuto dalla chiusura di X e si calcolano le sue dipendenze funzionali tramite la proiezione di $F_i$ su $T_a$; dopodiché si calcola l'insieme $T_b$ ottenuto dall'insieme degli attributi $T_i$ al quale si toglie la chiusura di X ma si aggiunge X e si calcola la sua proiezione delle dipendenze funzionali.\
Quindi ad ogni ciclo si va a rimuovere dall'insieme le relazioni che non rispettano la forma normale e si aggiunge $R_a$ e $R_b$ (i due insiemi appena calcolati).\

\textbf{Preserva i dati}, ma non necessariamente le dipendenze.\
Tuttavia, \textbf{in alcuni casi} la BCNF ``non è raggiungibile'':\ è necessario ricorrere a una forma normale indebolita.

\subsection{Terza forma normale}

Una relazione \textit{r} è in \textbf{terza forma normale} (\textbf{3NF}) se, per ogni DF (non banale) $X \rightarrow Y$ definita su \textit{r}, è verificata almeno una delle seguenti condizioni:\ X contiene una chiave K di \textit{r} (come nella FNBC), oppure ogni attributo in Y è contenuto in \textit{almeno} una chiave K di \textit{r}.

La 3FN è \textbf{meno restrittiva} della FNBC:\ tollera alcune ridondanze ed anomalie sui dati e certifica meno la qualità dello schema ottenuto; tuttavia, la 3FN è \textbf{sempre ottenibile}, qualsiasi sia lo schema di partenza.

\begin{theorem}
	$\mathrm{R\langle T,F \rangle}$ è in 3FN se per ogni $X\rightarrow A \in F^+$, con $A \not\in X$, \textbf{X è una superchiave} oppure \textbf{A è primo}.
\end{theorem}

\noindent \textbf{Nota}:\
\begin{itemize}
	\item la 3FN ammette una dipendenza non banale e non-da-chiave se gli attributi a destra sono primi;
	\item la NFBC non ammette mai nessuna dipendenza non banale e non-da-chiave.
\end{itemize}

\subsubsection{Algoritmo di Sintesi:\ versione base}

\begin{flushleft}
	Input:\ Un insieme R di attributi e un insieme F di dipendenze su R.\

	Output:\ Una decomposizione $\rho = \mathrm{\{S_i\}_{i=1\dots n}}$ di R tale che preservi dati e dipendenze e ogni $\mathrm{S_i}$ sia in 3NF, rispetto alle proiezioni di F su $\mathrm{S_i}$.\
\end{flushleft}

\begin{enumerate}
	\item Trova una copertura canonica G di F e poni $\rho = \{ \}$.
	\item Sostituisci in G ogni insieme $\mathrm{ X \rightarrow A_1, \dots, X \rightarrow A_h}$ di dipendenze con lo stesso determinante, con la dipendenza $ \mathrm{X \rightarrow A_1 \cdots A_h}$.
	\item Per ogni dipendenza $ \mathrm{X \rightarrow Y}$ in G, metti uno schema con attributi XY in $\rho$.
	\item Elimina ogni schema di $\rho$ contenuto in un altro schema di $\rho$.
	\item Se la decomposizione non contiene alcuno schema i cui attributi costituiscano una superchiave per R, aggiungi ad essa lo schema con attributi W, con W una chiave di R.
\end{enumerate}
