\chapter{Data Definition Language(DDL)}

Introduciamo il \textbf{Data Definition Language} (\textbf{DDL}) SQL, che consiste nell'insieme delle istruzioni SQL che permettono la creazione, modifica e cancellazione delle tabelle, dei domini e degli altri oggetti del database, al fine di definire il suo schema logico.\

Le tabelle (corrispondenti alle relazioni dell'algebra relazionale) vengono definite in SQL mediante l'istruzione \texttt{CREATE TABLE}.\
Questa istruzione
\begin{itemize}
	\item definisce uno schema di relazione e ne crea un'istanza vuota;
	\item specifica attributi, domini e vincoli.
\end{itemize}
L'istruzione \texttt{CREATE TABLE} è seguita da un nome che la caratterizza e dalla lista delle colonne (attributi) di cui si specificano le caratteristiche.\
Alla fine si possono anche specificare eventuali vincoli di tabella.

\begin{flushleft}

	$\mathtt{CREATE\ TABLE\ \langle nome\_tabella\rangle}$

	\quad $\mathtt{nomeColonna_1\ tipoColonna_1\ clausolaDefault_1\ vincoloDiColonna_1,}$

	\quad \dots

	\quad $\mathtt{nomeColonna_k\ tipoColonna_k\ clausolaDefault_k\ vincoloDiColonna_k,}$

	\quad \texttt{vincoli di tabella}

\end{flushleft}

\noindent L'effetto del comando \texttt{CREATE TABLE} è quello di definire uno schema di relazione e di crearne un'istanza vuota, specificandone attributi, domini e vincoli.\
Una volta creata, la tabella è pronta per l'inserimento dei dati che dovranno soddisfare i vincoli imposti.\

La visualizzazione dello schema di una tabella, dopo che è stata creata, può essere ottenuta mediante il comando SQL \texttt{DESCRIBE}:
\begin{center}
	$\mathtt{DESCRIBE\ \langle nomeTabella\rangle}$
\end{center}

\noindent Quindi SQL non è solo un linguaggio di interrogazione (Query Language), ma è anche un linguaggio per la definizione di basi di dati (Data-definition language (DDL)), un linguaggio per stabilire controlli sull'uso dei dati (GRANT) e un linguaggio per modificare i dati.

\subsubsection{I tipi}

I tipi più comuni per i valori degli attributi sono:
\begin{itemize}
	\item \texttt{CHAR(n)} per stringhe di caratteri di lunghezza fissa \texttt{n};
	\item \texttt{VARCHAR(n)} per stringhe di caratteri di lunghezza variabile di al massimo \texttt{n} caratteri;
	\item \texttt{INTEGER} per interi con la dimensione uguale alla parola di memoria standard dell'elaboratore;
	\item \texttt{REAL} per numeri reali con dimensione uguale alla parola di memoria standard dell'elaboratore;
	\item \texttt{NUMBER(p,s)} per numeri con \texttt{p} cifre, di cui \texttt{s} decimali;
	\item \texttt{FLOAT(p)} per numeri binari in virgola mobile, con almeno \texttt{p} cifre significative;
	\item \texttt{DATE} per valori che rappresentano istanti di tempo (in alcuni sistemi, come Oracle), oppure solo date (e quindi insieme ad un tipo \texttt{TIME} per indicare ora, minuti e secondi).
\end{itemize}

\subsubsection{Modificare una tabella}

Ciò che si crea con un \texttt{CREATE} si può eliminare con il comando \texttt{DROP} o cambiare con il comando \texttt{ALTER}.
\begin{flushleft}
	\texttt{CREATE TABLE Nome}

	\quad \texttt{(Attributo Tipo [ValoreDefault] [VincoloAttributo]}

	\qquad \texttt{\{Attributo Tipo [Default] [VincoloAttributo]\},}

	\qquad \texttt{\{VincoloTabella\})}

\end{flushleft}

\begin{flushleft}
	Default := \texttt{DEFAULT} {valore $|$ null $|$ username}
\end{flushleft}

\noindent  Nuovi attributi si possono aggiungere con:
\begin{flushleft}
	\texttt{ALTER TABLE Nome ADD COLUMN NuovoAttr Tipo}
\end{flushleft}

\noindent Con il comando \texttt{ALTER TABLE} è possibile (standard SQL):
\begin{enumerate}
	\item Aggiungere una colonna (\texttt{ADD [COLUMN]})
	\item Eliminare una colonna (\texttt{DROP [COLUMN]})
	\item Modificare la colonna (\texttt{MODIFY})
	\item Aggiungere l'assegnazione di valori di default (\texttt{SET DEFAULT})
	\item Eliminare l'assegnazione di valori di default (\texttt{DROP DEFAULT})
	\item Aggiungere vincoli di tabella (\texttt{ADD CONSTRAINT})
	\item Eliminare vincoli di tabella (\texttt{DROP CONSTRAINT})
	\item Altre opzioni sono possibili nei linguaggi specifici
\end{enumerate}

\subsubsection{Aggiungere una colonna}
Si può aggiungere una colonna in qualsiasi momento se non
viene specificato \texttt{NOT NULL}.\
Sintassi:
\begin{flushleft}
	\texttt{ALTER TABLE nome\_tabella}

	\texttt{ ADD [COLUMN] nome\_col tipo\_col default\_col vincolo\_col}
\end{flushleft}

\noindent In mancanza di altre specifiche, la nuova colonna viene inserita come ultima colonna della tabella.\
Altrimenti è possibile dare questa specifica:
\begin{flushleft}
	$\mathtt{ADD\ COLUMN\ \langle creaDefinizione\rangle}$

	\quad $\mathtt{[FIRST/AFTER\ \langle nomeColonna \rangle]}$
\end{flushleft}

\noindent \texttt{FIRST} permette di aggiungerla come prima colonna, mentre \texttt{AFTER} come colonna
subito dopo la colonna indicata.\

\subsubsection{Eliminare una colonna}

\begin{flushleft}
	\texttt{ALTER TABLE nome\_tabella}

	\texttt{DROP COLUMN nome\_colonna \{RESTRICT/CASCADE\}}
\end{flushleft}

\noindent In SQL standard le opzioni \texttt{RESTRICT}/\texttt{CASCADE} sono alternative ed è obbligatorio specificare l'una o l'altra.
\begin{itemize}
	\item \texttt{RESTRICT}:\ se un'altra tabella si ha un vincolo di integrità referenziale con questa colonna, l'esecuzione del comando drop fallisce.
	\item \texttt{CASCADE}:\ eliminando la colonna, vengono eliminate tutte le dipendenze logiche di altre colonne dello schema da questa.
\end{itemize}

\subsubsection{Modificare una colonna}

Se si vogliono modificare le caratteristiche di una colonna dopo averla definita, occorre eseguire l'istruzione:
\begin{flushleft}
	\texttt{ALTER TABLE nome\_tabella MODIFY}

	\texttt{nome colonna tipo\_col default\_col vincoli\_col}
\end{flushleft}

\subsubsection{Assegnare un valore di default}
Nell'SQL standard è possibile imporre un valore di default col comando specifico \texttt{SET DEFAULT}, con la seguente sintassi
\begin{flushleft}
	\texttt{ALTER TABLE nome\_tabella}

	\texttt{ALTER [COLUMN] nome\_colonna}

	\texttt{SET DEFAULT valore\_default}
\end{flushleft}

\subsubsection{Eliminare un valore di default}

In SQL standard è possibile eliminare un vincolo di default da una colonna mediante l'istruzione:
\begin{flushleft}
	\texttt{ALTER TABLE nome\_tabella}

	\texttt{ALTER [COLUMN] nome\_colonna}

	\texttt{DROP DEFAULT}
\end{flushleft}

\noindent Eseguendo questa istruzione il valore di default diventa automaticamente \texttt{NULL}.

\subsubsection{Aggiungere vincoli di tabella}

Se si vuole aggiungere un vincolo di tabella, si esegue il comando
\begin{flushleft}
	\texttt{ALTER TABLE nome\_tabella}

	\texttt{ADD CONSTRAINT nome\_vincolo vincolo\_di\_tabella  }
\end{flushleft}

\noindent\textbf{Nota bene}:\ Occorre assegnare un nome al vincolo.

\subsubsection{Eliminare vincoli di tabella}

Nello standard SQL, se si vuole eliminare un vincolo di tabella si esegue l'istruzione
\begin{flushleft}
	\texttt{ALTER TABLE nome\_tabella}

	\texttt{DROP CONSTRAINT nome\_vincolo\{RESTRICT/CASCADE\}}
\end{flushleft}

\noindent L'opzione \texttt{RESTRICT} non permette di eliminare vincoli di unicità e di chiave primaria su una colonna se esistono vincoli di chiave esterna che si riferiscono a tale colonna.\
L'opzione \texttt{CASCADE} non opera questa restrizione.\

Da notare che per eliminare un vincolo, esso deve essere definito mediante un identificatore.

\subsubsection{Drop Table}

Si può eliminare una tabella mediante l'istruzione \texttt{DROP TABLE}.\
Nello standard SQL si possono anche specificare le opzioni \texttt{RESTRICT}/\texttt{CASCADE}
\begin{itemize}
	\item \texttt{RESTRICT}:\ se la tabella è utilizzata nella definizione di altri oggetti dello schema, la sua eliminazione viene impedita.
	\item \texttt{CASCADE}:\ vengono eliminate tutte le dipendenze degli altri oggetti dello schema da questa tabella.
\end{itemize}

\section{I vincoli}

I \textbf{vincoli di integrità} consentono di limitare i valori ammissibili per una determinata colonna della tabella in base a specifici criteri.\
I vincoli di integrità intrarelazionali (ossia che non fanno riferimento ad altre relazioni) sono:
\begin{itemize}
	\item \texttt{NOT NULL}
	\item \texttt{UNIQUE} definisce chiavi
	\item \texttt{PRIMARY KEY}:\ chiave primaria (una sola, implica \texttt{NOT NULL})
	\item \texttt{CHECK}
\end{itemize}

\subsection{UNIQUE}

Può essere espresso in due forme:\ nella definizione di un attributo, se forma da solo la chiave, oppure come elemento separato.\
Il vincolo \texttt{unique} utilizzato nella definizione dell'attributo indica che non ci possono essere due valori uguali in quella colonna.\
È una chiave della relazione, ma non una chiave primaria.

\subsubsection{Vincolo unique per insiemi di attributi}

Il vincolo di unicità può anche essere riferito a coppie o insiemi di attributi.\
Ciò non significa che per gli attributi dell'insieme considerato ogni singolo valore deve apparire una sola volta, ma che non ci siano due dati (righe) per cui l'insieme dei valori corrispondenti a quegli attributi siano uguali.\

In questo caso il vincolo viene dichiarato dopo aver  dichiarato tutte le colonne mediante un vincolo di tabella, utilizzando il comando

\begin{center}
	\texttt{Unique (lista\_attributi)}
\end{center}

\subsection{Primary key}

Due forme:\ definizione di un attributo, se formato da solo la chiave oppure come elemento separato.\
Il vincolo \texttt{primary key} è simile a \texttt{unique}, ma definisce la chiave primaria della relazione, ossia un attributo che individua univocamente un dato.\

Implica sia il vincolo \texttt{unique} che il vincolo \texttt{not null} (non è ammesso che per uno degli elementi della tabella questo valore sia non definito).\
Serve ad identificare univocamente i soggetti del dominio.\
Questo vincolo permette spesso il collegamento fra due tabelle.\

\subsubsection{Chiave primaria con insiemi di attributi}

Analogamente al vincolo \texttt{unique}, anche il vincolo di chiave primaria può essere definito su un insieme di elementi.\
In tal caso la sintassi è simile a quella di \texttt{unique}:
\begin{center}
	\texttt{Primary key (lista\_attributi)}
\end{center}

\subsection{Vincoli interrelazionali}

I vincoli interrelazionali sono quei vincoli che vengono imposti quando gli \textit{attributi di due diverse tabelle devono essere messi in relazione}.\
Questo è fatto per soddisfare l'esigenza di un database di \textbf{non essere ridondante} e di avere i \textbf{dati sincronizzati}.\
Se due tabelle gestiscono gli stessi dati, è bene che di essi non ce ne siano più copie, sia allo scopo di non occupare troppa memoria, sia affinché le modifiche fatte su dati uguali utilizzati da due tabelle siano coerenti.

\textbf{References} e \textbf{Foreign Key} permettono di definire vincoli di integrità referenziale.\
Di nuovo \textbf{due sintassi}:\ per singoli attributi (come vincolo di colonna), oppure su più attributi (come vincolo di tabella).\
È possibile definire politiche di \textbf{reazione alla violazione} (ossia stabilire l'azione che il DBMS deve compiere quando si viola il vincolo).

\subsection{Check}

Un vincolo di \texttt{Check} richiede che una colonna, o una combinazione di colonne, soddisfi una condizione per ogni riga della tabella:\ deve essere una espressione booleana che è valutata usando i valori della colonna che vengono inseriti o aggiornati nella riga.\
Può essere espresso sia come \textbf{vincolo di riga} che come \textbf{vincolo di tabella}.\
Se è espresso come \textbf{vincolo di riga}, può coinvolgere solo l'attributo su cui è definito, mentre se serve eseguire un check che coinvolge due o più attributi, si deve definire come \textbf{vincolo di tabella}.

\subsection{Reazione alla violazione}

Quando si crea un vincolo \texttt{foreign key} in una tabella, in SQL standard si può specificare l'azione da intraprendere quando delle righe nella tabella riferita vengono cancellate o modificate.\
Tali reazioni alla violazione vengono dichiarate al momento della definizione dei vincoli di \texttt{foreign key} rispettivamente mediante i comandi
\begin{itemize}
	\item \texttt{On Delete}
	\item \texttt{On Update}
\end{itemize}

\subsubsection{Reazioni alla violazione On Delete}

\textbf{Impedire il delete} (\texttt{No Action}):\ blocca il delete delle righe dalla tabella riferita quando ci sono righe che dipendono da essa.\
Questa è l'azione che viene attivata per \textit{default}.

\noindent \textbf{Generare un delete a catena} (\texttt{Cascade}):\
cancella tutte le righe dipendenti dalla tabella quando la corrispondente riga è cancellata dalla tabella riferita.\

\noindent \textbf{Assegnare valore} \texttt{NULL} (\texttt{Set Null}):\ assegna \texttt{NULL} ai valori della colonna che ha il vincolo foreign key nella tabella quando la riga corrispondente viene cancellata dalla tabella riferita.

\noindent \textbf{Assegnare il valore di default} (\texttt{Set Default}):\ assegna il valore di default ai valori della colonna che ha il vincolo foreign key nella tabella quando la riga corrispondente viene cancellata dalla tabella riferita.

\subsubsection{Reazioni alla violazione On Update}

Nello standard SQL la reazione alla violazione può anche essere attivata quando i dati della tabella riferita vengono aggiornati.\
Viene attivato mediante il comando \texttt{ON UPDATE} seguito da:
\begin{itemize}
	\item \texttt{Cascade}:\ Le righe della tabella referente vengono impostati ai valori della tabella riferita.
	\item \texttt{Set Null}:\ i valori della tabella referente vengono  impostati a \texttt{NULL}.
	\item \texttt{Set Default}:\ i valori della tabella referente vengono impostati al valore di default.
	\item \texttt{No Action}:\ rifiuta gli aggiornamenti che violino l'integrità referenziale.
\end{itemize}

\section{Viste}

Le \textbf{viste logiche} o \textbf{viste} o \textbf{view} possono essere definite come delle tabelle virtuali, i cui dati sono riaggregazioni dei dati contenuti nelle tabelle ``fisiche'' presenti nel database.\
Le tabelle fisiche sono gli unici veri contenitori di dati.\
Le viste non contengono dati fisicamente diversi dai dati presenti nelle tabelle, ma forniscono una \textit{diversa visione}, \textit{dinamicamente aggiornata}, di quegli stessi dati.\
La vista appare all'utente come una normale tabella, in cui può effettuare \textbf{interrogazioni} e, limitatamente ai suoi privilegi, anche \textbf{modificare dei dati}.

\noindent Vantaggi
\begin{itemize}
	\item Le viste \textbf{semplificano la rappresentazione dei dati}.\ Oltre ad assegnare un nome alla vista, la sintassi dell'istruzione \texttt{CREATE VIEW} consente di cambiare i nomi delle colonne.\
	\item Le viste possono essere anche estremamente \textbf{convenienti per svolgere una serie di query molto complesse}.
	\item Le viste consentono di \textbf{proteggere i database}:\ le view ad accesso limitato possono essere utilizzate per controllare le informazioni alle quali accede un certo utente del database.
	\item Le viste consentono inoltre di \textbf{convertire le unità di misura e creare nuovi formati}.
\end{itemize}

\noindent Limitazioni

\begin{itemize}
	\item Non è possibile utilizzare gli operatori booleani \texttt{UNION}, \texttt{INTERSECT} ed \texttt{EXCEPT}.
	\item Gli operatori \texttt{INTERSECT} ed \texttt{EXCEPT} possono essere realizzati mediante una select semplice.\ La stessa cosa non si può dire dell'operatore \texttt{UNION}.
	\item Non è possibile utilizzare la clausola \texttt{ORDER BY}.
\end{itemize}

\subsubsection{Sintassi}

Il comando DDL che consente di definire una vista ha la seguente sintassi

\begin{flushleft}
	\texttt{CREATE VIEW NomeVista [(ListaAttributi)] \textbf{AS} SelectSQL}

	\texttt{[with [local $|$ cascaded] checkoption]}
\end{flushleft}

\noindent I nomi delle colonne indicati nella lista attributi sono i nomi assegnati alle colonne della vista, che corrispondono ordinatamente alle colonne elencate nella select.\
Se questi non sono specificati, le colonne della vista assumono gli stessi nomi di quelli della/e tabella/e a cui si riferisce.

\subsubsection{Modifica di una vista}

Sebbene il \textbf{contenuto} di una vista sia \textbf{dinamico}, \textit{la sua struttura non lo è}.\
Se una vista è definita su una subquery
\[\mathtt{Select^*\ From\ T_1}\]
e in seguito alla tabella $\mathtt{T_1}$ viene aggiunta una colonna, questa nuova definizione non si estende alla vista:\ la vista conterrà sempre le stesse colonne che aveva prima dell'inserimento della nuova colonna in $\mathtt{T_1}$.\

\subsubsection{Viste di gruppo}

Una \textbf{vista di gruppo} è una vista di cui una delle colonne è una funzione di gruppo.\
In questo caso è obbligatorio assegnare un nome alla colonna della vista corrispondente alla funzione.\

È una vista di gruppo anche una vista che è definita in base ad una vista di gruppo.

\subsubsection{Eliminazione delle vista}

Le viste si eliminano col comando \texttt{Drop View}.\
Sintassi:
\[\mathtt{Drop\ View\ nomeView\ \{Restrict/Cascade\}}\]

\noindent\texttt{Restrict}:\ la vista viene eliminata solo se non è riferita nella definizione di altri oggetti.

\noindent\texttt{Cascade}:\ oltre che essere eliminata la vista, vengono eliminate tutte le dipendenza da tale vista di altre definizioni dello schema.

\subsubsection{Viste modificabili}

Le tabelle delle viste si interrogano come le altre, ma in generale non  si possono modificare.\
Deve esistere una corrispondenza biunivoca fra le righe della vista e le righe di una tabella di base, ovvero:

\begin{enumerate}
	\item \texttt{SELECT} senza \texttt{DISTINCT} e solo di attributi
	\item \texttt{FROM} una sola tabella modificabile
	\item \texttt{WHERE} senza SottoSelect
	\item \texttt{GROUP BY} e \texttt{HAVING} non sono presenti nella definizione.
\end{enumerate}
Possono esistere anche delle restrizioni su \texttt{SELECT} su viste definite usando \texttt{GROUP BY}.

\subsubsection{Aggiornamento delle viste}

Le operazioni \texttt{INSERT/UPDATE/DELETE} sulle viste non erano permesse nelle prime edizioni di SQL.\
I nuovi DBMS permettono di farlo con certe limita\-zioni dovute alla definizione della vista stessa.

Ha senso aggiornare una vista?\
Dopotutto si potrebbe aggiornare la tabella di base direttamente\dots tuttavia è molto utile nel caso di accesso dati controllato.

L'opzione \texttt{With Check Option} messa alla fine della definizione di una vista assicura che le operazioni di inserimento e di modifica dei dati effettuate utilizzando la vista soddisfino la clausola \texttt{Where} della subquery.\

Supponiamo che una \textbf{vista V\textsubscript{1}} sia definita in termini di un'altra vista V\textsubscript{2}.\
Se si crea V\textsubscript{1} specificando la clausola \texttt{With Check Option}, il DBMS verifica che la nuova tupla \texttt{t} inserita soddisfi \textbf{sia la definizione di V\textsubscript{1} che quella di V\textsubscript{2}} (e di tutte le altre eventuali viste da cui V\textsubscript{1} dipende), \textbf{indipendentemente} dal fatto che V\textsubscript{2} sia stata a sua volta definita \texttt{With Check Option}.\
Questo comportamento di default è equivalente a definire V\textsubscript{1}
\begin{center}
	\texttt{WITH CASCADED CHECK OPTION}
\end{center}

\noindent Lo si può alterare definendo V\textsubscript{1}
\begin{center}
	\texttt{WITH LOCAL CHECK OPTION}
\end{center}

\noindent Ora il DBMS verifica solo che \texttt{t} soddisfi la specifica di V\textsubscript{1} e quelle di tutte e \textbf{sole le viste da cui V\textsubscript{1} dipende per cui è stata specificata} la clausola \texttt{With Check Option}.

\subsection{Vantaggi delle viste}

\subsubsection{Facilitazione nell'accesso ai dati}

In generale uno dei requisiti per la progettazione di un database relazionale è la \textbf{normalizzazione dei dati}.\
Sebbene la forma normalizzata del database permetta una corretta modellazione della realtà che il DB rappresenta, a volte dal punto di vista dell'utente comporta una \textit{maggiore difficoltà di comprensione} rispetto a una rappresentazione non normalizzata.\
Le viste permettono di fornire all'utente i dati in una forma \textit{più intuitiva}.

\subsubsection{Diverse visioni dei dati}

Esistono dei dati che sono presenti nelle tabelle del database, che sono \textbf{poco significativi per l'utente} e altri che \textbf{devono essere nascosti all'utente}.\
L'uso delle viste da parte dell'utente permette di \textbf{limitare il suo accesso ai dati del database}, eliminando quelli non interessanti per lui e quelli che devono essere tenuti nascosti.
L'uso delle viste può essere considerato come una \textbf{tecnica per assicurare la sicurezza dei dati}.

\subsubsection{Indipendenza logica}

Un vantaggio delle viste riguarda l'\textbf{indipendenza logica} delle applicazioni e delle operazioni eseguite dagli utenti rispetto alla struttura logica dei dati.\
Ciò significa che è possibile poter operare \textit{modifiche allo schema senza dover apportare modifiche alle applicazioni} che utilizzano il database.

\subsubsection{Utilità delle viste}

\begin{itemize}
	\item Per nascondere certe modifiche all'organizzazione logica dei dati (indipendenza logica)
	\item Per offrire visioni diverse degli stessi dati senza ricorrere a duplicazioni
	\item Per rendere più semplici, o per rendere possibili, alcune interrogazioni
\end{itemize}

\section{Procedure e Trigger}

\subsection{Trigger}

Un \textbf{trigger} definisce un'azione che il database deve attivare automaticamente quando si verifica (nel database) un determinato evento.\
Possono essere utilizzati
\begin{itemize}
	\item per \textbf{migliorare l'integrità} referenziale dichiarativa,
	\item per \textbf{imporre regole complesse} legate all'attività del database,
	\item per \textbf{effettuare revisioni} sulle modifiche dei dati.
\end{itemize}

\noindent L'esecuzione dei trigger è quindi trasparente all'utente.\
I trigger vengono eseguiti automaticamente dal database quando specifici tipi di comandi (\textbf{eventi}) di manipolazione dei dati vengono eseguiti su specifiche tabelle.\
Tali comandi comprendono i comandi DML \texttt{insert}, \texttt{update} e \texttt{delete}, ma gli ultimi DBMS prevedono anche trigger su istruzioni DDL come \texttt{Create View}.\
Anche gli aggiornamenti di specifiche colonne possono essere utilizzati come trigger di eventi.

\subsubsection{Trigger a livello di riga}

I trigger a livello di riga vengono eseguiti una volta per ciascuna riga modificata in una transazione; vengono spesso utilizzati in applicazioni di revisione dei dati e si rivelano utili per \textbf{operazioni di audit dei dati e per mantenere sincronizzati i dati distribuiti}.\
Per creare un trigger a livello di riga occorre specificare la clausola \[\mathtt{FOR\ EACH\ ROW}\] nell'istruzione \texttt{create trigger}.

\subsubsection{Trigger a livello di istruzione}

I trigger a livello di istruzione vengono eseguiti \textbf{una sola volta per ciascuna transazione}, indipendentemente dal numero di righe che vengono modificate (quindi anche se, ad esempio, in una tabella vengono inserite 100 righe, il trigger verrà eseguito solo una volta).\
Vengono pertanto utilizzati per \textbf{attività correlate ai dati}:\ vengono utilizzati di solito per \textbf{imporre misure aggiuntive di sicurezza sui tipi di transazione che possono essere eseguiti su una tabella}.\

È il tipo di trigger \textit{predefinito} nel comando \texttt{create trigger} (ossia non occorre specificare che è un trigger al livello di istruzione).

\subsubsection{Struttura}

I trigger si basano sul paradigma evento-condizione-azione (ECA).\

\noindent L'istruzione \texttt{Create Trigger} seguita dal \textbf{nome} assegnato al trigger.\

\noindent \textbf{Tipo di trigger}, \texttt{Before}/\texttt{After}.\

\noindent\textbf{Evento che scatena} il trigger \texttt{Insert}/\texttt{Delete}/\texttt{Update}.\

\noindent \texttt{For each row} se si vuole specificare trigger al livello di riga (altrimenti nulla per trigger al livello di istruzione).\

\noindent Specificare a quale \textbf{tabella} si applica.\

\noindent\textbf{Condizione} che si deve verificare perché il trigger sia eseguito.\

\noindent\textbf{Azione}, definita dal codice da eseguire se si verifica la condizione.

\subsubsection{Tipi di Trigger}

\texttt{Before} e \texttt{After}:\ i trigger possono essere eseguiti prima o dopo l'utilizzo dei comandi \texttt{insert}, \texttt{update} e \texttt{delete}; all'interno del trigger è possibile fare riferimento ai vecchi e nuovi valori coinvolti nella transizione.\
Occorre utilizzare la clausola
\begin{center}
	\[\mathtt{Before/After\ \langle tipoDiEvento\rangle}\]
\end{center}

\noindent Se si tratta di un trigger \texttt{before update}:
\begin{itemize}
	\item per valori \textbf{vecchi} intendiamo \textit{i valori che sono nella tabella} e che vogliamo modificare,
	\item per \textbf{nuovi} quelli che \textit{vogliamo inserire} al posto di quelli vecchi.
\end{itemize}

\noindent Se si tratta di un trigger \texttt{after update}

\begin{itemize}
	\item per \textbf{vecchi} intendiamo quelli \textit{che c'erano prima} dell'update,
	\item per \textbf{nuovi} quelli \textit{presenti nella tabella alla fine della modifica}.
\end{itemize}

\noindent Un \textbf{trigger} è \textbf{attivo} quando, in corrispondenza di certi eventi, \textit{modifica lo stato della base di dati}.\
Un \textbf{trigger} è \textbf{passivo} se serve a \textit{provocare il fallimento} della transazione corrente sotto certe condizioni.

\noindent\textbf{Instead Of}:\ per specificare che cosa fare invece di eseguire le azioni che hanno attivato il trigger.\
Ad esempio, è possibile utilizzare un trigger \texttt{INSTEAD OF} per reindirizzare le \texttt{INSERT} in una tabella verso una tabella differente o per aggiornare con update più tabelle che siano parte di una vista.

I trigger \texttt{instead-of} possono essere definiti su viste (relazionali od oggetto).\
I trigger instead-of devono essere a livello di riga.

\section{Controllo degli accessi}

Ogni componente dello schema (risorsa) può essere protetto (tabelle, attributi, viste, domini,\ \dots).\
Il possessore della risorsa (colui che la crea) assegna dei privilegi agli altri utenti; un utente predefinito (\verb|_system|) rappresenta l'amministratore della base di dati ed ha completo accesso alle risorse.\
Ogni privilegio è caratterizzato dalla risorsa a cui si riferisce, dall'utente che concede il privilegio, dall'utente che riceve il privilegio, dall'azione che viene permessa sulla risorsa e se il privilegio può esser trasmesso o meno ad altri utenti.\
Tipi di privilegi:

\begin{itemize}
	\item \texttt{SELECT}:\ lettura di dati.
	\item \texttt{INSERT [(Attributi)]}:\ inserire record (con valori non nulli per gli attributi).
	\item \texttt{DELETE}:\ cancellazione di record.
	\item \texttt{UPDATE [(Attributi)]}:\ modificare record (o solo gli attributi).
	\item \texttt{REFERENCES [(Attributi)]}:\ definire chiavi esterne in altre tabelle che riferiscono gli attributi.
	\item \texttt{WITH GRANT OPTION}:\ si possono trasferire i privilegi ad altri utenti.
\end{itemize}

\noindent Chi crea lo schema del DB è l'unico che può fare \texttt{CREATE}, \texttt{ALTER} e \texttt{DROP}.\
Chi crea la tabella stabilisce i modi in cui altri possono farne uso:

\begin{center}
	\texttt{GRANT Privilegi ON Oggetto TO Utenti [WITH GRANT OPTION]}
\end{center}

\noindent Per revocare il privilegio:
\begin{center}
	\texttt{revoke Privileges on Resource from Users [restrict $|$ cascade]}
\end{center}

\noindent La revoca deve essere fatta dall'utente che aveva concesso i privilegi:\ \texttt{re\-strict} (di default) specifica che il comando non deve essere eseguito qualora la revoca dei privilegi all'utente comporti qualche altra revoca (dovuta ad un precedente grant option), \texttt{cascade} invece forza l'esecuzione del comando.

Chi definisce una tabella o una view ottiene automaticamente tutti i privilegi su di esse ed è l'unico che può fare \texttt{DROP} e può autorizzare altri ad usarla con \texttt{GRANT}.\
Nel caso di viste il ``creatore'' ha i privilegi che ha sulle tabelle usate nella definizione.

\section{Indice e catalogo}

\subsubsection{Creazione di indici}
Non è un comando standard dell'SQL e quindi ci sono differenze nei vari sistemi.\

\begin{flushleft}
	$\mathtt{CREATE\ INDEX\ NomeIdx\ ON\ Tabella(Attributi)}$

	$\mathtt{CREATE\ INDEX\ NomeIdx\ ON\ Tabella }$

	\quad $\mathtt{WITH\ STRUCTURE = BTree,\ KEY = (Attributi)}$

	$\mathtt{DROP\ INDEX\ NomeIdx}$
\end{flushleft}

\subsubsection{Catalogo (dei metadati)}

Alcuni esempi di tabelle, delle quali si mostrano solo alcuni attributi, sono:
\begin{itemize}
	\item Tabella delle password:\ \texttt{PASSWORD(username,password)}.
	\item Tabella delle basi di dati:\ \texttt{SYSDB(dbname,creator,dbpath,remarks)}.
	\item Tabella delle tabelle (type = view or table):\ \texttt{SYSTABLES(name,crea\-tor,type,colcount,filename,remarks)}.
	\item Tabella degli attributi:\ \texttt{SYSCOLUMNS(name,tbname,tbcreator,colno, coltype,lenght,default,remarks)}.
	\item Tabella degli indici:\ \texttt{SYSINDEXES(name,tbname,creator,uniquerule, colcount)}.
\end{itemize}
E altre ancora sulle viste, vincoli, autorizzazioni, \dots (una decina).

