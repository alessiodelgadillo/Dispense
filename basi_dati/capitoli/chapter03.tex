\chapter{Il modello relazionale}

Proposto da \textbf{E.\ F.\ Codd} nel \textbf{1970} per favorire l'indipendenza dei dati.\
Disponibile in DBMS reali nel \textbf{1981} (non è facile implementare l'indipendenza con efficienza e affidabilità!).\
Si basa sul concetto matematico di \textbf{relazione} (con una variante).\
Le relazioni hanno naturale rappresentazione per mezzo di \textbf{tabelle}.

\noindent \textbf{Relazione matematica}:\ come nella teoria degli insiemi.
\begin{center}
	$D_1, \dots, D_n$ (\textit{n} insiemi anche non distinti)
\end{center}
\textbf{Prodotto cartesiano} $D_1 \times \dots \times D_n$:
\begin{center}
	l'insieme di tutte le \textit{n-uple} \textit{(d\textsubscript{1} , \dots, d\textsubscript{n})} tali che $d_1 \in D_1 , \dots, d_n \in D_n$
\end{center}
\textbf{Relazione matematica} su $D_1, \dots, D_n$:
\begin{center}
	un sottoinsieme di $D_1 \times \dots \times D_n$
\end{center}
$D_1, \dots, D_n$ sono i \textbf{domini} della relazione.

\textbf{Osservazione}:\ una relazione è un insieme quindi
\begin{itemize}
	\item non c'è ordinamento fra le \textit{n}-uple;
	\item le \textit{n}-uple sono distinte;
	\item ciascuna \textit{n}-upla è ordinata, l'\textit{i}-esimo valore proviene dall'\textit{i}-esimo dominio.
\end{itemize}

\subsubsection{Tabelle e relazioni}

Una tabella rappresenta una \textbf{relazione} se i valori di ogni colonna sono fra loro omogenei e se le righe e le intestazioni delle colonne sono diverse fra loro.

In una tabella che rappresenta una relazione l'ordinamento tra le righe e l'ordinamento tra le colonne sono irrilevanti.

\subsubsection{Modello basato su valori}

Il \textbf{modello relazionale è basato su valori}.\
Ciò significa che i riferimenti fra dati in relazioni diverse sono rappresentati per mezzo di valori dei domini che compaiono nelle ennuple.

\subsubsection{Vantaggi della struttura basata su valori}
\begin{itemize}
	\item \textbf{Indipendenza dalle strutture fisiche} (si potrebbe avere anche con puntatori di alto livello) che possono cambiare dinamicamente.\ La rappresentazione logica dei dati (costituita dai soli valori) non fa riferimento a quella fisica.
	\item Si rappresenta solo ciò che è \textbf{rilevante} dal punto di vista dell'applicazione.
	\item I dati sono \textbf{portabili} più facilmente da un sistema ad un altro.
	\item I \textbf{puntatori sono direzionali}.
\end{itemize}

\section{Il modello relazionale}

\begin{definition}
	I meccanismi per definire una base di dati con il modello relazionale sono l'\textbf{ennupla} e la \textbf{relazione}.\
\end{definition}

\noindent Un \textbf{tipo ennupla} T è un insieme finito di coppie \[\langle\mathtt{Attributo, Tipo\ elementare}\rangle\]
Se $T$ è un tipo ennupla, $R(T)$ è \textbf{lo schema della relazione \textit{R}}; lo schema di una base di dati è un insieme di schemi di relazione $R_i(T_i)$.\
Un'\textbf{istanza} di uno schema $R(T)$ è un insieme finito di ennuple di tipo $T$.

Uguaglianza di due tipi ennupla, due ennuple, due tipi relazione.

\subsubsection{Informazione incompleta}

Il modello relazionale impone ai dati una struttura rigida:\ le informazioni sono rappresentate per mezzo di ennuple, ma solo alcuni formati di ennuple sono ammessi, cioè quelli che corrispondono agli schemi di relazione.

\textbf{Valore nullo}:\ denota l'assenza di un valore del dominio (e non è un valore del dominio).\
\texttt{t[A]}, per ogni attributo $A$, è un valore del dominio $dom(A)$ oppure il valore nullo \texttt{NULL}.\
Si possono (e devono) imporre restrizioni sulla presenza di valori nulli.

\subsection{Vincoli di integrità}

Esistono istanze di basi di dati che, pur sintatticamente corrette, non rappresentano informazioni possibili per l'applicazione di interesse e che quindi generano informazioni senza significato.

Uno \textbf{schema relazionale} è costituito da un insieme di \textbf{schemi di relazione} e da un \textbf{insieme di vincoli} d'integrità sui possibili valori delle estensioni delle relazioni.\
Un \textbf{vincolo d'integrità} è una proprietà che deve essere soddisfatta dalle istanze che rappresentano informazioni corrette per l'applicazione.\
Un vincolo è espresso mediante una funzione booleana (un predicato):\ associa ad ogni istanza il valore vero o falso.

\noindent Perché usare i vincoli di integrità?
\begin{itemize}
	\item Descrizione più accurata della realtà.
	\item Contributo alla ``qualità dei dati''.
	\item Utili nella progettazione.
	\item Usati dai DBMS nella esecuzione delle interrogazioni.
	\item Non tutte le proprietà di interesse sono rappresentabili per mezzo di vincoli formulabili in modo esplicito.
\end{itemize}

\subsubsection{Tipi di vincoli}

Vincoli \textbf{intrarelazionali}:\ sono i vincoli che devono essere rispettati dai valori contenuti nella relazione considerata; vincoli su valori (o di dominio) e vincoli di ennupla.

\noindent Vincoli \textbf{interrelazionali}:\
sono i vincoli che devono essere rispettati da valori contenuti in relazioni diverse.

\subsubsection{Vincoli di ennupla}

I \textbf{Vincoli di ennupla} esprimono condizioni sui valori di ciascuna ennupla, indipendentemente dalle altre ennuple.\
\textbf{Vincoli di dominio}:\ caso particolare, coinvolgono un solo attributo.

\subsection{Chiave}

\textbf{Informalmente}:\ una \textbf{chiave} è un insieme di attributi che identifica le ennuple di una relazione.\

\noindent\textbf{Formalmente}:\ un insieme $K$ di attributi è \textbf{superchiave} per $r$ se $r$ non contiene due ennuple (distinte) $t_1\ \mathrm{e}\ t_2\ \mathrm{con}\ t_1[K] = t_2[K]$.\
$K$ è \textbf{chiave} per $r$ se è una superchiave minimale per $r$ (cioè non contiene un'altra superchiave).

\subsubsection{Esistenza delle chiavi}

Una relazione non può contenere ennuple distinte ma con valori uguali (una relazione è un sottoinsieme del prodotto cartesiano).\
Ogni relazione ha \textit{sicuramente} come \textbf{superchiave} l'insieme di tutti gli attributi su cui è definita e quindi ogni relazione \textit{ha (almeno) una chiave}.\
L'esistenza delle chiavi garantisce l'accessibilità a ciascun dato della base di dati.\
Le chiavi permettono di correlare i dati in relazioni diverse:
\begin{center}
	il modello relazionale è basato su valori
\end{center}
La presenza di valori nulli fra i valori di una chiave non permette di identificare le ennuple e di realizzare facilmente i riferimenti da altre relazioni.

Una \textbf{chiave primaria} è una chiave su cui non sono ammessi valori nulli.

\subsection{Integrità referenziale}

Nel modello relazionale le informazioni in relazioni diverse sono correlate attraverso valori comuni, in particolare vengono spesso presi in considerazione i valori delle chiavi (primarie).\

Le correlazioni debbono essere ``coerenti''.

\subsubsection{Vincolo di integrità referenziale}

Un vincolo di \textbf{integrità referenziale} (``\textbf{foreign key}'') fra gli attributi X di una relazione $R_1$ e un'altra relazione $R_2$ impone ai valori su $X$ in $R_1$ di comparire come valori della chiave primaria di $R_2$.

\section{Trasformazione di schemi}

\subsection{Progettazione logica relazionale}

Si tratta di ``\textit{tradurre}'' lo schema concettuale in uno schema logico relazionale che rappresenti gli stessi dati in maniera \textbf{corretta} ed \textbf{efficiente}; questo richiede una ristrutturazione del modello concettuale.

La trasformazione di uno schema ad oggetti in uno schema relazionale avviene eseguendo i seguenti passi:

\begin{enumerate}
	\item rappresentazione delle associazioni uno ad uno e uno a molti;
	\item rappresentazione delle associazioni molti a molti o non binarie;
	\item rappresentazione delle gerarchie di inclusione;
	\item identificazione delle chiavi primarie;
	\item rappresentazione degli attributi multivalore;
	\item appiattimento degli attributi composti.
\end{enumerate}
Obiettivo:
\begin{itemize}
	\item rappresentare le stesse informazioni;
	\item \textbf{minimizzare la ridondanza};
	\item produrre uno schema comprensibile, per facilitare la scrittura e manutenzione delle applicazioni.
\end{itemize}

\subsection{Rappresentazione delle associazioni uno a molti}

Le associazioni uno a molti si rappresentano aggiungendo agli attributi della relazione rispetto a cui l'associazione è univoca una chiave esterna che riferisce l'altra relazione.

\subsection{Rappresentazione delle associazioni uno ad uno}

Le associazioni uno a uno si rappresentano aggiungendo la chiave esterna ad una qualunque delle due relazioni che riferisce l'altra relazione, preferendo quella rispetto a cui l'associazione è \textbf{totale}, nel caso in cui esista un vincolo di totalità.\

\subsection{Vincoli sulla cardinalità delle associazioni uno a molti e uno ad uno}

La direzione dell'associazione rappresentata dalla chiave esterna è detta ``\textit{la diretta}'' dell'associazione.\
Vincoli sulla cardinalità delle associazioni uno a molti ed uno ad uno:
\begin{itemize}
	\item \textbf{univocità della diretta};
	\item \textbf{totalità della diretta}:\ si rappresenta imponendo un vincolo \texttt{not null} sulla chiave esterna;
	\item \textbf{univocità dell'inversa} e \textbf{totalità della diretta}:\ si rappresenta imponendo un vincolo not null ed un vincolo di chiave sulla chiave esterna.
\end{itemize}

\subsection{Rappresentazione delle associazioni molti a molti}

Un'associazione molti a molti tra due classi si rappresenta aggiungendo allo schema una nuova relazione che contiene due chiavi esterne che riferiscono le due relazioni coinvolte; la chiave primaria di questa relazione è costituita dall'insieme di tutti i suoi attributi.

\subsection{Rappresentazione delle gerarchie fra classi}

Il modello relazionale non può rappresentare direttamente le gerarchie, bisogna eliminarle, sostituirle con classi e
relazioni:
\begin{enumerate}
	\item accorpamento delle figlie della gerarchia nel genitore (\textbf{relazione unica});
	\item accorpamento del genitore della gerarchia nelle figlie (\textbf{partizionamen\-to orizzontale});
	\item sostituzione della gerarchia con relazioni (\textbf{partizionamento verticale}).
\end{enumerate}

\subsubsection{Accorpamento delle figlie della gerarchia nel genitore}

Se $A_{0}$ è la classe genitore di $A_{1}$ ed $A_{2}$, le classi $A_{1}$ ed $A_{2}$ vengono eliminate ed accorpate ad $A_{0}$.\
Ad $A_{0}$ viene aggiunto un \textbf{attributo} (\textit{discriminatore}) che indica da quale delle classi figlie deriva una certa istanza, e gli \textbf{attributi di \textit{A\textsubscript{1}} ed \textit{A\textsubscript{2}}} vengono assorbiti dalla classe genitore, e assumono valore nullo sulle istanze provenienti dall'altra classe.

Infine, una \textbf{relazione relativa a solo una delle classi figlie} viene acquisita dalla classe genitore e avrà comunque cardinalità minima uguale a 0, in quanto gli elementi dell'altra classe non contribuiscono alla relazione.

\subsubsection{Accorpamento del genitore della gerarchia nelle figlie }

La classe genitore $A_{0}$ viene eliminata e le classi figlie $A_{1}$ ed $A_{2}$ ereditano le proprietà (attributi, identificatore e relazioni) della classe genitore.\
Le relazioni che coinvolgono la classe genitore vengono sdoppiate, coinvolgendo ciascuna delle classe figlie.\

Il partizionamento orizzontale divide gli elementi della superclasse in più relazioni diverse, per cui \textit{non è possibile mantenere un vincolo referenziale verso la superclasse stessa}; in conclusione, questa tecnica non si usa se nello schema relazionale grafico c'è una freccia che entra nella superclasse;

\subsubsection{Sostituzione della gerarchia con relazioni}
La gerarchia si trasforma in due associazioni uno a uno che legano rispettivamente la classe genitore con le classi figlie.\
In questo caso non c'è un trasferimento di attributi o di associazioni e le classi figlie $A_{1}$ ed $A_{2}$ sono identificate esternamente dalla classe genitore $A_{0}$.

Nello schema ottenuto vanno aggiunti dei vincoli:\ ogni occorrenza di $A_{0}$ non può partecipare contemporaneamente alle due associazioni, e se la gerarchia è totale, deve partecipare ad almeno una delle due.\

