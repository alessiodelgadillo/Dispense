\chapter{From the Cloud to the IoT}

L'\textbf{Internet of Things} (\textbf{IoT}) è un'area in continua in espansione che prevede la connessione di dispositivi elettronici o cyber-fisici (Android wearables, Google Nest, Amazon Echo, Domotica) al fine di creare una serie di applicazioni innovative impressionanti come \textit{embedded AI}, \textit{sistemi di guida autonoma}, \textit{flotte di droni}, \textit{smart agricolture}, \textit{smart city},\ \dots

\vspace{12pt}
\noindent``\textit{More than 10 bilion active IoT devices in 2021, more than 25.4 bilion estimated for 2030}.''\quad [\texttt{dataprot.net}]
\vspace{12pt}

\noindent Tutte le applicazioni di IoT, non solo la parte di sensoristica o di attuazione, ma anche reti di sensori che sono dispiegate vicino al terreno, producono una quantità di dati enorme; fortunatamente il cloud è abbastanza grande per accoglierli.\
Tuttavia per alcune applicazioni questo tipo di architettura non è molto efficace:\ è necessario filtrare/processare tali dati prima di arrivare al cloud; la latenza, cioè il tempo che intercorre tra la rilevazione di un dato e l'attuazione determinata dalla sua elaborazione, deve essere veramente breve (si pensi ad esempio ai veicoli intelligenti).

Tradizionalmente il deployment delle applicazioni avveniva tramite due modelli principali:

\begin{enumerate}
	\item IoT + Edge (o edge computing):\ l'intera applicazione è collocata al confine (edge) della rete; l'idea è di processare i dati direttamente laddove vengono prodotti.\
	      \begin{itemize}
		      \item Latenze basse.
		      \item Limitata capacità di calcolo.
		      \item Difficoltà nella condivisione dei dati.
	      \end{itemize}
	\item IoT + Cloud:\ i dati vengono mandati al cloud per l'elaborazione.
	      \begin{itemize}
		      \item Enorme potenza di calcolo.
		      \item È necessaria una connessione fra i dispositivi coinvolti.
		      \item Latenza più alta.
		      \item Bottleneck dovuti all'ampiezza di banda
	      \end{itemize}
\end{enumerate}

\noindent A partire da tutto ciò è partita l'idea di questa continuità tra l'IoT e il Cloud che viene spesso chiamata \textit{Fog Computing}.

\vspace{12pt}
\noindent Il \textbf{Fog Computing} mira ad estendere il Cloud verso l'IoT per supportare meglio le applicazioni IoT \textit{latency-sensitive} e \textit{bandwith-hungry}.
\vspace{12pt}

\noindent C'è un interesse molto grande nel mondo industriale e in quello della ricerca.\
Ci sono molte sfide aperte:
\begin{itemize}
	\item deployment adattivo di applicazioni:\ mandare in esecuzione un'applicazione multi-servizio su un'infrastruttura grande, distribuita ed eterogenea non è semplice;
	\item la gestione delle applicazioni durante il ciclo di vita del sistema;
	\item il monitoraggio distribuito su infrastrutture offerte anche da provider diversi;
	\item problemi di trust, di privacy, di sicurezza in generale;
	\item fault resilience;
	\item testbeds:\ infrastrutture su cui è possibile sperimentare il fog davvero;
	\item aspetti di business model.
\end{itemize}

\subsubsection{Deployment delle applicazioni}

Le applicazioni di nuova generazione sono tipicamente multi-servizio (microservizi), quindi consistono di più ``pezzi'' che devono essere mandati in esecuzione sui nodi di un'infrastruttura eterogenea.\
In generale le applicazioni hanno dei requisiti, sia hardware che software, e dei vincoli di qualità.

Dall'altra parte abbiamo l'infrastruttura con le sue capabilities, che ha tre caratteristiche molto importanti:\ è eterogenea, i nodi non sono tutti uguali, è enorme ed è dinamica.

