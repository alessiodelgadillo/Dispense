\chapter{FaaS}

Il meccanismo di ``\textit{Function as a Service}'', o \textbf{FaaS}, conosciuto anche come ``\textit{serverless}\footnote{\textbf{Serverless} indica il fatto che la gestione del server è trasparente a chi ne usufruisce.} computing'', ha avuto molto successo nel mondo dell'ICT e dello sviluppo del software e molte altre compagnie hanno iniziato a offrirlo come servizio.

\vspace{12pt}
\noindent``More than 20\% of global enterprises will have deployed serverless computing technologies by 2020.'' [\texttt{Gartner, Dec 2018}]

\vspace{12pt}
\noindent``The most prominent manifestation of serverless computing is function platform ad a service, or fPaaS.\ Gartner predicts that half of global enterprises will have deployed fPaaS by 2025, up from 20\% today.'' [\texttt{Gartner, Dec 202}0]

\section{AWS Lambda}
Permette di \textbf{eseguire codice} senza eseguire il provisioning o gestire i server, il tutto \textbf{senza amministrazione}.\
Una volta caricato il codice, \textbf{Lambda si occuperà di eseguirlo e scalarlo} con disponibilità elevata.\
È possibile configurare il codice in modo che \textbf{si attivi automaticamente da altri servizi AWS} o chiamarlo direttamente da qualsiasi app Web o mobile.\
\textbf{Si paga solo per il tempo di elaborazione} che si consuma.\

\noindent Facile da usare:\
\begin{itemize}
	\item carica la tua funzione/progettala in AWS IDE/selezionala dall'elenco di esempi predefiniti;
	\item seleziona l'origine dell'evento da monitorare (ad es.\ Bucket S3).
\end{itemize}

\noindent Poi Lambda attiva automaticamente la funzione quando si verifica un evento; gestisce tutte le capacità, la scalabilità, l'applicazione di patch e l'amministrazione dell'infrastruttura per eseguire il codice e pubblica metriche e registri in tempo reale.\
Tutto ciò con un servizio low cost (canone basso per richiesta).\

\subsubsection{Prezzo}

\begin{itemize}
	\item Richieste:\ \$0,20 per 1 milione di richieste.
	\item Durata (dipende dalla quantità di memoria allocata alla funzione):\ \$0,0000166667 per ogni GB/sec.
\end{itemize}

\noindent Se hai assegnato 512 MB di memoria alla tua funzione, l'hai eseguita 3 milioni di volte in un mese e ha funzionato per 1 secondo ogni volta, i tuoi addebiti saranno:\
\[3.000.000/1.000.000 \times \$0,20 = \$0,60  \]
\[+\]
\[ (3.000.000 \times 1) \times (512/1024) \times \$ 0,0000166667 = \$ 25,00 \]

\noindent Livello di utilizzo gratuito

\begin{itemize}
	\item 1 milione di richieste gratuite al mese.
	\item 400.000 GB/sec. di tempo di elaborazione al mese.
\end{itemize}

\noindent Se hai assegnato 512 MB di memoria alla tua funzione, l'hai eseguita 3 milioni di volte in un mese e ha funzionato per 1 secondo ogni volta, e hai avuto l'utilizzo gratuito per sfruttare le tue spese saranno:

\[ (3.000.000-1.000.000)/1.000.000 \times \$ 0,20 = \$ 0,40\]
\[+\]
\[((3.000.000 \times 1) \times (512/1024) - 400.000) \times \$ 0,0000166667 = \$ 18,33\]

\subsubsection{Perché il nome ``lambda''?}

AWS Lambda prende il suo nome dal \textit{lambda calcolo}:\
\[ \lambda x.\ f(x)\]

\noindent $f$ è il supporto a tempo di esecuzione offerto da AWS, mentre \textit{x} è la funzione caricata su AWS.

\section{Confronto fra 10 piattaforme FaaS}

\begin{table}[H]
	\centering
	\begin{tabular}{|l|l|}
		\hline
		\textbf{Commercial}    & \textbf{Open source} \\\hline\hline
		AWS Lambda             & Apache Openwhisk     \\
		Google Cloud Functions & Fission              \\
		MS Azure Functions     & Fn                   \\
		                       & Knative              \\
		                       & Kubeless             \\
		                       & Nuclio               \\
		                       & OpenFaaS             \\\hline
	\end{tabular}
\end{table}

\subsection{Business view}

\subsubsection{Licensing}

Tutte le piattaforme open source usano una licenza permissiva:\ quella più usata è Apache 2.0.\

Tutte le piattaforme FaaS commerciali usano licenze proprietarie, con MS Azure Functions che rilascia anche alcune delle sue componenti come progetti open source.\

\subsubsection{Installazione}

Tra le piattaforme commerciali, solo Azure Functions ha alcune parti installabili on-premises.\

Le piattaforme FaaS open source sono tutte installabili e permettono di avere una certa portabilità nell'installazione (supportano più target):\ Kubernetes è la più supportata.\

\subsubsection{Codice sorgente}

Tutte le piattaforme open source sono ospitate su GitHub e la maggior parte sono implementate in Go.\

Azure Functions è l'unica piattaforma commerciale che è parzialmente open source.\

\subsubsection{Interface}

Tutte le piattaforme supportano CLI (\textit{Command Line Interface}), mentre non sempre sono disponibili le API e le GUI.\
Per quanto riguarda l'amministrazione, quindi dal punto di vista delle operazioni che possiamo eseguire sulla piattaforma per amministrarla (deployment, configurazione, terminazione), l'offerta varia considerevolmente.

\subsubsection{Community}

OpenFaaS, Apache Openwhisk e Knative hanno rispettivamente i rating più alti su GitHub in termini di stars, contributori e commits.\
Le domande su Stackoverflow mostrano una notevole differenza in interessi tra piattaforme commerciali e quelle open source.

\subsubsection{Documentazione}

Tutte le piattaforme forniscono la documentazione necessaria a eseguire il deployment di un'applicazione e usare la piattaforma stessa; tuttavia non sempre vengono documentati lo sviluppo e l'architettura della piattaforma.\

\subsection{Technical view}

\subsubsection{Development}

Java, Node.js e Python sono i runtime più supportati, con Docker che sta diventando piuttosto popolare per aiutare a personalizzare i runtime.\
La possibilità di usare IDEs e editor di testo è principalmente offerta dalle piattaforme commerciali.

\subsubsection{Versioning}

La maggior parte delle piattaforme open source, a differenza delle piattaforme commerciali, implementano l'implicit versioning senza offrire meccanismi dedicati.

\subsubsection{Event sources}

Tutte le piattaforme supportano l'invocazione sincrona di una funzione basata su HTTP, mentre le invocazioni asincrone sono supportate da poche piattaforme.\
Più di metà delle piattaforme open source non supportano i database come sorgenti di eventi.\
Sono ampiamente supportati meccanismi di scheduling per processare flussi di dati e messaggi.\
Soltanto metà delle piattaforme permette di avere sorgenti di eventi personalizzate.

\subsubsection{Orchestrazione delle funzioni}

Più della metà delle piattaforme FaaS supporta l'orchestrazione di funzioni:\ alcune offrono dei DSL (\textit{Domanin Specific Language}) \textit{ad-hoc}, altre dei workflow.

\subsubsection{Testing \& debugging}

Le piattaforme commerciali offrono delle opzioni più avanzate:\ funzioni di testing e debugging integrate all'interno dell'ambiente di sviluppo.

\subsubsection{Monitoraggio}

Le piattaforme commerciali offrono strumenti dedicati per eseguire monitoraggio e logging delle funzioni, mentre quelle open source si affidano all'integrazione di strumenti di terze parti.

\subsubsection{Application delivery}

La maggior parte delle piattaforme recensite segue un approccio dichiarativo per automatizzare la distribuzione delle applicazioni.\
Le piattaforme commerciali supportano nativamente CI/CD attraverso strumenti dedicati, mentre le piattaforme open source (a parte OpenFaasS) non documentano l'integrazione con le pipeline CI/CD.

\subsubsection{Code reuse}

Solo AWS Lambda e MS Azure Functions offrono un marketplace delle funzioni integrato.

\subsubsection{Access management}

Le piattaforme commerciali supportano nativamente autenticazione e controllo di accesso alle risorse.\

Le piattaforme open source si affidano, tipicamente, alle features offerte dall'ambiente che ospita il FaaS.
