\chapter{Datacenters}

\subsubsection{Adiabatic cooling}

L'idea è di utilizzare un sistema di raffreddamento basato sull'evaporazione dell'acqua per \textit{pre-raffreddare} l'aria dell'ambiente e che viene utilizzato in modo intermittente:\ soltanto nei periodi in cui la temperatura all'interno del datacenter è più alta, in particolare durante le ore più calde del giorno e in primavera inoltrata{\slash}estate.\
In questo modo si riesce a raggiungere l'85\% di secchezza dell'ambiente.\

\subsubsection{Power Usage Effectiveness}

Il Power Usage Effectiveness (PUE) è l'efficacia nell'utilizzo dell'energia e si misura come:
\[\mathrm{PUE = \frac{Total\ Facility\ Power}{IT\ Equipment\ Power}}\]
\noindent dove Total Facility Power indica tutta l'energia utilizzata all'interno del datacenter e IT Equipment Power l'energia usata solo per le apparecchiature informatiche.

\textbf{Nota bene}:\ PUE è un indicatore per l'energia utilizzata, ma non su quanto sia rinnovabile tale energia.

\subsubsection{Datacenter management}

L'aspetto di gestione di un datacenter coinvolge diverse attività:

\begin{itemize}
	\item pianificazione, sia per la manutenzione sia per l'aggiornamento di tutti i dispositivi hardware e software;
	\item installazione e gestione dei rack e delle connessioni;
	\item gestione del cabling.
\end{itemize}

