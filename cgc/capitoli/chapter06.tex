\chapter{Microservizi}

Due motivazioni principali per l'adozione di microservizi.\

\begin{enumerate}
	\item Cercare di abbreviare il cosiddetto \textbf{\textit{lead time}} (tempo che passa da quando si comincia a immaginare una funzionalità al momento in cui il primo utente ``fa il primo click'' su tale funzionalità) per nuove features o aggiornamenti a un'applicazione:
	      \begin{itemize}
		      \item accelerare \textit{rebuild} e \textit{redeployment};
		      \item ridurre le cosiddette \textit{corde} tra silos funzionali.
	      \end{itemize}
	\item Riuscire a \textbf{scalare} efficacemente un'applicazione con un numero di utenti enorme, anche nell'ordine delle centinaia di milioni.\
\end{enumerate}

\begin{center}
	\textit{Ma che cosa sono i microservizi?}
\end{center}

\noindent In passato una tipica applicazione enterprise consisteva in un'interfaccia utente (lato client), l'applicazione vera e propria (lato server) e il database.\
L'applicazione lato server, tipicamente chiamata \textbf{\textit{monolite}}, si occupava di gestire le richieste HTTP, di implementare la logica di dominio, di interrogare e aggiornare il database (se necessario) e, infine, di mandare le risposte alle richieste del client.\

Un'architettura monolitica comporta che anche un solo piccolo cambiamento di una parte richieda il \textbf{rebuild e il redeployment dell'intera applicazione}:\ tutto ciò potrebbe risultare costoso, non solo in termini di risorse, ma anche in termini di tempo necessario.\
Inoltre, scalare un monolite richiede di \textbf{scalare l'intera applicazione} piuttosto che scalare quella singola parte che necessita di più risorse.\

\subsubsection{Essenza dei microservizi}

Le applicazioni sono sviluppate come insieme di servizi, si tratta di un'architettura modulare.\
Ognuno di questi servizi viene eseguito in un ambiente (processo) separato, tipicamente in un container; la comunicazione fra tali processi avviene con meccanismi semplici e leggeri, solitamente si usano protocolli di tipo REST.\
Un'altra delle caratteristiche distintive è il \textbf{\textit{poliglottismo}}:\ servizi diversi possono essere scritti in linguaggi diversi; così come tipi di dati diversi possono essere memorizzati con tecnologie diverse.\

Si chiamano microservizi ma il termine ``\textit{micro}'' esiste per motivi storici:\ la dimensione non è veramente il punto più importante.\
Se si vuole sapere davvero la dimensione di un microservizio si può misurare la dimensione del team che lo gestisce.\

Uno degli aspetti più innovativi è l'idea di organizzare i servizi intorno alle \textbf{\textit{business capabilities}} e quindi avere dei \textbf{cross-functional teams}:\ team indipendenti (per quanto possibile) l'uno dall'altro e conseguentemente funzionalità di un'applicazione distribuite su questi microservizi.\

\vspace{12pt}
\noindent``\textit{Organizations which design systems [\dots] are constrained to produce designs which are copies of the communication structures of these organizations}''\qquad M.\ Conway, 1968
\vspace{12pt}

\noindent Un'ulteriore novità dei microservizi è la \textbf{decentralizzazione della gestione dei dati}:\
\begin{itemize}
	\item lasciare che ogni servizio gestisca il proprio database;
	\item ``\textit{eventual consistency and compensations}'' invece di transazioni distribuite;
\end{itemize}

\noindent Inoltre, alla base dei microservizi, vi è la \textbf{\textit{deployability indipendente}}, ovvero possibilità di eseguire il deployment di una funzionalità mentre l'intero applicativo è in esecuzione, e la \textbf{\textit{scalabilità orizzontale}}, cioè la possibilità di scalare un singolo servizio.\

Per concludere si può parlare di \textbf{\textit{fault resilient services}}:\ progettare le applicazioni in modo tale che riescano tollerare i fallimenti e quindi riuscire a garantire una \textit{resilienza} del sistema quanta più alta possibile di fronte a fault che certamente avverranno all'interno del sistema.\

\paragraph{Test (bravely)}

\textit{Chaos Monkey} è uno strumento che permette di testare in modo coraggioso il funzionamento di un'applicazione.\
Perché in modo ``\textit{coraggioso}''?
È possibile configurarlo in modo tale da uccidere istanze di macchine virtuali, o direttamente container, che vengono eseguiti nell'ambiente di \textbf{produzione}.\

\subsubsection{Dev-Ops}

Per anni lo sviluppo del software si è basato su un gruppo di \textit{developers} e un gruppo di \textit{operators}, le persone che gestiscono i rapporti con i clienti.\
In \textbf{Dev-Ops} questa distinzione cessa di esistere.\

\begin{center}
	``\textit{You build it, you run it}''
\end{center}

\noindent Strumenti a disposizione:
\begin{itemize}
	\item Version Control System (VCS), e.g.\ Git \& Github;
	\item Continuous Integration \& Continuos Development (CI{\&}CD), e.g.\ Jerkins;
	\item Infrastructure as Code (IaC), e.g.\ Puppet, Chef;
	\item Application Performance Monitoring (APM), e.g.\ NewRelic.
\end{itemize}


\subsubsection{Osservazioni conclusive}

\textit{Don't even consider microservices unless you have a system that's too complex to manage as a monolith}.\qquad [M.\ Fowler]
\vspace{12pt}

\noindent Lo stile architetturale basato sui microservizi è un'idea molto importante da tenere in considerazione per le applicazioni enterprise.\
Ci sono molti vantaggi, come abbreviare il \textbf{lead time} per aggiornare le applicazioni e riuscire a \textbf{scalare in maniera efficace} davanti a numeri drammatici di richieste.\

Tra gli svantaggi principali vi sono l'\textbf{overhead} dovuto alla comunicazione e la \textbf{complessità} del sistema finale; è facile ``sbagliare i tagli'' nella progettazione dell'architettura, quando si eliminano le dipendenze durante il  partizionamento (se ce ne sono troppe si parla di \textit{wrong cuts}).\
Inoltre la \textbf{decentralizzazione} dei dati comporta controlli per verificarne l'integrità e per ridurne al minimo la duplicazione.\

\begin{center}
	``\textit{A poor team will always create a poor system}.''
\end{center}
