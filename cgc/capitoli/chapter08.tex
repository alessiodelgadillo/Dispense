\chapter{Green Computing}

La produzione di dispositivi ICT causa \textbf{inquinamento} e genera \textbf{gas serra} e \textbf{e-waste}.\
Il \textbf{green computing} è un \textit{insieme di pratiche per arrivare ad avere un ICT sostenibile a livello ambientale}, minimizzando il consumo elettrico, progettando dispositivi efficienti (a livello energetico) e riducendo l'e-waste.\

\begin{flushleft}
	\textit{Per produrre un singolo computer sono necessari 1800 Kg di materiale grezzo}.\quad [\texttt{news.un.org}]
\end{flushleft}

\begin{flushleft}
	\textit{L'ICT rappresenta il 9\% del consumo europeo di elettricità e il 4\% delle emissioni europee di carbonio}.\quad [\texttt{ictfootprint.eu}]
\end{flushleft}

\noindent \textit{Nature} prevede che nel 2030 l'ICT rappresenterà il 20.9\% del consumo dell'elettricità mondiale.

Il $\sim$40\% del consumo di energia di un datacenter è solo per il raffreddamento.\
Si ricorda che il PUE non indica l'uso di energie rinnovabili.


\subsubsection{Obsolescenza programmata}

L'obsolescenza pianificata è quando un \textit{prodotto è deliberatamente progettato per avere una durata ridotta}:\ la durata della vita deve essere abbastanza lunga da sviluppare il bisogno duraturo di un cliente e il cliente deve considerare che il prodotto è un prodotto di qualità, anche se alla fine deve essere sostituito.

\subsubsection{Obsolescenza percepita}

L'obsolescenza percepita si verifica quando \textit{un cliente è convinto di aver bisogno di un prodotto aggiornato, anche se il suo prodotto esistente funziona bene}.

\subsubsection{The e-waste tragedy}

50 milioni di tonnellate di rifiuti elettronici ogni anno ($\approx$770 milioni di lavatrici).\
Il traffico di rifiuti elettronici è più grande del commercio di droga anche se l'esportazione di rifiuti elettronici è stata vietata da più di 30 anni.\

\subsubsection{Clicking clean}

GreenPeace periodicamente rilascia un rapporto chiamato \textit{Clicking Clean} in cui cerca di arrivare a una valutazione di come si comportano i grandi player del mondo ICT rispetto alla sostenibilità ambientale.

\begin{center}
	100\% \textit{renewably powered digital infrastructures}

	or

	\textit{dangerous climate changes}?
\end{center}

\noindent Greenpeace ha effettuato il primo benchmarking delle prestazioni energetiche del settore IT nel 2009, sfidando i più grandi attori di Internet a sostenere la propria crescita con il 100\% di energia rinnovabile.\

Apple, Facebook e Google hanno assunto impegni rinnovabili al 100\% nel 2013, ai quali si sono uniti nel 2017 quasi 20 società Internet.\
Perché?
\begin{itemize}
	\item Aumento della competitività in termini di costi delle energie rinnovabili (contratti a lungo termine).
	\item Crescente preoccupazione per il cambiamento climatico tra (dipendenti e) clienti.
\end{itemize}

\noindent L'acquisto diretto di energia rinnovabile da parte delle società negli Stati Uniti è aumentato notevolmente dal 2010, superando i 3,4 GW nel 2015.

\begin{flushleft}
	Ostacoli a Internet alimentato al 100\% da fonti rinnovabili:\ mancanza di trasparenza e mancanza di accesso alla fornitura di energia rinnovabile (RE).
\end{flushleft}

\noindent Nel 2017, Apple era leader per il 3° anno consecutivo tra gli operatori di piattaforme per quanto riguarda la sostenibilità.\
Sia Apple che Google continuano a guidare il settore nell'abbinare la loro crescita con una fornitura equivalente o maggiore di energia rinnovabile.\
Switch leader tra gli operatori di colocation per i suoi sforzi per trasferire la sua flotta di data center alle energie rinnovabili il più rapidamente possibile.\

Non è chiaro se Amazon Web Services (AWS) sia effettivamente sulla buona strada per diventare alimentato da fonti rinnovabili (mancanza di trasparenza e rapida crescita in Virginia e in altri mercati ampiamente serviti da energia sporca).\
Le società meno trasparenti, come AWS, Tencent, LG CNS, Baidu, sono anche tra le più dominanti nei rispettivi mercati.\

