\documentclass[12pt, a4paper]{report}
\usepackage[utf8]{inputenc}
\usepackage[hidelinks]{hyperref}
\usepackage[italian]{babel}
\usepackage{array}
\usepackage{caption}
\usepackage{graphicx}
\usepackage{float}
\usepackage{amssymb}
\usepackage{amsthm}
\usepackage{mathtools}
\usepackage{enumitem}

\graphicspath{{./images/}}

\newtheorem{theorem}{Teorema}[section]
\newtheorem{corollario}{Corollario}[section]
\newtheorem{lemma}{Lemma}[section]
\newtheorem{property}{Proprietà}[section]

\theoremstyle{definition}
\newtheorem{definition}{Definizione}[section]
\newtheorem{example}{Esempio}[section]

\theoremstyle{remark}
\newtheorem{question}{Domanda}[section]

\title{Elementi di Calcolabilità e Complessità}
\date{A.A. 2021-2022}
\author{Alessio Delgadillo}

\begin{document}

\maketitle
\newpage
\tableofcontents

\chapter{Calcolabilità}
\chapter{Introduzione}

\section{Intelligenza Artificiale: una doppia sfida}

L'Intelligenza Artificiale si occupa della
\begin{enumerate}
	\item comprensione
	\item riproduzione
\end{enumerate}
del comportamento \textit{intelligente}.
\subsubsection{L'IA come scienza empirica}

L'approccio della psicologia cognitiva:\\
\textbf{\textit{Obiettivo}}:\ comprensione dell'intelligenza umana.\\
\textbf{\textit{Metodo}}:\ costruzione di modelli computazionali, verifica sperimentale.\\
\textbf{\textit{Criterio di successo}}:\ risolvere i problemi con gli stessi processi usati dall'uomo.

\subsubsection{L'IA come disciplina informatica}
L'approccio ``costruttivo'':\\
\textbf{\textit{Obiettivo}}:\ costruzione di entità dotate di razionalità.\\
\textbf{\textit{Metodo}}:\ codifica del pensiero razionale; comportamento razionale.\\
\textbf{\textit{Criterio di successo}}:\ l'importante è risolvere i problemi che richiedono intelligenza.

\subsection{Definizioni di IA}
\begin{table}[H]
	\centering
	\begin{tabular}{|l|p{13.5em}|p{13em}|}
		\hline
		        & Umanamente                                                                                                                                                                      & Razionalmente                                                                                                                                                                                                                                  \\\hline
		Pensare & ``\textit{L'automazione delle attività che associamo al pensiero umano, come il processo decisionale, la risoluzione di problemi, l'apprendimento}\dots''\newline[Bellman 1978] & ``\textit{Lo studio delle facoltà mentali attraverso l'uso di modelli computazionali}''\newline[Charniak, McDermott, 1985]                                                                                                                     \\\hline
		Agire   & ``\textit{L'arte di creare macchine che svolgono funzioni che richiedono intelligenza quando svolte da esseri umani}''\newline[Kurzweil 1990]                                   & ``\textit{Il ramo della scienza dei calcolatori che si occupa dell'automazione del comportamento intelligente}''\newline [Luger-Stubblefield 1993] \newline ``\textit{L'impresa di costruire artifatti intelligenti}''\newline [Ginsberg 1993] \\\hline
	\end{tabular}
\end{table}
\subsubsection{Da ``\textit{Strategic directions in Artificial Intelligence}''}

Il settore dell'IA consiste nell'indagine tecnologica e intellettuale, a lungo termine, che mira al raggiungimento dei seguenti obiettivi scientifici e pratici:
\begin{itemize}
	\item costruzione di macchine intelligenti, sia che operino come l'uomo che diversamente;
	\item formalizzazione della conoscenza e meccanizzazione del ragionamento, in tutti i settori di azione dell'uomo;
	\item comprensione mediante modelli computazionali della psicologia e comportamento di uomini, animali e agenti artificiali;
	\item rendere il lavoro con il calcolatore altrettanto facile e utile che del lavoro con persone, capaci, cooperative e possibilmente esperte.
\end{itemize}
Prendiamo per buona questa definizione:
\begin{center}
	\textit{L'arte di creare macchine che svolgono funzioni che richiedono intelligenza quando svolte da esseri umani}

	- Kurzweil, 1990
\end{center}
Ma cosa significa ``intelligente''?\ Settanta definizioni di intelligenza, di cui 35 fuori dal settore AI.\
È una qualità intrinseca o un comportamento?\
Difficile arrivare ad una definizione condivisa e completa perché esistono diversi tipi di ``intelligenza''.

Che tipo di capacità?
\begin{itemize}
	\item Capacità di simulare il comportamento umano?
	\item Capacità di ragionamento logico/matematico?
	\item Intelligenza come competenza ``da esperto''?
	\item Intelligenza come ``buon senso'' (common sense)?
	\item Capacità di interagire efficacemente con un ambiente?
	\item Capacità sociali, di comunicazione e coordinamento?
	\item Capacità di comprendere e provare emozioni? (intelligenza emozionale)
	\item Capacità di ``immagazzinare'' esperienza? Apprendere?
	\item Creatività?
\end{itemize}
Capacità di imitazione dell'uomo?\
\textbf{Il test di Turing} (1950):\ un tentativo di definizione operativa di intelligenza.\
Le previsioni di Turing:

\vspace{12pt}
\noindent``\textit{Credo che tra circa 50 anni sarà possibile programmare computer con una memoria di un miliardo di byte in maniera tale che essi giochino il gioco dell'imitazione tanto bene che una persona comune non avrà più del 70\% di probabilità di identificarli dopo 5 minuti di interrogazione}''

\vspace{12pt}
\noindent Obiettivo raggiunto?\ Nel 2014 Eugene Goostman simula un ragazzo ucraino di 13 anni in maniera credibile ed è riuscito ad ingannare più del 30\% dei giudici.\
Molto rumore nella stampa, molte reazione scettiche.

\subsubsection{Flying like a pigeon?}

Sarebbe come se gli ingegneri aereonautici definissero il loro obiettivo come ``\textit{creare delle macchine in grado di volare come piccioni, in maniera così perfetta da ingannare gli altri piccioni}''.

\subsubsection{Definizione di ``intelligenza''}

``\textit{Qualità mentale che consiste nell'abilità di apprendere dall'esperienza, di adattarsi a nuove situazioni, comprendere e gestire concetti astratti.\ E utilizzare conoscenza per agire sul proprio ambiente}.''

[Enciclopedia britannica]

\section{Agenti Intelligenti: la visione ``moderna''}
Gli agenti sono \textbf{situati}
\begin{itemize}
	\item  ricevono \textit{percezioni} da un ambiente;
	\item  agiscono sull'ambiente mediante \textit{azioni}.
\end{itemize}
Gli agenti hanno \textbf{abilità sociale}:\ sono capaci di comunicare, collaborare, difendersi da altri agenti.\
Gli agenti hanno \textbf{opinioni}, \textbf{obiettivi}, \textbf{intenzioni}\dots\
Gli agenti sono \textbf{embodied}:\ hanno un \textbf{corpo} e forse provano ``\textbf{emozioni}''.
\begin{figure}[H]
	\centering
	\includegraphics[width=0.7\textwidth]{immagini/Agenti_intelligenti.jpg}
	\caption*{Agente intelligente}
\end{figure}

\subsubsection{La sfida: RoboCup}
La \textit{Robot World Cup Initiative} (RoboCup) è un problema di riferimento per la ricerca in I.A.\
Si tratta di realizzare agenti in grado di giocare a calcio (entro il 2050!).\
Un problema difficile, da usare come banco di prova per nuove idee e tecnologie.

Tecnologie da sviluppare e integrare

\begin{itemize}
	\item agenti autonomi
	\item collaborazione tra agenti
	\item acquisizione di strategie
	\item ragionamento e pianificazione in tempo reale
	\item robotica
	\item tecnologie hw e sw per infrastruttura
\end{itemize}

\subsubsection{Capacità di emozioni?}

\begin{center}
	``\textit{The question is not whether intelligent machines can have emotions, but whether machines can be intelligent without any emotions}.''

	[Minsky, The Society of Mind]

\end{center}
Capacità di comprendere e mostrare emozioni $\rightarrow$ ruolo delle emozioni nel meccanismo di decisione.

\section{The AI revolution?}

\begin{itemize}
	\item Algoritmi di apprendimento automatico che estraggono modelli statistici predittivi da immense quantità di esempi.
	\item Tecniche di estrazione di ``\textit{significati}'' da grande quantità di testi.
	\item Sistemi in grado di rispondere a domande in linguaggio naturale in un ``dominio aperto''.
\end{itemize}

\subsubsection{La disponibilità di grosse quantità di dati}
L'enfasi si sposta dagli algoritmi ai dati e agli algoritmi di apprendimento (ML e DM).\
Esempi dalle tecnologie del linguaggio:
\begin{itemize}
	\item La traduzione automatica di Google
	\item L'interazione in linguaggio naturale di SIRI
\end{itemize}
Più dati, maggiore l'accuratezza,\ \dots\ apparentemente senza limite (``Unreasonable Effectiveness of Big Data'')

La domanda diventa:\ l'intelligenza può essere estratta o inferita dai dati?

\subsubsection{The \textit{deep learning} tsunami}

L'idea di usare reti neurali con molti livelli (deep NN) era stata tentata per diversi anni, ma solo verso il 2011 si riescono ad avere successi evidenti, come riconoscimento di immagini e parlato.

C. Manning:\ ``le reti neurali profonde sono in giro da un po' di anni ma nel 2015 hanno colpito come uno tsunami il settore del NLP''.

I maggiori esperti (Le Cun, Hinton, Benjo) sono concordi nel ritenere che ci saranno sviluppi importanti a breve nella comprensione dei testi, video, traduzione automatica, QA,\ \dots

Apprendimento di Deep Networks diviene praticabile dal punto di vista computazionale:
\begin{itemize}
	\item Possibilità di usare GPU multiple (Graphic Processing Unit); ottimizzate per il prodotto di matrici.
	\item Se avessimo usato gli stessi algoritmi 10 anni fa, non avrebbero ancora finito.
\end{itemize}

\subsection{Priorità di ricerca}
L'intelligenza artificiale\dots ``deve fare solo quello che noi vogliamo che faccia''.\
Servono ricerche non solo per rendere l'IA più \textbf{capace} ma anche fare in modo che sia \textbf{robusta} e \textbf{benefica} per la società (\textit{Trustworthy AI})
\begin{itemize}
	\item Impatto economico sul mercato del lavoro e sulla società
	\item Responsabilità dei veicoli autonomi, etica delle macchine, armi autonome, privacy
	\item Verifica (il sistema è ``corretto''?), validità (il sistema è ``giusto''?)
	\item Sicurezza (protezione da terzi), controllo (dei sistemi autonomi)
\end{itemize}

\section{In sintesi}
È difficile dare una definizione univoca di ``intelligenza'' e quindi di ``intelligenza artificiale''.\
A seconda dei periodi storici gli approcci sono diversi, l'enfasi è diversa, gli obiettivi stessi sono diversi.\
Parafrasando la definizione di Kurzweil

\begin{center}
	\textit{A.I è l'arte di creare macchine che svolgono funzioni che tuttora richiedono intelligenza umana per essere svolte}

	\dots \textit{un ``bersaglio mobile''}.
\end{center}
A.I. (ML) come alternativa all'approccio algoritmico tradizionale in presenza di incertezza.
\begin{itemize}
	\item \textit{Sorgenti di incertezza:\ sensori imperfetti, dati incompleti, limiti alle capacità di calcolo}
	\item \dots
	\item \textit{Thrun:\ AI is the technique of uncertainty management in computer software.\ AI is the discipline that you apply when you don't know what to do.}
\end{itemize}
Ma ricordate:\ l'IA non coincide con il ML;\ la General AI è ancora lontana e dovrà integrare diverse capacità e in particolare apprendimento e ragionamento.

\chapter{nDPI}

\section{Introduzione}

La nDPI è una liberia che permette di fare la \textbf{Deep Packet Inspection}:\ un'attività che analizza i pacchetti di rete oltre agli header di protocollo; con altri tool di analisi dei pacchetti di rete come Wireshark è possibile analizzare principalmente gli header dei pacchetti (dove ci sono IP e porte).\
L'analisi di Wireshark è molto interessante però non è esaustiva (specialmente in questi ultimi anni):\ il traffico sta diventando sempre più complicato da gestire perché i protocolli esistenti sono continuamente in decrescita.\

Un tempo veniva creato un protocollo per ogni cosa ma, con l'avvento del Web, HTTP è diventato uno dei protocolli predominanti; tuttavia, col tempo HTTP ha perso il senso per cui è stato inventato, ovvero per risolvere gli \textit{hypertext}, ed è diventato una sorta di protocollo di trasferimento.\
Quindi, dato che i protocolli si stanno appiattendo, cioè vengono impiegati sempre gli stessi e talvolta vengono stravolti dal loro utilizzo originale, è necessario analizzare il traffico in maniera più profonda ovvero andare a guardare il payload.\
Il tool nDPI permette questo tipo d'analisi.

I motivi per cui si vuole monitorare il payload sono vari, ma il più importante è \textbf{\textit{classificare il traffico}} (per capire se ci sono contenuti malevoli o non consentiti sulla nostra rete).\
Parlare di HTTP o UDP non ha più alcun senso poiché è troppo generico:\ è necessario capire cosa sia contenuto nei pacchetti per capire che tipo di traffico attraversi una rete e, di conseguenza, prendere delle decisioni.\
Ad esempio, potremmo rallentare, bloccare o velocizzare un determinato traffico per proteggere la rete o aumentarne le prestazioni.\

\section{Classificare il traffico di rete}

Classificare il traffico non significa spiarlo, ma capirne la natura, sia per sicurezza che per migliorare l'esperienza dell'utente ottimizzando parametri di rete specifici.\
I principali metodi di classificazione includono
\begin{itemize}
    \item \textbf{Classificazione della porta} TCP/UDP:\ agli albori di Internet, i protocolli di traffico di rete venivano identificati per protocollo e porta.\ Può classificare solo protocolli applicativi che operano su porte note (no rpcbind o portmap):\ facile da imbrogliare e quindi inaffidabile (TCP/80 $\neq$ HTTP).
    \item \textbf{Classificazione basata su QoS} (DSCP):\ simile alla classificazione delle porte ma basata sui tag QoS.\ Di solito ignorato in quanto è facile da imbrogliare e falsificare.\
    \item \textbf{Classificazione statistica}:\ classificazione dei pacchetti IP (dimensione, porta, flag, indirizzi IP) e dei flussi (durata, frequenza,\ \dots) basato su regole scritte manualmente, o automaticamente utilizzando algoritmi di \textit{machine learning} (ML).\ Il machine learning richiede un training set di ottima qualità e di solito è intensivo dal punto di vista computazionale.\
          Il tasso di rilevamento può arrivare fino al 95\% per i casi che erano coperti dal training set e scarsa precisione per tutti gli altri casi.\
    \item \textbf{Deep Packet Inspection}:\ tecnica che ispeziona il carico utile del pacchetto; computazionalmente intensivo rispetto alla semplice analisi dell'intestazione del pacchetto.\ Preoccupazioni per la privacy e la riservatezza dei dati ispezionati.\ La crittografia sta diventando pervasiva, sfidando così le tecniche DPI.
          Nessun falso positivo a meno che gli strumenti DPI non utilizzino metodi statistici o analisi dell'intervallo/flusso IP.
\end{itemize}

\subsubsection{Usare DPI nel monitoraggio del traffico}

L'analisi dell'intestazione del pacchetto non è più sufficiente in quanto è \textit{inaffidabile} e quindi \textit{inutile}.\
Gli amministratori di sicurezza e di rete vogliono sapere quali sono i veri protocolli che scorrono su una rete, indipendentemente dalla porta utilizzata.\

L'estrazione selettiva dei metadati (ad es.\ \texttt{URL HTTP} o User-Agent) è necessaria per eseguire un monitoraggio accurato e quindi questa attività dovrebbe essere eseguita dal toolkit DPI senza replicarlo sulle applicazioni di monitoraggio.\

\subsection{Welcome to nDPI}

Nel 2012 è stato deciso di sviluppare un toolkit GNU LGPL DPI (basato su un progetto non mantenuto chiamato OpenDPI) al fine di costruire un livello DPI \textit{aperto} per applicazioni \texttt{ntop} e di terze parti (Wireshark, netfilter, strumenti ML,\ \dots).\
Ci sono più di 240 protocolli supportati divise in famiglie (è importante sapere la famiglia poiché il protocollo non è sempre conosciuto a livello globale):\ P2P (Skype, BitTorrent), messaging (Viber, Whatsapp, MSN, Facebook), multimedia (YouTube, Last.gm, iTunes), conferenza (Webex, CitrixOnLine), streaming (Zattoo, Icecast, Shoutcast, Netflix), business (VNC, RDP, Citrix, *SQL).

\subsubsection{Cos'è un protocollo in nDPI?}

Ogni protocollo è identificato come $\langle \mathtt{principale} \rangle . \langle \mathtt{minore} \rangle$, per esempio:\ \texttt{DNS.Facebook} o \texttt{QUIC.YouTube} e \texttt{QUIC.YouTubeUpload}.\
\textbf{Nota}:\ Skype o Facebook sono protocolli nel mondo nDPI ma non per IETF.\

La prima domanda che le persone chiedono quando devono valutare un toolkit DPI è:\ quanti protocolli sono supportati?\ Ma questa non è la domanda giusta.

Oggi la maggior parte dei protocolli sono basati su HTTP/TLS.\
nDPI include il supporto per il rilevamento di protocolli basati su stringhe:\ nome della query DNS, campi di intestazione host/server HTTP, TLS/QUIC SNI (indicazione del nome del server).

\subsubsection{Categorie nDPI}

I protocolli sono troppi e aumentano ogni giorno; molte persone non hanno familiarità con i nomi di protocollo.\
Spesso le persone si fanno domande del tipo ``Come posso impedire ai miei figli di utilizzare i social network?''.\
Soluzione:\
nDPI consente di raggruppare i protocolli in categorie definite dall'utente come VoIP, P2P, Cloud,\ \dots le categorie possono includere migliaia di voci e possono essere (ri)caricate dinamicamente.\

\subsubsection{nDPI Internals}

Le applicazioni che utilizzano nDPI sono responsabili della ``cattura (inoltro in modalità inline) dei pacchetti'' e del mantenimento dello stato del flusso.\
In base al protocollo/porta di flusso, tutti i dissettori che possono potenzialmente corrispondere al flusso vengono applicati in sequenza a partire da quello che più probabilmente corrisponde.\
Ogni dissettore è codificato in un file ``\texttt{.c}'' diverso per motivi di modularità ed estensibilità.\
C'è un file \texttt{.c} aggiuntivo per la corrispondenza IP (ad esempio, identificare il traffico Spotify basato su Spotify AS).

\subsubsection{Ciclo di vita della classificazione}

In base al tipo di traffico i dissettori vengono applicati in sequenza a partire da quello che ha più probabilità di corrispondere al flusso; ad esempio, per TCP/80 viene provato per primo il dissettore HTTP.\
Ogni flusso mantiene lo stato per i dissettori non corrispondenti per saltarli nelle iterazioni future.\
L'analisi dura fino a quando non viene trovata una corrispondenza o dopo troppi tentativi (8 pacchetti è il limite superiore nella nostra esperienza).\

\subsubsection{Cattura del traffico di rete}

Come detto in precedenza nDPI non cattura il traffico, ma ne esegue un'analisi:\ questo significa che il traffico gli deve essere passato in qualche modo.\
La libreria ha anche la necessità di dividere il traffico in connessioni, quindi quando gli viene passato un pacchetto gli deve essere specificato anche il suo flusso di appartenenza.\

La cattura del traffico di rete che non è locale può essere eseguita tramite un \textit{port mirror} (copia del traffico di rete) oppure si utilizza un \textit{network-tap}.\
\begin{itemize}
    \item\textbf{Port mirror}:\ su una determinata porta dove viene attaccato l'analizzatore di rete, si duplica il traffico (SPAN) della porta che si vuole analizzare (ad esempio tramite nDPI).\ Se la porta di destinazione non ha una capacità di destinazione superiore (quantomeno doppia) a quella sorgente, alcuni pacchetti possono essere persi; per nDPI questo è un problema visto che analizza solamente la parte iniziale di una comunicazione (se mi perdo dei pacchetti diventa quindi difficile ricostruire il traffico successivo).\
    \item\textbf{Network Tap}:\ il traffico entra ed esce dal tap per poi essere indirizzato sulle porte collegate al dispositivo di cattura/analisi del traffico.\ Il vantaggio dell'utilizzo del tap è che non si ha la perdita di pacchetti, questo perché si cattura una direzione per volta; il vero problema è che si necessita di due porte per l'analisi (una per ogni direzione del traffico, quindi una per il traffico in entrata e una per il traffico in uscita).
\end{itemize}

\chapter{Teoria della Complessità}

\subsubsection{Decidibilità e Trattabilità}

Ci sono problemi (problema dell'arresto) che non possono essere risolti da nessun calcolatore, indipendentemente dal tempo a disposizione:\ problemi indecidibili.\
Ci sono problemi decidibili che possono richiedere tempi di risoluzione esponenziali nella dimensione dell'istanza (torri di Hanoi, generazione delle sequenze binarie e delle permutazioni):\ \textbf{problemi intrattabili}.\
Ci sono problemi che possono essere risolti con algoritmi di costo polinomiale (ordinamento; ricerca di chiavi in array, liste, alberi; problemi su grafi:\ OT di DAG, connettività, ricerca di cicli, ricerca di un ciclo euleriano):\ \textbf{problemi trattabili} (``facili'').\
Ci sono infine problemi il cui stato non è noto (clique, cammino hamiltoniano); abbiamo a disposizione solo algoritmi di costo esponenziale, ma nessuno ha dimostrato che non possano esistere algoritmi di costo polinomiale:\ \textbf{problemi presumibilmente intrattabili}.\

Studiamo la dimensione dei dati trattabili in funzione dell'incremento della velocità dei calcolatori.\

\vspace{12pt}

\noindent Calcolatori:\ $C_1$, $C_2$ ($k$ volte più veloce di $C_1$).\

\noindent Tempo di calcolo a disposizione: $t$
\begin{itemize}
    \item $n_1$ = dati trattabili nel tempo $t$ su $C_1$
    \item $n_2$ = dati trattabili nel tempo $t$ su $C_2$
\end{itemize}
\textbf{Osservazione}:\ usare $C_2$ per un tempo $t$, equivale a usare $C_1$ per un tempo $k \cdot t$.\

Algoritmo \textbf{polinomiale} che risolve il problema in $c n^s$ secondi ($c,s$ costanti)
\[C_1:\ cn_1^s = t \rightarrow n_1 = \left(\frac{t}{c}\right)^{\frac{1}{s}}\]
\[C_2:\ cn_2^s = kt \rightarrow n_2 = \left(\frac{kt}{c}\right)^{\frac{1}{s}} = k^{\frac{1}{s}}\left(\frac{t}{c}\right)^\frac{1}{s}\]
\[n_2 =k^{\frac{1}{s}}n_1\]
Miglioramento di un fattore moltiplicativo $k^{\frac{1}{s}}$.\

Algoritmo \textbf{esponenziale} che risolve il problema in $c 2^n$ secondi ($c$ costante)
\[C_1:\ c2^{n_1} = t \rightarrow 2^{n_1} = \frac{t}{c}\]
\[C_2:\ c2^{n_2} = kt \rightarrow 2^{n_2} = \frac{kt}{c} = k2^{n_1}\]
\[n_2 = n_1 + \log_2k\]
Miglioramento di un fattore moltiplicativo $\log_2k$.\

\section{Problemi}

Problema $\Pi$
\begin{itemize}
    \item $I$:\ insieme delle \textbf{istanze} in ingresso
    \item $S$:\ insieme delle \textbf{soluzioni}
\end{itemize}

\noindent \textbf{Problemi decisionali}:\ richiedono una risposta binaria ($S = \{0,1\}$)
\begin{table}[H]
    \centering
    \begin{tabular}{l l}
        Istanze positive (accettabili): & $x \in I, \mathrm{t.c.}\ \Pi(x) = 1$ \\
        Istanze negative:               & $x \in I, \mathrm{t.c.}\ \Pi(x) = 0$ \\
    \end{tabular}
\end{table}
\noindent \textbf{Problemi di ricerca}:\ data un'istanza $x$, richiedono di restituire una soluzione $s$.\

\noindent\textbf{Problemi di ottimizzazione}:\ data un'istanza $x$, si vuole trovare la migliore soluzione $s$ tra tutte le soluzioni possibili.\

\subsubsection{Problemi decisionali}

La teoria della complessità computazionale è definita principalmente in termini di \textit{problemi di decisione}:\ essendo la risposta binaria, non ci si deve preoccupare del tempo richiesto per restituire la soluzione e tutto il tempo è speso esclusivamente per il calcolo.\
La difficoltà di un problema è già presente nella sua versione decisionale.\

Molti problemi di interesse pratico sono però problemi di ottimizzazione:\ è possibile esprimere un problema di ottimizzazione in forma decisionale, chiedendo l'esistenza di una soluzione che soddisfi una certa proprietà.\
Il problema di ottimizzazione è \textbf{almeno tanto difficile quanto} il corrispondente problema decisionale.\

Caratterizzare la complessità del problema decisionale permette quindi di dare almeno una \textbf{limitazione inferiore} alla complessità del problema di ottimizzazione.\

\subsection{Classi di complessità}

Dato un problema decisionale $\Pi$ ed un algoritmo $A$, diciamo che $A$ risolve $\Pi$ se, data un'istanza di input $x$
\[A(x) = 1 \Leftrightarrow \Pi(x) = 1\]
$A$ risolve $\Pi$ in tempo $t(n)$ e spazio $s(n)$ se il tempo di esecuzione e l'occupazione di memoria di $A$ sono rispettivamente $t(n)$ e $s(n)$.\

\subsubsection{Classi Time e Space}

Data una qualunque funzione $f(n)$

\begin{itemize}
    \item \textbf{\textit{Time(f(n))}}:\ insiemi dei \textit{problemi decisionali} che possono essere risolti in tempo $O(f(n))$.\
    \item \textbf{\textit{Space(f(n))}}:\ insiemi dei \textit{problemi decisionali} che possono essere risolti in spazio $O(f(n))$.\
\end{itemize}

\subsubsection{Classe P}

Algoritmo polinomiale (tempo):\ esistono due costanti $c, n_0 >0$ t.c.\ il numero di passi elementari è al più $n^c$ per ogni input di dimensione $n$ e $\forall n>n_0$.\
La classe P è la classe dei problemi \textbf{risolvibili in tempo polinomiale} nella dimensione $n$ dell'istanza di ingresso.\

\subsubsection{Classe PSpace}

Algoritmo polinomiale (spazio):\ esistono due costanti $c, n_0 >0$ t.c.\ il numero di celle di memoria utilizzate è al più $n^c$ per ogni input di dimensione $n$ e $\forall n>n_0$.\
La classe PSpace è la classe dei problemi \textbf{risolvibili in tempo polinomiale} nella dimensione $n$ dell'istanza di ingresso.\

\subsubsection{Classe Exp-Time}

La classe Exp-Time è la classe dei problemi \textbf{risolvibili in tempo esponenziale} nella dimensione $n$ dell'istanza di ingresso.\

\subsubsection{Relazioni tra classi}

\[\mathrm{P} \subseteq \mathrm{PSpace}\]
infatti un \textbf{algoritmo polinomiale} può avere accesso al più ad un numero polinomiale di locazioni di memoria diverse (in ordine di grandezza).\
\[\mathrm{PSpace} \subseteq \mathrm{ExpTime}\]
Non è noto (ad oggi) se le inclusioni siano proprie, l'unico risultato di separazione dimostrato finora riguarda P e ExpTime:\ esiste un problema che può essere risolto in tempo esponenziale, ma per cui tempo polinomiale non è sufficiente (per esempio le ``Torri di Hanoi'').\

\subsubsection{Algoritmo per Clique}

Si considerano tutti i sottoinsiemi di vertici, in ordine di cardinalità decrescente, e si verifica se formano una clique di dimensione almeno k.\
Se $n$ è il numero di vertici, al caso peggiore l'algoritmo esamina $2^n$ sottoinsiemi diversi.\
\[\mathrm{Clique} \in \mathrm{ExpTime}\]
Algoritmo polinomiale non noto.

\subsubsection{Algoritmo per Cammino Hamiltoniano}

Si considerano tutte le permutazioni di vertici, e si verifica se i vertici in quell'ordine sono a due a due adiacenti.\
Se $n$ è il numero di vertici, al caso peggiore l'algoritmo esamina $n!$ permutazioni diverse.\
\[\mathrm{Cammino Hamiltoniano} \in \mathrm{ExpTime}\]
Algoritmo polinomiale non noto.\

\subsubsection{SAT}

Insieme $V$ di variabili booleane
\begin{itemize}
    \item Letterale:\ variabile o sua negazione.
    \item Clausola:\ disgiunzione (OR) di letterali.
\end{itemize}
Un'espressione booleana su $V$ si dice in forma normale congiuntiva (FNC) se è espressa come \textit{congiunzione di clausole} (AND di OR di letterali), per esempio:
\[V = \{x,y,z,w\}\qquad \mathit{FNC}: (x \lor \neg y \lor z) \land (\neg x \lor w) \land y\]
Data una espressione in forma normale congiuntiva verificare se esiste una assegnazione di valori di verità alle variabili che rende l'espressione vera.\
L'algoritmo per SAT considera tutti i $2^n$ assegnamenti di valore alle $n$ variabili, e per ciascuno verifica se la formula è vera.\
\[\mathrm{SAT} \in \mathrm{ExpTime}\]
Algoritmo polinomiale non noto.\

\vspace{12pt}
\noindent La ricerca esaustiva è necessaria?\
Non lo sappiamo.\

\subsection{Problemi decisionali e certificati}
In un \textit{problema decisionale} siamo interessati a verificare se una istanza del problema soddisfa una certa proprietà.
\begin{itemize}
    \item Esiste una clique di $k$ vertici?
    \item Esiste un cammino hamiltoniano?
    \item Esiste un assegnamento di valori che rende vera la formula?
\end{itemize}
Per alcuni problemi, per le \textbf{istanze accettabili} (positive) $x$ è possibile fornire un \textbf{certificato} $y$ che possa convincerci del fatto che l'istanza soddisfa la proprietà e dunque è un'istanza accettabile.\

\subsubsection{Certificato}

\begin{itemize}
    \item Certificato per Clique:\ sottoinsieme di $k$ vertici, che forma la clique.\
    \item Certificato per Cammino Hamiltoniano:\ permutazione degli $n$ vertici che definisce un cammino semplice.\
    \item Certificato per SAT:\ un'assegnazione di verità alle variabili che renda vera l'espressione.\
\end{itemize}

\noindent Un certificato è un \textbf{attestato breve di esistenza} di una soluzione con determinate proprietà.\
Si definisce solo per le istanza \textit{accettabili}, infatti, in generale non è facile costruire \textit{attestati di non esistenza}.\

\subsubsection{Verifica}

Idea:\ \textbf{utilizzare il costo della verifica} di un certificato (una soluzione) per un'istanza accettabile (positiva) per \textbf{caratterizzare la complessità} del problema stesso.\

Un problema $\Pi$ è verificabile in tempo polinomiale se:\
\begin{enumerate}
    \item ogni {istanza accettabile} $x$ di $\Pi$ di lunghezza $n$ ammette un \textbf{certificato $y$ di lunghezza polinomiale} in $n$;
    \item esiste un \textbf{algoritmo di verifica polinomiale} in $n$ e applicabile a ogni coppia $\langle x,y\rangle$ che permette di attestare che $x$ è accettabile.\
\end{enumerate}

\subsection{Classe NP}

NP è la classe dei problemi decisionali \textbf{verificabili in tempo polinomiale}.\
Cosa vuol dire NP? P sta per polinomiale, ma N non vuol dire NON...
La classe NP è la classe dei problemi risolvibili in tempo \textbf{\textit{polinomiale non deterministico}}.\

\textbf{Osservazione}:\ un certificato contiene un'informazione molto prossima alla soluzione, quindi qual è l'interesse di questa definizione?\
Dubbio legittimo:
\begin{itemize}
    \item La teoria della verifica è utile per far luce sulle gerarchie di complessità dei problemi, non aggiunge nulla alla possibilità di risolverli efficientemente.
    \item Chi ha una soluzione può verificare in tempo polinomiale che l'istanza è accettabile.
    \item Chi non ha una soluzione (certificato), può individuarla in tempo esponenziale considerando tutti i casi possibili con una ricerca esaustiva.
\end{itemize}

\subsubsection{Le classi P e NP}

\[P \subset NP\]
Ogni problema in P ammette un certificato verificabile in tempo polinomiale:\ eseguo l'algoritmo che risolve il problema per costruire il certificato.\
\[\mathit{P} = \mathit{NP}\ \mathrm{oppure}\ \mathit{P} \neq \mathit{NP}?\]
Non sappiamo se per risolvere un problema della classe \textit{NP} dobbiamo per forza passare attraverso una ricerca esaustiva oppure no.\
Si congettura che $\mathit{P} \neq \mathit{NP}$.\
È possibile individuare i problemi più difficili all'interno della classe \textit{NP}, ovvero quelli candidati ad appartenere a \textit{NP} se $\mathit{P} \neq \mathit{NP}$.\

\subsection{Problemi NP-Completi}
Sono i problemi più difficili all'interno della classe NP:\ se esistesse un algoritmo polinomiale per risolvere uno solo di questi problemi, allora tutti i problemi in NP potrebbero essere risolti in tempo polinomiale e dunque $\mathit{P} = \mathit{NP}$.\
Quindi tutti i problemi NP-Completi sono risolvibili in tempo polinomiale oppure nessuno lo è.\

\subsubsection{Riduzioni polinomiali}

$\Pi_1$ e $\Pi_2$ sono problemi decisionali, $I_1$ e $I_2$ insiemi delle istanze di input di $\Pi_1$ e $\Pi_2$.\
$\Pi_1$ \textit{si riduce in tempo polinomiale} a $\Pi_2$
\[\Pi_1 \leq_p \Pi_2\]
se esiste una funzione $f: I_1 \rightarrow I_2$ calcolabile in tempo polinomiale t.c. $\forall x \in \Pi_1$, $x$ è un'istanza accettabile di $\Pi_1$ se e solo se $f(x)$ è un'istanza accettabile di $\Pi_2$.\

Se esistesse un algoritmo per risolvere $\Pi_2$ potremmo utilizzarlo per risolvere $\Pi_1$
\[\Pi_1 \leq_p \Pi_2\ \mathrm{e}\ \Pi_2 \in \mathit{P} \Rightarrow \Pi_1 \in \mathit{P}\]

\noindent Un problema $\Pi$ si dice \textbf{NP-arduo} se
\[\forall\ \Pi' \in \mathit{NP}, \Pi' \leq_p \Pi\]

\noindent Un problema decisionale $\Pi$ si dice \textbf{NP-Completo} se
\[\Pi \in \mathit{NP}\ \mathrm{e}\ \forall\ \Pi' \in \mathit{NP}, \Pi' \leq_p \Pi\]

\noindent Dimostrare che un problema è in NP può essere facile, basta esibire un certificato polinomiale.\
Non è altrettanto facile dimostrare che un problema $\Pi$ è NP-arduo o NP-completo:\ bisogna dimostrare che TUTTI i problemi in NP si riducono polinomialmente a $\Pi$.\
In realtà la prima dimostrazione di NP-completezza aggira il problema.\

\begin{theorem}[Cook, 1971]
    SAT è NP completo.\
\end{theorem}

\noindent Cook ha mostrato che dati un qualunque problema $\Pi \in \mathit{NP}$ ed una qualunque istanza $x$ per $\Pi$ si può costruire una espressione booleana in forma normale congiuntiva che descrive il calcolo di un algoritmo per risolvere $\Pi$ su $x$:\ l'espressione è vera se e solo se l'algoritmo restituisce 1.\

Un problema decisionale $\Pi$ è NP-completo se
\[ \Pi \in \mathit{NP}\qquad \mathrm{SAT} \leq_p \Pi\]
(o un qualsiasi altro problema NP-completo); infatti,
\[\forall\ \Pi' \in \mathit{NP},\ \Pi' \leq_p \mathrm{SAT\ e\ SAT} \leq_p \Pi\]
Quindi $\Pi' \leq_p \Pi$.\

\[\mathrm{SAT} \leq_p \mathrm{Clique} \]
data un'espressione booleana $F$ in forma normale congiuntiva con $k$ clausole è possibile costruire in \textit{tempo polinomiale} un grafo $G$ che contiene una \textbf{clique di $k$ vertici se esolo se $F$ è soddisfacibile}.\

\[\mathrm{SAT} \leq_p \mathrm{Clique} \Rightarrow \mathrm{Clique\ \grave{e}\ NP\textrm{-}Completo}\]
\[\mathrm{SAT\ \grave{e}\ NP\textrm{-}Completo} \Rightarrow \mathrm{Clique} \leq_p \mathrm{SAT}\]
SAT e Clique sono \textbf{NP-equivalenti}.\
Tutti i problemi NP completi sono tra loro NP-equivalenti; sono tutti facili, o tutti difficili.\

\subsubsection{Problemi di ottimizzazione NP-arduo}

Se la soluzione ottima è troppo difficile da ottenere, una soluzione quasi ottima ottenibile facilmente forse è buona abbastanza.\
A volte, avere una so\-luzione \textbf{esatta} non è strettamente necessario:\ ci si accontenta di una soluzione che non si discosti troppo da quella ottima e che si possa calcolare in tempo polinomiale.\

\subsection{Le classi Co-P e Co-NP}

Profonda differenza tra certificare l'esistenza o la non-esistenza di una soluzione, per esempio nel problema del ciclo Hamiltoniano:
\begin{itemize}
    \item una permutazione di vertici (per certificare esistenza)
    \item per la non-esistenza è difficile dare un certificato polinomiale che indichi direttamente questa proprietà
\end{itemize}

\begin{flushleft}
    $\Pi$:\ problema decisionale.

    $\mathrm{co}\Pi$:\ problema \textit{complementare} (accetta tutte e sole le istanze rifiutate da P.
\end{flushleft}

\noindent La classe co-P è la classe dei \textbf{problemi decisionali} $\Pi$ per cui \[\mathrm{co}\Pi \in \mathit{P}\]
P = co-P (risolvo il problema, e complemento il risultato).\

Si congettura che NP e co-NP siano diverse; se questa congettura è vera, allora $\mathit{P \neq NP}$.\

\chapter{Metriche e rilevamento delle anomalie}

\section{Metriche}

Con sistemi come NetFlow è possibile andare a monitorare il traffico di rete, tuttavia è importante anche riuscire a capire che cosa monitorare, cioè quali sono le metriche importanti da monitorare.\
Dal punto di vista di un ISP è importante andare a monitorare il Service Level Agreements:\ il provider deve capire se la rete sta fuzionando, se le linee internet a sua disposizione sono adeguate per il traffico che sta passando e deve fare delle previsioni per riuscire a capire se ha abbastanza banda per crescere nel futuro.\
Dal punto di vista dell'utente non si parla di linea, di connettività, di Autonomous Systems\dots\ ma di servizi:\ riuscire a connettersi alla propria mailbox, connettersi a Google tramite il proprio browser,\ \dots

Il traffico può essere monitorato:
\begin{itemize}
    \item sugli end-system (poiché richiede di installarvi una sonda di solito viene evitato);
    \item sfruttando sistemi esistenti, per esempio usando NetFlow su un router;
    \item usando delle macchine appositamente configurate per lo scopo.
\end{itemize}

\subsection{Benchmarking}

Al contario delle metriche della vita quotidiana (kg, litri,\ \dots), le metriche usate per il traffico di rete spesso non sono standardizzate e le misure dei fornitori sono eseguite in modi leggermente diversi, rendendo i risultati difficili da confrontare.\
Per questo gli operatori e il mondo delle reti hanno definito degli standard di \textit{benchmarking} che permettono di creare dei termini di paragone.\

\begin{itemize}
    \item \textbf{Disponibilità}:\ può essere espressa come la percentuale di tempo che un sistema, un componente o un'applicazione di rete è disponibile per l'utente.\ È basata sull'affidabilità dei singoli componenti di una rete.
          \[\%\ \mathrm{Availability} = \frac{\mathrm{MTBF}}{\mathrm{MTBF + MTTR}}\]
          MTBF è il tempo medio tra i fallimenti e MTTR è il tempo medio per riparare il fallimento.
    \item \textbf{Tempo di risposta}:\ tempo che intercorre in un sistema tra un input e la sua reazione.\ È desiderabile che sia il più breve possibile.
          \begin{itemize}
              \item \textbf{Network delay}:\ quanto una rete ritarda l'esecuzione di un'azione.
              \item \textbf{Application delay}:\ quanto un'applicazione ritarda l'esecuzione di un'azione.
          \end{itemize}
    \item \textbf{Throughput}:\ misura la quantità di dati che può essere inviata su un link in un certo momento.\ Spesso viene utilizzato per stimare la larghezza di banda disponibile sul link.\ Si noti che throughput e banda sono misure diverse:\ il primo è orientato all'applicazione.
    \item \textbf{Goodput}:\ come il throughput, ma considera solo il payload del pacchetto, cioè i ``dati reali'' che sono trasportati dalla rete.\ È utile per identificare le cosiddette \textit{stealth connections}.
    \item \textbf{Utilizzo}:\ è una misura più precisa rispetto al throughput, si riferisce alla determinazione della percentuale di tempo in cui una risorsa è in uso in un determinato periodo di tempo.\ Spesso un utilizzo molto scarso significa che qualcosa non funziona come previsto.
\end{itemize}

\noindent Per interpretare tali metriche sono vengono utilizzate la media, la varianza e la deviazione standard.\
Viene anche definito un intervallo di confidenza ($\mathrm{media\ \pm\ margine\ di\ errore}$), i dati al di fuori di tale intervallo vengono scartati.

\vspace{12pt}
\noindent\textbf{Percentile}:\ il valore al di sotto del quale cade una determinata percentuale di osservazioni in un gruppo di osservazioni.

\vspace{12pt}
\noindent Nel networking un percentile popolare è il 95esimo.\

\vspace{12pt}
\noindent\textbf{Quartile}:\ è una divisione di serie in quarti.

\begin{itemize}
    \item $\mathrm{Q_1}$ è definito come il 25esimo percentile.
    \item $\mathrm{Q_2}$ è definito come il 50esimo percentile.
    \item $\mathrm{Q_3}$ è definito come il 75esimo percentile.
\end{itemize}

\subsubsection{Outlier}

Gli outliers (valori anomali) sono i valori che ``si trovano al di fuori'' degli altri.\
In statistica un valore è outlier quando cade al di fuori del

\begin{itemize}
    \item limite inferiore:\ $\mathrm{Q_1} - 1,5 \cdot \mathrm{IQR}$
    \item limite superiore:\ $\mathrm{Q_3} + 1,5 \cdot \mathrm{IQR}$
\end{itemize}
dove IQR (interquartile range):\ $\mathrm{Q_3 - Q_1}$.\
Gli outliers vengono utilizzati per individuare valori anomali, spesso denominati ``aberranti'', ovvero che si discostano da uno standard accettato, tra cui:
\begin{itemize}
    \item errori sperimentali/di misurazione;
    \item distribuzioni a coda pesante (cioè che vanno a zero molto lentamente) in cui uno o più valori molto grandi influenzeranno le statistiche.
\end{itemize}

\subsubsection{Latenza e jitter}

La \textbf{latenza} è la quantità di tempo che impiega un pacchetto per andare da sorgente a destinazione ed è molto importante per le applicazioni interattive.\
$\frac{\mathrm{RTT}}{2}$ non è la latenza di una rete, ma la somma della latenza client-to-server e della latenza server-to-client è l'RTT.\

Il \textbf{jitter} è la varianza dell'intra-packet delay su un link monodirezionale ed è molto importante per le applicazioni multimediali.\
\[\mathrm{jitter} = \frac{\sum_i (|x_i - x_{i-1}|)}{n-1}\]
È un modo per misurare quanto irregolare è un segnale.\
\textbf{Nota}:\ il jitter tiene conto dell'ordine degli eventi mentre deviazione standard, media\dots\ non lo fanno.

\subsubsection{Bandwidth}

Per misurare la banda va definito inanzitutto l'intervallo di misura (TC), solitamente viene fissato a un secondo.\
Si tratta dell'intervallo di tempo, o ``intervallo di larghezza di banda'', utilizzato per controllare i picchi di traffico.\
Il Burst Committed (BC) è il numero massimo di bit che la rete accetta di trasferire durante qualsiasi TC, mentre il CIR (Committed Information Rate) è la velocità con cui una rete accetta di trasferire informazioni in condizioni normali, mediata su un incremento minimo di tempo:\ si tratta di una delle principali metriche tariffarie negoziate ed è misurato in bit al secondo.\
\[\mathrm{CIR} = \frac{\mathrm{BC}}{\mathrm{TC}}\]
Inoltre viene definito il Burst Excess (BE), ossia il numero di bit che si tenta di trasmettere dopo aver raggiunto il valore BC.\
La velocità massima che può essere raggiunta (Maximum Data Rate) viene calcolata come
\[\frac{\mathrm{BC + BE}}{\mathrm{BC}}\cdot\mathrm{CIR}=\frac{\mathrm{BC + BE}}{\mathrm{TC}}\]
Si noti che la banda è una misura per link, mentre il throughput è una misura per comunicazione.\

\subsubsection{Approcci di monitoraggio}

\begin{itemize}
    \item Misure attive:\ si inietta del traffico e si studia come la rete reagisce a tale traffico.
    \item Misure passive:\ non si inietta nessun traffico, ma si osserva ciò che passa tramite la rete.
\end{itemize}

\noindent Le misure attive sono spesso end-to-end, mentre quelle passive sono limitate al link dove viene osservato il traffico.\

\section{Rilevamento delle anomalie}

Un'\textbf{anomalia} è un'osservazione che si discosta tanto dalle altre osservazioni da far sospettare che sia stata generata da un meccanismo diverso.\
Data una misura e trovati gli outliers è necessario capire se quest'ultimi rappresentino un evento interessante o dei dati non voluti (sbagliati):\ nel primo caso l'obiettivo è quello di analizzare l'outlier, nel secondo è eseguire un \textit{data cleaning}.\

\begin{table}[H]
    \centering
    \begin{tabular}{|l|l|}
        \hline
        \multicolumn{2}{|c|}{\textit{\textbf{Definizioni}}}                             \\\hline
        \textbf{Serie}          & sequenza ordinata di numeri                           \\
        \textbf{Ordine}         & indice di un numero nella serie                       \\
        \textbf{Timeseries}     & serie di punti ordinati nel tempo                     \\
        \textbf{Osservazione}   & valore numerico osservato in un tempo specifico       \\
        \textbf{Forecast}       & stima di un valore atteso in un momento specifico     \\
        \textbf{Forecast Error} & differenza dell'osservazione rispetto alla previsione \\
        \textbf{SSE}            & la somma degli errori di una serie al quadrato        \\\hline
    \end{tabular}
\end{table}

\noindent Le serie temporali possono essere di due tipi, \textbf{univariate} e \textbf{multivariate}.\
Una \textit{serie temporale univariata} è costituita da singole osservazioni (scalari) registrate in sequenza su incrementi di tempo uguali, per esempio la temperatura giornaliera, mentre una \textit{serie temporale multivariata} ha più di una variabile dipendente dal tempo:\ ogni variabile dipende non solo dai suoi valori passati, ma ha anche una certa dipendenza da altre variabili; questa dipendenza viene utilizzata per prevedere i valori futuri, ad esempio le previsioni del tempo che dipendono da temperatura, vento, umidità, copertura nuvolosa.

Una serie temporale è \textit{stazionaria} quando le sue proprietà statistiche, media e varianza, non cambiano nel tempo, cioè non hanno stagionalità o trend.\
Un contatore non è stazionario, ma può essere reso stazionario scrivendo la serie temporale come $\mathrm{osservazione}(t) - \mathrm{osservazione}(t-1)$ [gauge].\
La stazionarietà è importante poiché per la previsione abbiamo bisogno di una sorta di invarianza:\ intuitivamente una variabile della serie temporale è stazionaria rispetto a un qualche percorso di equilibrio se dopo uno shock tende a ritornare su quel percorso, invece, una serie non è stazionaria se si sposta su un nuovo percorso dopo uno shock.\
È molto difficile modellare il percorso di una variabile che cambia percorso se è soggetta a qualche shock.

\subsection{Medie e previsioni}

Data una serie ci sono diversi modi per prevedere il valore del punto $x+1$.\
\begin{itemize}
    \item Media semplice:\ il prossimo punto è la media dei punti della serie.
    \item Media mobile:\ come la media semplice ma calcolata sugli ultimi $n$ punti più rilevanti di quelli precedenti.
    \item Media mobile ponderata:\ uguale alla media mobile con un peso assegnato a ciascun punto in base alla loro età.
\end{itemize}

\subsubsection{Single Exponential Smoothing}

La seguente formula (Poisson, Holt o Roberts) afferma che il valore atteso/livellato (smoothed) è la somma di due prodotti (formula ricorsiva):
\[\hat{y}_x = \alpha\cdot y_x + (1-\alpha)\cdot \hat{y}_{x-1}\]
$\alpha$ (\textit{smoothing factor}) è un ``decadimento della memoria'':\ indica l'influenza del passato.\
Il processo per trovare il migliore valore di $\alpha$ si chiama \textit{fitting}.\

\vspace{12pt}
\noindent\textbf{Nota}:\ questa è essenzialmente una funzione di livellamento piuttosto che una previsione.\

\vspace{12pt}
\noindent In questo caso si fa una previsione sul futuro andando a considerare i valori passati:\ invece di dare un peso in base alla finestra temporale passata, si vuole dare un peso anche al trend (\textit{double exponential smoothing}).

Chiamiamo $\hat{y}_x$ \textit{level} $\mathcal{L}_x$ e definiamo il trend (o slope) di due punti consecutivi come la differenza tra tali valori:\ $b = y_x - y_{x-1}$.\
Otteniamo quindi
\[\mathcal{L}_x = \alpha\cdot y_x + (1-\alpha)\cdot(\mathcal{L}_{x-1} + b_{x-1})\]
\[b_x = \beta\cdot(\mathcal{L}_x - \mathcal{L}_{x-1}) + (1-\beta)\cdot b_{x-1}\]
\[\hat{y}_{x+1} = \mathcal{L}_x + b_x\]
dove $\beta$ è detto \textit{trend factor}.

Nel caso in cui si voglia tenere in considerazione anche la stagionalità dei valori, è necessario introdurre il \textit{triple exponential smoothing} (o \textit{Holt-Winters method}).\

\subsubsection{Triple Exponential Smoothing}

Quando una serie si ripete a intervalli regolari è detta \textit{stagionale}.\
La durata di una stagione è il numero di punti nella stagione.\
La componente stagionale è una deviazione rispetto a $\hat{y}_{x+1}$ calcolato nel double exponential smoothing che si ripete per ogni stagione; quindi per ogni datapoint stagionale viene definita una componente stagionale $s$.
\[s_x = \gamma\cdot(y_x - \mathcal{L}_x) + (1-\gamma)\cdot s_{x-L} \]
\[\hat{y}_{x+m} = \mathcal{L}_x + m\cdot b_x + s_{x-L+1+(m-1)\%L}\]
dove $\gamma$ è detto \textit{seasonal smoothing factor} e $S_{x-L+1+(m-1)\%L}$ indica che il seasonal factor deve essere ripetuto in loop per ogni durata della stagione.\
Poiché la serie temporale ha una stagione, il numero di pronostici è arbitrario, consentendo di prevedere un numero arbitrario di punti (con errore crescente).\

$\alpha$, $\beta$ e $\gamma$ sono i parametri di adattamento dell'algoritmo e $0 < \alpha$, $\beta, \gamma < 1$.\
Valori maggiori indicano che l'algoritmo si adatta più velocemente e le previsioni riflettono osservazioni recenti nelle serie temporali; valori più piccoli significano che l'algoritmo si adatta più lentamente, dando più peso alla cronologia passata delle serie temporali.\
I valori di $\alpha,\ \beta\ \mathrm{e}\ \gamma$ possono essere determinati cercando di determinare il più piccolo SSE (somma degli errori al quadrato) con un processo iterativo chiamato \textit{fitting}.\

\subsection{Misurare una deviazione}

Se Holt-Winters può prevedere un punto, possiamo rilevare anomalie quando l'osservazione si discosta troppo dalla previsione.\
La deviazione può essere rilevata quando l'osservazione non rientra nelle bande di confidenza minima e massima per un dato punto:
\[d_t= \gamma \cdot |(y_t-\hat{y}_t)| + (1-\gamma) \cdot d_{t-m}\]
Bande di fiducia
\begin{itemize}
    \item Banda superiore:\ $\hat{y}_t+\delta \cdot d_{t-m}$
    \item Banda inferiore:\  $\hat{y}_t - \delta \cdot d_{t-m}$
\end{itemize}
dove $\delta$ è detto \textit{confidence scaling factor} e tipicamente è compreso tra 2 e 3.\

\chapter{Il livello della Macchina Assembler}

\section{Modello a programma memorizzato}

Come già introdotto, il modello di calcolatore adottato comunemente è quello di \textit{Von Neumann} o a programma memorizzato, caratterizzato dal fatto che il linguaggio assembler è di tipo imperativo, quindi facente uso

\begin{enumerate}
    \item del concetto di variabile, e della tecnica di assegnamento e riassegnamento di variabili,
    \item del concetto di sequenzializzazione esplicita delle istruzioni, implementato da una particolare variabile detta contatore di istruzioni, IC (instruction counter).
\end{enumerate}

\noindent Ogni istruzione assembler è rappresentata in binario secondo un certo formato dell'istruzione.\
Tipicamente, il formato dell'istruzione contiene elementi come:

\begin{itemize}
    \item codice operativo dell'istruzione,
    \item informazioni per ricavare indirizzi di variabili o di altre istruzioni,
    \item eventuali costanti,
    \item eventuali opzioni legate all'ottimizzazione del programma e dell'architettura.
\end{itemize}

\noindent Al momento della loro utilizzazione durante l'esecuzione del programma, le variabili possono risiedere in:

\begin{itemize}
    \item  locazioni di memoria principale,
    \item  registri generali.
\end{itemize}

\noindent Il linguaggio assembler deve prevedere un insieme di regole per indirizzare una locazione da parte di un'istruzione:\ tali regole saranno dette \textbf{modi di indirizzamento}.

I registri generali rappresentano, in un certo senso, un'estensione della memoria disponibile al programma.\
Essendo implementati, a livello firmware, nella Parte Operativa del processore, essi sono caratterizzati da un tempo di accesso molto basso, di diversi ordini di grandezza inferiore a quello necessario per leggere o scrivere parole residenti in memoria principale.\
Normalmente, i registri generali sono organizzati in una (piccola) memoria di registri; la sua capacità può variare da poche unità a qualche decina fino alle centinaia (noi supporremo che sia di 64 unità).

Il significato dell'aggettivo ``\textit{generali}'' sta a indicare che l'utilizzo di tali registri non è specifico di una certa funzionalità.\
Un registro generale verrà usato per contenere il valore di una variabile riferita dal programma, ma anche un indirizzo di memoria o una quantità che viene utilizzata per calcolare un indirizzo di memoria, o altre eventuali informazioni manipolabili a livello assembler.

I registri generali sono, ovviamente, visibili anche a livello firmware secondo la specifica struttura che
assumono nella Parte Operativa del processore.

In alcune macchine, il contatore istruzioni è, come i registri generali, \textit{visibile a livello assembler} e su esso sono possibili tutte le operazioni consentite sui registri generali.\
In altre macchine, \textit{IC è visibile solo a livello firmware}, nel senso che la sua manipolazione (modifica) è effettuata dal microprogramma del processore nell'interpretare ogni istruzione assembler.\
In questo secondo caso, esistono comunque istruzioni assembler (\textit{istruzioni di salto}) il cui effetto finale è di modificare il contenuto di IC per fargli assumere un valore fuori sequenza, pur se IC non è riferito esplicitamente dall'istruzione; allo stesso modo, tutte le istruzioni non di salto modificano IC nel senso di incrementarlo per puntare all'istruzione immediatamente successiva in sequenza, ma, a maggior ragione, questo effetto è implicito e IC non è riferito esplicitamente dall'istruzione.

In ogni caso, \textit{la semantica di un'istruzione} deve contenere sempre la manipolazione di IC; si ricorda infatti che la semantica di un comando deve fornire tutti gli elementi che permettano di definirne il supporto (interprete).

\section{Compilazione ed esecuzione}

\subsection{Programmi e processi}

I \textit{moduli di elaborazione} a livello delle applicazioni e a livello del sistemo operativo sono i \textit{processi}:\ programmi sequenziali con propria capacità di controllo, eseguibili con concorrentemente ad altri e cooperanti con essi.

Un qualunque programma \textit{viene compilato in un processo} o, nel caso che il programma stesso sia parallelo, in una collezione di processi.

La fase di linking ha il compito di collegare a ogni processo applicativo i \textit{servizi} di sistema necessari all'esecuzione del programma corrispondente e non visibili esplicitamente nel codice sorgente e di arricchire il processo con un insieme di \textit{informazioni di utilità}, cioè informazioni che sono essenziali per l'esecuzione e la gestione del processo, che sono contenute in una \textit{struttura dati} che chiameremo \textbf{descrittore di processo} (PCB, \textit{Process Control Back}).\
Il PCB stesso fa parte delle informazioni di un processo.

\subsection{Compilazione e caricamento}

La compilazione di un programma è effettuata da un compilatore, il quale, prendendo in ingresso il programma sorgente, lo trasforma in un processo \textit{rappresentato da un \textbf{file oggetto} binario}, o \textbf{codice eseguibile}, contenete tutte le informazioni necessarie per l'esecuzione del processo stesso.

Il procedimento di compilazione costa, operativamente, di più fasi:

\begin{itemize}
    \item traduzione da linguaggio sergente in assembler simbolico;
    \item traduzione da assembler simbolico a codice binario;
    \item linking.
\end{itemize}

\noindent In base alle regole di compilazione legate alla struttura del linguaggio sorgente, il compilatore provvede a generare una lista di istruzioni assembler.\
Nel far questo, un importante compito del compilatore è quello:

\begin{itemize}
    \item dell'allocazione statica della memoria virtuale e dei registri generali,
    \item dell'inizializzazione di una parte delle locazioni di memoria virtuale e registri generali.
\end{itemize}

\noindent Gli indirizzi generati dal processore, durante l'esecuzione di un processo, non sono direttamente indirizzi di memoria principale (\textit{indirizzi fisici}), bensì \textbf{indirizzi logici}, cioè indirizzi riferiti a un'\textit{astrazione} della memoria del processo, detta \textbf{memoria virtuale}.\
Questa può essere vista come un array unidimensionale, con indici (indirizzi logici) a partire da zero fino al massimo necessario per rappresentare il programma o fino al massimo consentito dall'ampiezza dell'indirizzo logico in bit; per una macchina a 32 bit e con indirizzamento alla parola, la massima ampiezza della memoria virtuale di ogni programma è 4G parole.

L'insieme degli indirizzi logici di un processo è detto il suo \textbf{spazio logico di indirizzamento}.

Il codice eseguibile del processo, generato dal compilatore, è quindi riferito alla memoria virtuale.\
Il processore genera indirizzi logici sia per le istruzioni sia per i dati.

Quando un processo viene allocato (o riallocato) in memoria principale, viene stabilita una corrispondenza tra gli indirizzi logici della parte allocata e gli indirizzi fisici in cui viene allocata.\
Questa funzione, detta \textit{funzione di rilocazione} o \textit{di traduzione dell'indirizzo}, è di norma implementata come una tabella associata al processo (\textbf{Tabella di Rilocazione}).

La funzione di rilocazione viene aggiornata ogni volta che l'allocazione del processo viene modificata.

La traduzione dell'indirizzo deve essere effettuata in modo molto efficiente e quindi l'accesso alla Tabelle di Rilocazione viene effettuata con opportune soluzioni hardware-firmware delegate a un'unità interposta tra il Processore e la memoria principale (\textit{Memory Management Unit}, o \textit{MMU}).

\newpage

\section{D-RISC:\ un assembler didattico di tipo Risc}

Le principali caratteristiche della macchina assembler, che chiameremo D-RISC, sono le seguenti:

\begin{itemize}
    \item sono presenti 64 \textit{registri generali} RG[0..63], mentre il registro \textit{contatore istruzioni} (IC) è visibile solo a livello firmware; solo alcune istruzioni, tipicamente di salto o speciali, hanno effetto sul contenuto di IC, ma, a livello assembler, non esiste alcun altro modo di modificare tale registro;
    \item \textit{la parola è di} 32 \textit{bit}:\ questa caratteristica non è comunque vincolante per quanto riguarda la rappresentazione dei dati, che possono essere a 64 bit senza alterare le altre caratteristiche della macchina.\ Come detto sotto, vincolante è invece la lunghezza dell'istruzione e la lunghezza dell'indirizzo logico;
    \item la memoria, sia virtuale sia fisica, è \textit{indirizzabile} alla parola;
    \item \textit{lo spazio di indirizzamento logico di un processo è unidimensionale} e gli indirizzi logici sono di 32 bit.\ Lo spazio di indirizzamento logico di un processo è dunque più ampio al più 4G parole;
    \item le istruzioni sono tutte rappresentate su 32 bit.\ Supporremo che il \textit{codice operativo} sia codificato negli 8 bit più significativi; sono dunque disponibili fino a 256 istruzioni, anche se di fatto ne potranno essere implementate meno.
\end{itemize}

\section{Regole di compilazione e composizione}

\subsection{Comandi condizionali e iterativi}

Indicheremo con \textit{C} un predicato (guardia) e con \textit{B} un comando o blocco (sequenza) di comandi.\ Con IF\_\textit{C} ETICHETTA e IF\_N\_\textit{C} ETICHETTA indicheremo, in generale, le istruzioni assembler di salto condizionato \textit{C} e not \textit{C} rispettivamente.

La compilazione di un comando/blocco \textit{B} nella corrispondente sequenza di istruzioni sarà indicata con \textit{compile} (\textit{B}).

Abbiamo allora le seguenti regole:

\begin{figure}[H]
    \centering
    \includegraphics[width=\textwidth]{immagini/Regole.png}
\end{figure}

\subsection{Compilazioni di procedure e funzioni}

Si consideri una computazione che contenga una procedura o una funzione, nel seguito indicata con \textit{S}.

I problemi da risolvere per la sua compilazione sono essenzialmente:

\begin{enumerate}
    \item \textit{meccanismo di chiamata} a \textit{S} e \textit{di ritorno} da \textit{S} al programma chiamante;
    \item \textit{passaggio di parametri e risultati} tra programma chiamante e programma chiamato \textit{S}.
\end{enumerate}
\begin{enumerate}
    \item All'atto della \textit{chiamata}, nel programma chiamante occorre salvare l'indirizzo di ritorno ed effettuare un salto all'indirizzo di \textit{S}:\ queste due azioni sono effettuate dall'unica istruzione $\mathtt{CALL\ R_{proc}}, \mathtt{R_{ret}}$.\ Il \textit{ritorno} da \textit{S} al programma chiamante avviene semplicemente con l'istruzione (già nota) $\mathtt{GOTO\ R_{ret}}$ dove $\mathtt{R_{ret}}$ è l'indirizzo del registro nel quale è stato salvato il valore di IC+1 al momento della chiamata.
    \item \textbf{Passaggio dei parametri}:\ nel caso più semplice, ma piuttosto frequente, in cui i parametri di ingresso e di uscita di \textit{S} siano relativamente pochi, il passaggio di tali parametri può essere effettuato attraverso registri generali.\ Altrimenti, e comunque qualora le procedure/funzioni siano annidate o ricorsive, occorre utilizzare una struttura dati in memoria di tipo pila gestita mediante istruzioni di \texttt{Load} e \texttt{Store}.
\end{enumerate}

\begin{figure}[H]
    \centering
    \includegraphics[width=\textwidth]{immagini/Istruzioni.png}
\end{figure}
\includepdf[pages=-]{istruzioni ass DRISC}


\chapter{Data Definition Language(DDL)}

Introduciamo il \textbf{Data Definition Language} (\textbf{DDL}) SQL, che consiste nell'insieme delle istruzioni SQL che permettono la creazione, modifica e cancellazione delle tabelle, dei domini e degli altri oggetti del database, al fine di definire il suo schema logico.\

Le tabelle (corrispondenti alle relazioni dell'algebra relazionale) vengono definite in SQL mediante l'istruzione \texttt{CREATE TABLE}.\
Questa istruzione
\begin{itemize}
	\item definisce uno schema di relazione e ne crea un'istanza vuota;
	\item specifica attributi, domini e vincoli.
\end{itemize}
L'istruzione \texttt{CREATE TABLE} è seguita da un nome che la caratterizza e dalla lista delle colonne (attributi) di cui si specificano le caratteristiche.\
Alla fine si possono anche specificare eventuali vincoli di tabella.

\begin{flushleft}

	$\mathtt{CREATE\ TABLE\ \langle nome\_tabella\rangle}$

	\quad $\mathtt{nomeColonna_1\ tipoColonna_1\ clausolaDefault_1\ vincoloDiColonna_1,}$

	\quad \dots

	\quad $\mathtt{nomeColonna_k\ tipoColonna_k\ clausolaDefault_k\ vincoloDiColonna_k,}$

	\quad \texttt{vincoli di tabella}

\end{flushleft}

\noindent L'effetto del comando \texttt{CREATE TABLE} è quello di definire uno schema di relazione e di crearne un'istanza vuota, specificandone attributi, domini e vincoli.\
Una volta creata, la tabella è pronta per l'inserimento dei dati che dovranno soddisfare i vincoli imposti.\

La visualizzazione dello schema di una tabella, dopo che è stata creata, può essere ottenuta mediante il comando SQL \texttt{DESCRIBE}:
\begin{center}
	$\mathtt{DESCRIBE\ \langle nomeTabella\rangle}$
\end{center}

\noindent Quindi SQL non è solo un linguaggio di interrogazione (Query Language), ma è anche un linguaggio per la definizione di basi di dati (Data-definition language (DDL)), un linguaggio per stabilire controlli sull'uso dei dati (GRANT) e un linguaggio per modificare i dati.

\subsubsection{I tipi}

I tipi più comuni per i valori degli attributi sono:
\begin{itemize}
	\item \texttt{CHAR(n)} per stringhe di caratteri di lunghezza fissa \texttt{n};
	\item \texttt{VARCHAR(n)} per stringhe di caratteri di lunghezza variabile di al massimo \texttt{n} caratteri;
	\item \texttt{INTEGER} per interi con la dimensione uguale alla parola di memoria standard dell'elaboratore;
	\item \texttt{REAL} per numeri reali con dimensione uguale alla parola di memoria standard dell'elaboratore;
	\item \texttt{NUMBER(p,s)} per numeri con \texttt{p} cifre, di cui \texttt{s} decimali;
	\item \texttt{FLOAT(p)} per numeri binari in virgola mobile, con almeno \texttt{p} cifre significative;
	\item \texttt{DATE} per valori che rappresentano istanti di tempo (in alcuni sistemi, come Oracle), oppure solo date (e quindi insieme ad un tipo \texttt{TIME} per indicare ora, minuti e secondi).
\end{itemize}

\subsubsection{Modificare una tabella}

Ciò che si crea con un \texttt{CREATE} si può eliminare con il comando \texttt{DROP} o cambiare con il comando \texttt{ALTER}.
\begin{flushleft}
	\texttt{CREATE TABLE Nome}

	\quad \texttt{(Attributo Tipo [ValoreDefault] [VincoloAttributo]}

	\qquad \texttt{\{Attributo Tipo [Default] [VincoloAttributo]\},}

	\qquad \texttt{\{VincoloTabella\})}

\end{flushleft}

\begin{flushleft}
	Default := \texttt{DEFAULT} {valore $|$ null $|$ username}
\end{flushleft}

\noindent  Nuovi attributi si possono aggiungere con:
\begin{flushleft}
	\texttt{ALTER TABLE Nome ADD COLUMN NuovoAttr Tipo}
\end{flushleft}

\noindent Con il comando \texttt{ALTER TABLE} è possibile (standard SQL):
\begin{enumerate}
	\item Aggiungere una colonna (\texttt{ADD [COLUMN]})
	\item Eliminare una colonna (\texttt{DROP [COLUMN]})
	\item Modificare la colonna (\texttt{MODIFY})
	\item Aggiungere l'assegnazione di valori di default (\texttt{SET DEFAULT})
	\item Eliminare l'assegnazione di valori di default (\texttt{DROP DEFAULT})
	\item Aggiungere vincoli di tabella (\texttt{ADD CONSTRAINT})
	\item Eliminare vincoli di tabella (\texttt{DROP CONSTRAINT})
	\item Altre opzioni sono possibili nei linguaggi specifici
\end{enumerate}

\subsubsection{Aggiungere una colonna}
Si può aggiungere una colonna in qualsiasi momento se non
viene specificato \texttt{NOT NULL}.\
Sintassi:
\begin{flushleft}
	\texttt{ALTER TABLE nome\_tabella}

	\texttt{ ADD [COLUMN] nome\_col tipo\_col default\_col vincolo\_col}
\end{flushleft}

\noindent In mancanza di altre specifiche, la nuova colonna viene inserita come ultima colonna della tabella.\
Altrimenti è possibile dare questa specifica:
\begin{flushleft}
	$\mathtt{ADD\ COLUMN\ \langle creaDefinizione\rangle}$

	\quad $\mathtt{[FIRST/AFTER\ \langle nomeColonna \rangle]}$
\end{flushleft}

\noindent \texttt{FIRST} permette di aggiungerla come prima colonna, mentre \texttt{AFTER} come colonna
subito dopo la colonna indicata.\

\subsubsection{Eliminare una colonna}

\begin{flushleft}
	\texttt{ALTER TABLE nome\_tabella}

	\texttt{DROP COLUMN nome\_colonna \{RESTRICT/CASCADE\}}
\end{flushleft}

\noindent In SQL standard le opzioni \texttt{RESTRICT}/\texttt{CASCADE} sono alternative ed è obbligatorio specificare l'una o l'altra.
\begin{itemize}
	\item \texttt{RESTRICT}:\ se un'altra tabella si ha un vincolo di integrità referenziale con questa colonna, l'esecuzione del comando drop fallisce.
	\item \texttt{CASCADE}:\ eliminando la colonna, vengono eliminate tutte le dipendenze logiche di altre colonne dello schema da questa.
\end{itemize}

\subsubsection{Modificare una colonna}

Se si vogliono modificare le caratteristiche di una colonna dopo averla definita, occorre eseguire l'istruzione:
\begin{flushleft}
	\texttt{ALTER TABLE nome\_tabella MODIFY}

	\texttt{nome colonna tipo\_col default\_col vincoli\_col}
\end{flushleft}

\subsubsection{Assegnare un valore di default}
Nell'SQL standard è possibile imporre un valore di default col comando specifico \texttt{SET DEFAULT}, con la seguente sintassi
\begin{flushleft}
	\texttt{ALTER TABLE nome\_tabella}

	\texttt{ALTER [COLUMN] nome\_colonna}

	\texttt{SET DEFAULT valore\_default}
\end{flushleft}

\subsubsection{Eliminare un valore di default}

In SQL standard è possibile eliminare un vincolo di default da una colonna mediante l'istruzione:
\begin{flushleft}
	\texttt{ALTER TABLE nome\_tabella}

	\texttt{ALTER [COLUMN] nome\_colonna}

	\texttt{DROP DEFAULT}
\end{flushleft}

\noindent Eseguendo questa istruzione il valore di default diventa automaticamente \texttt{NULL}.

\subsubsection{Aggiungere vincoli di tabella}

Se si vuole aggiungere un vincolo di tabella, si esegue il comando
\begin{flushleft}
	\texttt{ALTER TABLE nome\_tabella}

	\texttt{ADD CONSTRAINT nome\_vincolo vincolo\_di\_tabella  }
\end{flushleft}

\noindent\textbf{Nota bene}:\ Occorre assegnare un nome al vincolo.

\subsubsection{Eliminare vincoli di tabella}

Nello standard SQL, se si vuole eliminare un vincolo di tabella si esegue l'istruzione
\begin{flushleft}
	\texttt{ALTER TABLE nome\_tabella}

	\texttt{DROP CONSTRAINT nome\_vincolo\{RESTRICT/CASCADE\}}
\end{flushleft}

\noindent L'opzione \texttt{RESTRICT} non permette di eliminare vincoli di unicità e di chiave primaria su una colonna se esistono vincoli di chiave esterna che si riferiscono a tale colonna.\
L'opzione \texttt{CASCADE} non opera questa restrizione.\

Da notare che per eliminare un vincolo, esso deve essere definito mediante un identificatore.

\subsubsection{Drop Table}

Si può eliminare una tabella mediante l'istruzione \texttt{DROP TABLE}.\
Nello standard SQL si possono anche specificare le opzioni \texttt{RESTRICT}/\texttt{CASCADE}
\begin{itemize}
	\item \texttt{RESTRICT}:\ se la tabella è utilizzata nella definizione di altri oggetti dello schema, la sua eliminazione viene impedita.
	\item \texttt{CASCADE}:\ vengono eliminate tutte le dipendenze degli altri oggetti dello schema da questa tabella.
\end{itemize}

\section{I vincoli}

I \textbf{vincoli di integrità} consentono di limitare i valori ammissibili per una determinata colonna della tabella in base a specifici criteri.\
I vincoli di integrità intrarelazionali (ossia che non fanno riferimento ad altre relazioni) sono:
\begin{itemize}
	\item \texttt{NOT NULL}
	\item \texttt{UNIQUE} definisce chiavi
	\item \texttt{PRIMARY KEY}:\ chiave primaria (una sola, implica \texttt{NOT NULL})
	\item \texttt{CHECK}
\end{itemize}

\subsection{UNIQUE}

Può essere espresso in due forme:\ nella definizione di un attributo, se forma da solo la chiave, oppure come elemento separato.\
Il vincolo \texttt{unique} utilizzato nella definizione dell'attributo indica che non ci possono essere due valori uguali in quella colonna.\
È una chiave della relazione, ma non una chiave primaria.

\subsubsection{Vincolo unique per insiemi di attributi}

Il vincolo di unicità può anche essere riferito a coppie o insiemi di attributi.\
Ciò non significa che per gli attributi dell'insieme considerato ogni singolo valore deve apparire una sola volta, ma che non ci siano due dati (righe) per cui l'insieme dei valori corrispondenti a quegli attributi siano uguali.\

In questo caso il vincolo viene dichiarato dopo aver  dichiarato tutte le colonne mediante un vincolo di tabella, utilizzando il comando

\begin{center}
	\texttt{Unique (lista\_attributi)}
\end{center}

\subsection{Primary key}

Due forme:\ definizione di un attributo, se formato da solo la chiave oppure come elemento separato.\
Il vincolo \texttt{primary key} è simile a \texttt{unique}, ma definisce la chiave primaria della relazione, ossia un attributo che individua univocamente un dato.\

Implica sia il vincolo \texttt{unique} che il vincolo \texttt{not null} (non è ammesso che per uno degli elementi della tabella questo valore sia non definito).\
Serve ad identificare univocamente i soggetti del dominio.\
Questo vincolo permette spesso il collegamento fra due tabelle.\

\subsubsection{Chiave primaria con insiemi di attributi}

Analogamente al vincolo \texttt{unique}, anche il vincolo di chiave primaria può essere definito su un insieme di elementi.\
In tal caso la sintassi è simile a quella di \texttt{unique}:
\begin{center}
	\texttt{Primary key (lista\_attributi)}
\end{center}

\subsection{Vincoli interrelazionali}

I vincoli interrelazionali sono quei vincoli che vengono imposti quando gli \textit{attributi di due diverse tabelle devono essere messi in relazione}.\
Questo è fatto per soddisfare l'esigenza di un database di \textbf{non essere ridondante} e di avere i \textbf{dati sincronizzati}.\
Se due tabelle gestiscono gli stessi dati, è bene che di essi non ce ne siano più copie, sia allo scopo di non occupare troppa memoria, sia affinché le modifiche fatte su dati uguali utilizzati da due tabelle siano coerenti.

\textbf{References} e \textbf{Foreign Key} permettono di definire vincoli di integrità referenziale.\
Di nuovo \textbf{due sintassi}:\ per singoli attributi (come vincolo di colonna), oppure su più attributi (come vincolo di tabella).\
È possibile definire politiche di \textbf{reazione alla violazione} (ossia stabilire l'azione che il DBMS deve compiere quando si viola il vincolo).

\subsection{Check}

Un vincolo di \texttt{Check} richiede che una colonna, o una combinazione di colonne, soddisfi una condizione per ogni riga della tabella:\ deve essere una espressione booleana che è valutata usando i valori della colonna che vengono inseriti o aggiornati nella riga.\
Può essere espresso sia come \textbf{vincolo di riga} che come \textbf{vincolo di tabella}.\
Se è espresso come \textbf{vincolo di riga}, può coinvolgere solo l'attributo su cui è definito, mentre se serve eseguire un check che coinvolge due o più attributi, si deve definire come \textbf{vincolo di tabella}.

\subsection{Reazione alla violazione}

Quando si crea un vincolo \texttt{foreign key} in una tabella, in SQL standard si può specificare l'azione da intraprendere quando delle righe nella tabella riferita vengono cancellate o modificate.\
Tali reazioni alla violazione vengono dichiarate al momento della definizione dei vincoli di \texttt{foreign key} rispettivamente mediante i comandi
\begin{itemize}
	\item \texttt{On Delete}
	\item \texttt{On Update}
\end{itemize}

\subsubsection{Reazioni alla violazione On Delete}

\textbf{Impedire il delete} (\texttt{No Action}):\ blocca il delete delle righe dalla tabella riferita quando ci sono righe che dipendono da essa.\
Questa è l'azione che viene attivata per \textit{default}.

\noindent \textbf{Generare un delete a catena} (\texttt{Cascade}):\
cancella tutte le righe dipendenti dalla tabella quando la corrispondente riga è cancellata dalla tabella riferita.\

\noindent \textbf{Assegnare valore} \texttt{NULL} (\texttt{Set Null}):\ assegna \texttt{NULL} ai valori della colonna che ha il vincolo foreign key nella tabella quando la riga corrispondente viene cancellata dalla tabella riferita.

\noindent \textbf{Assegnare il valore di default} (\texttt{Set Default}):\ assegna il valore di default ai valori della colonna che ha il vincolo foreign key nella tabella quando la riga corrispondente viene cancellata dalla tabella riferita.

\subsubsection{Reazioni alla violazione On Update}

Nello standard SQL la reazione alla violazione può anche essere attivata quando i dati della tabella riferita vengono aggiornati.\
Viene attivato mediante il comando \texttt{ON UPDATE} seguito da:
\begin{itemize}
	\item \texttt{Cascade}:\ Le righe della tabella referente vengono impostati ai valori della tabella riferita.
	\item \texttt{Set Null}:\ i valori della tabella referente vengono  impostati a \texttt{NULL}.
	\item \texttt{Set Default}:\ i valori della tabella referente vengono impostati al valore di default.
	\item \texttt{No Action}:\ rifiuta gli aggiornamenti che violino l'integrità referenziale.
\end{itemize}

\section{Viste}

Le \textbf{viste logiche} o \textbf{viste} o \textbf{view} possono essere definite come delle tabelle virtuali, i cui dati sono riaggregazioni dei dati contenuti nelle tabelle ``fisiche'' presenti nel database.\
Le tabelle fisiche sono gli unici veri contenitori di dati.\
Le viste non contengono dati fisicamente diversi dai dati presenti nelle tabelle, ma forniscono una \textit{diversa visione}, \textit{dinamicamente aggiornata}, di quegli stessi dati.\
La vista appare all'utente come una normale tabella, in cui può effettuare \textbf{interrogazioni} e, limitatamente ai suoi privilegi, anche \textbf{modificare dei dati}.

\noindent Vantaggi
\begin{itemize}
	\item Le viste \textbf{semplificano la rappresentazione dei dati}.\ Oltre ad assegnare un nome alla vista, la sintassi dell'istruzione \texttt{CREATE VIEW} consente di cambiare i nomi delle colonne.\
	\item Le viste possono essere anche estremamente \textbf{convenienti per svolgere una serie di query molto complesse}.
	\item Le viste consentono di \textbf{proteggere i database}:\ le view ad accesso limitato possono essere utilizzate per controllare le informazioni alle quali accede un certo utente del database.
	\item Le viste consentono inoltre di \textbf{convertire le unità di misura e creare nuovi formati}.
\end{itemize}

\noindent Limitazioni

\begin{itemize}
	\item Non è possibile utilizzare gli operatori booleani \texttt{UNION}, \texttt{INTERSECT} ed \texttt{EXCEPT}.
	\item Gli operatori \texttt{INTERSECT} ed \texttt{EXCEPT} possono essere realizzati mediante una select semplice.\ La stessa cosa non si può dire dell'operatore \texttt{UNION}.
	\item Non è possibile utilizzare la clausola \texttt{ORDER BY}.
\end{itemize}

\subsubsection{Sintassi}

Il comando DDL che consente di definire una vista ha la seguente sintassi

\begin{flushleft}
	\texttt{CREATE VIEW NomeVista [(ListaAttributi)] \textbf{AS} SelectSQL}

	\texttt{[with [local $|$ cascaded] checkoption]}
\end{flushleft}

\noindent I nomi delle colonne indicati nella lista attributi sono i nomi assegnati alle colonne della vista, che corrispondono ordinatamente alle colonne elencate nella select.\
Se questi non sono specificati, le colonne della vista assumono gli stessi nomi di quelli della/e tabella/e a cui si riferisce.

\subsubsection{Modifica di una vista}

Sebbene il \textbf{contenuto} di una vista sia \textbf{dinamico}, \textit{la sua struttura non lo è}.\
Se una vista è definita su una subquery
\[\mathtt{Select^*\ From\ T_1}\]
e in seguito alla tabella $\mathtt{T_1}$ viene aggiunta una colonna, questa nuova definizione non si estende alla vista:\ la vista conterrà sempre le stesse colonne che aveva prima dell'inserimento della nuova colonna in $\mathtt{T_1}$.\

\subsubsection{Viste di gruppo}

Una \textbf{vista di gruppo} è una vista di cui una delle colonne è una funzione di gruppo.\
In questo caso è obbligatorio assegnare un nome alla colonna della vista corrispondente alla funzione.\

È una vista di gruppo anche una vista che è definita in base ad una vista di gruppo.

\subsubsection{Eliminazione delle vista}

Le viste si eliminano col comando \texttt{Drop View}.\
Sintassi:
\[\mathtt{Drop\ View\ nomeView\ \{Restrict/Cascade\}}\]

\noindent\texttt{Restrict}:\ la vista viene eliminata solo se non è riferita nella definizione di altri oggetti.

\noindent\texttt{Cascade}:\ oltre che essere eliminata la vista, vengono eliminate tutte le dipendenza da tale vista di altre definizioni dello schema.

\subsubsection{Viste modificabili}

Le tabelle delle viste si interrogano come le altre, ma in generale non  si possono modificare.\
Deve esistere una corrispondenza biunivoca fra le righe della vista e le righe di una tabella di base, ovvero:

\begin{enumerate}
	\item \texttt{SELECT} senza \texttt{DISTINCT} e solo di attributi
	\item \texttt{FROM} una sola tabella modificabile
	\item \texttt{WHERE} senza SottoSelect
	\item \texttt{GROUP BY} e \texttt{HAVING} non sono presenti nella definizione.
\end{enumerate}
Possono esistere anche delle restrizioni su \texttt{SELECT} su viste definite usando \texttt{GROUP BY}.

\subsubsection{Aggiornamento delle viste}

Le operazioni \texttt{INSERT/UPDATE/DELETE} sulle viste non erano permesse nelle prime edizioni di SQL.\
I nuovi DBMS permettono di farlo con certe limita\-zioni dovute alla definizione della vista stessa.

Ha senso aggiornare una vista?\
Dopotutto si potrebbe aggiornare la tabella di base direttamente\dots tuttavia è molto utile nel caso di accesso dati controllato.

L'opzione \texttt{With Check Option} messa alla fine della definizione di una vista assicura che le operazioni di inserimento e di modifica dei dati effettuate utilizzando la vista soddisfino la clausola \texttt{Where} della subquery.\

Supponiamo che una \textbf{vista V\textsubscript{1}} sia definita in termini di un'altra vista V\textsubscript{2}.\
Se si crea V\textsubscript{1} specificando la clausola \texttt{With Check Option}, il DBMS verifica che la nuova tupla \texttt{t} inserita soddisfi \textbf{sia la definizione di V\textsubscript{1} che quella di V\textsubscript{2}} (e di tutte le altre eventuali viste da cui V\textsubscript{1} dipende), \textbf{indipendentemente} dal fatto che V\textsubscript{2} sia stata a sua volta definita \texttt{With Check Option}.\
Questo comportamento di default è equivalente a definire V\textsubscript{1}
\begin{center}
	\texttt{WITH CASCADED CHECK OPTION}
\end{center}

\noindent Lo si può alterare definendo V\textsubscript{1}
\begin{center}
	\texttt{WITH LOCAL CHECK OPTION}
\end{center}

\noindent Ora il DBMS verifica solo che \texttt{t} soddisfi la specifica di V\textsubscript{1} e quelle di tutte e \textbf{sole le viste da cui V\textsubscript{1} dipende per cui è stata specificata} la clausola \texttt{With Check Option}.

\subsection{Vantaggi delle viste}

\subsubsection{Facilitazione nell'accesso ai dati}

In generale uno dei requisiti per la progettazione di un database relazionale è la \textbf{normalizzazione dei dati}.\
Sebbene la forma normalizzata del database permetta una corretta modellazione della realtà che il DB rappresenta, a volte dal punto di vista dell'utente comporta una \textit{maggiore difficoltà di comprensione} rispetto a una rappresentazione non normalizzata.\
Le viste permettono di fornire all'utente i dati in una forma \textit{più intuitiva}.

\subsubsection{Diverse visioni dei dati}

Esistono dei dati che sono presenti nelle tabelle del database, che sono \textbf{poco significativi per l'utente} e altri che \textbf{devono essere nascosti all'utente}.\
L'uso delle viste da parte dell'utente permette di \textbf{limitare il suo accesso ai dati del database}, eliminando quelli non interessanti per lui e quelli che devono essere tenuti nascosti.
L'uso delle viste può essere considerato come una \textbf{tecnica per assicurare la sicurezza dei dati}.

\subsubsection{Indipendenza logica}

Un vantaggio delle viste riguarda l'\textbf{indipendenza logica} delle applicazioni e delle operazioni eseguite dagli utenti rispetto alla struttura logica dei dati.\
Ciò significa che è possibile poter operare \textit{modifiche allo schema senza dover apportare modifiche alle applicazioni} che utilizzano il database.

\subsubsection{Utilità delle viste}

\begin{itemize}
	\item Per nascondere certe modifiche all'organizzazione logica dei dati (indipendenza logica)
	\item Per offrire visioni diverse degli stessi dati senza ricorrere a duplicazioni
	\item Per rendere più semplici, o per rendere possibili, alcune interrogazioni
\end{itemize}

\section{Procedure e Trigger}

\subsection{Trigger}

Un \textbf{trigger} definisce un'azione che il database deve attivare automaticamente quando si verifica (nel database) un determinato evento.\
Possono essere utilizzati
\begin{itemize}
	\item per \textbf{migliorare l'integrità} referenziale dichiarativa,
	\item per \textbf{imporre regole complesse} legate all'attività del database,
	\item per \textbf{effettuare revisioni} sulle modifiche dei dati.
\end{itemize}

\noindent L'esecuzione dei trigger è quindi trasparente all'utente.\
I trigger vengono eseguiti automaticamente dal database quando specifici tipi di comandi (\textbf{eventi}) di manipolazione dei dati vengono eseguiti su specifiche tabelle.\
Tali comandi comprendono i comandi DML \texttt{insert}, \texttt{update} e \texttt{delete}, ma gli ultimi DBMS prevedono anche trigger su istruzioni DDL come \texttt{Create View}.\
Anche gli aggiornamenti di specifiche colonne possono essere utilizzati come trigger di eventi.

\subsubsection{Trigger a livello di riga}

I trigger a livello di riga vengono eseguiti una volta per ciascuna riga modificata in una transazione; vengono spesso utilizzati in applicazioni di revisione dei dati e si rivelano utili per \textbf{operazioni di audit dei dati e per mantenere sincronizzati i dati distribuiti}.\
Per creare un trigger a livello di riga occorre specificare la clausola \[\mathtt{FOR\ EACH\ ROW}\] nell'istruzione \texttt{create trigger}.

\subsubsection{Trigger a livello di istruzione}

I trigger a livello di istruzione vengono eseguiti \textbf{una sola volta per ciascuna transazione}, indipendentemente dal numero di righe che vengono modificate (quindi anche se, ad esempio, in una tabella vengono inserite 100 righe, il trigger verrà eseguito solo una volta).\
Vengono pertanto utilizzati per \textbf{attività correlate ai dati}:\ vengono utilizzati di solito per \textbf{imporre misure aggiuntive di sicurezza sui tipi di transazione che possono essere eseguiti su una tabella}.\

È il tipo di trigger \textit{predefinito} nel comando \texttt{create trigger} (ossia non occorre specificare che è un trigger al livello di istruzione).

\subsubsection{Struttura}

I trigger si basano sul paradigma evento-condizione-azione (ECA).\

\noindent L'istruzione \texttt{Create Trigger} seguita dal \textbf{nome} assegnato al trigger.\

\noindent \textbf{Tipo di trigger}, \texttt{Before}/\texttt{After}.\

\noindent\textbf{Evento che scatena} il trigger \texttt{Insert}/\texttt{Delete}/\texttt{Update}.\

\noindent \texttt{For each row} se si vuole specificare trigger al livello di riga (altrimenti nulla per trigger al livello di istruzione).\

\noindent Specificare a quale \textbf{tabella} si applica.\

\noindent\textbf{Condizione} che si deve verificare perché il trigger sia eseguito.\

\noindent\textbf{Azione}, definita dal codice da eseguire se si verifica la condizione.

\subsubsection{Tipi di Trigger}

\texttt{Before} e \texttt{After}:\ i trigger possono essere eseguiti prima o dopo l'utilizzo dei comandi \texttt{insert}, \texttt{update} e \texttt{delete}; all'interno del trigger è possibile fare riferimento ai vecchi e nuovi valori coinvolti nella transizione.\
Occorre utilizzare la clausola
\begin{center}
	\[\mathtt{Before/After\ \langle tipoDiEvento\rangle}\]
\end{center}

\noindent Se si tratta di un trigger \texttt{before update}:
\begin{itemize}
	\item per valori \textbf{vecchi} intendiamo \textit{i valori che sono nella tabella} e che vogliamo modificare,
	\item per \textbf{nuovi} quelli che \textit{vogliamo inserire} al posto di quelli vecchi.
\end{itemize}

\noindent Se si tratta di un trigger \texttt{after update}

\begin{itemize}
	\item per \textbf{vecchi} intendiamo quelli \textit{che c'erano prima} dell'update,
	\item per \textbf{nuovi} quelli \textit{presenti nella tabella alla fine della modifica}.
\end{itemize}

\noindent Un \textbf{trigger} è \textbf{attivo} quando, in corrispondenza di certi eventi, \textit{modifica lo stato della base di dati}.\
Un \textbf{trigger} è \textbf{passivo} se serve a \textit{provocare il fallimento} della transazione corrente sotto certe condizioni.

\noindent\textbf{Instead Of}:\ per specificare che cosa fare invece di eseguire le azioni che hanno attivato il trigger.\
Ad esempio, è possibile utilizzare un trigger \texttt{INSTEAD OF} per reindirizzare le \texttt{INSERT} in una tabella verso una tabella differente o per aggiornare con update più tabelle che siano parte di una vista.

I trigger \texttt{instead-of} possono essere definiti su viste (relazionali od oggetto).\
I trigger instead-of devono essere a livello di riga.

\section{Controllo degli accessi}

Ogni componente dello schema (risorsa) può essere protetto (tabelle, attributi, viste, domini,\ \dots).\
Il possessore della risorsa (colui che la crea) assegna dei privilegi agli altri utenti; un utente predefinito (\verb|_system|) rappresenta l'amministratore della base di dati ed ha completo accesso alle risorse.\
Ogni privilegio è caratterizzato dalla risorsa a cui si riferisce, dall'utente che concede il privilegio, dall'utente che riceve il privilegio, dall'azione che viene permessa sulla risorsa e se il privilegio può esser trasmesso o meno ad altri utenti.\
Tipi di privilegi:

\begin{itemize}
	\item \texttt{SELECT}:\ lettura di dati.
	\item \texttt{INSERT [(Attributi)]}:\ inserire record (con valori non nulli per gli attributi).
	\item \texttt{DELETE}:\ cancellazione di record.
	\item \texttt{UPDATE [(Attributi)]}:\ modificare record (o solo gli attributi).
	\item \texttt{REFERENCES [(Attributi)]}:\ definire chiavi esterne in altre tabelle che riferiscono gli attributi.
	\item \texttt{WITH GRANT OPTION}:\ si possono trasferire i privilegi ad altri utenti.
\end{itemize}

\noindent Chi crea lo schema del DB è l'unico che può fare \texttt{CREATE}, \texttt{ALTER} e \texttt{DROP}.\
Chi crea la tabella stabilisce i modi in cui altri possono farne uso:

\begin{center}
	\texttt{GRANT Privilegi ON Oggetto TO Utenti [WITH GRANT OPTION]}
\end{center}

\noindent Per revocare il privilegio:
\begin{center}
	\texttt{revoke Privileges on Resource from Users [restrict $|$ cascade]}
\end{center}

\noindent La revoca deve essere fatta dall'utente che aveva concesso i privilegi:\ \texttt{re\-strict} (di default) specifica che il comando non deve essere eseguito qualora la revoca dei privilegi all'utente comporti qualche altra revoca (dovuta ad un precedente grant option), \texttt{cascade} invece forza l'esecuzione del comando.

Chi definisce una tabella o una view ottiene automaticamente tutti i privilegi su di esse ed è l'unico che può fare \texttt{DROP} e può autorizzare altri ad usarla con \texttt{GRANT}.\
Nel caso di viste il ``creatore'' ha i privilegi che ha sulle tabelle usate nella definizione.

\section{Indice e catalogo}

\subsubsection{Creazione di indici}
Non è un comando standard dell'SQL e quindi ci sono differenze nei vari sistemi.\

\begin{flushleft}
	$\mathtt{CREATE\ INDEX\ NomeIdx\ ON\ Tabella(Attributi)}$

	$\mathtt{CREATE\ INDEX\ NomeIdx\ ON\ Tabella }$

	\quad $\mathtt{WITH\ STRUCTURE = BTree,\ KEY = (Attributi)}$

	$\mathtt{DROP\ INDEX\ NomeIdx}$
\end{flushleft}

\subsubsection{Catalogo (dei metadati)}

Alcuni esempi di tabelle, delle quali si mostrano solo alcuni attributi, sono:
\begin{itemize}
	\item Tabella delle password:\ \texttt{PASSWORD(username,password)}.
	\item Tabella delle basi di dati:\ \texttt{SYSDB(dbname,creator,dbpath,remarks)}.
	\item Tabella delle tabelle (type = view or table):\ \texttt{SYSTABLES(name,crea\-tor,type,colcount,filename,remarks)}.
	\item Tabella degli attributi:\ \texttt{SYSCOLUMNS(name,tbname,tbcreator,colno, coltype,lenght,default,remarks)}.
	\item Tabella degli indici:\ \texttt{SYSINDEXES(name,tbname,creator,uniquerule, colcount)}.
\end{itemize}
E altre ancora sulle viste, vincoli, autorizzazioni, \dots (una decina).


\chapter{Agenti basati su conoscenza}

Abbiamo trattato:
\begin{itemize}
	\item Agenti con stato e con obiettivo in mondi osservabili con stati atomici e azioni descrivibili in maniera semplice; enfasi sul processo di ricerca.
	\item Descrizioni ``fattorizzate'' (come nei CSP) che consentono di iniziare a ``guardare dentro'' lo stato,
	      descritto come un insieme di caratteristiche rilevanti (o \textbf{\textit{feature}}).
\end{itemize}
Vogliamo adesso migliorare le \textbf{capacità di ragionamento} dei nostri agenti dotandoli di rappresentazioni di mondi più complessi e \textbf{astratti}, non descrivibili semplicemente:\ agenti \textbf{\textit{basati su conoscenza}}, dotati di una KB (\textit{Knowledge Base}) con conoscenza espressa in maniera esplicita e dichiarativa.

\begin{figure}[H]
	\centering
	\includegraphics[width=0.7\textwidth]{immagini/AgentiConoscenti.png}
	\caption*{Agenti basati su modello}
\end{figure}

\subsubsection{Agenti ``Knowledge Based''}

La maggior parte dei problemi di I.A. sono ``\textit{knowledge intensive}''.\
Il mondo è tipicamente complesso:\ ci serve una rappresentazione \textbf{\textit{parziale}} e \textbf{\textit{incompleta}} (un'astrazione) del mondo utile agli scopi dell'agente.\

Per ambienti parzialmente osservabili e complessi ci servono linguaggi di rappresentazione della conoscenza più espressivi e capacità inferenziali.\
La conoscenza può essere codificata a mano ma anche estratta dai testi o appresa dall'esperienza.

\subsubsection{Approccio dichiarativo vs procedurale}

La KB racchiude tutta la conoscenza necessaria a decidere l'azione da compiere in forma \textbf{\textit{dichiarativa}}.\
L'alternativa (\textit{approccio procedurale}) è scrivere un programma che implementa il processo decisionale, una volta per tutte.

Un agente KB è più flessibile:\ più semplice acquisire conoscenza incrementalmente e modificare il comportamento con l'esperienza.

\section{Agente basato su conoscenza}

Un agente basato su conoscenza mantiene una \textbf{\textit{base di conoscenza}} (KB):\ un insieme di enunciati espressi in un linguaggio di rappresentazione.\
Interagisce con la KB mediante una interfaccia funzionale \textit{Tell-Ask}:
\begin{itemize}
	\item \textit{Tell}:\ per aggiungere nuovi enunciati a KB
	\item \textit{Ask}:\ per interrogare la KB
	\item \textit{Retract}:\ per eliminare enunciati
\end{itemize}
Gli enunciati nella KB rappresentano le \textbf{opinioni}{\slash}\textbf{credenze dell'agente}.\
Le risposte $\alpha$ devono essere tali che $\alpha$ è una conseguenza (discende necessariamente) della KB.\

\textit{Il problema}:\ data una base di conoscenza KB, contenente una rappresentazione dei fatti che si \textbf{ritengono veri}, vorrei sapere se un certo fatto $\alpha$ è vero di conseguenza
\[\mathrm{KB} \models \alpha\quad (\mathit{conseguenza\ logica})\]

\subsubsection{Base di conoscenza vs base di dati}

\textit{Base di conoscenza}:\ una rappresentazione esplicita, parziale e compatta, in un linguaggio simbolico, che contiene fatti di tipo specifico e fatti di tipo generale, o regole.\

\noindent\textit{Base di dati}:\ solo fatti specifici, solo recupero.\

Quello che caratterizza una KB è la \textbf{\textit{capacità inferenziale}}, cioè la capacità di derivare nuovi fatti da quelli memorizzati esplicitamente.\

\subsubsection{Il trade-off fondamentale della R.C.}
Sfortunatamente più il linguaggio è \textit{espressivo}, meno \textit{efficiente} è il meccanismo inferenziale.\
Il problema ``fondamentale'' nella rappresentazione della conoscenza (R.C.) è trovare il giusto compromesso tra:\ espressività del linguaggio di rappresentazione e complessità del meccanismo inferenziale.\
Questi due obiettivi sono in contrasto e si tratta di mediare tra queste due esigenze.

\subsection{Formalismi per la R.C.}

Un formalismo per la rappresentazione della conoscenza ha tre componenti:
\begin{enumerate}
	\item Una \textbf{\textit{sintassi}}:\ un linguaggio composto da un vocabolario e regole per la formazione delle frasi (\textit{enunciati}).
	\item Una \textbf{\textit{semantica}} che stabilisce una corrispondenza tra gli enunciati e fatti del mondo; se un agente ha un enunciato $\alpha$ nella sua KB, crede che il fatto corrispondente sia vero nel mondo.
	\item Un \textbf{\textit{meccanismo inferenziale}} (codificato, o meno, tramite regole di inferenza come nella logica) che ci consente di inferire nuovi fatti.
\end{enumerate}

\subsubsection{Logica come linguaggio per la R.C.}

Qual è la complessità computazionale del problema KB $\models \alpha$ nei vari linguaggi logici?\
Quali sono gli algoritmi di decisione e le strategie di ottimizzazione?\
I linguaggi logici come calcolo proposizionale (PROP) e logica dei predicati (FOL) sono adatti per la rappresentazione della conoscenza?

\section{Agenti logici:\ calcolo proposizionale}

\subsubsection{Sintassi}

La sintassi definisce quali sono le frasi legittime (\textbf{ben formate}) del linguaggio:

\begin{table}[H]
	\centering
	\begin{tabular}{l l p{16em}}
		\textit{formula}          & $\rightarrow$ & \textit{formulaAtomica} $|$ \textit{formulaComplessa}                        \\
		\textit{formulaAtomica}   & $\rightarrow$ & \textbf{\textit{True}} $|$ \textbf{\textit{False} }$|$ \textit{simbolo}      \\
		\textit{simbolo}          & $\rightarrow$ & \textbf{\textit{P}} $|$\textbf{\textit{Q}} $|$ \textbf{\textit{R}} $|$ \dots \\
		\textit{formulaComplessa} & $\rightarrow$ & $\lnot formula$                                                              \\
		                          &               & $|$ $(formula\ \land\ formula)$                                              \\
		                          &               & $|$ $(formula\ \lor\ formula)$                                               \\
		                          &               & $|$ $(formula\ \Rightarrow\ formula)$                                        \\
		                          &               & $|$ $(formula\ \Leftrightarrow\ formula)$                                    \\
	\end{tabular}
\end{table}

\noindent Esempio:\ $((\mathrm{A} \land \mathrm{B}) \Rightarrow \mathrm{C})$

\noindent Possiamo omettere le parentesi assumendo questa \textbf{precedenza} tra gli operatori:
\[
	\lnot\ >\ \land\ >\ \lor\ >\ \Rightarrow\ >\ \Leftrightarrow
\]

\subsubsection{Semantica e mondi possibili (modelli)}

La semantica ha a che fare col significato delle frasi:\ definisce se un enunciato è vero o falso rispetto ad una \textbf{\textit{interpretazione}} (mondo possibile).\
Un'interpretazione definisce un valore di verità per tutti i simboli proposizionali.\

\vspace{12pt}
Per esempio:\ $\{\mathrm{P}_{1,1}\ \mathit{vero},\ \mathrm{P}_{1,2}\ \mathit{falso},\ \mathrm{W}_{2,3}\ \mathit{vero}\}$

\vspace{12pt}

\noindent Il valore di una formula complessa è fissato di conseguenza:
\[\mathrm{P}_{1,1} \Rightarrow \mathrm{W}_{2,3} \lor \mathrm{P}_{1,2}\ \grave{e}\ \mathrm{vera\ in\ questa\ interpretazione}.\]
Un \textbf{\textit{modello}} è un'interpretazione che \textit{rende vera} una formula o un insieme di formule.\

\subsubsection{Semantica composizionale}
Il significato di una frase è determinato dal significato dei suoi componenti, a partire dalle frasi atomiche (i simboli proposizionali).\
\begin{itemize}
	\item \textit{True} sempre vero; \textit{False} sempre falso
	\item P $\land$ Q, vero se P e Q sono veri
	\item P $\lor$ Q, vero se P oppure Q, o entrambi, sono veri
	\item $\lnot$P  vero se P è falso
	\item P $ \Rightarrow$ Q, vero se P è falso oppure Q è vero
	\item P $ \Leftrightarrow$ Q,  vero se entrambi veri o entrambi falsi
\end{itemize}

\subsubsection{Conseguenza logica}

Una formula $\alpha$ è \textbf{\textit{conseguenza logica}} di un insieme di formule KB se e solo se in ogni modello di KB, anche $\alpha$ è vera (KB $\models \alpha$).\
Indicando con $M(\alpha)$ l'insieme delle interpretazioni che rendono $\alpha$ vera (i \textbf{modelli} di $\alpha$) e con \textit{M}(KB) i modelli dell'insieme di formule in KB\dots
\[
	\mathrm{KB} \models \alpha\  sse\  M(\mathrm{KB}) \subseteq M(\alpha)
\]

\subsubsection{Equivalenza logica:\ leggi}

\textbf{Equivalenza logica}:\ A $\equiv$ B se e solo se A $\models$ B e B $\models$ A.

\begin{table}[H]
	\centering
	\begin{tabular}{r c l}
		$(\alpha\ \land\ \beta)$                  & $\equiv$ & $(\beta\ \land\ \alpha)$ \quad commutatività di $\land$                                             \\
		$(\alpha\ \lor\ \beta)$                   & $\equiv$ & $(\beta\ \lor\ \alpha)$ \quad commutatività di $\lor$                                               \\
		$((\alpha\ \land\ \beta)\ \land\ \gamma)$ & $\equiv$ & $(\alpha\ \land\ (\beta\ \land\ \gamma))$ \quad associatività di $\land$                            \\
		$((\alpha\ \lor\ \beta)\ \lor\ \gamma)$   & $\equiv$ & $(\alpha\ \lor\ (\beta\ \lor\ \gamma))$ \quad associatività di $\lor$                               \\
		$\lnot(\lnot\alpha)$                      & $\equiv$ & $\alpha$ \quad doppia negazione                                                                     \\
		$(\alpha\ \Rightarrow\ \beta)$            & $\equiv$ & $(\lnot \beta\ \Rightarrow\ \lnot \alpha)$ \quad contrapposizione                                   \\
		$(\alpha\ \Rightarrow\ \beta)$            & $\equiv$ & $(\lnot \alpha\ \lor\ \beta)$ \quad eliminazione dell'implicazione                                  \\
		$(\alpha\ \Leftrightarrow\ \beta)$        & $\equiv$ & $(\alpha\ \Rightarrow\ \beta)\ \land\ (\beta\ \Rightarrow\ \alpha)$ \quad eliminazione del sse      \\
		$\lnot(\alpha\ \land\ \beta)$             & $\equiv$ & $(\lnot \alpha\ \lor\ \lnot \beta)$ \quad De Morgan                                                 \\
		$\lnot(\alpha\ \lor\ \beta)$              & $\equiv$ & $(\lnot \alpha\ \land\ \lnot \beta)$ \quad De Morgan                                                \\
		$(\alpha\ \land\ (\beta\ \lor\ \gamma))$  & $\equiv$ & $((\alpha\ \land\ \beta\ )\ \lor\ (\alpha \land \gamma))$ \quad distributività di $\land$ su $\lor$ \\
		$(\alpha\ \lor\ (\beta\ \land\ \gamma))$  & $\equiv$ & $((\alpha\ \lor\ \beta\ )\ \land\ (\alpha \lor \gamma))$ \quad distributività di $\lor$ su $\land$  \\
	\end{tabular}
\end{table}

\subsubsection{Validità, soddisfacibilità}

A è \textbf{valida} \textit{sse} è vera in tutte le interpretazioni (anche detta tautologia).

\noindent A è \textbf{soddisfacibile} \textit{sse} esiste un'interpretazione in cui A è vera.\

Ne discende che
\begin{center}
	A è valida sse $\lnot$A è insoddisfacibile
\end{center}

\subsubsection{Inferenza per calcolo proposizionale}

\begin{itemize}
	\item \textit{Model checking}:\ una forma di inferenza che fa riferimento alla definizione di conseguenza logica (si enumerano i possibili modelli), per esempio usando la tecnica delle tabelle di verità.\
	\item Algoritmi per la \textit{soddisfacibilità} (SAT):\ KB $\models$ A \textit{sse} (KB $\land\ \lnot$A) è insoddisfacibile.\ Un problema può essere ricondotto all'altro.
\end{itemize}

\subsection{L'algoritmo TT-entails}
\[\mathrm{KB} \models\alpha?\]
Enumera tutte le possibili interpretazioni di KB (\textit{k simboli}, $2^k$ possibili interpretazioni).\
Per ciascuna interpretazione
\begin{itemize}
	\item Se non soddisfa KB, OK.
	\item Se soddisfa KB, si controlla che soddisfi anche $\alpha$.
\end{itemize}
Se si trova anche solo un'interpretazione che soddisfa KB e non $\alpha$ la risposta sarà NO.

\begin{figure}[H]
	\centering
	\includegraphics[width=0.8\textwidth]{immagini/TTEntails.png}
\end{figure}

\subsection{Algoritmi per la soddisfacibilità (SAT)}

Usano KB in \textbf{forma a clausole} (insiemi di letterali)
\[\{\mathrm{A},\ \mathrm{B}\}\ \{\lnot \mathrm{B},\ \mathrm{C},\ \mathrm{D}\}\ \{\lnot \mathrm{A},\ \mathrm{F}\}\]
La forma a clausole è la forma normale congiuntiva (CNF):\ una congiunzione di disgiunzioni di letterali
\[(\mathrm{A} \lor \mathrm{B})\ \land\ (\lnot \mathrm{B} \lor \mathrm{C} \lor \mathrm{D}) \land (\lnot \mathrm{A}  \lor \mathrm{F})\]
Non è restrittiva:\ è sempre possibile ottenerla con trasformazioni che preservano l'equivalenza logica.\

\subsubsection{Trasformazione in forma a clausole}
I passi sono:
\begin{enumerate}
	\item Eliminazione del $\Leftrightarrow$: (A $\Leftrightarrow$ B) $\equiv$ (A $\Rightarrow$ B) $\land$ (B $\Rightarrow$ A)
	\item Eliminazione dell' $\Rightarrow$: (A $\Rightarrow$ B) $\equiv$ ($\lnot$A $\lor$ B)
	\item Negazioni all'interno:
	      \begin{table}[H]
		      \centering
		      \begin{tabular}{l l}
			      $\lnot$(A $\lor$ B) $\equiv$ ($\lnot$A $\land$ $\lnot$B) & (De Morgan) \\
			      $\lnot$(A $\land$ B) $\equiv$ ($\lnot$A $\lor$ $\lnot$B) &             \\
		      \end{tabular}
	      \end{table}
	\item Distribuzione di $\lor$ su $\land$:\ 	(A $\lor$ (B $\land$ C)) $\equiv$ (A $\lor$ B) $\land$ (A $\lor$ C)
\end{enumerate}

\subsection{L'algoritmo DPLL per la soddisfacibilità}

DPLL:\ Davis, Putman, e poi Lovemann, Loveland

Parte da una KB in \textbf{forma a clausole}.\
È un'enumerazione \textit{in profondità} di tutte le possibili interpretazioni alla ricerca di un modello.\
Tre miglioramenti rispetto a TTEntails:
\begin{enumerate}
	\item terminazione anticipata,
	\item euristica dei simboli (o letterali) puri,
	\item euristica delle clausole unitarie.
\end{enumerate}

\subsubsection{Terminazione anticipata}

Si può decidere sulla verità di una clausola anche con interpretazioni parziali:\ basta che un \textit{letterale} sia vero.\ Per esempio se A è vero, lo sono anche \{A, B\} e \{A, C\} indipendentemente dai valori di B e C.\

Se anche una sola clausola è falsa l'interpretazione non può essere un modello dell'insieme di clausole.\

\subsubsection{Simboli puri}

\textit{Simbolo puro}:\ un simbolo che appare con lo stesso segno in tutte le clausole, per esempio
\[ \{\mathrm{A}, \lnot \mathrm{B}\}\ \{\lnot \mathrm{B}, \lnot \mathrm{C}\}\ \{\mathrm{C}, \mathrm{A}\} \quad \mathrm{A\ puro,\ B\ anche}\]
Nel determinare se un simbolo è puro se ne possono trascurare le occorrenze in clausole già rese vere.\

I simboli puri possono essere assegnati a \textit{True} se il letterale è positivo, \textit{False} se negativo.\
Non si eliminano modelli utili:\ se le clausole hanno un modello continuano ad averlo dopo questo assegnamento.\ L'assegnamento è obbligato.

\subsubsection{Clausole unitarie}

\textit{Clausola unitaria}:\ una clausola con un solo letterale \textbf{\textit{non assegnato}}.\ Per esempio \{B\} è unitaria ma anche \{B, $\lnot$C\} è unitaria quando C = \textit{True}.

Conviene assegnare prima valori al letterale in clausole unitarie.\ L'assegnamento è univoco (\textit{True} se positivo, \textit{False} se negativo).
\begin{figure}[H]
	\centering
	\includegraphics[width=0.9\textwidth]{immagini/DPLL.png}
\end{figure}

\subsubsection{Miglioramenti di DPLL}

DPLL è completo e termina sempre.\
Alcuni miglioramenti:
\begin{itemize}
	\item Analisi di componenti (sotto-problemi indipendenti):\ se le variabili possono essere suddivise in sotto-insiemi disgiunti (senza simboli in comune).
	\item Ordinamento di variabili e valori:\ scegliere la variabile che compare in più clausole.
	\item Backtracking intelligente e altre ottimizzazioni\dots
\end{itemize}

\subsection{Metodi locali per SAT}

Gli stati sono assegnamenti completi, l'obiettivo è un assegnamento che soddisfi tutte le clausole (un modello).\
Si parte da un assegnamento casuale e ad ogni passo si cambia il valore di un simbolo proposizionale (\textbf{\textit{flip}}).\
Gli stati sono valutati contando il numero di clausole \textbf{non soddisfatte} (meno sono meglio è) [o soddisfatte].\

Ci sono molti minimi locali per sfuggire ai quali serve introdurre perturbazioni casuali
\begin{itemize}
	\item Hill climbing con riavvio casuale
	\item Simulated Annealing
\end{itemize}
Molta sperimentazione per trovare il miglior compromesso tra il grado di ``avidità'' e casualità.\
WALK-SAT è uno degli algoritmi più semplici ed efficaci.

\subsubsection{WalkSAT}

WalkSAT ad ogni passo sceglie a caso una clausola non ancora soddisfatta, sceglie un simbolo da modificare (\textit{flip}) con probabilità \textit{p} (di solito 0,5) tra una delle due:
\begin{itemize}
	\item sceglie un simbolo a caso (\textbf{passo casuale}),
	\item sceglie quello che rende più clausole soddisfatte (\textbf{passo di ottimizzazione}).
\end{itemize}
Il passo casuale corrisponde a un \textbf{random walk} locale, quello di ottimizzazione a un tentativo di andare in salita.

Si arrende dopo un certo numero di \textit{flip} predefinito.

\begin{figure}[H]
	\centering
	\includegraphics[width=0.9\textwidth]{immagini/WalkSAT.jpg}
\end{figure}

\subsubsection{Analisi di WalkSAT}

Se \textit{max-flips} $= \infty$ e l'insieme di clausole è soddisfacibile, prima o poi termina.\
Va bene per cercare un modello, sapendo che c'è, ma se è insoddisfacibile non termina quindi non può essere usato per verificare l'insoddisfacibilità.\

Il problema è decidibile ma l'algoritmo non è completo.

\subsubsection{Problemi SAT difficili}

Se un problema ha molte soluzioni (problema sotto-vincolato) è più probabile che WalkSAT ne trovi una in tempi brevi.\
Per esempio se ho 16 soluzioni su 32; un assegnamento ha il 50\% di probabilità di essere giusto:\ 2 passi in media!

\noindent Esempio:\ Istanza di 3SAT

\begin{center}
	($\lnot$D $\lor$ $\lnot$B $\lor$ C) $\land$ (B $\lor$ $\lnot$A $\lor$ $\lnot$C) $\land$ ($\lnot$C $\lor$ $\lnot$B $\lor$ E) $\land$ (E $\lor$ $\lnot$D $\lor$ B) $\land$ (B $\lor$ E $\lor$ $\lnot$C)
\end{center}

\noindent Quello che conta è il rapporto $m/n$, dove \textit{m} è il numero di clausole (vincoli) e \textit{n} il numero di simboli.\
In questo caso, $5/5=1$.\
Più grande il rapporto, più vincolato è il problema.\

Le regine sono facili perché il problema è sotto-vincolato.

\subsection{Inferenza come deduzione}

Un altro modo per decidere se KB $\models$ A è usare un \textbf{sistema di deduzione}; si scrive KB $\vdash$ A  (A è deducibile da KB).\
La deduzione avviene specificando delle \textbf{regole di inferenza}.

In un sistema di inferenza le regole
\begin{itemize}
	\item dovrebbero derivare \textit{solo} formule che sono conseguenza logica,
	\item dovrebbero derivare \textit{tutte} le formule che sono conseguenza logica.
\end{itemize}

\subsubsection{Correttezza e completezza}
\begin{center}
	\textbf{Correttezza}:\ Se KB $\vdash$ A allora KB $\models$ A
\end{center}
Tutto ciò che è derivabile è conseguenza logica.\ Le regole preservano la verità.
\begin{center}
	\textbf{Completezza}:\ Se KB $\models$ A allora KB $\vdash$ A
\end{center}
Tutto ciò che è conseguenza logica è ottenibile tramite il sistema deduttivo.

\subsubsection{Dimostrazione come ricerca}

\textit{Problema}:\ come decidere ad ogni passo qual è la regola di inferenza da applicare?\ \dots e a quali premesse?\ Come evitare l'esplosione combinatoria?

Si tratta di un problema di esplorazione di uno spazio di stati.\
Una \textit{procedura di dimostrazione} definisce:
\begin{itemize}
	\item la direzione della ricerca,
	\item la strategia di ricerca.
\end{itemize}

\subsubsection{Direzione della ricerca}
Nella dimostrazione di teoremi conviene procedere all'indietro.\
Con una applicazione \textit{in avanti} delle regole di inferenza non controllata:\ da A, B posso derivare A $\land$ B, A $\land$ (A $\land$ B), \dots, A $\land$ (A $\land$ (A $\land$ B)).

Meglio \textit{all'indietro}:
\begin{itemize}
	\item se si vuole dimostrare A $\land$ B, si cerchi di dimostrare A e poi B
	\item se si vuole dimostrare A $\Rightarrow$ B, si assuma A e si cerchi di dimostrare B
	      %\item \dots
\end{itemize}

\subsubsection{Strategia di ricerca}

\textit{Completezza}
\begin{itemize}
	\item Le regole della deduzione naturale sono un insieme di regole di inferenza completo (2 per ogni connettivo)
	\item Se l'algoritmo di ricerca è completo siamo a posto
\end{itemize}
\textit{Efficienza} $\rightarrow$ la complessità è alta:\ è un problema decidibile ma NP-completo.

\subsubsection{Regola di risoluzione per PROP}

Meno regole ci sono e meglio è, senza rinunciare alla completezza.\
Un'unica regola:\ la regola di risoluzione (presuppone forma a clausole).

\[
	\frac{\{P,Q\}\quad \{\lnot,R\}}{\{Q,R\}} \qquad \frac{P\lor Q\quad \lnot P \lor R}{Q \lor R}
\]

\noindent È corretta?\ Basta pensare ai modelli.\
Il motivo per cui viene preferita la notazione insiemistica è che gli eventuali duplicati si eliminano.

\subsubsection{La regola di risoluzione generale}

\[
	\frac{\{l_1,l_2,\dots, l_i, \dots, l_k\} \quad \{m_1, m_2, \dots, m_j, \dots, m_n\}}{\{l_1,l_2,\dots, l_{i-1}, l_{i+1}, \dots, l_k, m_1, m_2, \dots, m_{j-1}, m_{j+1}, \dots, m_n\}}
\]
Gli \textit{l} e \textit{m} sono \textbf{letterali}, simboli di proposizione positivi o negativi; \textit{\textbf{l}\textsubscript{i}} e \textit{\textbf{m}\textsubscript{j}} sono uguali e di segno opposto.\

\vspace{12pt}

\noindent Caso particolare:\ clausola vuota $\rightarrow$ \textit{contraddizione}.

\[
	\frac{\{P\}\quad \{\lnot P\}}{\{\ \}}
\]
\noindent La regola è sufficiente?\ È sicuro che applicando la regola in tutti i modi possibili si riesca a dedurre A quando è conseguenza logica?

\textbf{Completezza}:\ se KB $\models \alpha$ allora KB $\vdash$\textsubscript{res} $\alpha$?\ Non sempre, per esempio KB $\models$ \{A, $\lnot$A\}, ma non è vero che KB $\vdash$\textsubscript{res} \{A, $\lnot$A\}.\

Nella versione proposizionale è di aiuto il teorema di risoluzione [ground]:
\begin{center}
	KB insoddisfacibile \textit{sse} KB $\vdash$\textsubscript{res} \{ \}
\end{center}
in qualche modo si tratta di una garanzia di \textit{completezza}.\
Il teorema di refutazione offre un modo alternativo:\

\begin{center}
	KB $\models \alpha$ \textit{sse} (KB $\cup\ \{\lnot \alpha \}$) è insoddisfacibile
\end{center}

\noindent Nell'esempio:\ KB $\cup$ FC ($\lnot$(A $\lor$ $\lnot$A)) è insoddisfacibile?\ Sì, perché\dots KB $\cup$ \{A\} $\cup$ \{$\lnot$A\} $\vdash$\textsubscript{res} \{ \} in un passo.\ Quindi KB $\models$  \{A, $\lnot$A\}.

\subsubsection{Conclusioni}

\begin{itemize}
	\item Abbiamo visto come gli agenti KB che usano PROP come linguaggio di rappresentazione possono decidere se KB $\models \alpha$.
	\item Il problema è decidibile, ma intrattabile (NP) nel caso peggiore.
	\item Esistono algoritmi efficienti e completi che consentono di affrontare problemi di grosse dimensioni.
	\item I metodi locali sono particolarmente efficienti ma non completi.
\end{itemize}


\section{Funzioni ricorsive generali}

Arricchiamo di seguito gli schemi per la definizione delle funzioni ricorsive primitive con un nuovo schema, attraverso il quale è possibile esprimere anche funzioni parziali.\
In esso, si fa uso dell'operatore $\mu$, detto di \textit{minimizzazione}, il quale applicato a un insieme di numeri naturali ne restituisce il minimo (se c'è!\ ovvero se l'insieme in questione non è vuoto).

\begin{definition}[Funzioni $\mu$-ricorsive]

    La classe delle \textit{funzioni $\mu$-ricorsive} (o \textit{ricorsive generali}) è la minima classe $\mathcal{R}$ tale che soddisfa le condizioni
    \begin{itemize}
        \item[I-V] per le ricorsive primitive e
        \item[VI] (\textit{Minimizzazione}).\ Se $\varphi(x_1,...,x_n,y)\in \mathcal{R}$ in $n+1$ variabili, allora la funzione $\psi$ in $n$ variabili è in $\mathcal{R}$ se è definita da
              \subitem $\psi(x_1,..,x_n)=\mu y[\varphi(x_1,\dots,x_n,y)=0$ e
                              \subsubitem  $\forall z \leq y.\ \varphi(x_1,\dots,x_n,z)\downarrow]$ \hfill ($*$)
    \end{itemize}

    \noindent A volte le funzioni $\mu$-ricorsive \textit{totali} le chiameremo semplicemente \textit{ricorsive}, soprattutto per ragioni storiche.\
    Nota bene:\ \textit{non} sono \textit{solo} le funzioni ricorsive primitive!\ (in cui, per esempio manca la funzione di Ackermann che è ricorsiva ma non ricorsiva primitiva.)
\end{definition}

\noindent Una funzione $\mu$-ricorsiva è intuitivamente calcolabile?\
\textbf{Sì}:\ l'algoritmo ``intuitivo'' che la calcola è composto da un ciclo in cui si incrementa la variabile $y$ (inizialmente posta a 0), si calcola la $\varphi$ e si ripetono questi passi finché il risultato non è 0.\
I primi passi dell'esecuzione di questo algoritmo potrebbero essere dunque:

\begin{enumerate}
    \item calcola $\phi(x_1, \dots, x_n, 0)$; se il risultato è 0 allora $\psi(x_1, \dots, x_n) = 0$;
    \item altrimenti calcola $\varphi(x_1,\dots, x_n,1)$; se il risultato è 0 allora\\$\psi(x_1,\dots,x_n) = 1$;
    \item \dots\\ \vdots
\end{enumerate}

\noindent Intuitivamente, potrei non finire mai o perché per ogni valore di $y$ esiste
un $m_y$ tale che $\varphi(x_1,\dots, x_n, y) = m_y \neq 0$, o perché per i primi $k$ numeri naturali $\varphi(x_1,\dots, x_n, z) = n_z \neq 0$ e $\varphi(x_1, \dots, x_n, k) \uparrow$.\
Infatti nel primo caso continuiamo a calcolare la $\varphi(x_1,\dots, x_n, y)$ per valori crescenti di $y$ senza terminare mai, e nel secondo caso non ci arrestiamo mai nel calcolo di $\varphi(x_1,\dots, x_n, k)$:\ da qui la parzialità di $\psi$.\
Se dovessi scrivere un programma, userei un comando di tipo \texttt{while}.

Vediamo adesso un semplicissima definizione $\mu$-ricorsiva, tramite la quale rappresentiamo l'archetipo delle funzioni parziali:\ la funzione ovunque indefinita, che è calcolabilissima!

\begin{example}
    La seguente è una delle possibili derivazioni per la funzione ovunque indefinita $\psi_{\uparrow}(x)$:
    \begin{itemize}
        \item[] $\varphi = \lambda x,y.\ 3$
        \item[] $\psi_\uparrow = \lambda x.\ (\mu y.\ \varphi(x,y)= 0)$
    \end{itemize}
    Per verificare che quanto scritto sopra è davvero una derivazione basta controllare che la funzione (ricorsiva primitiva) $\varphi$ è definita per tutti i suoi argomenti, ovvero che il calcolo di $\varphi(x, y)$ termina per ogni $x$; il che è banale.\
    In questo caso è altrettanto facile vedere che per nessun $x$ esiste un $y$ per cui $\varphi(x, y) = 0$ e che quindi la funzione $\psi(x)$ è indefinita per ogni valore di $x$; attenzione però:\ nella maggior parte dei casi questo controllo è molto molto difficile, in un senso che sarà precisato formalmente tra poco.
\end{example}

\noindent Ricapitolando:\ si comincia con le ricorsive primitive, si applica l'operatore di minimizzazione $\mu$ e si ottiene una funzione $\mu$-ricorsiva, che può essere ora usata nella prossima definizione.\
Tutto ciò a patto che la condizione (*) valga, altrimenti si può uscire dalla classe, cioè, se $\varphi$ non termina per qualche valore $z$ minore del minimo $y$ su cui $\varphi$ vale 0, allora la funzione $\psi$ potrebbe non essere $\mu$-ricorsiva:\ terminazione e non terminazione sono \textit{importantissime}!

Si noti anche che
\[f(x) =\left\{\begin{array}{l l}
        \mu y[y<g(x), h(x,y) = 0] & \mathrm{se\ esiste\ tale}\ y \\
        0                         & \mathrm{altrimenti}          \\
    \end{array}\right.\]
è ricorsiva primitiva se $g$ e $h$ lo sono.\
La ragione è che $g$ impone un limite ai tentativi di ricercare il minimo $y$, e quindi o lo troviamo in meno di $g(x)$ applicazioni di $h$ o diamo come risultato 0 dopo al più $g(x)$ applicazioni di $h$, che è un numero determinabile in tempo \textit{finito} perché sia $g$ che $h$ sono totali in quanto ricorsive primitive.\
In altre parole anche la $f$ sarebbe totale, e quindi avremmo definito un formalismo che definisce solo funzioni totali e quindi inadatto a rappresentare \textit{tutte} le funzioni calcolabili.

Diamo ora alcune definizioni ausiliarie che ci saranno utili in seguito.

\begin{definition}
    Una relazione $I \subseteq \mathbb{N}^n$, $n \geq 1$ è \textit{ricorsiva} (come sinonimo di totale) (o rispettivamente è \textit{ricorsiva primitiva}, ha la proprietà $P$) se la sua funzione caratteristica $\chi_I$ è ricorsiva totale (è ricorsiva primitiva, ha la proprietà $P$).\
    Un caso particolare e interessante si ha con gli \textit{insiemi ricorsivi} $I \subseteq \mathbb{N}$, cioè quando $n = 1$.
\end{definition}

\noindent In analogia a quanto fatto con le funzioni T-calcolabili, diciamo che una funzione è $\mu$-\textit{calcolabile} se la sua definizione è $\mu$-ricorsiva.\

Adesso abbiamo le funzioni T-calcolabili, quelle \textit{\footnotesize WHILE}-calcolabili e quelle $\mu$-calcolabili e il bello è che formano \textit{esattamente la stessa classe di funzioni calcolabili}, ciò che è stato accuratamente dimostrato.\
Abbiamo già annunciato che molti altri formalismi sono stati proposti e che tutti questi (quando siano sufficientemente potenti in un senso che renderemo preciso tra poco) definiscono la \textit{stessa classe di funzioni}; in altre parole sono \textit{Turing equivalenti}.\
Pertanto possiamo, o meglio vogliamo stipulare come vera la

\begin{table}[H]
    \centering
    \begin{tabular}{|c|}
        \hline
        \textbf{Tesi di Church-Turing}:\ Le funzioni (\textit{intuitivamente}) calcolabili sono \\
        tutte e sole le funzioni (parziali) T-calcolabili.                                      \\\hline
    \end{tabular}
\end{table}

\noindent In realtà questa è un'ipotesi, ma è talmente forte che la prendiamo come tesi.\
In termini informatici, questo significa che non importa quale linguaggio di programmazione usiamo, né su quale macchina facciamo girare i nostri programmi, purché si abbia a disposizione memoria e tempo illimitati:\ ciò che possiamo calcolare \textit{non} cambia --- può forse cambiare \textit{come} lo si calcola.

Chiaramente è dimostrabile solo l'equivalenza tra i formalismi esistenti, ed è certamente molto difficile immaginare una dimostrazione di equivalenza tra tutti i \textit{possibili} formalismi, inclusi quelli ancora da inventare.

La tesi di Church-Turing postula che la nozione di calcolabilità ``intuitiva'' è \textit{robusta}.\
Inoltre, la tesi cade se si rilascia anche una sola delle ipotesi fatte sulla natura degli algoritmi.

Bene, di qui in avanti parleremo solo di \textit{funzioni calcolabili}, senza qualificare ulteriormente il formalismo usato per definirle.\
Quante sono?\ E ce ne sono di non calcolabili?\ Se sì, ne vedremo una interessante?

\section{Alcuni risultati classici}

Introdurremo brevemente alcuni risultati basilari della teoria della calcolabilità che ne illustrano l'essenza, caratterizzando la classe delle funzioni, ovvero dei problemi calcolabili, mediante alcuni teoremi di ``chiusura''.\
Privilegeremo una presentazione orientata ai fondamenti dell'informatica, a volte purtroppo senza la profondità e l'accuratezza che sarebbero necessari nel presentare una teoria in cui precisione e attenzione ai dettagli giocano un ruolo essenziale.\
Prima di enunciare questi risultati, insistiamo a ricordare che, grazie alla tesi di Church, possiamo chiamare \textit{calcolabili} indifferentemente le funzioni esprimibili nel formalismo delle macchine di Turing o le funzioni $\mu$-ricorsive o i programmi \textit{\footnotesize WHILE} o ciò che volete voi, purché la loro definizione rispetti le cinque condizioni intuitive poste agli algoritmi che sono state espresse nel primo capitolo:\ le nostre ipotesi di lavoro.

Cominciamo con un semplice risultato sulla cardinalità\footnote{Dato un insieme $A$, indicheremo con $\#(A)$ la sua cardinalità, ovvero il numero dei suoi elementi.} dell'insieme delle funzioni calcolabili, da cui segue che vi sono funzioni \textit{non calcolabili}.

\begin{theorem}
    \label{n_fcalcolabili}
    \hfill
    \begin{itemize}
        \item[i)] Le funzioni calcolabili sono $\#(\mathbb{N})$; inoltre anche le funzioni calcolabili totali sono $\#(\mathbb{N})$.
        \item[ii)] Esistono funzioni non calcolabili.
    \end{itemize}
\end{theorem}

\begin{proof}
    \hfill
    \begin{itemize}
        \item[i)] Costruisci $\#(\mathbb{N})$ MdT $M_i$ che svuotano il nastro dall'input, ci scrivono la stringa $|^i$ e si arrestano (sono tutte le funzioni costanti).\ Che non siano più di $\#(\mathbb{N})$ segue dal fatto che le MdT si possono enumerare, come fatto intuire a pagina 26.\
        \item[ii)] Con una costruzione analoga a quella di Cantor (la classe dei sottoinsiemi di $\mathbb{N}$ non è numerabile) si vede che $\{f : \mathbb{N} \rightarrow \mathbb{N}\}$ ha cardinalità $2^{\#(\mathbb{N})}$.
    \end{itemize}
\end{proof}

\noindent Abbiamo già visto nel capitolo 1.6 che possiamo associare un indice alle macchine di Turing codificandole (veramente avevamo una funzione solamente iniettiva, ma sappiamo che Gödel ne ha definito una anche surgettiva); analogamente nel capitolo successivo abbiamo accennato a come enumerare le funzioni ricorsive primitive e non è difficile immaginare una sua estensione alle funzioni $\mu$-ricorsive.\
Oltre alla superficialità con cui sono stati presentati, i due modi hanno in comune il fatto che si basano \textit{solamente} sui simboli usati nel definire gli algoritmi, il che è bene.\

Infatti, sotto ipotesi molto ragionevoli, per i nostri scopi non c'è sostanziale differenza tra una enumerazione e un'altra, purché sia \textit{effettiva}.\
Quindi siamo liberi di scegliere quella che più ci aggrada.\
Basti qui dire che una buona enumerazione deve essere una funzione biunivoca che dipende \textit{solo} dalla sintassi con cui scriviamo gli algoritmi e \textit{non} dal significato che attribuiamo loro.\
Di nuovo, la già ricordata enumerazione del capitolo 1.6 andrebbe bene se fosse surgettiva; invece una enumerazione che pretendesse, ad esempio, di elencare prima tutte le funzioni costanti e poi le altre non sarebbe effettiva.\
L'osservazione appena fatta ci consente anche di fissare una volta per tutte un elenco di MdT (programmi \textit{\footnotesize WHILE}, funzioni $\mu$-ricorsive, \dots) e di indicare con $M_i$ la MdT (il programma, la funzione $\mu$-ricorsiva, \dots) che vi appare in posizione $i$-ma, o meglio l'algoritmo $i$-mo.\
Ancor meglio, si può usare la seguente notazione, che finalmente evidenzia la differenza tra \textit{funzione} e \textit{algoritmo} che la calcola.

\vspace{12pt}

\noindent\textbf{NOTAZIONE}\quad Data un'enumerazione effettiva, indicheremo con $\varphi$ la funzione (parziale) che la macchina, o meglio l'algoritmo, $M_i$ calcola e chiameremo $i$ \textit{indice} (non della funzione, bensì della macchina!\ quindi può darsi benissimo che per $i \neq j$ sia $\varphi_i = \varphi_j$, mentre sicuramente $M_i \neq M_j$).

\vspace{12pt}

\noindent Entriamo adesso nel vivo della presentazione dei teoremi più importanti di questa prima parte del corso, ribadendo ancora una volta il seguente fatto:\ i risultati che riportiamo nel seguito sono tutti \textit{invarianti} rispetto all'enumerazione scelta.\

\medskip

\noindent Il primo teorema, che spesso è chiamato \textit{padding lemma}, ci dice che ci sono infinite (numerabili) MdT, ovvero infiniti numerabili algoritmi che calcolano la stessa funzione, e che \textit{alcuni} di essi si possono costruire ``facilmente'' da un algoritmo dato (ossia esprimibile con una funzione primitiva ricorsiva o un programma \textit{\footnotesize FOR}).

\begin{theorem}[Padding Lemma]
    \label{padding lemma}
    Ogni funzione calcolabile $\varphi_x$ ha $\#(\mathbb{N})$ indici.\
    Inoltre $\forall x$ si può costruire, mediante una funzione ricorsiva primitiva, un insieme infinito $A_x$ di indici tale che
    \[\forall y \in  A_x.\ \varphi_y = \varphi_x\]
    cioè $\varphi_y(n) = m$ sse $\varphi_x(n) = m$ (e ovviamente $\varphi_y(n) \uparrow$ sse $\varphi_x(n) \uparrow$).
\end{theorem}

\begin{proof}
    Per ogni macchina $M_x$, se $Q = \{q_0,\dots,q_k\}$, ottieni la prima macchina $M_{x_1}$ con $x_1 \in A_x$ aggiungendo lo stato $q_{k+1} \notin Q$ e la quintupla $(q_{k+1}, \#, q_{k+1}, \#, -)$; ottieni la seconda $M_{x_2}$ aggiungendo lo stato $q_{k+2}$ e la quintupla $(q_{k+2}, \#, q_{k+2}, \#, -)$, \dots.
\end{proof}

\noindent Il prossimo teorema dice che tra tutti gli algoritmi che calcolano una data funzione ce n'è uno privilegiato, nel senso che ha una forma speciale.\
Di conseguenza, ogni funzione ha una rappresentazione privilegiata.\

\begin{theorem}[Forma normale]
    \label{forma normale}
    Esistono un predicato $T(i,x,y)$ e una funzione $U(y)$ calcolabili totali tali che $\forall i,x.\ \varphi_i(x) = U(\mu y.\ T(i,x,y))$.\
    Inoltre, $T$ e $U$ sono primitive ricorsive.\
\end{theorem}

\begin{proof}
    Definisci $T(i, x, y)$, detto comunemente \textit{predicato di Kleene},
    vero se e solamente se $y$ è la codifica di una computazione terminante di $M_i$ con dato iniziale $x$.\
    Per calcolare $T$, dato $i$ recupera $M_i$ dalla lista e comincia a scandire i valori $y$.\
    Decodifica ognuno di essi uno alla volta e, avendo come ingresso $x$, controlla se il risultato è una computazione terminante della    forma $M_i(x) = c_0, c_1,\dots, c_n$.\
    Questa sequenza di passi termina sempre e quindi $T$ è totale.\
    Se $y$ è la codifica di una computazione terminante di $M_i(x)$, allora $c_n = (h, \triangleright z \underline{\#})$ e definisci $U$ in modo che $U(y) = z$ e i passi necessari a questo calcolo sono finiti e terminano tutti:\ quindi $U$ è totale.\
    L'intero procedimento seguito è effettivo, e quindi $T$ e $U$ sono calcolabili per la tesi di Church-Turing.

    Inoltre $U$ e $T$ sono ricorsivi primitivi perché le codifiche che usiamo lo sono e perché lo sono i controlli effettuati.
\end{proof}

\noindent Si noti che la versione del teorema di forma normale sulle MdT deterministiche è tale per cui se esiste un $y$ tale che $T(i, x, y)$ risulta vero, allora tale $y$ è anche unico.\
Invece, se le MdT considerate fossero non-deterministiche (vedi Def.\ \ref{MdT_non-det}), ci potrebbero essere più computazioni che terminano nello stato \textit{h}, per cui l'operatore di minimizzazione darebbe come risultato il minimo intero che codifica una di esse (si veda la discussione sulle regole di valutazione fatta nel capitolo 1.6).\

Un'immediata conseguenza del teorema di forma normale è che ogni funzione calcolata da una MdT ammette una definizione $\mu$-ricorsiva.\
In altre parole, vale il seguente corollario.

\begin{corollario}
    Le funzioni T-calcolabili sono $\mu$-ricorsive.
\end{corollario}

\begin{lemma}
    Le funzioni $\mu$-calcolabili sono T-calcolabili.
\end{lemma}

\noindent Ora è facile concludere l'equivalenza tra MdT e funzioni $\mu$-ricorsive.\
\begin{theorem}
    Una funzione è T-calcolabile se e solo se è $\mu$-calcolabile.\
\end{theorem}

\noindent Il teorema di forma normale e quello d'equivalenza tra MdT e funzioni $\mu$-ricorsive ha il seguente corollario interessante dal punto di vista informatico.\
La sua rilevanza nel nostro campo è legata al fatto che le funzioni primitive ricorsive si possono rappresentare con un programma nel linguaggio \textit{\footnotesize FOR} e quelle $\mu$-ricorsive con uno nel linguaggio \textit{\footnotesize WHILE} (v.\ capitolo precedente); pertanto, si deduce che \textit{ogni programma} può essere scritto (in forma normale) usando un comando di tipo \texttt{while} e due di tipo \texttt{for} (ciò è particolarmente rilevante quando il programma in questione sia l'interprete di un linguaggio, come vedremo più avanti).

\begin{corollario}
    Ogni funzione calcolabile parziale può essere ottenuta da due funzioni primitive ricorsive e una sola applicazione dell'operatore $\mu$.
\end{corollario}

\noindent Adesso arriva un teorema molto importante.\
Dice che un formalismo universale, cioè uno che esprima \textit{tutte} le funzioni calcolabili, è così potente da riuscire a esprimere \textit{l'interprete dei propri programmi}.\
Vedremo più avanti che questa capacità può essere usata ``alla rovescia'', cioè per mostrare che un formalismo è universale.\
Vediamo il semplice caso con funzioni a una variabile e la sua dimostrazione; l'estensione al caso generale per funzioni a $n$ variabili è immediata.\

\begin{theorem}[Enumerazione]
    \label{teo-enumerazione}
    Esiste una funzione calcolabile parziale $\varphi_z(i,x)$ tale che $\varphi_z(i,x) = \varphi_i(x)$.\

    (Si noti l'ordine dei quantificatori:\ $\exists z$ tale che $\forall i,x$ si ha $\varphi_z(i,x) = \varphi_i(x)$.)
\end{theorem}

\begin{proof}
    Poiché la funzione $U(\mu y.\ T(i, x, y))$ usata nel teorema di forma normale è definita per composizione e $\mu$-ricorsione a partire da funzioni (meglio da una funzione e un predicato) primitive ricorsive, essa stessa è una funzione calcolabile in due argomenti $i$ e $x$.\
    Avrà quindi un indice che chiamiamo $z$, cioè sia $\varphi_z(i, x) = U(\mu y.\ T(i, x, y))$.\
    Applichiamo allora il teorema di forma normale, da cui $U(\mu y.\ T(i, x, y)) = \varphi_i(x)$.\
    Per ottenere la tesi basta la transitività dell'uguaglianza.\

    Più intuitivamente e informalmente:\ $M_z$ recupera la descrizione di $M_i$ e la applica a $x$.\
\end{proof}

\noindent Il teorema di enumerazione garantisce che esiste la MdT Universale, $M_z$, che ha in ingresso la descrizione di una MdT $M_i$, il dato $x$ e ``si comporta come $M_i$''.\
È esattamente quanto avviene ogni giorno con i nostri programmi:\ li diamo in pasto a una macchina realizzata in parte H/W, in parte S/W (si tratta di una macchina reale, e quindi solo ``quasi-universale'', perché, tra l'altro, ha memoria limitata).\
La macchina in questione esegue i nostri programmi, ovvero si comporta esattamente come dettato dai suoi dati, i programmi in ingresso:\ finché l'istruzione corrente non è {\small STOP} prende i suoi argomenti, esegue l'istruzione, ne memorizza il risultato e aggiorna il puntatore all'istruzione corrente.

Intuitivamente, il teorema di enumerazione ci ``libera'' dalla necessità di avere un esecutore umano delle MdT, così come previsto da Alan Turing:\ \textit{esiste una macchina che esegue gli algoritmi}.

\begin{theorem}[Parametro, s-1-1]
    Esiste una funzione calcolabile totale (iniettiva) $s_1^1$ con due argomenti, tale che $\forall i,x$
    \[\varphi_{s_1^1(i,x)} = \lambda y.\ \varphi_i(x,y)\]
\end{theorem}

\noindent Intuitivamente, la macchina $M_{s(i,x)}$, o più liberamente l'algoritmo o il programma $P_{s(i,x)}$ (omettiamo d'ora in poi l'indice e l'apice 1, per leggibilità) opera su $y$ soltanto, mentre $P_i$ opera su $x$ e $y$.\
Quindi $x$ è un \textit{parametro} di $P_i$.\
Ad esempio, sia $\varphi_i(x,y)$ la funzione $x \times f(y)$ (con $f$ qualunque); allora, a partire da $i$ e da 2, mediante la $s$ trovo in \textit{modo effettivo} l'indice del programma che calcola la funzione $2 \times f(y)$, cioè determino $\varphi_{s(i,2)}$.\

Il teorema \textit{s-m-n} è importante in informatica perché è la base per la tecnica di ``valutazione parziale'' secondo la quale si specializza via via un programma generale per ottenerne versioni più efficienti in casi particolari (compilatori/interpreti per architetture parametriche, ecc.).\
Questo teorema è inoltre utilissimo nella teoria della calcolabilità sia perché ci offre uno strumento potentissimo da usare nelle dimostrazioni, sia per le ragioni espresse dal teorema di espressività, riportato più avanti.\
Vediamo la dimostrazione della sua versione \textit{s-1-1}, cioè con $m, n = 1$, e poi l'enunciato generale.

\begin{proof}[Dimostrazione intuitiva del teorema s-1-1]
    Per calcolare $\varphi_{s_1^1(i,x)}(y)$ si prenda $M_i$ decodificando $i$ e si predisponga lo stato iniziale, cioè $M_i(x, y)$, dove $x$ è fissato in anticipo.\
    (Se vi fosse più familiare, prendete il programma $P_i$ che avrà un'istruzione per leggere il valore $k$ del parametro $x$:\ allora rimuovetela e inserite al suo posto l'assegnamento $x := k$.)
    Quella delineata (in entrambi i casi) è una procedura effettiva, cioè algoritmica, che termina sempre, quindi per la tesi di Church-Turing esiste una funzione calcolabile totale $s = s_1^1$.\
    Se tale funzione non fosse iniettiva, allora si costruisca $s'$ con $\varphi_{s(i,x)} = \varphi_{s'(i,x)}$ in modo tale che $s'(i, x)$ generi indici (che esistono perché gli indici delle MdT che calcolano $s(i, x)$ sono $\#(\mathbb{N})$) in modo strettamente crescente, cioè tali che $s'(i_0, x_0 ) > s'(i_1, x_1)$ se la codifica della coppia $(i_0, x_0)$ è maggiore di quella di $(i_1, x_1)$.\
    A questo punto basta notare che una funzione totale strettamente crescente è iniettiva.
\end{proof}

\begin{theorem}[Parametro, s-m-n]
    \label{parametro}
    $\forall m,n > 0$ esiste una funzione calcolabile totale (iniettiva) $s_n^m$ con $m+1$ argomenti tale che $\forall i, x_1, \dots, x_m$
    \[\varphi_{s_n^m(i, x_1,\dots x_m)}^{(n)} = \lambda y_1, \dots y_n.\ \varphi_i^{(m+n)}(x_1,\dots, x_m, y_1, \dots, y_n)\]
\end{theorem}

\noindent Si noti come il teorema del parametro e quello di enumerazione siano in un certo senso l'inverso l'uno dell'altro.\
Infatti uno ``abbassa'' un argomento nella posizione di indice, mentre l'altro ``innalza'' un indice nella posizione di argomento.

L'importanza dei teoremi del parametro e di enumerazione può essere compresa ancora meglio considerando il seguente teorema che non dimostreremo.\

\begin{theorem}[Espressività]
    Un formalismo è Turing-equivalente (calcola tutte e sole le funzioni T-calcolabili, è universale) se e solamente se
    \begin{itemize}
        \item ha un algoritmo universale (cioè vale il teorema di enumerazione),
        \item vale il teorema del parametro.
    \end{itemize}
\end{theorem}

\noindent Grazie al teorema \textit{s-m-n} si dimostra un teorema molto elegante, che ha un ruolo fondamentale, sia in informatica che nella pura teoria della calcolabilità.\

\begin{theorem}[Ricorsione, Kleene II]
    $\forall f$ funzione calcolabile totale $\exists n$ tale che $\varphi_n = \varphi_{f(n)}$.

    (Un tale indice viene detto \textbf{punto fisso} di $f$.)
\end{theorem}

\noindent Prima della dimostrazione, diamo un po' di intuizione:\ la funzione $f$ ``trasforma'' programmi in programmi, come fanno i compilatori.\
Infatti $f$ trasforma indici:\ dato $n$, ovvero il programma $P_n$ lo trasforma in $P_{f(n)}$.\
Quando consideriamo il punto fisso, la trasformazione operata da $f$ \textit{non cambia} la funzione calcolata, ovvero trasforma un programma $P_n$ nel programma $P_{f(n)}$ con la \textit{stessa} semantica.\
Si noti che l'accezione ``punto fisso'' usata qui è diversa da quella solita in cui il punto fisso $x$ di un funzione $g$ è tale che $g(x) = x$:\ qui il punto fisso riguarda la \textit{funzione}, e non l'indice delle macchine che la calcolano.\

Questo teorema fornisce in un certo senso la ``base'' della semantica denotazionale; garantisce la realizzabilità di macchine che eseguono programmi ricorsivi o delle funzioni di crittografia o di molte altre diavolerie infernatiche.\
Nella dimostrazione del teorema si fa uso del fatto che le funzioni calcolabili sono chiuse rispetto alla trasformazioni di indici introdotte dal teorema del parametro.\

\begin{proof}
    Definiamo la seguente funzione calcolabile ``diagonale''
    \[\psi(u,z) = \left\{\begin{array}{l l}
            \varphi_{\varphi_u(u)}(z) & \mathrm{se}\ \varphi_u(u) \downarrow \\
            \mathrm{indefinita}       & \mathrm{altrimenti}                  \\
        \end{array}\right.\]
    Poiché $\psi$ è calcolabile, per Church-Turing avrà un indice $\varphi_i(u,z) = \psi(u,z)$.\
    A questo punto per il teorema del parametro si ha $\varphi_i(u,z) = \varphi_{s(i,u)}(z)$, ma $s(i,u)$ dipende \textit{solo} da $u$, quindi ponendo $d(u) = \lambda u.\ s(i,u)$ si ottiene che
    \begin{equation}
        \tag{1}
        \psi(u,z) = \varphi_i(u,z) = \varphi_{s(i,u)}(z) = \varphi_{d(u)}(z) = \left\{\begin{array}{l l}
            \varphi_{\varphi_u(u)}(z) & \mathrm{se}\ \varphi_u(u) \downarrow \\
            \mathrm{indefinita}       & \mathrm{altrimenti}                  \\
        \end{array}\right.\
    \end{equation}
    Per il teorema \textit{s-m-n}, $d(u)$ è totale e iniettiva (e \textit{non} dipende da $f$).\
    Data $f$, $f \circ d$ è calcolabile e sia $v$ proprio un indice tale che
    \begin{equation}
        \tag{2}
        \varphi_v(x) = f(d(x))
    \end{equation}
    Tale funzione è totale (perché sia $d$ che $f$ lo sono), e quindi $\varphi_v(v)\downarrow$.\
    Pertanto, in accordo con la definizione (1) abbiamo $\varphi_{d(v)} = \varphi_{\varphi_v(v)}$.\
    Calcoliamo adesso $d(v)$ e supponiamo che il risultato sia $n$, cioè poniamo
    \begin{equation}
        \tag{3}
        n = d(v)
    \end{equation}
    Dimostriamo che $n$ è un punto fisso di $f$.\
    Infatti
    \[\varphi_n \stackrel{(3)}{=} \varphi_{d(v)} \stackrel{(1)}{=} \varphi_{\varphi_v(v)} \stackrel{(2)}{=} \varphi_{f(d(v))} \stackrel{(3)}{=} \varphi_{f(n)}\]
    Si noti come nell'eguaglianza più a sinistra si sfrutti l'iniettività della funzione $d$ garantitaci dal teorema del parametro.\

\end{proof}

\noindent Ci sono due fatti interessanti che sono correlati con il teorema di ricorsione.\
\begin{property}
    Nelle ipotesi del teorema di ricorsione,
    \begin{itemize}
        \item il punto fisso è calcolabile mediante una funzione totale (iniettiva) $g$ e a partire da(ll'indice di) $f$;
        \item ci sono $\#(\mathbb{N})$ punti fissi di $f$.
    \end{itemize}
\end{property}

\noindent C'è un altro modo per dimostrare il teorema di ricorsione, o meglio per specificare come deve essere implementata la ricorsione nei linguaggi di programmazione.\
Supponete di avere una procedura ricorsiva $P$ il cui corpo sia $C$, all'interno del quale ovviamente appare la chiamata a $P$ stessa.\
La tecnica usata per definire la semantica operazionale consiste nel memorizzare in un componente, di solito chiamato \textit{ambiente} e rappresentato da una funzione $\rho$, l'associazione tra $P$ e $C$, cioè $\rho(P) = C$.\
Al momento della chiamata si cerca nell'ambiente il significato di $P$, che appunto è $C$, e si trasferisce il controllo all'inizio di $C$, dopo aver ovviamente legato i parametri formali con quelli attuali.\
Il significato di $P$ viene mantenuto nell'ambiente, cosicché alla successiva chiamata si possa recuperare $\rho(P) = C$.\

Questa tecnica a volte viene chiamata \textit{copy rule}, perché in effetti si copia il corpo $C$ della procedura $P$ al posto di $P$ stesso, tante volte quanto è necessario.\
Si noti che non si tratta di una macro-espansione, né si potrebbe macro-espandere per sempre la chiamata a $P$, perché non è noto a priori quante volte si debba chiamare la $P$ stessa.\
Per rendersene conto, si consideri di nuovo la funzione fattoriale.\
Il suo corpo viene ``usato'' $n + 1$ volte se $n$ è l'argomento:\ di fatto si approssima la definizione della funzione dove la macro-espansione è ripetuta per sempre, avendo definito il suo comportamento sull'intervallo $[0\dots n]$, mentre per valori maggiori di $n$ il risultato è lasciato indefinito.

\chapter{Architetture con Parallelismo a Livello di Istruzioni}

\section{Architettura della CPU pipeline}

Il concetto alla base di quest'architettura è la parallelizzazione della CPU mediante la \textit{parallelizzazione dell'interprete firmware} delle istruzioni, eseguite dal processore con la collaborazione delle altre unità della CPU stessa.\
È noto che tale interprete costa di più fasi, alcune delle quali eventualmente effettuate in uno stesso ciclo di clock da parte di un processore elementare:

\begin{enumerate}
    \item chiamata istruzione, contenente la lettura dell'istruzione dalla memoria con la cooperazione di MMU e memoria,
    \item decodifica dell'istruzione, preparazione degli indirizzi di eventuali operandi in memoria, preparazione degli eventuali operandi presenti in registri generali,
    \item eventuale lettura degli operandi in memoria, ancora con la cooperazione di MMU e memoria,
    \item fase di esecuzione vera e propria e aggiornamento del contatore istruzioni,
    \item eventuale scrittura dei risultati nei registri generali o in memoria, con la cooperazione di MMU e memoria,
    \item trattamento delle interruzioni.
\end{enumerate}

\noindent Queste fasi hanno un ordinamento totale e operano concettualmente su uno \textit{stream} d'istruzioni:\ di conseguenza l'interprete si presta a una parallelizzazione in \textit{pipeline}.

\subsection{Memoria e MMU}

La fase 1 e la fase 3 (5) necessitano, per essere eseguite in parallelo su istruzioni diversa, di due unità di memoria distinte, una dedicata a memorizzare solo istruzioni (\textit{Memoria Istruzioni}, \textbf{IM}) e l'altra a memorizzare solo dati (\textit{Memoria Dati}, \textbf{DM}).\
Entrambe sono dotate di propria MMU.

IM e DM sono partizioni disgiunte della \textit{cache primaria}.

L'unità MINF (\textbf{Memory Interface}) mette in comunicazione, e arbitra, le richieste per il trasferimento di blocchi nei confronti del livello superiore della memoria esterna.

Il Bus di I/O è connesso alla $\mathrm{MMU}_D$, principalmente allo scopo di implementare la tecnica del Memory Mapped I/O.

\textit{Per non introdurre comunicazioni a domanda-risposta} con il processore nella chiamata dell'istruzione, deleghiamo a IM il compito di leggere, senza soluzione di continuità, istruzioni a indirizzi logici consecutivi, generando di fatto lo \textbf{\textit{stream d'ingresso}} della pipeline.\
A tale scopo, MMUI contiene una copia (IC1) del contatore istruzioni (IC) del processore.

I contenuti di $\mathrm{IC}_1$ e IC sono identici e consistenti \textit{finché} non è eseguita un'istruzione di salto che fa effettivamente saltare, dopodiché verrà ripristinato il nuovo stato consistente.

\subsection{Processore}

Lo stream d'istruzione viene raccolto dallo stadio che chiameremo \textit{Unità Istruzioni} (\textbf{IU}), delegato alle fasi 2, parte della 4 e 6.\
IU \textit{decodifica} ogni istruzione e delega ad altri stadi della pipeline il suo proseguimento.

IU contiene la copia ``affidabile'' del \textit{contatore istruzioni IC}.\
Di conseguenza, è naturale pensare che tutte le \textbf{\textit{istruzioni di salto}} siano eseguite interamente da IU.\
In caso di salto, è assolutamente necessario introdurre una prima \textit{retroazione nella pipeline}, che quindi ``non è una pipeline pura'':\ IU invia l'indirizzo logico di salto a IM che \textit{aggiorna IC1 e lo stream di istruzioni riprende a partire dal nuovo indirizzo}.\
È inevitabile che, in caso di salto, IM possa aver già inviato a IU una o più istruzioni, che quindi dovranno essere scartate finché IU non riceve l'istruzione che ha richiesto esplicitamente.\
Questo fenomeno introduce una prima \textit{degradazione} delle prestazioni.

Inoltre, IU provvede a \textit{calcolare gli indirizzi di memoria} delle \texttt{Load} e \texttt{Store} e a chiedere a DM l'esecuzione delle rispettive operazioni di lettura e scrittura.\
Nel caso di \texttt{Load}, il valore letto verrà inviato da DM a uno stadio successivo della pipeline (EU), nel caso di Store IU provvederà anche a inviare a DM il dato da scrivere.\
Di conseguenza, la \texttt{Store} \textit{termina nella DM}.

Poiché il contatore istruzioni è contenuto in IU, quest'ultima è collegata a UNINT e provvede ad effettuare il \textit{trattamento delle interruzioni}.

Tutte le \textit{istruzioni aritmetico-logiche} sono delegate allo stadio chiamato Unità Esecutiva (EU), che provvede a scrivere il risultato nel registro generale destinazione (fasi 4, 5).\
La \texttt{Load} è vista come un'istruzione ``identità'' con un operando in memoria, ragione per cui è EU a occuparsi di ricevere il dato in memoria e a scriverlo nel registro destinazione.

Tutte le comunicazioni tra unità della pipeline avvengono su collegamenti dedicati con protocollo asincrono.

Occupiamoci ora dei Registri Generali.\
Una soluzione efficiente consiste nel tenerne una doppia copia:\ una in IU ($\mathrm{RG}_1$) e l'altra in EU (RG).\
IU necessità solo di \textit{leggere} i registri generali di $\mathrm{RG}_1$.\
Le modifiche dei registri generali, in istruzioni aritmetico-logiche e di \texttt{Load}, sono effettuate da EU su RG e sono, da quest'unità, comunicate a IU che provvede a scrivere i valori ricevuti in $\mathrm{RG}_1$.\
Di conseguenza, la copia sempre aggiornata è quella in EU (RG), mentre ci saranno dei momenti in cui la copia in IU ($\mathrm{RG}_1$) non sarà aggiornata:\ un meccanismo di sincronizzazione provvede al corretto funzionamento di IU, in particolare a impedire a IU la lettura di registri non ancora aggiornati.

Oltre alla \textit{retroazione dovuta alla presenzadi istruzioni di salto}, un'altra seria causa di retroazione è data dalle \textbf{\textit{dipendenze logiche sui registri generali}}:\ IU può voler leggere in $\mathrm{RG}_1$ il contenuto di un registro che deve ancora essere aggiornato dall'esecuzione in EU di un'istruzione precedente dello stream.\
Entrambe le retroazioni

\begin{itemize}
    \item richiedono che, per ragioni di correttezza, vengano introdotti degli opportuni meccanismi di sincronizzazione che ordinino gli eventi in modo consistente con la semantica del programma,
    \item rappresentano cause di degradazione delle prestazioni.
\end{itemize}

\section{Architettura pipeline astratta}

Adotteremo una visione semplificata dell'architettura, che rappresenta l'\textbf{\textit{ar\-chitettura astratta}}:\ questa è costituita da soli quattro stadi, \textbf{IM}, \textbf{IU}, \textbf{DM}, \textbf{EU}.

\begin{itemize}
    \item Sappiamo sia IM sia DM possono essere realizzati come pipeline di 2 o 3 stadi; si tratta di strutture pipeline ``pure'', che non introducono alcun problema di retroazione o dipendenza locica.\ Considerare sia IM sia DM come un unico sottosistema, avente il tempo di servizio effettivo uguale a quello della scrittura interna, semplifica la valutazione delle prestazioni senza nascondere eventi significativi o introdurre approssimazioni;
    \item come in precedenza, supporremo che il registro IC sia presente in doppia copia in IU e IM e che i registri generali siano presenti in doppia copia in IU ed EU.\ In entrambi i casi, opportuni meccanismi di sincronizzazione assicurano la correttezza di funzionamento quando si verifichino momentanee inconsistenze.
\end{itemize}

\noindent L'architettura astratta ha inoltre le seguenti caratteristiche:

\begin{itemize}
    \item \textit{ogni stadio (sottosistema) ha \textbf{tempo di servizio ideale per istruzione}}: \[t= T_{id}= 2\tau\]
    \item \textit{tutti i canali di comunicazione sono asincroni con grado di asincronia uguale a uno}.
\end{itemize}

\section{Implementazione delle unità della CPU pipeline}

Risolviamo anzitutto il problema della \textit{sincronizzazione tra IU e IM in seguito a istruzioni di salto}.\
La richiesta da IU a IM è accompagnata da un \textit{identificatore unico}, che IM associa all'istruzione inviata a IU, mantenendo lo stesso valore dell'identificatore in tutte le successive richieste finché IU non effettua una nuova richiesta.\
A ogni ricezione di istruzione, IU confronta il proprio valore dell'identificatore con quello ricevuto:\ in caso di concordanza l'istruzione è valida, altrimenti IU scarta l'istruzione ricevuta e attende altre istruzioni finché non trova concordanza.

\subsection{Registri generali}

La sincronizzazione per il corretto uso dei registri generali $\mathrm{RG}_1$ può essere così implementata all'interno dell'unità IU:

\begin{itemize}
    \item a ogni registro $\mathrm{RG}_1$ è associato un \textit{semaforo interno}, non negativo, inizializzato a zero;
    \item per ogni istruzione aritmetico-logica o di \texttt{Load}, IU incrementa di uno il semaforo associato al registro destinazione in $\mathrm{RG}_1$;
    \item ogni volta che scrive in un registro di $\mathrm{RG}_1$ su richiesta di EU, IU decrementa di uno il semaforo associato;
    \item quando IU intende leggere un registro di $\mathrm{RG}_1$ la lettura può avere luogo solo se il semaforo associato al registro stesso è uguale a zero; in caso contrario, IU:
          \begin{enumerate}
              \item si mette in attesa degli aggiornamenti del valore del registro, finché il semaforo non ritorna a zero
                    \begin{center}
                        oppure
                    \end{center}
              \item ricorda la situazione dell'istruzione corrente e continua a servire nuove istruzioni in arrivo.
          \end{enumerate}
\end{itemize}

\subsection{Funzionamento in-order e out-of-order}

Consideriamo le due situazioni 1), 2) descritte a proposito dell'attesa che un registro generale divenga aggiornato.

Il funzionamento 1) non introduce modifiche all'ordinamento delle istruzioni elaborate in IU rispetto allo stream d'ingresso.\
Le architetture che adottano questo funzionamento sono dette ``\textbf{in-order}''.

Il funzionamento 2) introduce modifiche all'ordinamento delle istruzioni elaborate dalla IU rispetto allo stream d'ingresso:\ alcune istruzioni, purché abilitate (tutti i registri utili a IU sono aggiornati), ``scavalcano'' altre istruzioni in attesa.\
Le architetture che adottano questo funzionamento sono dette ``\textbf{out-of-order}''.

\section{Ottimizzazioni}

Avendo individuato le cause di degradazione delle performance, le ottimizzazioni tendenti a ridurne l'effetto devono occuparsi di:
\begin{itemize}
    \item \textit{per le degradazioni dovute ai salti}:
          \begin{itemize}
              \item ridurre la probabilità di salto;
              \item sfruttare i tempi morti, introdotti dalle bolle, per effettuare lavoro utile;
              \item ``predire'' l'esito di salti condizionati in macchine ``out-of-order'';
          \end{itemize}
    \item \textit{per le degradazioni dovute alle dipendenze logiche}:
          \begin{itemize}
              \item ridurre la probabilità di dipendenza logica;
              \item aumentare la distanza delle dipendenze logiche.
          \end{itemize}
\end{itemize}

\subsection{Minimizzazioni delle degradazioni dovute ai salti}

Ciò significa ridurre la frequenza stessa con cui vengono utilizzate istruzioni di salto.\
Si tratta di tipiche ottimizzazioni \textit{a tempo di compilazione}, come:

\begin{itemize}
    \item espansione di \textit{macro}, al posto di subroutine o procedure,
    \item \textit{loop unfolding},
\end{itemize}

\noindent che comportano un aumento della memoria \textit{logica} del processo.\
In termini di memoria \textit{fisica}, queste tecniche possono comportare un aumento del working set del processo.

Un'interessante tecnica a tempo di compilazione è quella chiamata \textit{Delayed Branch}, che può essere considerata un caso particolare di \textit{spostamento di codice}, cioè basata sul concetto di cercare di sfruttare i tempi morti, introdotti dalle potenziali bolle, per effettuare invece lavoro utile.

\subsection{Minimizzazione delle degradazioni dovute alle dipendenze logiche}

A tempo di compilazione possono essere applicate tecniche dirette ad ``allontanare'' il più possibile le istruzioni che sono legate da dipendenze logiche.\
Questo comporta, principalmente, l'adozione di tecniche di \textit{spostamento di codice}.

\subsection{Unità Esecutiva parallela}

L'Unità Esecutiva può essere definita secondo lo schema seguente, contenente:

\begin{itemize}
    \item una Unità Funzionale pipeline \textit{moltiplicatore}/\textit{divisore} in virgola fissa,
    \item una Unità Funzionale pipeline \textit{addizionatore} in virgola mobile,
    \item una Unità Funzionale pipeline \textit{moltiplicatore}/\textit{divisore in virgola mobile}, che può anche includere altre operazioni in virgola mobile, come la radice quadrata,
    \item l'unità \textit{EU\textsubscript{MASTER}}.
\end{itemize}

\noindent L'unità indicata con $\mathit{EU_{MASTER}}$ interfaccia IU e DM e \textit{contiene la copia principale dei registri generali} (\textit{RG}) \textit{e dei registri in virgola mobile RF}.\
Ricevendo una istruzione da IU, il funzionamento è il seguente:

\begin{itemize}
    \item se l'operazione richiesta da IU è una aritmetica corta o una {\ttfamily LOAD}, la \textit{esegue direttamente} (per la {\ttfamily LOAD} attende il dato da DM) e invia a IU il valore del registro generale destinazione e il suo indirizzo;
    \item se l'operazione è lunga in virgola fissa, oppure è in virgola mobile, la distribuisce all'Unità Funzionale corrispondente \textit{insieme ai valori degli operandi} letti dai registri RG o RF rispettivamente.
\end{itemize}

\begin{figure}[H]
    \centering
    \includegraphics[width=\textwidth]{immagini/Architettura_pipeline.png}
\end{figure}


\chapter{Complessità}
\chapter{Cifratura a blocchi e a chiave pubblica}

\section{Cifratura a blocchi}

Cifrari come l'AES e il DES lavorano a blocchi:\ se si ha un messaggio di $n$ bit, questo \textbf{viene diviso e cifrato in vari blocchi} sempre con la stessa chiave.\
Si noti che un messaggio ha la seguente forma:\ \[m = m_1 m_2 m_3 \dots m_l\ \mathrm{t.c.}\ |m_i| = 128\ \mathrm{bit}\]
Qualora non valga che $|m_l| < 128$ bit allora si aggiunge una concatenazione t.c.\ $|m_l 100\dots 0| = 128\ \mathrm{bit}$.\
Se invece $|m_l| = 128$ bit allora si aggiunge un blocco terminatore $|10\dots 0| = 128$.\
\textit{Per evitare che messaggi diversi diano crittogrammi uguali}, al messaggio viene concatenata una stringa iniziale $c_0$ che potrebbe rappresentare anche il timestamp in cui viene cifrato il messaggio.\

La \textbf{cifratura} è semplice:\
\[c_i = C(c_{i-1} \oplus m_i, k)\]
si prende l'$i$-esimo blocco del messaggio da cifrare, si esegue lo $\oplus$ col crittogramma $c_{i-1}$ calcolato per il blocco precedente e infine si cifra il risultato ottenuto con la chiave $k$.\
Sostanzialmente per cifrare il blocco di messaggio $(i + 1)$-esimo è necessario aver già cifrato il blocco di messaggio $i$-esimo.\

La \textbf{decifrazione} è anch'essa semplice:\ dato che la doppia applicazione di una stringa con lo $\oplus$ ha l'effetto di annullare l'effetto stesso dello $\oplus$, allora per decifrare $c_i$ si esegue
\[m_i = D(c_i, k) \oplus c_{i-1}\]

\noindent Ci sono proprietà molto interessanti nella cifratura a blocchi:\
\begin{itemize}
    \item La decifrazione può essere fatta \textit{in parallelo} perché il blocco $c_{i-1}$ è già noto nel momento in cui bisogna decifrare il blocco $c_i$.
    \item La cifratura a blocchi prevede che il crittogramma $c_i$ influenzi solo se stesso e $c_{i+1}$.\ Questo significa che se alcuni bit di $c_i$ sono trasmessi male, la decifratura stessa di $c_i$ non influenzerà tutto il messaggio $m$ ma solo i blocchi $m_i$ e $m_{i+1}$.\
\end{itemize}

\section{Crittografia a chiave pubblica}

Per tanto tempo, il problema principale della crittografia è stato lo \textbf{scambio della chiave segreta}.\
Questo problema è rimasto tale fino al 1976, quando Diffie e Hellman hanno proposto due metodi per generare e scambiare una chiave segreta su un canale insicuro senza che le due parti si debbano incontrare precedentemente.\
Il primo metodo è un algoritmo chiamato \textbf{protocollo DH} ed è ancora usato nella crittografia su Internet; il secondo è la \textbf{crittografia a chiave pubblica} che non prevede affatto alcuno scambio di chiave.\
In quest'ultimo metodo sostanzialmente le due parti cifrano il messaggio con una chiave pubblica e lo decifrano rispettivamente con una chiave privata.\

\vspace{12pt}
\noindent Nei cifrari a chiave privata (o simmetrici) si ha una chiave segreta per ogni coppia di utenti che vuole parlare, facendo crescere il numero di chiavi con un fattore esponenziale.
\vspace{12pt}

\noindent Nei cifrari a chiave pubblica o cifrari asimmetrici, invece, tutti possono inviare messaggi cifrati e solo il ricevente può decifrarli.\
Le operazioni di cifratura e decifrazione sono pubbliche e utilizzano due chiavi diverse:
\begin{itemize}
    \item  $k_{\mathit{pub}}$ per cifrare:\ è pubblica, nota a tutti.
    \item  $k_{\mathit{priv}}$ per decifrare:\ è privata, nota solo al destinatario.
\end{itemize}

\noindent In questo caso, ogni utente possiede una coppia $\langle k_{\mathit{pub}}, k_{\mathit{priv}}\rangle$:\ la chiave pubblica è nota a chiunque e qualsiasi persona interessata a inviare un messaggio all'utente che possiede $k_{\mathit{pub}}$ come chiave pubblica, può e deve cifrare il proprio messaggio esattamente con $k_{\mathit{pub}}$.\
Il destinatario provvederà a decifrare con la sua chiave privata $k_{\mathit{priv}}$ nota solo a lui.\
In questo modo il numero di chiavi è $2n$ se si hanno $n$ utenti.

\subsubsection{Cifratura e decifrazione in un cifrario a chiave pubblica}

\begin{flushleft}
    Cifratura:\ $c = C(m, k_{\mathit{pub}})$.\

    Decifrazione:\ $m = D(c, k_{\mathit{priv}})$.\
\end{flushleft}

\noindent I requisiti perché un cifrario a chiave pubblica funzioni sono:

\begin{enumerate}
    \item Correttezza:\ per ogni possibile messaggio $m = D(C(m, k_{\mathit{pub}}), k_{\mathit{priv}})$.\
    \item Efficiente e sicuro:\ tutto ciò che è lecito, legale, deve richiedere tempo polinomiale, mentre tutto ciò che è ``illegale'' deve richiedere tempo esponenziale.\ Quindi:
          \begin{itemize}
              \item la \textbf{coppia di chiavi è facile da generare} e deve risultare pressoché impossibile che due utenti scelgano la stessa chiave (\textit{generazione casuale delle chiavi});
              \item dati $m$ e $k_{\mathit{pub}}$, \textbf{è facile calcolare il crittogramma} $c = C(m, k_{\mathit{pub}})$ (\textit{adottabilità del sistema});
              \item dati $c$ e $k_{\mathit{priv}}$, \textbf{è facile calcolare il messaggio in chiaro} $m = D(c, k_{\mathit{priv}})$ (\textit{adottabilità del sistema});
              \item pur conoscendo il crittogramma $c,\ k_{\mathit{pub}},\ C\ \mathrm{e}\ D$ la \textbf{decifrazione} di $c$ deve essere \textbf{difficile per il crittoanalista} (\textit{sicurezza del cifrario}).
          \end{itemize}
\end{enumerate}

\noindent Al fine di poter soddisfare questi requisiti, per la funzione di cifratura $C$ si deve ricorrere a una funzione \textbf{one-way trapdoor}, cioè calcolare $c = C(m, k_{\mathit{pub}})$ è \textit{computazionalmente facile}, ma calcolare $m = D(c, k_{\mathit{priv}})$ è \textit{computazionalmente difficile} se non si conosce la trapdoor ($k_{\mathit{priv}}$).\
Tuttavia, Diffie e Hellman non riuscirono mai a trovare una funzione one-way trapdoor.\

\subsection{RSA}

Rivest, Adleman e Shamir proposero un sistema a chiave pubblica riuscendo a trovare una funzione one-way trapdoor.\
RSA si basa sulla moltiplicazione di due numeri primi $p$ e $q$:

\begin{itemize}
    \item Calcolare $n = p\cdot q$ è facile.\ Una moltiplicazione richiede sempre tempo polinomiale.
    \item Calcolare $p$ e $q$ conoscendo $n$ è difficile a meno che non si conosca uno dei due fattori.\ In pratica, fattorizzare $n$ senza conoscere $p$ o $q$ richiede tempo esponenziale.
\end{itemize}

\noindent RSA \textbf{utilizza algebra modulare} ($\mathit{mod}\ n$).

\subsection{Algebra modulare}

Usata in molti algoritmi crittografici per
\begin{itemize}
    \item ridurre lo spazio dei numeri su cui si opera e quindi aumentare la velocità di calcolo;
    \item rendere difficili problemi computazionali che sono semplici (o anche banali) nell'algebra non modulare.
\end{itemize}

\noindent In algebra modulare le funzioni tendono a comportarsi in modo ``\textit{imprevedibile}''.\
Si prenda, per esempio, la funzione $2^x$.\
Il suo comportamento nell'algebra ordinaria è monotono crescente:

\begin{table}[H]
    \centering
    \begin{tabular}{|c|c|c|c|c|c|c|c|c|c|c|c|c|c|}
        \hline
        x     & 1 & 2 & 3 & 4  & 5  & 6  & 7   & 8   & 9   & 10   & 11   & 12   \\\hline
        $2^x$ & 2 & 4 & 8 & 16 & 32 & 64 & 128 & 256 & 512 & 1024 & 2048 & 4096 \\\hline
    \end{tabular}
\end{table}

\noindent Se invece si prende $2^x\ \mathit{mod}\ 13$, il comportamento è:
\begin{table}[H]
    \centering
    \begin{tabular}{|c|c|c|c|c|c|c|c|c|c|c|c|c|c|}
        \hline
        x     & 1 & 2 & 3 & 4 & 5 & 6  & 7  & 8 & 9 & 10 & 11 & 12 \\\hline
        $2^x$ & 2 & 4 & 8 & 3 & 6 & 12 & 11 & 9 & 5 & 10 & 7  & 1  \\\hline
    \end{tabular}
\end{table}

\noindent Sale, scende, sale\dots ha un andamento caotico:\ perde struttura rispetto alla precedente e non dà alcun suggerimento.\

\subsubsection{Caratteristiche}

Preso $n \in \mathbb{N}$, intero positivo
\[\mathbb{Z}_n = \{0,1,2.\dots, n-1\}\qquad \mathbb{Z}_n^* \subseteq \mathbb{Z}_n\]
è l'insieme degli elementi di $\mathbb{Z}_n$ co-primi con $n$.\
Se $n$ è primo, $\mathbb{Z}_n^* = \mathbb{Z}_n$.\
Se $n$ non è primo, calcolare $\mathbb{Z}_n^*$ è computazionalmente difficile:\ richiede tempo proporzionale al valore di $n$ (per confrontare $n$ con gli elementi di $\mathbb{Z}_n$) quindi esponenziale nella lunghezza della sua rappresentazione.\

\vspace{12pt}
\noindent Dati due interi $a,b \geq 0$ e $n\geq0$, $a$ è congruo a $b$ modulo $n$
\[a \equiv b\ \mathit{mod}\ n\]
se e solo se esiste $k$ intero per cui
\[a = b + kn\]
Nelle relazioni di congruenza la notazione $\mathit{mod}\ n$ si riferisce all'intera relazione, nelle relazioni di uguaglianza la stessa notazione si riferisce solo al membro dove appare:\ $5\equiv 8\ \mathit{mod}\ 3$, ma $5\neq 8\ \mathit{mod}\ 3$

\subsubsection{Proprietà}

\[(a+b)\ \mathit{mod}\ m = (a\ \mathit{mod}\ m + b\ \mathit{mod}\ m)\ \mathit{mod}\ m\]
\[(a-b)\ \mathit{mod}\ m = (a\ \mathit{mod}\ m - b\ \mathit{mod}\ m)\ \mathit{mod}\ m\]
\[(a\times b)\ \mathit{mod}\ m = (a\ \mathit{mod}\ m \times b\ \mathit{mod}\ m)\ \mathit{mod}\ m\]
\[a^{r \times s}\ \mathit{mod}\ m = (a^r\ \mathit{mod}\ m)^s\ \mathit{mod}\ m\quad (r, s\ \mathrm{interi\ positivi})\]

\subsubsection{Funzione di Eulero}

Il numero di interi minori di $n$ e co-primi con esso
\[\phi(n) = |\mathbb{Z}_n^*|\]
Se $n$ è primo $\Rightarrow \phi (n) = n-1$.\

\begin{theorem}

    \[n\ \mathrm{composto} \Rightarrow \phi(n) = n\left(1-\frac{1}{p_1}\right)\dots \left(1-\frac{1}{p_k}\right)\]
    $p_1, \dots, p_k$ fattori primi di $n$, presi senza molteplicità.\
\end{theorem}

\begin{theorem}
    $n$ prodotto di due primi (semi-primo)
    \[n = p\cdot q \Rightarrow \phi(n) = (p-1)(q-1)\]
\end{theorem}

\begin{theorem}[Eulero]
    Per $n>1$ e per ogni $a$ primo con $n$
    \[a^{\phi(n)} \equiv 1\ \mathit{mod}\ n\]
\end{theorem}

\begin{theorem}[Fermat]
    Per $n$ primo e per ogni $a \in \mathbb{Z}_n^*$
    \[a^{n-1} \equiv 1\ \mathit{mod}\ n\]
\end{theorem}

\subsubsection{Conseguenze}

Per qualunque $a$ primo con $n$
\[a \times a^{\phi(n)-1}\equiv 1\ \mathit{mod}\ n \quad (\mathrm{Teorema\ di\ Eulero})\]
\[a \times a^{-1}\equiv 1\ \mathit{mod}\ n \quad (\mathrm{definizione\ di\ inverso})\]
quindi,
\[a^{-1} = a^{\phi(n)-1}\ \mathit{mod}\ n \]
L'inverso $a^{-1}$ di $a$ modulo $n$ si può dunque calcolare per esponenziazione di $a$ se si conosce $\phi(n)$.\
In generale, nell'algebra modulare l'esistenza dell'inverso non è garantita perché $a^{-1}$ deve essere intero.\

\begin{theorem}
    L'equazione $ax \equiv b\ \mathit{mod}\ n$ ammette soluzione se e solo se $\mathrm{MCD}(a,n)$ divide $b$.\
    In questo caso si hanno esattamente $\mathrm{MCD}(a,n)$ soluzioni distinte.\
\end{theorem}

\begin{corollario}
    L'equazione $ax \equiv b\ \mathit{mod}\ n$ ammette un'unica soluzione se e solo se $a$ e $n$ sono co-primi ($\mathrm{MCD}(a,n) = 1 $) $\Leftrightarrow$ esiste l'inverso $a^{-1}$ di $a$.\
\end{corollario}

\noindent Se nel corollario si sostituisce 1 con $b$, si ottiene $ax \equiv 1\ \mathit{mod}\ n$.\
Ne consegue che ammette esattamente una soluzione (l'inverso di $a$) se e solo se $a$ e $n$ sono primi tra loro.\
L'inverso si può calcolare come
\[a^{-1} = a^{\phi(n)-1}\ \mathit{mod}\ n \]
ma occorre conoscere $\phi(n)$, cioè fattorizzare $n$:\ è un problema ``difficile''.\

\subsubsection{Algoritmo di Euclide Esteso}

L'algoritmo di Euclide per il calcolo del MCD si può estendere per risolvere l'equazione in due incognite $ax+by=\mathrm{MCD}(a,b)$.\

\begin{flushleft}
    \ttfamily
    Function Extended\_Euclid(a,b)

    \quad if ($b=0$) then return $\langle a, 1, 0\rangle$

    \quad else

    \qquad $\langle d', x', y' \rangle = $ Extended\_Euclid($b, a\ \mathit{mod}\ b$);

    \qquad $\langle d, x, y \rangle = \langle d', y', x' - \left\lfloor \frac{a}{b}\right\rfloor y' \rangle$

    \qquad return $\langle d, x, y\rangle$
\end{flushleft}

\noindent \textbf{Osservazioni}:\

\begin{itemize}
    \item La funzione \texttt{Extended\_Euclid} restituisce una delle triple di valori \[\langle\mathrm{MCD} (a, b), x, y\rangle\] con $x,y$ tali che $ax + by = \mathrm{MCD}(a,b)$.\ Quindi $d = \mathrm{MCD}(a,b)$.\
    \item complessità logaritmica nel valore di $a$ e $b$, quindi \textit{polinomiale} nella dimensione dell'input.\
\end{itemize}

\noindent L'algoritmo di Euclide esteso si può applicare al \textbf{calcolo dell'inverso}.\

$ax \equiv 1\ \mathit{mod}\ b \Leftrightarrow ax = bz + 1$ per un opportuno valore di $z$ se e solo se $ax +by = \mathrm{MCD}(a,b)$, dove $y =-z$ e $\mathrm{MCD}(a,b) =1$.\

\subsubsection{Generatori}

$a \in \mathbb{Z}_n^*$ è un \textbf{generatore} di $\mathbb{Z}_n^*$ se la funzione
\[a^k\ \mathit{mod}\ n\qquad 1 \leq k \leq \phi(n)\]
\textbf{genera tutti e soli} gli elementi di $\mathbb{Z}_n^*$.\
Produce come risultati tutti gli elementi di $\mathbb{Z}_n^*$, ma in un ordine difficile da prevedere.\

\begin{theorem}[Eulero]
    $a^{\phi(n)} \equiv 1\ \mathit{mod}\ n \Rightarrow 1 \in \mathbb{Z}_n^*$ è generato per $k = \phi(n)$.\
    Per ogni generatore \[a^k \not\equiv 1\ \mathit{mod}\ n\qquad 1\leq k < \phi(n)\]
\end{theorem}

\begin{theorem}
    Se $n$ è un numero primo, $\mathbb{Z}_n^*$ ha almeno un generatore.\
\end{theorem}

\begin{itemize}
    \item Per $n$ primo, non tutti gli elementi di $\mathbb{Z}_n^*$ sono suoi generatori (1 non è mai generatore e altri elementi non possono esserlo).
    \item Per $n$ primo, i generatori di $\mathbb{Z}_n^*$ sono in totale $\phi(n-1)$.
\end{itemize}

\subsubsection{Problemi sui generatori rilevanti in crittografia}

Risolvere nell'incognita $x$ l'equazione $a^x = b\ \mathit{mod}\ n$, con $n$ primo.

L'equazione ammette una soluzione per ogni valore di $b$ se e solo se $a$ è un generatore di $\mathbb{Z}_n^*$.\
Tuttavia non è noto a priori in che ordine sono generati gli elementi di $\mathbb{Z}_n^*$, quindi non è noto per quale valore di $x$ si genera $b\ \mathit{mod}\ n$.\
Un esame diretto della successione richiede tempo esponenziale nella dimensione di $n$:\ non è noto un algoritmo polinomiale di soluzione.\

\subsection{Funzioni one-way trap-door}

Esistono funzioni matematiche che sembrano possedere i requisiti richiesti:\ il loro calcolo risulta incondizionatamente semplice e la loro inversione semplice se si dispone di un'informazione aggiuntiva sui dati (cioè una chiave privata).\
Senza questa informazione, l'inversione richiede la soluzione di un problema NP-hard, o comunque di un problema noto per cui non si conosce un algoritmo polinomiale.\

\subsubsection{Fattorizzazione}

Calcolare $n = p \times q$ è facile e richiede tempo quadratico nella lunghezza della loro rappresentazione.\
Invertire la funzione per trovare $p$ e $q$ a partire da $n$ (univocamente possibile solo se $p$ e $q$ sono primi) richiede tempo (sub)esponenziale.\
Per quanto noto fino a oggi, non è mai stato dimostrato che il problema è NP hard ma non è nemmeno mai stato dimostrato che il problema ammette un algoritmo risolutivo polinomiale (cosa da non escludere).\
La trap door in questo caso è uno qualsiasi dei due fattori $p$ e $q$.\

\subsubsection{Calcolo della radice in modulo}

Calcolare $y = x^z\ \mathit{mod}\ s$ con $x, z, s$ interi, richiede tempo polinomiale e si può fare con l'algoritmo delle esponenziazioni successive eseguendo $\Theta(\log_2 z)$ moltiplicazioni.\
Se $s$ non è primo, invertire la funzione e calcolare $x = y^{\frac{1}{z}}\ \mathit{mod}\ s$ richiede tempo esponenziale per quanto noto ad oggi.\
La trap door la vedremo successivamente.\

\subsubsection{Calcolo del logaritmo discreto}

Calcolare la potenza $y = x^z\ \mathit{mod}\ s$ è facile.\
Invertire rispetto a $z$, cioè trovare $z$ t.c.\ $y = x^z\ \mathit{mod}\ s$ dati $x$, $y$ e $s$ è computazionalmente difficile.\
Tutti gli algoritmi noti hanno la stessa complessità della fattorizzazione.\

\subsection{Vantaggi e svantaggi}

Vantaggi
\begin{itemize}
    \item Se ho $n$ utenti, il numero di chiavi del sistema sono $2n$ anziché $\frac{n(n - 1)}{2}$.
    \item Non è richiesto alcuno scambio di chiavi.
\end{itemize}
Svantaggi
\begin{itemize}
    \item Molto più lenti dei cifrari simmetrici.
    \item Sono esposti per natura ad attacchi di tipo \textit{chosen plain-text}.\
\end{itemize}

\subsubsection{Attacchi chosen plain-text}

Un crittoanalista può crearsi svariati crittogrammi $c = C(m, k_{\mathit{pub}})$ di un determinato destinatario e successivamente può mettersi in ascolto sul canale verso il quale sono diretti i crittogrammi per quel determinato destinatario.\
A questo punto il crittoanalista può confrontare i crittogrammi che passano sul canale con quelli che si era preparato in precedenza e, conoscendone il testo in chiaro, se ne trova due uguali ne conosce automaticamente la decifrazione.\
Qualora non trovasse alcuni dei suoi crittogrammi uguali a quelli passati sul canale, saprebbe comunque che il messaggio in chiaro è diverso da quelli da lui preparati.

\section{Cifrari ibridi}

Si usa un cifrario a chiave segreta (AES) per le comunicazioni di massa e un cifrario a chiave pubblica per scambiare le chiavi segrete relative al primo, senza incontri fisici tra gli utenti.\

La trasmissione dei messaggi lunghi avviene ad alta velocità, mentre lo scambio delle chiavi segrete è lento (sono composte al massimo da qualche decina di byte).\
L'attacco \textit{chosen plain-text} è risolto se l'informazione cifrata con la chiave pubblica (chiave segreta dell'AES) è scelta in modo da risultare imprevedibile al crittoanalista.\

La chiave pubblica deve essere estratta da un certificato digitale valido, per evitare attacchi \textit{man-in-the-middle}.\

\section{Misure di complessità deterministiche}

In questo capitolo studieremo brevemente come associare a una macchina di Turing una funzione che stimi il tempo che le è necessario per risolvere un caso $x \in I$ del problema $I$ che decide.\
Poi vedremo un paio di teoremi che mostrano come e di quanto questo tempo possa essere ridotto, al prezzo di usare macchine con un hardware più ``efficiente''.\
Preliminarmente introdurremo una variante ``parallela'' delle macchine di Turing, permettendo alla macchina di operare simultaneamente su molti nastri.\
Questa estensione naturalmente non modifica la classe dei problemi decidibili, tuttavia ci consente in alcuni casi un trattamento più agevole.\
Inoltre è un buon modello delle attuali macchine parallele sincrone (ma non di quelle concorrenti e distribuite e men che meno di quelle impiegate nel paradigma chiamato ``mobile computing''!), nella stessa misura in cui le macchine di Turing a un nastro lo sono delle macchine sequenziali mono-processore.\
Vedremo infine quale sia il prezzo in termini di tempo che si deve pagare per simulare una macchina a molti nastri su una a un nastro solo, dando così un preciso limite teorico ai vantaggi ottenibili dall'introduzione di calcolatori paralleli.\

Lo stesso schema verrà seguito per misure che riguardano lo spazio necessario al calcolo.\
Introdurremo un'ulteriore variante delle macchine di Turing in cui si identificano i nastri di lavoro (ignorando quelli destinati ai dati in ingresso e in uscita), in modo da definire lo spazio necessario per risolvere un problema.\
Anche in questo caso, la potenza espressiva non cambia.\
Infine, mostreremo che lo spazio può essere compresso in modo analogo a quanto fatto per ridurre il tempo.\

L'aggettivo \textit{deterministico} che compare nel titolo è legato al fatto che le macchine di Turing che impieghiamo usano una \textit{funzione} di transizione, come quelle usate nella prima parte del corso.\
Ciò garantisce che ogni passo di computazione è univocamente determinato dallo stato e dal simbolo correnti, in altri termini, la macchina è \textit{deterministica}.\
In seguito vedremo cosa succede impiegando \textit{relazioni} di transizione piuttosto che funzioni, il che rende le macchine di Turing \textit{non deterministiche}, in un senso che sarà precisato.\

\subsection{Macchine di Turing a k-nastri}

Ricordiamo che per le macchine di Turing introdotte nella definizione \ref{Macchina di Turing} si postula l'esistenza di un nastro semi-infinito e di una funzione di transizione $\delta$ che opera su di esso.\
Adesso arricchiamo l'hardware delle macchine di Turing fornendole di $k$ nastri.\
Poiché trattiamo solo problemi di decisione, ci prendiamo la libertà di spezzare lo stato di arresto $h$ in due nuovi stati di arresto \textit{\footnotesize SI}, \textit{\footnotesize NO}, a rappresentare che la macchina si ferma con successo nel primo caso e con insuccesso nel secondo.\
Formalmente:

\begin{definition} [MdT a $k$ nastri]
    \label{MdT_k-nastri}
    Dato un numero naturale $k$, una \textit{Macchina di Turing a k nastri} è una quadrupla $M = (Q, \Sigma, \delta, q_0)$, con
    \begin{itemize}
        \item $\#, \triangleright \in \Sigma$ e $L, R, - \notin \Sigma$
        \item \textit{\footnotesize SI}, \textit{\footnotesize NO} $\notin Q$
        \item$\delta : Q \times \Sigma^k \rightarrow Q \cup \{$\textit{\footnotesize SI}, \textit{\footnotesize NO}$\} \times (\Sigma \times \{L,R,-\})^k$ è la funzione di transizione, soggetta alle stesse condizioni della definizione \ref{Macchina di Turing} sull'unicità della stringa in $(\Sigma \times \{ L,R, -\})^k$, in modo che $\delta$ sia una funzione, e sull'uso del carattere di inizio stringa $\triangleright$.
    \end{itemize}
\end{definition}

\noindent La funzione di transizione $\delta$ per uno stato $q$ e $k$ simboli $\sigma_1, \dots, \sigma_k$ ha allora la forma seguente
\[\delta (q,\sigma_1, \dots,\sigma_k) = (q', (\sigma_1', D_1), (\sigma_2', D_2), \dots, (\sigma_k', D_k))\]
Una configurazione di una macchina a $k$ nastri ha la forma:
\[(q, u_1 \sigma_1 v_1, u_2 \sigma_2 v_2 , \dots, u_k \sigma_k v_k)\]
dove il carattere corrente sull'$i$-esimo nastro è $\sigma_i$, che abbiamo evitato di sottolineare per non appesantire la notazione; le sue computazioni lunghe $n$ saranno rappresentate da
\[(q, w_1 , w_2, \dots, w_k) \rightarrow^n (q', w_1', w_2', \dots, w_k')\]
le cui mosse, derivabili in accordo con la funzione di transizione $\delta$, hanno la forma
\[(q, u_1 \sigma_1 v_1, u_2 \sigma_2 v_2 , \dots, u_k \sigma_k v_k) \rightarrow (q, u_1 \sigma_1 v_1', u_2 \sigma_2 v_2', \dots, u_k\sigma_k v_k')\]
Si noti che se si volesse rappresentare una situazione in cui ci sono davvero $k$ processori che evolvono in sincronia, ciascuno con il suo carattere corrente e con il suo stato preso da un insieme $Q_i$, basterebbe definire l'insieme della macchina a $k$ nastri come $Q = Q_1 \times Q_2 \times \dots \times Q_k$ e interpretare nel modo ovvio la funzione di transizione in modo da tenerne conto.\

Vediamo adesso un esempio di macchina di Turing con due nastri.\

\begin{example}
    \label{ex_k-nastri}
    La seguente MdT con 2 nastri riconosce le stringhe palindrome costruite sull'alfabeto $\{a, b\}$.\
    La funzione di transizione può essere suddivisa in tre ``blocchi omogenei''.\
    Nel primo blocco ci sono le istruzioni che ricopiano la stringa in ingresso sul secondo nastro; nel secondo la testina del primo nastro vien portata sul simbolo di inizio nastro, mentre quella del secondo nastro è lasciata sul primo carattere \# dopo la stringa.\
    Il terzo blocco di istruzioni effettua il controllo vero e proprio, cancellando dal secondo nastro a partire da \textit{destra} i caratteri della stringa di ingresso se e solamente se corrispondono a quelli della stringa originale, incontrati muovendo il cursore del primo nastro da \textit{sinistra}.\
    Ovviamente, la stringa è palindroma se le due testine si trovano su caratteri sempre uguali e il secondo nastro viene svuotato.\
    \begin{table}[H]
        \centering
        \begin{tabular}{ |c c c|c c c| }
            \hline
            $q$   & $\sigma_1$       & $\sigma_2$       & \multicolumn{3}{c|}{$\delta(q,\sigma_1, \sigma_2)$}                                               \\\hline\hline
            $q_0$ & $\triangleright$ & $\triangleright$ & $q_0$                                               & $(\triangleright,R)$ & $(\triangleright,R)$ \\
            $q_0$ & $a$              & $\#$             & $q_0$                                               & $(a,R)$              & $(a,R)$              \\
            $q_0$ & $b$              & $\#$             & $q_0$                                               & $(b,R)$              & $(b,R)$              \\
            $q_0$ & $\#$             & $\#$             & $q_0$                                               & $(\#,L)$             & $(\#,-)$             \\
            \hline
            $q_1$ & $a$              & $\#$             & $q_1$                                               & $(a,L)$              & $(\#,-)$             \\
            $q_1$ & $b$              & $\#$             & $q_1$                                               & $(b,L)$              & $(\#,-)$             \\
            $q_1$ & $\triangleright$ & $\#$             & $q_2$                                               & $(\triangleright,R)$ & $(\#,L)$             \\
            \hline
            $q_2$ & $a$              & $a$              & $q_2$                                               & $(a,R)$              & $(\#,L)$             \\
            $q_2$ & $b$              & $b$              & $q_2$                                               & $(b,R)$              & $(\#,L)$             \\
            $q_2$ & $a$              & $b$              & \textit{\footnotesize NO}                           & $(a,R)$              & $(a,-)$              \\
            $q_2$ & $b$              & $a$              & \textit{\footnotesize NO}                           & $(b,R)$              & $(b,-)$              \\
            $q_2$ & $\#$             & $\triangleright$ & \textit{\footnotesize SI}                           & $(\#,-)$             & $(\triangleright,R)$ \\
            \hline
        \end{tabular}
    \end{table}

    \noindent Come esempio di calcolo applichiamo la macchina alla stringa \textit{abba}, raggruppando per quanto possibile i passi della computazione in blocchi omogenei.
    \item $(q_0,\underline{\triangleright} abba, \underline{\triangleright} \#) \rightarrow (q_0,\triangleright \underline{a}bba, \triangleright \underline{\#}) \rightarrow$
    \item\qquad $ (q_0,\triangleright a\underline{b}ba, \triangleright a \underline{\#}) \rightarrow^3 (q_0,\triangleright abba\underline{\#}, \triangleright abba \underline{\#})\rightarrow $
    \item
    \item $(q_1,\triangleright ab\underline{b}a, \triangleright abba\underline{\#}) \rightarrow (q_1, \triangleright abb\underline{a}\#, \triangleright abba \underline{\#})\rightarrow^4 (q_1,\underline{\triangleright} abba, \triangleright abba \underline{\#})$
    \item
    \item $(q_2,\triangleright \underline{a}bba, \triangleright abb\underline{a}\#) \rightarrow (q_2,\triangleright a\underline{b}ba, \triangleright ab\underline{b}\#) \rightarrow (q_2,\triangleright ab\underline{b}a, \triangleright a\underline{b}\#) \rightarrow$
    \item \qquad$(q_2,\triangleright abb\underline{a}, \triangleright \underline{a}\#) \rightarrow (q_2,\triangleright abba\underline{\#}, \underline{\triangleright}) \rightarrow (\mbox{\textit{\footnotesize SI}},\triangleright abba\underline{\#}, \triangleright \underline{\#})$
\end{example}

\subsection{Complessità in tempo deterministico}

Introduciamo adesso il modo che useremo per determinare il tempo necessario alla soluzione di un problema, ricordando che per problema qui intendiamo l'appartenenza o meno a un insieme, ovvero a un linguaggio di cui le macchine di Turing sono gli automi accettori; poiché le macchine usate sono \textit{deterministiche}, anche le misure che introdurremo sono tali e spesso ometteremo tale aggettivo, dandolo per inteso.

\begin{definition}
    Diciamo che $t$ è il \textit{tempo richiesto} da una MdT $M$ a $k$ nastri\footnote{Se volessimo considerare il tipo di macchine viste in precedenza, cioè se $H = \{h\}$, allora si potrebbe definire che il \textit{tempo richiesto} da $M$ su $x$ è $t$.\ Inoltre, questa definizione
        sarebbe accettabile anche per la complessità di problemi \textit{tout court} e non solo di problemi decidibili come facciamo, ponendo $\infty$ il tempo richiesto, se $M(x) \uparrow$.} per \textit{decidere} il caso $x \in I$ se
    \[(q_0, \underline{\triangleright}x, \underline{\triangleright}, \dots, \underline{\triangleright}) \rightarrow^t (H, w_1,w_2, \dots, w_k),\ \mathrm{con}\ H \in \{\mbox{\textit{\footnotesize SI}, \textit{\footnotesize NO}}\}\]
\end{definition}

\noindent In realtà vorremmo ottenere una misura del tempo necessario a risolvere un problema mediante la macchina $M$ come una funzione della taglia dei suoi possibili dati di ingresso $x$; cioè, indicando la taglia di $x$ con $|x|$, vorremmo una funzione $f(|x|)$.\
La definizione della taglia dei dati è arbitraria, ma spesso è molto naturale.\
Per esempio, spesso la taglia di un grafo è il numero dei suoi nodi (e/o dei suoi archi), quella di una stringa o di un vettore la sua lunghezza, indipendentemente dagli elementi costitutivi.\
La funzione \textit{taglia} deve essere ovviamente calcolabile totale e \textit{facile}; in ogni caso deve restituire un numero naturale.\
In queste note useremo funzioni di taglia spesso senza definirle e nel modo che ci sarà più conveniente.\
Nella maggior parte dei casi e senza avvertenza contraria, misureremo i dati di ingresso $x$ in relazione alle caselle della MdT necessarie a contenerli.\

Nel seguito, non pretenderemo che la funzione $f(|x|)$ dia il numero \textit{esatto} di passi necessari al calcolo di $M(x)$, perché ciò potrebbe rivelarsi troppo complicato; ci contenteremo allora di approssimare tale numero per eccesso:\ la macchina non richiederà un tempo maggiore di quello stimato.\
In conclusione, la funzione che determina la complessità di $M$ è una funzione calcolabile totale $f : N \rightarrow N$, la quale limita superiormente il numero dei passi che $M$ compie per risolvere il problema in questione --- torneremo sulle caratteristiche delle funzioni di misura nella definizione \ref{def_appropriata}.\
Si noti che questo non contrasta con la richiesta di avere classi di complessità, indotte da funzioni di misura, che esprimono la quantità \textit{minima} di risorse necessarie alla decisione dei problemi in esse contenuti:\ basta trovare la minima funzione che limita superiormente i passi di $M$.\
L'aggettivo \textit{deterministico} che compare nella definizione dipende dal fatto che le macchine di Turing usate sono deterministiche, nel senso che sarà più chiaro dopo la definizione \ref{MdT_non-det} (la componente $\delta$ è una funzione e non una relazione); quando sarà chiaro dal contesto che parliamo di questo tipo di macchine, ometteremo tale aggettivo.\

\begin{definition}
    $M$ \textit{decide} $I$ \textit{in tempo deterministico} $f$ se per ogni dato di ingresso $x \in I$ il tempo $t$ richiesto da $M$ per decidere $x$ è $\leq f(|x|)$.
\end{definition}

\noindent Adesso possiamo introdurre il concetto di \textit{classe di complessità} in tempo \textit{deterministico}.\

\begin{definition} [Classe di complessità in tempo deterministico]
    \[\mathrm{TIME}(f) = \{ I \mid \exists M\ \mathrm{che\ decide}\ I\ \mathrm{in\ tempo\ deterministico}\ f\}\]
\end{definition}

\noindent La classe appena introdotta contiene tutti e soli i problemi risolvibili in tempo deterministico $f$, ovvero affinché un problema vi appartenga occorre e basta che vi sia una macchina $M$ che lo decide in tempo deterministico $f$.\

Prendiamo ad esempio la macchina $M$ dell'esempio \ref{ex_k-nastri} e calcoliamo la sua complessità in tempo, suppondendo che la stringa di ingresso sia lunga $n$.\
Come abbiamo visto dianzi, il funzionamento della macchina può essere suddiviso in tre blocchi di operazioni ``omogenee'':
\begin{enumerate}
    \itemsep0px
    \item copia il dato in ingresso sul secondo nastro in \hfill $n+1$ passi
    \item rimette la prima testina sul $\triangleright$ in \hfill $n+1$ passi
    \item sposta le due testine se i simboli sono uguali in \hfill $\underline{n+1}$ passi
    \item[]\hfill $3n + 3$ passi
\end{enumerate}
cui va aggiunto il passo per l'accettazione.\
Quindi il problema di verificare se una stringa è palindroma appartiene a TIME$(3n + 4)$ o, scordandosi le costanti, è dell'ordine di $n$.\

\medskip
\noindent \textbf{NOTAZIONE}\quad Nella discussione precedente abbiamo menzionato l'ordine di una funzione.\
Poiché questo concetto viene ripetutamente usato in seguito, in quanto in questa porzione di teoria della complessità si preferisce ignorare le costanti (ne discuteremo più avanti le ragioni), vale la pena di introdurre esplicitamente la seguente abbreviazione, dove ``quasi ovunque'' significa per ogni argomento, eccetto che per un insieme finito di essi:
\[\mathcal{O}(f) = \{g \mid \exists r \in \mathbb{R}^+.\ g(n) < r \times f (n)\ \mathrm{quasi\ ovunque}\}\]
a indicare che la funzione $f$ cresce allo stesso modo o più velocemente delle funzioni $g$ appartenenti alla classe $\mathcal{O}(f)$, la quale viene quindi chiamata \textit{ordine} di $f$.\footnote{Di solito si introducono anche le classi
    \begin{itemize}
        \itemsep0px
        \item[] $\Omega(f) = \{g \mid f \in \mathcal{O}(g)\}$ --- $f$ cresce più lentamente di $g$ e
        \item[] $\Theta(f) = \mathcal{O}(f) \cap \Omega(f)$ --- $f$ cresce come $g$.
    \end{itemize}}
(Inutile notare che ``più velocemente'' dipende solo dal fattore moltiplicativo $r$.)\

\medskip

\noindent In effetti, il calcolo di complessità fatto sopra non tiene conto del fatto che, se la stringa non è palindroma, il numero di passi necessari è minore.\
Abbiamo infatti definito una misura della complessità nel \textit{caso pessimo}.\
Ci sono altri modi per misurare la complessità che tengono conto della distribuzione dei dati di ingresso.\
Un esempio particolarmente rilevante è quello della complessità nel \textit{caso medio}, che tuttavia non tratteremo in queste note.

Può essere interessante confrontare adesso, dal punto di vista della complessità, le macchine che decidono se una stringa è palindroma nelle loro versioni ``parallela'', appena vista, e ``sequenziale''.\
Il tempo delle computazioni di quest'ultima è in $\mathcal{O}(n^2)$, perché servono $n$ passi per controllare se il primo simbolo è uguale all'ultimo e questo controllo va ripetuto per $n/2$ volte.\
Una prima osservazione che possiamo fare è che il problema di decidere se una stringa è palindroma sta \textit{anche} in $\mathcal{O}(n^2)$, il che non sorprende perché abbiamo appena visto che sta in $\mathcal{O}(n)$ e se una funzione è superiormente dominata da $g \in \mathcal{O}(n)$ lo è a maggior ragione da $h \in \mathcal{O}(n^2)$ --- se avendo poco tempo a disposizione risolvo un problema, lo risolverò a maggior ragione se ne avessi di più.\

Un'altra osservazione, forse più interessante, è che, usando una macchina parallela, abbiamo ``guadagnato'' tempo in modo quadratico (meglio:\ perduto da parallelo a sequenziale).\
Questo è un fatto vero in generale.

\begin{theorem} [Riduzione del numero di nastri]
    \label{riduzione_nastri}
    Data una macchina di Turing $M$ con $k$ nastri che decide $I$ in tempo deterministico $f$, allora $\exists M'$ con un solo nastro che decide $I$ in tempo deterministico $\mathcal{O}(f^2)$.
\end{theorem}

\begin{proof}
    Riportiamo solo una traccia della dimostrazione, confidando nella diligenza dei lettori che vorranno certamente precisare i passi appena accennati.\
    Costruiamo $M'$ in modo che simuli la data $M$, in modo analogo a quanto fatto nella costruzione della macchina di Turing universale.\
    Ogni configurazione di $M$ della forma
    \[(q, \triangleright w_1 \sigma_1 u_1, \triangleright w_2 \sigma_2 u_2, \dots, \triangleright w_k \sigma_k u_k)\]
    viene simulata da:
    \[(q', \triangleright \triangleright' w_1 \overline{\sigma}_1 u_1 \triangleleft' \triangleright' w_2 \overline{\sigma}_2 u_2 \triangleleft' \dots  \triangleright' w_k \overline{\sigma}_k u_k \triangleleft')\]
    per qualche $q'$, cioè racchiudiamo ciascun nastro $w_i \sigma_i u_i$ tra due nuove parentesi $\triangleright'$ e $\triangleleft'$ e usiamo $\#\Sigma$ nuovi simboli $\overline{\sigma}_i$ per ricordarci la posizione della testina sull'$i$-esimo nastro.\

    Per cominciare, la macchina $M'$ applicata a $x$ dovrà generare la configurazione che simula la configurazione iniziale di $M$, cioè dobbiamo passare dalla configurazione iniziale di $M$ $(q_0, \underline{\triangleright} x, \triangleright, \dots, \triangleright)$ a
    \[(q', \triangleright \triangleright' x \triangleleft' (\triangleright' \triangleleft')^{k-1})\]
    per qualche $q'$.\
    Per far ciò, bastano $2k + \#\Sigma$ nuovi stati\footnote{Supponendo che il carattere \# non appaia in $x$, un modo per farlo è il seguente:\
        arrivare al primo carattere \# (il che richiede $|x| + 1$ passi e un nuovo stato); cambiare
        stato; tornare indietro di una casella e ricordarsi, codificandolo in un nuovo stato il simbolo ($\neq \#$) corrente, sia $a$, scriverci \#, spostarsi a destra e scrivere $a$; poi bisogna ripetere le ultime due mosse per $|x| - 1$ volte.\ In questo modo abbiamo ``spostato'' $x$ di una casella a destra, impiegando $2 \times |x|$ passi e $2 + \#\Sigma$ nuovi stati.\ A questo punto si scrive sulla casella corrente, che è vuota la parentesi $\triangleright'$; si torna sulla prima casella vuota muovendosi a destra e si scrivono $k - 1$ coppie $\triangleright' \triangleleft'$, usando altri $2 \times (k - 1)$ nuovi stati.} e un certo numero di passi che, essendo dell'ordine di $|x|$, non influenza la complessità asintotica (consideriamo infatti il caso pessimo, quindi tutto il dato iniziale va letto).\

    Per simulare una mossa di $M$, la macchina $M'$ scorre l'intero nastro da sinistra a destra e \textit{viceversa} due volte:
    \begin{itemize}
        \item la prima volta $M'$ determina quali sono i simboli correnti di $M$, $\overline{\sigma}_i$ (si noti che, per ricordare quale sia la stringa $\overline{\sigma}_1 \dots \overline{\sigma}_k$ sono sufficienti $(\#\Sigma)^k$ nuovi stati);
        \item la seconda volta $M'$ scrive i nuovi simboli nel posto giusto --- attenzione! se un $\triangleleft'$ deve essere spostato a destra, per far posto a un nuovo simbolo da scrivere, si verifica una cascata di spostamenti a destra!
    \end{itemize}
    Infine quando $M$ si ferma, anche $M'$ si ferma, eventualmente rimuovendo tutte le parentesi $\triangleright'$ e $\triangleleft'$ e sostituendo i caratteri $\overline{\sigma}_i$ con $\sigma_i$.\

    Adesso ricordiamo un fatto generale e ovvio:\ una macchina non può toccare un numero di caselle maggiore del numero dei passi che compie.\
    Di conseguenza, la lunghezza totale del nastro scritto è al più $K = k \times (f(n) + 2) +1$ (l'addendo 2 è dovuto alle parentesi $\triangleright'$ e $\triangleleft'$, l'addendo 1 al simbolo $\triangleright$).\
    Allora, andare due volte avanti e indietro costa, in termini di tempo, per ogni stringa simulata $4K$ più al massimo $3K$ per gli spostamenti a destra, nel caso in cui la casella corrente sia all'estrema sinistra (è il caso pessimo nel quale ci poniamo sempre):\ $K$ per arrivare alla fine del nastro scritto e $2K$ per spostare a destra i $K$ caratteri, come descritto nella nota precedente.\
    Poiché né $k$ né le altri costanti sono rilevanti, possiamo concludere che per simulare un \textit{singolo} passo di $M$ la macchina $M'$ richiede $\mathcal{O}(f|x|)$ passi sul dato $x$.\
    Il numero di passi di $M'$ sull'intera computazione è quindi in $\mathcal{O}(f(|x|)^2)$, perché $M$ richiede tempo $f(|x|)$ e perché $M'$ impiega $\mathcal{O}(f(|x|))$ per simulare ogni passo di $M$.\
    Infine, per costruzione $M'$ è equivalente a $M$, e quindi le due macchine decidono lo stesso problema; allora $M'$, che ha un nastro solo, decide tale problema in tempo deterministico $\mathcal{O}(f(|x|)^2)$.\
\end{proof}

\noindent Il teorema appena dimostrato mostra che le MdT sono molto stabili!\
Infatti, miglioramenti che siano accettabili ``algoritmicamente'', come aggiungere nastri e processori che operano in parallelo, non solo non cambiano le funzioni calcolate, come ci aspettavamo, ma non modificano il tempo deterministico richiesto se \textit{non polinomialmente}, quindi le MdT appaiono stabili anche rispetto la tesi di Cook-Karp (ancora da vedere con precisione!).\

Ribadiamo adesso l'osservazione fatta nella dimostrazione di sopra, che mette in relazione il tempo e lo spazio necessari alla soluzione di un problema.\
Se una macchina di Turing $M$ richiede tempo $f(|x|)$ per decidere $x \in I$, significa che si arresta in un numero di passi inferiore a $f(|x|)$, e quindi non può aver visitato, in alcuno dei suoi nastri, un numero di caselle maggiore di $f(|x|)$.\
Abbiamo quindi il seguente fatto basilare:

\begin{center}
    \textbf{Osservazione 1}:\ non si può usare più spazio che tempo!
\end{center}

\noindent Usiamo ancora la dimostrazione appena riportata per dedurre un ulteriore osservazione.\
Infatti, il ragionamento fatto ci suggerisce come misurare, beninteso solo da un punto di vista teorico, i vantaggi che derivano dall'introduzione di macchine parallele.\
Negli ultimi passi della dimostrazione abbiamo potuto ottenere l'ordine $\mathcal{O}(f(|x|)^2)$ eliminando il fattore $k^2$, perché $k$ è indipendente da $x$.\
Però l'elevamento al quadrato del fattore $f(|x|)$ non potrà mai essere eliminato!\
Quindi possiamo dedurre una stima del vantaggio che deriva dall'uso del parallelismo.

\begin{corollario}
    Le macchine parallele sono polinomialmente più veloci di quelle sequenziali.
\end{corollario}

\noindent Finora, nei nostri conti abbiamo impiegato solo gli ordini di crescita \textit{trascurando} le costanti.\
Ovviamente, quando si cerchino stime più precise, le costanti contano terribilmente, tanto che l'astuzia dei progettisti di algoritmi si dispiega spesso proprio nello scoprire come ridurle.\
Ciò nonostante, continueremo nel seguito a trascurare le costanti, a meno di casi particolari in cui esse saranno menzionate espressamente.\
Due sono le ragioni:
\begin{itemize}
    \item[i)] la teoria che si sviluppa è molto più semplice; inoltre per valori grandi di $n$, cioè per dati di grandi dimensioni, le costanti tendono a valere ``poco'';
    \item[ii)] macchine sempre più potenti tendono a far rimpicciolire le costanti.
\end{itemize}

\noindent L'ultima osservazione è sostenuta da un teorema che riportiamo qui sotto, detto di accelerazione lineare.\
L'idea è che se $I \in \mathrm{TIME}(f)$, (ovvero se esiste una MdT $M$ che lo risolve in tempo deterministico $f(n)$) allora $I$ appartiene anche a $\mathrm{TIME}(\epsilon \times f)$, qualunque sia la scelta per $\epsilon > 0$ (attenzione:\ poiché la nostra complessità è nel caso pessimo, quanto detto è impreciso e ci sarà bisogno di una correzione per mantenere lineare la misura del tempo).\
In altre parole, dato un algoritmo che decide un problema, se ne può sempre trovare uno equivalente che è più veloce per una costante moltiplicativa $\epsilon$ (supponendo ovviamente che questa sia minore di 1).\
Attenzione però:\ se p.e.\ $I \in \mathrm{TIME}(2^n)$, ovvero se il problema $I$ è deciso da un algoritmo in tempo esponenziale, non è possibile trovare un algoritmo che lo risolva in tempo deterministico \textit{polinomiale}, per mezzo del \textit{solo} teorema di accelerazione.\
Un'analoga osservazione vale ovviamente per lo spazio, come vedremo.\
Quindi il teorema non inficia una eventuale gerarchia, ancora da stabilire.

Il trucco fondamentale è quello di codificare l'alfabeto $\Sigma$ in un alfabeto ``più ricco'' $\Sigma^m$, con $m$ arbitrario.\
In pratica, questo significa avere macchine con parole di dimensioni via via crescenti (32, 64, 128, \dots, $2^m$ bit).\
Si vede quindi che l'accelerazione è legata al \textit{cambio di hardware}.

Questo non è chiaramente del tutto fattibile in pratica, e mostra che le costanti \textit{sono} importanti quando la macchina sia fissata; inoltre, non si può aumentare a piacere l'efficienza di un tuo programma cambiando semplicemente l'hardware delle macchine!

Nell'enunciato del teorema seguente compare un addendo $n + 2$ il quale dipende unicamente dal tipo di MdT usata (a 1 o a $k$ nastri, con nastro semi-infinito o infinito, ecc.).\
La presenza dell'addendo $n$ garantisce che, anche se la $f(n)$ fosse lineare, la complessità risultante rimarrebbe tale.\
In enunciati diversi del teorema si possono trovare diversi addendi, che variano in funzione dei vari tipi delle MdT; in tutti i casi essi garantiscono che il risultato sia una funzione almeno lineare.\
Si noti anche che in questa dimostrazione e in altre che seguiranno, si misura la taglia del dato con il numero di caselle del nastro di ingresso che servono a memorizzarlo.\

\begin{theorem} [Accelerazione lineare MdT]
    \hfill

    Se $I \in \mathrm{TIME}(f)$, allora $\forall \epsilon < 1 \in \mathbb{R}^+$ si ha che $I \in \mathrm{TIME}( \epsilon \times f(n) + n + 2)$.
\end{theorem}
\begin{proof}
    Omessa, perché lunga e piena di dettagli insidiosi:\ si tratta di simulare una data macchina $M$ con una macchina $M'$, sulla falsariga di quanto fatto nella costruzione della macchina universale o nella dimostrazione \ref{riduzione_nastri}.\
    Può essere tuttavia interessante notare un fatto che potrebbe guidare il lettore a una maggiore comprensione del teorema stesso e dell'uso che si può fare degli stati per ``ricordare'' porzioni di nastro.\
    Quello che faremo è vedere che si può determinare il numero $m$ di simboli di $M$ da compattare in un unico simbolo di $M'$ in funzione del \textit{solo} $\epsilon$.\

    Il primo passo ``condensa il dato di ingresso'' (in $n + 2$ passi, con $n = |x| \leq m \times \left\lceil \frac{|x|}{m}\right\rceil + 2)$:\ ogni sequenza di $m$ simboli di $M$ origina un singolo simbolo di $M'$, cioè $\sigma_{i_1} \dots \sigma_{i_m}$ viene codificata come il singolo simbolo $[\sigma_{i_1} \dots \sigma_{i_m}]$ (si noti che non c'è alcun problema se esiste $m' > 1$ tale che $\sigma_{k_h} = \#$ per $h \geq m' > 1$).\
    In maniera analoga, gli stati di $M'$ saranno formati da triple $[q, \sigma_{i_1} \dots \sigma_{i_m}, k]$, con $1 \leq k \leq m$, in modo da ``rappresentare'' il fatto che $M$ si trova nello stato $q$ e ha il cursore sul $k$-esimo simbolo della stringa $\sigma_{i_1} \dots \sigma_{i_m}$.\
    Data una configurazione, alla macchina $M'$ bastano 6 passi per simularne $m$ della macchina $M$.\
    Nei primi 4 passi $M'$ va a sinistra, poi a destra, poi ancora a destra e infine ritorna sul carattere corrente $s = \sigma_{i_1} \dots \sigma_{i_m}$, in modo da raccogliere i simboli che $M$ potrebbe visitare e codificarli nel suo stato.\
    Infatti, $M$ con $m$ mosse può spostare il suo cursore all'interno della stringa di $m$ caratteri che si trova a sinistra del carattere corrente, o di quella a destra o lasciarlo all'interno della stringa $s$ considerata.\
    Quando $M$ compie $m$ mosse, $M'$ le simula ``a blocchi'' muovendosi a sinistra, oppure a destra del simbolo corrente $s$, ma in ogni caso ne modifica solo due, incluso $s$ -- le altre 2 mosse.\
    Basta quindi ``prevedere'' il risultato di ciascun blocco di 6 transizioni di $M'$, che dipende \textit{solo} dalla funzione di transizione di $M$ e non dal tipo di mosse fatte e men che meno dalla taglia del dato di ingresso.\
    Allora $M'$ farà $|x| + 2 + 6 \times \left\lceil \frac{f(|x|)}{m}\right\rceil$ passi e la traccia della dimostrazione si conclude scegliendo $m$ in modo tale che $m = \left\lceil \frac{6}{\epsilon} \right\rceil$.\

\end{proof}

\noindent Prima di introdurre una delle classi di complessità più importanti, quella dei problemi decidibili in tempo polinomiale deterministico, usiamo i teoremi precedenti, per fare alcune osservazioni che giustificano ulteriormente la scelta fatta di usare solo ordini di grandezza trascurando le costanti.

Preliminarmente, notiamo che vi sono misure di complessità in tempo che sono sub-lineari, cioè vi sono macchine che richiedono un tempo $f(n) < n$ per risolvere un problema di taglia $n$.\
Per esempio, la ricerca di una parola in un dizionario effettuata con un metodo dicotomico porta a leggere $\log n$ parole, certo non tutte quelle contenute nel dizionario stesso (ma tale misura è invariante rispetto al cambiamento di rappresentazione dei dati o si basa proprio su una caratteristica specifica della rappresentazione?).\
Non considereremo nel seguito misure in tempo sub-lineari, perché per ipotesi vogliamo ottenere la complessità nel caso pessimo, e quindi le macchine leggono sempre l'intero dato di ingresso $x$, il che richiede appunto $n = |x|$ passi e quindi ogni funzione di complessità in tempo $f$ è tale che $f(n) \geq n$.\

Adesso supponiamo di avere una funzione $f$.\
Se $f(n) = c \times n$ (cioè $f$ è lineare), allora il teorema di accelerazione ci consente di ``rimpicciolire'' la costante fino a renderla uguale 1, ponendo $\epsilon = \frac{1}{c}$.\
Se invece $f(n) = c_1n^k + c_2n^{k-1} + \dots + c_k$ (cioè è un polinomio), ancora una volta il teorema ci dice che possiamo rendere $c_1$ uguale 1 e inoltre gli addendi con esponente minori di $k$ si possono trascurare perché quello di grado massimo li domina per $n$ sufficientemente grande:\ ecco allora giustificato l'uso di $\mathcal{O}\left(n^k\right)$.\
Infine, quanto detto sopra ci porta a concludere che, se $I$ è decidibile polinomialmente, allora esiste un $k$ tale che $I \in \mathcal{O}\left(n^k\right)$.\
Analoghe considerazioni si possono applicare quando la funzione $f$ considerata maggiori ogni polinomio, per esempio sia una funzione esponenziale.

Quanto osservato sopra basta per introdurre la classe dei problemi decidibili in tempo polinomiale deterministico, ovvero di quei problemi per cui esiste una macchina deterministica che li decide in tempo deterministico limitato da un polinomio.

\begin{definition}
    La classe dei problemi decidibili (da MdT) in tempo polinomiale deterministico è
    \[\mathcal{P} = \bigcup_{k \geq 1} \mathrm{TIME}\left(n^k\right)\]
\end{definition}

\noindent Il prossimo passo dovrebbe essere quello di dimostrare che la classe $\mathcal{P}$ appena introdotta è invariante rispetto al cambio di modelli.\
Dopo aver fatto questo, potremmo anche eliminare nella definizione precedente il riferimento alle macchine di Turing, ciò che implicitamente abbiamo già fatto.\
Tuttavia la mancanza di tempo ci impedisce di affrontare il problema della robustezza delle classi di complessità con la dovuta attenzione.\
Ci limiteremo allora a ritornare di sfuggita su questo punto più avanti, affermando senza dimostrarlo che si può passare \textit{in tempo polinomiale} da un algoritmo rappresentato in un modello a uno equivalente rappresentato in un altro modello.\
In altre parole, la classe $\mathcal{P}$ è chiusa rispetto a trasformazioni di modelli, il che ne garantisce la robustezza.\

Tuttavia, come accennato in precedenza, bisogna far molta attenzione al modo in cui si misura il tempo necessario a risolvere un problema.

\subsection{Macchine di Turing I/O}

Per studiare la complessità in spazio è conveniente usare un'ulteriore ragionevole e ben motivata variante di macchine di Turing, quelle che hanno un nastro dedicato a contenere il dato di ingresso, che sarà di sola lettura, uno destinato a memorizzare il risultato, che sarà di sola scrittura, e $k - 2$ nastri di lavoro, gli unici rilevanti ai fini della complessità.\

\begin{definition}
    \label{MdT_I/O}
    Una MdT con $k$ nastri $M= (Q,\Sigma, \delta, q_0)$ è di tipo I/O se e solamente se la funzione di transizione $\delta$ è tale che, tutte le volte che $\delta(q, \sigma_1, \dots, \sigma_k) = (q', (\sigma_1', D_1), \dots, (\sigma_k', D_k))$
    \begin{itemize}
        \item $\sigma_1'=\sigma_1$ --- quindi il primo nastro è a sola lettura;
        \item o $D_k = R$ o, quando $D_k = -, \sigma_k' = \sigma_k$ --- quindi il $k$-esimo nastro è a sola scrittura;
        \item se $\sigma_1 = \#$ allora $D_1 \in \{L,-\}$ --- la macchina visita al massimo una cella bianca a destra del dato di ingresso (che ipotizziamo non contenere \# al suo interno, o che abbia una marca di fine stringa, v.\ dopo).
    \end{itemize}
\end{definition}

\noindent Si noti che nulla cambia nella definizione di funzione calcolata, né rispetto alle macchine di Turing usate nella prima parte, né rispetto quelle con $k$ nastri.\
Inoltre, le relazioni delle macchine con $k$ nastri di tipo I/O con quelle non di tipo I/O sono facili da stabilire.\

\begin{property}
    Per ogni MdT a $k$ nastri $M$ che decide $I$ in tempo deterministico $f$, esiste una MdT a $k+2$ nastri $M'$ di tipo I/O che decide $I$ in tempo deterministico $c \times f$, per qualche costante $c$.
\end{property}

\begin{proof}
    La macchina $M'$ copia il primo nastro di $M$ sul proprio secondo nastro, impiegando $|x| + 1$ passi; opera come $M$ senza più toccare il proprio primo nastro; e quando $M$ si arresta, $M'$ si arresta dopo aver copiato il risultato sul proprio $k + 2$-esimo nastro, in al più $f(|x|)$ passi.\
    In totale, la macchina $M'$ ha richiesto su $x$ un numero di passi inferiore o uguale a $2 \times f (|x|) + |x| + 1$.\
    Determinare la costante $c$ è ora immediato.\
\end{proof}

\noindent Infine, per stabilire l'equivalenza tra le MdT a $k$ nastri di tipo I/O o le MdT usate nella prima parte, basta ricorrere alla simulazione vista nel teorema \ref{riduzione_nastri}.\
Quindi, ancora una volta le macchine di Turing si dimostrano estremamente robuste:\ modifiche algoritmicamente ``ragionevoli'' non ne alterano il potere espressivo.

\subsection{Complessità in spazio deterministico}

Al fine di avere una nozione di misura di spazio sensata, modifichiamo la definizione di configurazione in modo da ricordare \textit{tutte} le celle visitate, incluse quelle che erano o sono diventate bianche.\
A esser pignoli, questa modifica richiederebbe l'introduzione di un nuovo simbolo ausiliario, per esempio $\triangleleft \notin \Sigma$, da usare su ciascun nastro come delimitatore destro della parte scritta, e una semplice modifica alla funzione di transizione perché ne tenga conto e lo sposti, \textit{ma solo a destra}, quando necessario.\
Per esempio, prendiamo la macchina ``parallela'' per decidere se una stringa è palindroma ed estendiamola con un nastro di ingresso e uno di uscita.\
Alcuni passi della computazione su $aba$ verranno allora rappresentati così, dove $q_i, q_j , q_k$ e $q_h$ sono stati opportuni:
\begin{itemize}
    \itemsep0px
    \item[] $(q_0, \underline{\triangleright} aba \triangleleft, \triangleright \triangleleft) \rightarrow^* (q_i, \triangleright aba \underline{\triangleleft}, \triangleright aba \underline{\triangleleft}) \rightarrow^*$
    \item[] $(q_j, \underline{\triangleright} aba \triangleleft, \triangleright aba \underline{\triangleleft}) \rightarrow (q_k, \triangleright \underline{a}ba \triangleleft, \triangleright ab\underline{a} \triangleleft) \rightarrow (q_h, \triangleright a\underline{b}a \triangleleft, \triangleright a\underline{b}\# \triangleleft)$
\end{itemize}

\noindent Ci sentiamo liberi di non apportare queste modifiche e di immaginare che, una volta toccata, una casella del nastro apparirà sempre nella rappresentazione del nastro.\
Quindi in una configurazione $(q, u_1 \sigma_1 v_1 , \dots, u_k \sigma_k v_k )$, $u_i$ comincia per $\triangleright$ e $v_i$ può finire con (molti) \#, tutti quelli su cui l'$i$-mo cursore è venuto a posizionarsi durante il calcolo.\
In questo modo il numero delle caselle in uso nei nastri non diminuisce mai, né

\begin{itemize}
    \item nel nastro di ingresso, il primo, perché è di sola lettura;
    \item nel nastro di uscita, il $k$-esimo, perché è di sola scrittura;
    \item nei nastri di lavoro $1 < i < k$, perché i caratteri bianchi a destra non scompaiono mai.
\end{itemize}

\noindent Adesso siamo pronti per definire lo spazio necessario a una computazione come il numero totale delle caselle toccate solamente sui nastri di lavoro.\
Come per il tempo, anche qui si parla di spazio deterministico perché usiamo macchine di Turing deterministiche.\

\begin{definition}
    Sia $M$ una MdT a $k$ nastri di tipo I/O tale per cui $\forall x$
    \[(q_0, \underline{\triangleright} x, \underline{\triangleright}, \dots, \underline{\triangleright}) \rightarrow^* (H, w_1,w_2, \dots, w_k) \text{ con } H \in \{\mbox{\textit{\footnotesize SI, NO}}\} \]
    Lo \textit{spazio richiesto} da $M$ per decidere $x$ è
    \[\sum^{k-1}_{i=2} |w_i|\]
    Inoltre $M$ \textit{decide} $I$ \textit{in spazio deterministico} $f(n)$ se $\forall x$ lo spazio richiesto da $M$ per decidere $x$ è minore o uguale a $f(x)$.\
    Infine, se $M$ decide $I$ in spazio deterministico $f(n)$, allora $I \in \mathrm{SPACE}(f(n))$.
\end{definition}

\noindent Ovviamente, la definizione appena vista può essere immediatamente adattata nel caso generale, in cui si consideri anche lo stato di arresto $h$ come elemento dell'insieme $H$.

Alcuni autori preferiscono definire lo spazio richiesto come
\[\max_{2 \leq i \leq k-1} |w_i|\]
ma la sola differenza con la nostra definizione è un fattore $k$, il quale viene trascurato quando si considerano solamente ordini di grandezza.\

\medskip
\noindent La ragione principale per cui si trascura lo spazio necessario a contenere i dati di ingresso e quelli di uscita è che si vogliono misure abbastanza fini per la complessità in spazio.\
Infatti, se uno considerasse sempre anche la dimensione dei dati di ingresso, cioè la sommatoria di sopra partisse da 1 anziché da 2, si avrebbero sempre complessità almeno lineari.\
Ciò perché la misura del primo nastro è proprio $|x| + 1$, in quanto tale nastro è di sola lettura e contiene la stringa $w_1 = \triangleright x$.\
La dimensione del nastro d'uscita non è rilevante nel caso di problemi di decisione $I$ considerati:\ il risultato è semplicemente un segnale che il caso $x \in I$ è risolto positivamente o meno.\
Inoltre, a differenza di quanto accade per le classi di complessità in tempo, ci sono delle classi interessanti e importanti che sono sub-lineari e che rivestono un ruolo rilevante nella trattazione successiva, per esempio, $\mathrm{LOGSPACE}(n)$, la classe dei problemi decisi in spazio deterministico logaritmico che definiamo più precisamente qui sotto (la quale potrebbe coincidere con $\mathcal{P}$:\ si veda il frammento di gerarchia introdotto a pagina 61).\
In questo caso, sommare anche lo spazio per il dato di ingresso porterebbe a trascurare l'addendo $\log n$ che cresce meno di $n$ e a schiacciare così $\mathrm{LOGSPACE}(n)$ su $\mathrm{PSPACE}(n)$.\

\medskip
\noindent Passiamo ora a vedere che anche lo spazio è suscettibile di essere ridotto linearmente, come già visto per il tempo.\
Si noti che un teorema analogo a quello di riduzione dei nastri sarebbe banale:\ avremmo ancora la stessa misura, dopo aver ricopiato fianco a fianco i $k$ nastri di lavoro su un solo nastro.\

\begin{theorem}[Compressione Lineare in Spazio]
    \hfill

    Se $I \in \mathrm{SPACE}(f(n))$ allora $\forall \epsilon \in \mathbb{R}^+.\ I \in \mathrm{SPACE}(2 + \epsilon \times f(n))$
\end{theorem}

\noindent Come fatto precedentemente per $\mathcal{P}$, i problemi decidibili in tempo polinomiale deterministico, introduciamo ora la classe dei problemi decidibili in spazio deterministico polinomiale; poi, come promesso, definiremo quella dei problemi decidibili in spazio deterministico logaritmico, che giocheranno un ruolo importante nel capitolo 2.5.

\begin{definition}
    La classe dei problemi decidibili (da MdT) in spazio polinomiale deterministico è
    \[\mathrm{PSPACE} = \bigcup_{k \geq 1} \mathrm{SPACE}\left(n^k\right)\]
    La classe dei problemi decidibili (da MdT) in spazio logaritmico deterministico è
    \[\mathrm{LOGSPACE} = \bigcup_{k \geq 1} \mathrm{SPACE}(k \times \log n)\]
\end{definition}

\noindent Per fortuna anche queste classi sono invarianti rispetto al cambio di modelli, e quindi come già fatto per $\mathcal{P}$ possiamo eliminare nella definizione precedente il riferimento alle macchine di Turing.\
In altre parole, le classi PSPACE e LOGSPACE sono chiuse rispetto a trasformazioni di modelli, il che garantisce la loro robustezza, oltre a quella della teoria che stiamo passando in rassegna.\

Concludiamo questo capitolo stabilendo alcune relazioni tra le classi di complessità appena introdotte e facendo un'ulteriore osservazione su come tempo e spazio siano correlati.\
Iniziamo enunciando senza dimostrazione il teorema che stabilisce la relazione di stretta contenenza tra le due classi in spazio appena viste:

\begin{theorem}
    $\mathrm{LOGSPACE} \subsetneq \mathrm{PSPACE}$
\end{theorem}

\noindent Inoltre, confrontiamo LOGSPACE con $\mathcal{P}$, stabilendo un altro piccolissimo frammento della gerarchia che intercorre tra classi di complessità in spazio e in tempo; ulteriori risultati si trovano nel capitolo 2.4.

\begin{theorem}
    \label{logspaceInP}
    $\mathrm{LOGSPACE} \subseteq \mathcal{P}$
\end{theorem}

\begin{proof}

    Poiché il problema $I$ appartiene a LOGSPACE, c'è una macchina di Turing $M$ che decide ogni sua istanza $x \in I$ in $\mathcal{O}(\log|x|)$ spazio deterministico, basta notare che $M$ può attraversare al massimo $\mathcal{O}(|x| \times \log|x| \times \#Q \times \#\Sigma^{\log|x|})$ configurazioni non terminali diverse.\
    Una computazione non può ripassare su una stessa configurazione, altrimenti va in ciclo, quindi una computazione ha al massimo $\mathcal{O}\left(|x|^k\right)$ passi per qualche $k$.

\end{proof}

\noindent Infine, dalla dimostrazione di sopra, si può vedere che, seppure in modo meno preciso, vale anche il ``duale'' di quanto affermato a pagina 69, e cioè che, nel caso di algoritmi per problemi decidibili

\begin{center}
    \textbf{Osservazione 2}:\ lo spazio limita il tempo di calcolo!
\end{center}

\section{Misure di complessità non deterministiche}

C'è un modo per risolvere i problemi che è un po' ottuso, ma funziona sempre benissimo, complessità a parte, soprattutto quando uno non conosca o non sappia definire l'algoritmo giusto.\
Prendiamo come semplicissimo esempio il problema di calcolare il massimo numero naturale che sia minore della radice quadrata di 29 senza sapere né come fare né avere a disposizione la tabellina pitagorica:\ basterà allora cominciare a moltiplicare tra loro tutti i numeri tra 1 e 6 per scoprire che il risultato è 5 (se lo volete vedere come un problema di decisione, chiedetevi se $n \in \left\{\left\lfloor \sqrt{29}\right\rfloor\right\}$).\
In alternativa, si può tirare un dado, calcolare il quadrato del numero uscito e controllare se è una soluzione, magari ripetendo il lancio (avendo per magia rimosso il numero appena uscito dalle facce del dado) e confrontando l'ultimo numero uscito con i precedenti e con 29 --- basta quindi che esista un lancio ``fortunato'' e la soluzione è trovata.\
Quest'ultima modalità (e anche l'altra!) origina un albero di scelte i cui livelli rappresentano i lanci e nei cui nodi immaginiamo di aver scritto il numero uscito --- basta che esista un cammino nell'albero che porta a una soluzione.\

Questo metodo di soluzione, seppure ottuso, non è così assurdo come può apparire a prima vista e ritorneremo più avanti sui problemi che sembra si possano risolvere solo ricorrendovi, perché non si conoscono o non si sanno sfruttare proprietà matematiche, strutturali del problema.\
In entrambi i casi delineati nell'esempio, vi è un procedimento di tipo \textit{non deterministico} --- l'unica proprietà matematica usata per giustificare l'impiego di un dato a sei facce è che $\sqrt{29} \leq 6$.\
Nel primo metodo, il non determinismo è meno evidente, essendo stato ridotto a una soluzione operazionalmente accettabile; infatti, si generano tutte le scelte, ovvero l'intero spazio di ricerca, e ciascuna di queste viene esaminata, cioè si fa una ricerca esaustiva della soluzione; questo metodo viene anche detto di \textit{forza bruta}.\
Seguendo il secondo procedimento si prende una potenziale soluzione \textit{a caso} (si ricordi che basta che ne esista una!) e si controlla se essa lo è davvero; in inglese questo metodo si chiama \textit{guess-and-try} --- se c'è una soluzione, ci viene fornita per magia.\
Si noti che questo metodo non richiede né di generare né di visitare l'intero spazio di ricerca se non \textit{implicitamente}, come sarà forse più chiaro in seguito.\
Inoltre, l'albero delle possibili scelte è sempre finito, perché stiamo considerando solo problemi decidibili.\
Quindi, una soluzione appare sempre a profondità finita o, sempre a profondità finita, \textit{tutti} i rami dell'albero di scelta finiscono su nodi che riportano fallimento (non è il caso nel nostro esempio, in cui tutti i rami terminano con successo al massimo dopo sei lanci del dado, cioè a profondità sei nell'albero; non è difficile immaginare casi in cui vi siano situazioni di successo e situazioni di fallimento e ne vedremo in seguito).\
Si noti infine che un albero delle possibili scelte, per brevità \textit{albero non deterministico delle computazioni} o semplicemente \textit{albero non deterministico}, può sempre essere visitato in modo deterministico, livello per livello.\

Per formalizzare le intuizioni descritte sopra, è opportuno introdurre una variante delle macchine di Turing, dette \textit{non deterministiche}.\
Essenzialmente, una macchina non deterministica differisce da una deterministica per il fatto che la relazione di transizione $\delta$ non è necessariamente una funzione, cioè una configurazione può evolvere in più di una configurazione successiva, originando per così dire un albero (di computazioni) non deterministico (sia ben chiaro però che una computazione continua a essere una \textit{singola successione} di configurazioni).\
L'osservazione fatta prima, che tale albero può esser visitato per livelli, ci assicura che introdurre le macchine non deterministiche non cambia affatto la classe dei problemi decidibili, né alcuno dei risultati di calcolabilità presentati nella prima parte del corso.\

Con la stessa osservazione ci si può facilmente convincere che le macchine di Turing deterministiche simulano quelle non deterministiche con una perdita di efficienza \textit{esponenziale} (si ricordi che i nodi di un albero sono in numero esponenziale rispetto la profondità dell'albero stesso); per una formulazione esatta, si veda il teorema \ref{simulazione_non-deterministica}.\
Quanto detto giustifica, almeno in parte, la tesi di Cook-Karp, cioè che la classe dei problemi decidibili in tempo polinomiale \textit{deterministico}, $\mathcal{P}$, è formata dai problemi \textit{facili}, mentre la classe dei problemi decidibili in tempo polinomiale \textit{non deterministico}, $\mathcal{NP}$, che definiremo precisamente in seguito, è quella dei problemi \textit{difficili}.\
Torneremo più avanti su questa distinzione, analizzando più in dettaglio $\mathcal{P}$ e $\mathcal{NP}$.\

Però a questo punto non possiamo non chiederci se l'introduzione del non-determinismo dia davvero un potere maggiore alle MdT dal punto di vista della complessità del calcolo, ovvero quanto sia fondata la tesi di Cook-Karp.\
Abbiamo già toccato questo argomento introducendo un frammento di una gerarchia di complessità, concludendo col dire che è ancora irrisolto il famoso problema $\mathcal{P} \stackrel{?}{=} \mathcal{NP}$, ovvero se non vi sia differenza tra il tempo polinomiale non deterministico e quello deterministico, nel qual caso aver introdotto il non determinismo non separerebbe i problemi ``difficili'' da quelli ``facili''.\
Per quanto riguarda lo spazio, preannunciamo che la classe dei problemi che richiedono spazio polinomiale \textit{deterministico} PSPACE coincide con NPSPACE, quella dei problemi risolubili in spazio polinomiale \textit{non deterministico}, ancora da definire.\
Se valesse anche l'eguaglianza $\mathcal{P} = \mathcal{NP}$, il meccanismo del non determinismo, che sembra avere scarso significato computazionale, almeno dal punto di vista della sua concreta realizzabilità, sarebbe completamente irrilevante anche dal punto di vista della complessità.\

\subsection{Macchine di Turing non deterministiche}

Introduciamo di seguito l'estensione non deterministica alle macchine di Turing come definite in \ref{Macchina di Turing}, dando esplicitamente conto solo delle cose che cambiano.\
Come detto sopra, una macchina non deterministica differisce da una deterministica per il fatto che la relazione di transizione $\delta$ non è necessariamente una funzione.\
A differenza di altre estensioni viste, questo nuovo modello non appare però sufficientemente ``realistico''.\
Infatti, le estensioni viste nel capitolo precedente si basano su meccanismi che hanno un'interpretazione computazionale immediata; ad esempio, le macchine con molti nastri sono una semplificazione accettabile dei calcolatori paralleli.\
Invece, non si conoscono, almeno per ora, né ci riesce di immaginare, allo stato presente della tecnologia, macchine che siano davvero non deterministiche.
\footnote{Tra le differenti varianti della macchina di Turing, introdotte allo scopo di confermare o confutare la tesi di Church-Turing, vanno citati alcuni modelli di computazione introdotti abbastanza recentemente e ispirati a meccanismi fisici o biologici.\

Più precisamente, in alcuni articoli di Deutsch, Feynman ed altri autori, apparsi verso la metà degli anni ottanta, viene esaminata criticamente l'adeguatezza del sistema fisico (computer o agente umano) alla base del modello di macchina di Turing.\
Senza entrare nei dettagli, appare chiaro che tale sistema fisico obbedisce alle leggi della meccanica classica laddove la comunità fisica è abbastanza concorde che il comportamento di un sistema fisico vada descritto sulla base delle leggi della fisica quantistica [magari facendo riferimento alla teoria delle stringhe].\
Una macchina che modella un tale tipo di sistema (macchina di Turing quantistica) è stato formalizzato in un articolo di Deutsch.\
Meccanismi computazionali basati su fenomeni biologici hanno invece conosciuto un particolare sviluppo soprattutto dopo un esperimento effettuato da Adleman e che consiste essenzialmente nell'utilizzare reazioni biologiche su molecole di DNA per ``codificare'' un algoritmo per la soluzione del problema del cammino hamiltoniano in un grafo.\
C'è da dire che né tali meccanismi, né la macchina di Turing quantistica ampliano le capacità espressive delle macchine di Turing:\ la classe delle funzioni calcolate è quella delle funzioni T- [o $\mu$- o WHILE-] calcolabili.\
Dato però il parallelismo estremo di tali dispositivi, quello che esse alterano è piuttosto la classificazione di alcuni problemi nelle varie classi di complessità; ad esempio, l'algoritmo quantistico di Shor che, con complessità di tempo polinomiale, fattorizza un intero in numeri primi.}

L'importanza di queste macchine però, sia come le presentiamo qui, e a maggior ragione in versioni assai più elaborate studiate in letteratura, è rilevantissima per organizzare in modo adeguato una teoria quantitativa degli algoritmi, per cui, pur consci della loro astrattezza e apparente (?) irrealizzabilità, non esitiamo a usarle.

\begin{definition}
    \label{MdT_non-det}
    Una MdT \textit{non deterministica} (a $k$ nastri, di tipo I/O) è una quadrupla $N = (Q, \Sigma, \Delta, q_0)$ dove
    \begin{itemize}
        \item $Q, \Sigma, q_0$ sono come nella definizione \ref{Macchina di Turing} (nella \ref{MdT_k-nastri} delle macchine con $k$ nastri, nella \ref{MdT_I/O} di quelle I/O);
        \item $\Delta \subseteq (Q \times \Sigma) \times ((Q \cup \{\mbox{\textit{\footnotesize SI, NO}}\}) \times \Sigma \times \{L,R,-\})$ è la \textit{relazione} di transizione (estesa nel modo ovvio nel caso delle macchine a $k$ nastri e di tipo I/O).
    \end{itemize}
    Le configurazioni non vengono affatto modificate:\ esse hanno la stessa forma di quelle già viste:\ $(q, w\sigma u)$.\
    Allo stesso modo non cambia il passo di computazione $(q, w \sigma v) \rightarrow_N (q', w'\sigma'v')$, e quindi le computazioni sono anche qui una \textit{successione} (\textit{non} un albero!)\ di configurazioni; infine continueremo a usare gli apici $n$ e $*$ per indicare computazioni di lunghezza $n$ o qualunque.\
\end{definition}

\noindent A rimarcare che nelle macchine di Turing non deterministiche si usa una relazione piuttosto che una funzione, abbiamo usato la lettera maiuscola $\Delta$ al posto di quella minuscola $\delta$.\
Poiché la relazione di transizione $\Delta$ può contenere più quintuple associate allo stesso stato e allo stesso simbolo, ci possono essere molte configurazioni $(q', w' \sigma' u')$ che sono raggiungibili da $(q, w \sigma u)$ in un \textit{solo} passo.\
Dovrebbe essere adesso più chiaro che le computazioni possono venir organizzate in un albero non deterministico del genere visto sopra.\
Inoltre, la vera potenza del non determinismo appare nel modo in cui si accetta (o se preferite si calcola, il che richiede una semplice e ovvia estensione alla definizione seguente).\

\begin{definition}
    La macchina non deterministica $N$ (a $k$ nastri) decide $I$ tutte e sole le volte che
    \begin{itemize}
        \itemsep0px
        \item[] $x \in I$ se e solamente se
        \item[] esiste una computazione tale che $(q_0, \underline{\triangleright}x, \triangleright, \dots,\triangleright) \rightarrow^*_N (\mbox{\textit{\footnotesize SI}}, w_1, \dots,w_k)$
    \end{itemize}
\end{definition}

\noindent Per accettare una stringa di ingresso, \textit{basta che esista} una computazione che porti a una configurazione il cui stato sia \textit{\footnotesize SI} -- ecco il pizzico di magia:\ basta che ci sia una configurazione che accetta e non è affatto rilevante che ci siano altre computazioni che rifiutano, finendo in configurazioni con stato \textit{\footnotesize NO} (o nel caso generale che non terminano).
\footnote{Per questa ragione il non determinismo che abbiamo introdotto si chiama anche angelico; la versione demoniaca prevede che tutte le computazioni raggiungano uno stato di successo.}
Si noti come questa definizione di accettazione introduca un'asimmetria rispetto a quella di non accettazione.\
In quest'ultimo caso, infatti, per rifiutare una stringa di ingresso bisogna che \textit{tutte} le computazioni della macchina portino a configurazioni con stato \textit{\footnotesize NO} o siano non terminanti.\
La situazione di non terminazione verrà esclusa nel seguito, perché parliamo solo di problemi decidibili; per avere un'idea di come ci possano essere computazioni non terminanti, si consideri di nuovo il banale esempio fatto all'inizio del capitolo, e si cerchi di determinare che 5 è una soluzione lanciando ripetutamente il dado da cui però nessun mago ha sottratto le facce che portano i numeri già usciti:\ vi saranno allora sequenze di lanci in cui esce sempre lo stesso numero.\
Si noti tuttavia che, anche senza magia, le soluzioni si trovano \textit{sempre} a profondità finita.\
Si noti anche che una computazione è \textit{completamente determinata} da una sequenza di scelte tra le varie configurazioni raggiungibili passo dopo passo.

\subsection{Complessità in tempo e in spazio non deterministici}

Introduciamo ora i corrispettivi non deterministici delle classi di complessità definite nel precedente capitolo, i cui nomi inizieranno tutti con la lettera $N$, per non determinismo.\
Non si dimentichi che nel seguito considereremo \textit{solo} problemi decidibili, quindi le macchine che useremo terminano per ogni ingresso; ciò implica che se una di queste macchine $N$ decide $I$, non solo si ha che $\forall x \in I$ esistono $w_1, w_2, \dots, w_k$ tali che $N(x) = (q_0, \underline{\triangleright}x, \triangleright, \dots,\triangleright) \rightarrow_N^*(\mbox{\textit{\footnotesize SI}}, w_1, w_2, \dots, w_k)$, ma anche che $\forall x \notin I,\ \forall w_1, w_2, \dots, w_k$ tali che $N(x) \rightarrow_N^* (q, w_1, w_2, \dots, w_k) \not\rightarrow_N$ si ha che $q = \mbox{\textit{\footnotesize NO}}$, cioè si raggiunge lo stato di rifiuto.\

\begin{definition}
    La macchina non deterministica $N$ \textit{decide} $I$ \textit{in tempo non deterministico} $f(n)$ se e solamente se
    \begin{itemize}
        \itemsep 0px
        \item $N$ decide $I$ e
        \item $\forall x \in I,\ \exists t$ tale che $(q_0, \underline{\triangleright}x,\triangleright, \dots, \triangleright ) \rightarrow^t_N (\mbox{\textit{\footnotesize SI}}, w_1, \dots,w_k)$, $t \leq f(|x|)$.
    \end{itemize}
\end{definition}

\noindent Intuitivamente, una MdT non deterministica $N$ accetta $x$ in tempo non deterministico $f(|x|)$ se e solo se c'è \textit{almeno una} computazione che termina in uno stato \textit{\footnotesize SI} in $t$ passi, $t \leq f(|x|)$; di fatto, occorre e basta che la più breve delle computazioni accettanti ci metta meno passi di $f(|x|)$, o altrettanti.\
Invece, $N$ non accetta $x$ se \textit{tutte} le sue computazioni lunghe al massimo $f(|x|)$ conducono allo stato \textit{\footnotesize NO}.\
Ecco di nuovo l'asimmetria:\ non basta che per un elemento $x \notin I$ ci sia una computazione che porta allo stato \textit{\footnotesize NO} in meno di $f(|x|) + 1$ passi, ma si richiede che lo debbano fare \textit{tutte} in meno di $f(|x|) + 1$ passi.\
Quindi l'asimmetria tra il decidere $I$ e il decidere $\bar{I}$ che abbiamo visto nel caso generale viene rafforzata dal richiedere che ciò venga svolto in tempo $f(|x|)$.\
Ripetiamo:\ affinché $N$ decida il problema $I$ basta che per \textit{ogni} suo elemento $x$ ci sia \textit{una} computazione che lo accetta.\
Invece, affinché $N$ decida $\bar{I}$ bisogna che \textit{tutte} le computazioni sul dato di ingresso $x \notin I$ conducano allo stato \textit{\footnotesize NO} in meno di $f(|x|) + 1$ passi.\

Come fatto per TIME$(f(n))$ possiamo ora definire la classe dei problemi decidibili da MdT (e da altri modelli ``ragionevolmente'' equivalenti) in tempo \textit{non deterministico} $f(n)$.

\begin{definition}
    Se una macchina di Turing non deterministica decide il problema $I$ in tempo $f$, allora $I \in \mathrm{NTIME}(f)$
\end{definition}

\noindent Adesso possiamo finalmente introdurre la classe dei problemi decidibili in tempo polinomiale non determistico.

\begin{definition}
    La classe dei problemi decidibili in tempo non determistico polinomiale è
    \[\mathcal{NP} = \bigcup_{k \geq 1} \mathrm{NTIME}\left(n^k\right)\]
\end{definition}

\noindent Ovviamente $\mathcal{P} \subseteq \mathcal{NP}$, perché una macchina deterministica $M$ è anche non deterministica.\
Sarebbe bello se fosse vero anche il contrario, cioè $\mathcal{P} \supseteq \mathcal{NP}$, da cui si dedurrebbe che $\mathcal{P} = \mathcal{NP}$.\
Se così fosse, ci sarebbe un modo per trasformare un algoritmo non deterministico polinomiale in uno deterministico polinomiale, quindi ``facile e fattibile''.

Come abbiamo già più volte detto, quello che si sa fare per ora è di simulare una macchina non deterministica $N$ che decide un problema in tempo polinomiale con una macchina deterministica $M$ con una perdita di efficienza in tempo \textit{esponenziale}.\
L'idea è di generare prima le potenziali soluzioni e poi verificare se lo sono davvero; vedremo anche che il primo passo è difficile, nel senso che (solitamente) richiede un numero di passi esponenziale, mentre il secondo è facile e può essere eseguito in tempo deterministico polinomiale.\

\begin{theorem}
    \label{simulazione_non-deterministica}
    Se I è deciso in tempo non deterministico $f(n)$ dalla macchina non deterministica $N$ (a $k$ nastri), allora, con una perdita esponenziale, è deciso in tempo $\mathcal{O}\left(c^{f(n)}\right)$ da una macchina deterministica $M$ (a $k+1$ nastri), con $c > 1$ dipendente solo da $N$.\
    Più precisamente:
    \[\mathrm{NTIME}\left(f(n)\right) \subseteq \mathrm{TIME}\left(c^{f(n)}\right)\]
\end{theorem}

\begin{proof}

    Sia $d$ il grado di non-determinismo di $N$, cioè poniamo
    \[d = \max \{\mathit{Grado}(q,\sigma) \mid q \in Q, \sigma \in \Sigma\}\]
    dove $Grado(q, \sigma) = \#\{(q',\sigma',D) \mid ((q,\sigma),(q',\sigma',D)) \in \Delta\}$.\

    \medskip
    \noindent(Per semplicità di trattazione, supponiamo nel seguito che la macchina abbia sempre $d$ scelte; non è difficile modificare quanto segue al caso generale).
    \medskip

    \noindent Per ogni stato $q \in Q$ e ogni simbolo $\sigma \in \Sigma$, ordiniamo totalmente, p.e.\ lessicograficamente, l'insieme $\Delta (q, \sigma)$.\
    Ogni computazione di $N$ è una sequenza di scelte; se tale sequenza è lunga $t$, la possiamo vedere come una successione di numeri naturali minori di $t$ nell'intervallo $[0 \dots d - 1]$, rappresentando con 0 la prima scelta.\
    La macchina $M$ considera queste successioni una alla volta, in ordine crescente (ovvero visita l'albero per livelli) e per ciascuna di esse simula $N$.\
    (Nota bene:\ la costruzione di $M$ deve essere indipendente da $f$ e quindi deve esser fatta a prescindere dal valore corrente di $f(n)$ --- se così non fosse, basterebbe generare tutte le computazioni lunghe al più $f(|x|)$).

    La macchina $M$ riproduce la successione di scelte $(c_1 , \dots, c_t )$, tenendo l'ultima successione sul nastro aggiuntivo, ovvero vi mantiene il numero $t'$ in base $d$ che gli corrisponde.\
    Se durante questa simulazione $M$ arriva in uno stato \textit{\footnotesize SI} allora $M$ termina accettando, altrimenti genera la prossima successione, usando come guida il prossimo numero in base $d$, cioè $t' + 1$.\
    Se tutte le successioni terminano (ovvero è stato generato il numero $t$) portando allo stato \textit{\footnotesize NO}, allora $M$ termina rifiutando.\
    È chiaro che $M$ termina con successo se e solamente se $N$ fa altrettanto.\

    Rimane da verificare che il tempo impiegato da $M$ nella simulazione è quello dell'enunciato.\
    Quante sono le successioni da visitare?\ al massimo $\mathcal{O}\left(d^{f(n)+1}\right)$, quindi il teorema è dimostrato ponendo $c = d$.\

\end{proof}

\noindent Terminiamo con la definizione di spazio non deterministico e della classe di problemi decidibili in spazio non deterministico polinomiale.\
Si noti il quantificatore esistenziale nel secondo punto.

\begin{definition}
    La macchina di Turing $N$, non deterministica a $k$-nastri di tipo I/O, \textit{decide} $I$ \textit{in spazio non deterministico} $f(n)$ se e solamente se
    \begin{itemize}
        \itemsep0px
        \item $N$ decide $I$
        \item $\forall x \in I\ \exists w_1,\dots,w_k$ tali che $(q_0, \underline{\triangleright} x, \dots,\underline{\triangleright}) \rightarrow^*_N (\mbox{\textit{\footnotesize SI}}, w_1,w_2,\dots,w_k)$ e $\sum_{2\leq i \leq k-1} |w_i| \leq f(n)$
    \end{itemize}
    Se $N$ decide $I$ in spazio non deterministico $f(n)$, allora $I \in \mathrm{NSPACE}(f(n))$.\
    Infine, la classe dei problemi decidibili (da MdT) in spazio non deterministico polinomiale è
    \[\mathrm{NPSPACE} = \bigcup_{k \geq 1} \mathrm{NSPACE}\left(n^k\right)\]

\end{definition}

\noindent Abbiamo già preannunciato che l'estensione con il non determinismo non allarga la classe dei problemi trattabili polinomialmente, quando la misura della complessità riguardi lo spazio.\
Enunciamo ora tale teorema, senza dimostrarlo.\

\begin{theorem} [Savitch]
    $\mathrm{NPSPACE} = \mathrm{PSPACE}$.
\end{theorem}

\noindent Nonostante nella soluzione di un problema $I$ ottenuta per forza bruta o procedendo per tentativi non si sfrutti affatto la struttura matematica di $I$, forse perché essa è troppo complessa, o perché essa sfugge al solutore del problema, vi sono numerosi esempi di problemi per cui quello è l'unico modo conosciuto di arrivare a una soluzione; ne menzioneremo alcuni più avanti e vi ritorneremo ancora nel prossimo capitolo.\
Per ora limitiamoci a discutere brevemente un esempio paradigmatico per la soluzione del quale si conoscono solo algoritmi polinomiali non deterministici o esponenziali deterministici:\ il problema del commesso viaggiatore.
\footnote{Stiamo parlando di soluzione esatta; nel caso in cui ci si accontenti di una soluzione approssimata vi sono algoritmi molto più furbi, sulla cui natura ritorneremo brevemente più avanti.}

\begin{example} [Problema del commesso viaggiatore]
    Vi sono $n$ città, ciascuna individuata da un intero e collegata a tutte le altre da una strada percorribile nei due sensi; sia allora $d(i, j)$ la distanza tra le città $i$ e $j$.\
    \footnote{Una distanza $d$ è una funzione da coppie di punti, nel nostro caso città, nei reali positivi, tale che gode delle seguenti tre proprietà
        \itemsep0px
        \item[-]  riflessiva:\ $d(i, j) = 0$ se e solamente se $i = j$;
        \item[-]  simmetrica:\ $d(i, j) = d(j, i)$;
        \item[-]  triangolare:\ $d(i, j) \leq d(i, k) + d(k, j)$.
    }
    Il problema consiste nel trovare il cammino di costo minimo che attraversa tutte le città una e una volta sola --- ovvero dobbiamo trovare una permutazione di indici (o città) $\Pi : [1\dots n] \rightarrow [1\dots n]$ che minimizza la seguente quantità, in cui intendiamo $i = \Pi(h), i + 1 = \Pi(k)$ per qualche $h$ e $k$:
    \[\sum_{1 \leq i \leq n-1} d(i,i+1)\]
    Ovviamente, il problema si può rappresentare come un grafo (non diretto) i cui nodi sono le città e in cui c'è un arco tra ogni coppia di nodi $i$ e $j$, pesato da $d(i,j)$.\

    Per vedere questo problema come un problema di decisione si abbia un valore limite $B$, da interpretarsi come il rimborso viaggi assegnato al commesso viaggiatore; allora il problema è risolto positivamente se c'è un cammino che tocca tutte le città una e una sola volta di costo minore o uguale a $B$.\

    \medskip
    \noindent Vediamo adesso di calcolare, in modo assai spiccio, la complessità prima di una (tipica) procedura che impiega il metodo di forza bruta per risolvere il problema del commesso viaggiatore e poi di una (tipica) MdT non deterministica che risolve la sua variante di decisione; entrambe procedono in due fasi separate.\
    Il dato iniziale delle due macchine è ovviamente la rappresentazione della rete stradale e del costo $B$ (per esempio come matrice di incidenza determinata dalle distanze).

    Come tutte le procedure a forza bruta, il primo metodo ha la seguente struttura:
    \begin{itemize}
        \item[i)] genera tutte le potenziali soluzioni come permutazioni di tutti gli interi fino a $n$ il che costa $\frac{(n-1)!}{2}$
        \item[ii)] scegli tra tutte le permutazioni la prima che ha costo accettabile, ovvero minore di $B$, e questo controllo può essere fatto in tempo deterministico cubico (bisogna accedere $\mathcal{O}(n)$ volte alle $\mathcal{O}(n^2)$ coppie $(i,j)$ per ottenere la distanza $d(i,j)$ memorizzata nel nastro di ingresso).
    \end{itemize}
    (Può essere interessante vedere che lo spazio necessario alla procedura sopra delineata è in $\mathcal{O}(n)$, perché è sufficiente generare una permutazione alla volta e memorizzare in uno spazio di lavoro la permutazione in esame.)

    Costruiamo adesso una macchina di Turing non deterministica $N$ che decide il problema in $\mathcal{O}(n^3)$.\
    $N$ procede con le due fasi seguenti:\
    \begin{itemize}
        \item[i)] $N$ scrive su un nastro di lavoro una stringa di $n$ numeri naturali compresi tra 1 e $n$; cioè vi sono $n$ transizioni possibili a partire dalla configurazione iniziale, ciascuna che scrive un numero nell'intervallo $[1 \dots n]$ sul nastro di lavoro e dalle configurazioni così raggiunte vi sono $n$ transizioni, che scrivono un numero in $[1 \dots n]$, e così via --- ovviamente la scelta della transizione da effettuare viene fatta a caso (naturalmente la macchina potrebbe esser più furba).\ Questa fase richiede $\mathcal{O}(n)$ passi;
        \item[ii)] $N$ verifica se la stringa è un cammino accettabile, cioè è una permutazione degli indici, usando un altro nastro di lavoro, in tempo $\mathcal{O}(n^2)$; $N$ verifica di seguito (o contemporaneamente) se il costo del cammino è minore o uguale a $B$, di nuovo in tempo $\mathcal{O}(n^3)$, nel qual caso risponde positivamente; altrimenti risponde negativamente.
    \end{itemize}

    \noindent Si noti come la prima fase che compiono entrambe le procedure consiste nel generare una delle soluzioni, le quali sono in numero esponenziale; la differenza è che nel primo algoritmo si procede costruttivamente, generandole tutte, mentre nel secondo si tira a indovinare e si sfrutta il meccanismo non deterministico della macchina usata.\
    In altre parole, in entrambi i casi si genera esaustivamente tutto lo spazio del problema su cui fare poi la ricerca della soluzione, ma nel primo modo ciò avviene \textit{esplicitamente}, nel secondo \textit{implicitamente}.\
    È importante osservare che la seconda fase è una \textit{certificazione in tempo polinomiale} fatta in modo \textit{deterministico} che la stringa di naturali sia davvero una soluzione.\
    Uno potrebbe, e spesso viene fatto, definire allora la classe $\mathcal{NP}$ come l'insieme dei problemi che ammettono una certificazione in tempo polinomiale.\
    Tuttavia, dovrebbe a questo punto essere chiaro che i due modi per definire tale classe, o via MdT non deterministiche o via certificazione polinomiale, sono del tutto equivalenti e differiscono solo dal punto di vista con cui si esaminano i problemi che vi appartengono.\
    Infine, notiamo che, quando il cammino in esame non è una soluzione, il metodo esaustivo esplicito ci dice che dobbiamo visitare tutto lo spazio degli stati, che sono in numero esponenziale, quindi la certificazione del fallimento avviene in tempo esponenziale:\ \textit{tutti} i tentativi falliscono.\

    Come già detto più volte, questo modo di procedere trova applicazioni in molti campi.\
    Ad esempio, in logica per dimostrare che una formula, decidibile e lunga $n$, è un teorema, uno può generare ``tutte le dimostrazioni'' le cui ultime formule sono lunghe $n$ e se ci trova la formula di partenza, allora questa è un teorema.\
    Oppure, per decidere se c'è un assegnamento di valori di verità alle variabili di una formula che la rende vera, si considerano tutti i possibili assegnamenti e poi si calcola il valore della formula in ciascun caso.\
    Strettamente correlato a quest'ultimo problema, c'è, in teoria dei circuiti, il problema di decidere quando in un dato circuito passa un segnale.\
    Nell'area dell'ottimizzazione combinatoria, oltre al problema del commesso viaggiatore e di quelli ad esso correlati, si risolvono in modo esatto con questa tecnica i problemi di assegnazione di risorse, tra i quali l'allocazione dei task (non pre-rilasciabili) ai processori, compito fondamentale dei sistemi operativi.\
    Infine, nel campo dell'intelligenza artificiale, ci sono tutti i problemi legati ai giochi o alla ricerca e all'estrazione di informazione da grosse collezioni di dati, anche non strutturate.
\end{example}


\chapter{Crittografia su curve ellittiche}

Oggi si parla di una nuova generazione di cifrari a chiave pubblica facendo riferimento alla crittografia sulle \textbf{curve ellittiche}, la quale sta via via sostituendosi ai sistemi basati sull'algebra modulare (RSA, Diffie-Hellman, El Gamal) per i problemi relativi alla pesantezza computazionale delle elaborazioni.\
Gli attacchi a questi cifrari non sono più attacchi puramente esponenziali, ci sono delle tecniche che permettono di ridurre il costo della fattorizzazione e del calcolo del logaritmo discreto.\
Quindi si è diffuso un nuovo tipo di crittografia che usa delle operazioni definite sulle curve ellittiche:\ la funzione one-way trap-door è facile da calcolare, ma per l'inversione si hanno a disposizione solo algoritmi con costo esponenziale nel numero di bit e quindi sono più sicuri.\

La crittografia su curve ellittiche (\textit{Elliptic Curve Cryptography}) permette di avere un livello di sicurezza molto più elevato a parità di lunghezza della chiave e di usare chiavi più corte a parità di sicurezza.\

\section{Curve ellittiche}

Si chiamano curve ellittiche, ma non sono delle ellissi:\ sono delle curve algebriche descritte da equazioni che somigliano a quelle usate per calcolare la lunghezza degli archi delle ellissi.\
Sono uno strumento matematico abbastanza recente; sono studiate dalla metà del XIX secolo, mentre le applicazioni alla crittografia sono più recenti.\

A metà degli anni `80, Miller (IBM) e Koblitz (Università di Washington) hanno proposto di prendere i protocolli noti della cifratura a chiave pubblica e modificare gli algoritmi sostituendo le operazioni dell'algebra modulare con le operazioni sui punti delle curve ellittiche.\

\begin{definition}
    Preso un campo $k$, una \textbf{curva ellittica} è un insieme di punti $(x, y) \in k^2$ tale che
    \[y^2 + axy + by = x^3 + cx^2 + dx + e\qquad a,b,c,d,e \in k \]
\end{definition}

\noindent Se la \textit{caratteristica}\footnote{La \textbf{caratteristica} di un campo è il numero di volte che l'elemento neutro moltiplicativo (1) deve essere sommato per ottenere l'elemento neutro additivo (0):\ nel campo $\mathbb{Z}_p$, la caratteristica è proprio $p$} di $k \neq 2,3$ allora è possibile ridurre l'equazione in \textit{forma normale di Weierstrass}:
\[y^2 =  x^3 + ax + b \qquad a,b \in k\]
Quindi scriveremo
\[E_k(a,b) = \{(x,y) \in k^2 \mid y^2 = x^3 + ax + b\} \cup \{O\}\]
È interessante lavorare sulle curve ellittiche perché è possibile dare a $E_k(a,b)$ la struttura algebrica di \textit{gruppo abeliano} (non di campo) ed quindi è possibile definirvi un'operazione interna associativa e commutativa con esistenza dell'inverso:\ combinando due punti della curva tramite questa funzione si ottiene un terzo punto appartenente alla curva.\
Per poter poter essere un gruppo abeliano è necessario anche l'elemento neutro $\{O\}$, chiamato \textit{punto all'infinito} (a seconda del campo $k$ tale elemento potrebbe essere già presente).\

Supponiamo $k=\mathbb{R}$
\[E_{\mathbb{R}}(a,b) = \{(x,y) \in \mathbb{R}^2 \mid y^2 = x^3 + ax + b\} \cup \{O\}\]
Per poter definire la struttura di gruppo abeliano è richiesto che $x^3 + ax + b$ non abbia radici multiple.\
Quindi assumiamo che \[4a^3 +27b^2 \neq 0\]
in questo caso la cubica non ha radici multiple e perciò è garantita l'esistenza della tangente in ogni punto della curva ellittica.\

Nel caso in cui vi siano radici multiple la curva potrebbe assumere una forma a ``nodo'' o a ``cuspide''.\

\subsubsection{Simmetria orizzontale}

Assumiamo di avere un punto $P = (x,y) \in E(a,b)$.\
Allora $P$ soddisfa $y^2 = x^3 + ax + b$ e quindi anche il punto $-P = (x,y) \in E(a,b)$, infatti
\[\left(-y \right)^2 = y^2 = x^3 + ax + b\]
$-P$ è detto ``\textit{opposto}'', l'inverso di $P$.\

Per il \textit{punto all'infinito} si pone $O = -O$.\

\subsection{Somma sulle curve ellittiche}

Si chiama \textit{somma}, ma non ha relazioni con la somma delle coordinate.\

Se si prende una curva ellittica e se ne studia l'intersezione si vede che i punti di intersezione sono al massimo tre; infatti, mettendo a sistema una curva ellittica di Weierstrass con una retta, si ottiene un sistema in $x^3$ che ha al massimo tre soluzioni.\
\[ \left\{\begin{array}{l}
        y = mx + q \\
        y^2 = x^3 + ax + b
    \end{array}\right.\]
Esiste anche il caso in cui c'è una sola soluzione reale e le altre due sono complesse coniugate.\
Una cosa molto interessante è che se una retta interseca $E(a,b)$ in due punti, allora la interseca anche in un terzo punto:\ questa caratteristica viene usata per definire l'operazione di ``somma''.\

\begin{definition}
    Siano $P, Q, R \in E(a,b)$.\
    Se $P, Q, R$ sono disposti su una retta, si pone $P+Q+R = O$.\
\end{definition}

\noindent Siano $P,Q \in E(a,b)$ e supponiamo che $Q \neq \pm P$.\
Si consideri la retta $\overline{PQ}$ e il punto generato $R$ con l'intersezione di $E(a,b)$.\
Si pone $P+Q = -R$ e
\[\left\{ \begin{array}{l}
        R \in E(a,b)  \\
        -R \in E(a,b) \\
    \end{array}\right.\]
Se $Q = -P$, allora la retta $\overline{PQ}$ è parallela all'asse $y$ e quindi incontra $E(a,b)$ nel punto all'infinito, perciò $P + (-P) = O$ ($O$ è l'elemento neutro).\

Nel caso in cui $Q = P$ si hanno due radici uguali nella risoluzione del sistema fra curva e retta.\
Si considera la tangente alla curva nel punto $P$ (sempre definita per costruzione, in quanto $4a^3 +27b^2 \neq 0$) e si prende l'opposto del punto di intersezione tra la tangente in $P$ e la curva ellittica (questo punto può essere $O$).\

\subsubsection{Proprietà della somma}

\begin{itemize}
    \item \textbf{Chiusura}:\ $\forall\ P,Q \in E(a,b) \Rightarrow P + Q \in E(a,b)$
    \item \textbf{Elemento neutro}:\ $\forall\ P \in E(a,b) \Rightarrow P + O = O + P = P$
    \item \textbf{Inverso}:\ $\forall\ P \in E(a,b),\ \exists !\ Q \in E(a,b)$ t.c. $P + Q = O = Q + P$
    \item \textbf{Proprietà commutativa}:\ $\forall\ P,Q \in E(a,b),\ P + Q = Q + P$
    \item \textbf{Proprietà associativa}:\ $\forall\ P,Q,R \in E(a,b),\ (P + Q) + R = P + (Q + R)$
\end{itemize}

\subsubsection{Formulazione algebrica}

Prendiamo i punti $P=(x_p, y_p)$, $Q=(x_q, y_q)$ e si vuole formulare $S = P + Q$.\

\vspace{12pt}

\noindent Se $Q \neq \pm P \Rightarrow S = (x_s, y_s)$
\[x_s = \lambda^2 - x_p - x_q \quad \mathrm{e} \quad y_s = -y_p + \lambda (x_p-x_s), \qquad \lambda = \frac{y_q - y_p}{x_q -x_p}\]

\vspace{12pt}

\noindent Se $Q = P$ le formule sono le stesse di prima ma
\[\lambda = \frac{3x_p^2 + a}{2 y_p}\]

\vspace{12pt}

\noindent Si noti che se $y_p = 0$ allora siamo nel caso in cui la retta che passa per i due punti è proprio quella tangente in $P$ che ha come terzo punto il punto all'infinito, perciò $P + P = O$

\vspace{12pt}
\noindent Se $Q = -P$ allora $P + Q = O$.\

\section{Curve ellittiche su campi finiti}

La crittografia ha bisogno di un'aritmetica veloce e assolutamente precisa, senza errori di arrotondamento; di conseguenza, non possono essere usate le curve ellittiche sui reali, ma vengono utilizzate le curve definite su un campo finito:\

\begin{itemize}
    \item \textbf{Prime}:\ $k = \mathbb{Z}_p$, dove $p$ è un numero primo.
    \item \textbf{Binarie}:\ $k = \mathit{GF}(2^m)$, dove $m \in \mathbb{N}$.
\end{itemize}

\noindent Si osservi che nel caso delle curve binarie, siccome la caratteristica di campo è uguale a due ($k=2$), la forma normale di Weierstrass non può essere usata.\

Per semplicità si considereranno solo le curve prime e poiché la caratteristica di $\mathbb{Z}_p = p$, si pone $p > 3$, primo.\
E quindi
\[E_{p}(a,b) = \{(x,y) \in \mathbb{Z}_p^2 \mid y^2\ \mathit{mod}\ p = x^3 + ax + b\ \mathit{mod}\ p\} \cup \{O\}\]
Perché il gruppo di punti della curva sia un gruppo abeliano è necessario che $4a^3 +27b^2 \neq 0$.\
A questo punto, la curva non è più una figura continua, ma una nuvola di punti perché i punti stessi sono presi da un campo finito e si lavora solo nel quadrante positivo (intervallo $[0, p - 1]$).\
Inoltre, se $P \in E_p(a,b) \Rightarrow -P = (x, p-y) \in E_p(a,b)$.\
Si noti che non essendo più un normale piano cartesiano, l'asse di simmetria viene spostato in $y =\frac{p}{2}$.

\subsubsection{Formule}

Le formule sono identiche a quelle viste sui campi non finiti; le uniche differenze sono la presenza di $\mathit{mod}\ p$ e il fatto che la divisione equivale a moltiplicare per l'inverso a meno della presenza di una semplificazione (e l'inverso si può sempre fare perché si lavora modulo numero primo).\

\subsection{Ordine}

La cardinalità del numero di punti appartenenti alla curva è detto \textbf{ordine}.\
Non esiste un modo per definirlo, ma dipende tutto da come è fatta la cubica.\

Considerando \[y^2 \equiv x^3 + ax + b\ \mathit{mod}\ p\quad x\in\mathbb{Z}_p\]
si hanno $p$ valori per $x$ e per ogni $x$ esiste anche il suo opposto, quindi ci si aspetterebbe che il numero di punti $\simeq 2p + 1$.\
Tuttavia, non tutti i $p$ valori di $x$ della cubica danno origine a un \textit{residuo quadratico}\footnote{I \textbf{residui quadratici} sono gli elementi del campo $\mathbb{Z}_p$ che ammettono radici nel campo.}:\ i valori di $x$ che non danno luogo a un residuo quadratico, non definiscono punti sulla curva.\

\begin{center}
    In un campo finito $\mathbb{Z}_p$, solo $\frac{p-1}{2}$ sono residui quadratici.\
\end{center}

\begin{theorem}[Hasse]
    Preso $N =$ ordine di una curva prima $E_p(a,b)$
    \[ | N- (p+1) | \leq 2\sqrt p\]
\end{theorem}

\begin{example}

    \[ y^2 \equiv x^3 + 4x + 4\ \mathit{mod}\ 5\]
    Calcoliamo prima di tutto i punti che possono essere residui quadratici nel campo:

    \begin{table}[H]
        \centering
        \begin{tabular}{|c|c|}
            \hline
            $y$ & $y^2$ \\\hline\hline
            0   & 0     \\
            1   & 1     \\
            2   & 4     \\
            3   & 4     \\
            4   & 1     \\\hline
        \end{tabular}
    \end{table}

    \noindent Solo 0, 1 e 4 sono residui quadratici.\

    \begin{table}[H]
        \centering
        \begin{tabular}{ l c l }
            $x = 0,\ y = 4$ & $\rightarrow$ & $(0,2), (0,3) \in E_5(4,4)$ \\
            $x = 1,\ y = 4$ & $\rightarrow$ & $(1,2), (1,3) \in E_5(4,4)$ \\
            $x = 2,\ y = 0$ & $\rightarrow$ & $(2,0) \in E_5(4,4)$        \\
            $x = 3,\ y = 3$ & $\rightarrow$ & nessuna soluzione           \\
            $x = 4,\ y = 4$ & $\rightarrow$ & $(4,2), (4,3) \in E_5(4,4)$ \\
        \end{tabular}
    \end{table}

    \noindent $\rm Ordine = 7 + 1 = 8$ (compreso il punto $O$).\

\end{example}

\subsection{Funzione one-way trap-door}

Esiste una sorta di parallelismo tra le operazioni dell'algebra modulare e le operazioni sulle curve ellittiche:

\begin{table}[H]
    \centering
    \begin{tabular}{ l c l }
        moltiplicazione                         & $\rightarrow$ & somma di punti                   \\
                                                &               &                                  \\
        \multirow{3}{9em}{elevazione a potenza} &               & \textit{moltiplicazione scalare} \\
                                                & $\rightarrow$ & di un punto $P$ della curva      \\
                                                &               & per un intero $k$                \\
    \end{tabular}
\end{table}

\noindent e quindi
\[ y^k = y \times y \times \dots \times y \quad \rightarrow \quad kP = P+ P+ \dots + P\]
Come l'elevamento a potenza, anche la \textit{moltiplicazione scalare} ha un costo polinomiale.\
È one-way perché calcolare $Q = kP$ dati $k$ e $P$ è facile e si può fare in $\Theta(\log k)$ operazioni grazie all'algoritmo dei \textbf{raddoppi ripetuti}.\

\subsubsection{Algoritmo generale}

\[ k =\sum_{i=0}^t k_i2^i \quad \rightarrow \quad k= (k_t k_{t-1}\dots k_{1} k_{0})_2\]
\[\Rightarrow \# \mathrm{bit}: t + 1 = \lfloor \log_2 k \rfloor + 1\]
\begin{enumerate}
    \item Si calcolano i punti $2P, 4P,\dots, 2^tP$ ciascuno come raddoppio del punto precedente.\ Quindi si eseguono $t = \Theta(\log k)$ raddoppi.\
    \item Si calcola $Q$ come \[Q = \sum_{i: k_i = 1} 2^iP\] Si eseguono al più $O(t)$ somme:\ $O(\log k)$.\
\end{enumerate}

\subsubsection{Operazione inversa della moltiplicazione}

Dati $P$ e $Q$ sulla curva $E_p(a, b)$, trovare se esiste il più piccolo $k$ tale che $Q = kP$.\

Sostanzialmente si tratta del problema di trovare $k = log_p Q$ noto come \textbf{problema del logaritmo discreto per le curve ellittiche}, nonché un problema difficile in quanto non esistono algoritmi polinomiali e addirittura sub-esponenziali in grado di risolverlo, ne esistono solo esponenziali.\

\section{Protocollo DH su curve ellittiche}

Alice e Bob scelgono una curva ellittica che soddisfa la condizione che tutti i suoi punti formino un gruppo abeliano e un punto $B$ della curva di ordine molto grande.\

\vspace{12pt}
\noindent\textbf{Osservazioni}
\begin{itemize}
    \item Per ordine $n$ di un punto $B$ si intende il più piccolo intero tale che $nB = 0$.\
    \item $B$ ``corrisponde'' al generatore $g$ in Diffie-Hellman standard.\
    \item Curva e punto $B$ sono pubblici.\
\end{itemize}

\begin{table}[H]
    \centering
    \begin{tabular}{l|l}
        \multicolumn{1}{c}{Alice}                      & \multicolumn{1}{c}{Bob}                        \\\hline
        estrae $n_A < n$ casuale ($k_{\mathit{priv}}$) & estrae $n_B < n$ casuale ($k_{\mathit{priv}}$) \\
        calcola $k_{\mathit{pub}}$:\ $P_A = n_AB$      & calcola $k_{\mathit{pub}}$:\ $P_B = n_BA$      \\
                                                       &                                                \\
        manda $P_A$ in chiaro a Bob                    & manda $P_B$ in chiaro a Alice                  \\
                                                       &                                                \\
        riceve $P_B$ e calcola                         & riceve $P_A$ e calcola                         \\
        $S = n_A P_B = n_A n_B B$                      & $S = n_B P_A = n_B n_A B$                      \\
                                                       &                                                \\
        \multicolumn{2}{c}{$k_{\mathit{sessione}} = x_S\ \mathit{mod}\ 2^{256}$}                        \\
    \end{tabular}
\end{table}

\subsubsection{Crittoanalista}

Il crittoanalista conosce $E_p(a, b), B, P_A, P_B$, ma non può ricavarne niente di interessante in tempo polinomiale:\ per calcolare $S$ deve trovare $n_A$ tale che $n_A B = P_A $ (oppure $n_B \mid n_B B = P_B $) quindi deve risolvere il problema del logaritmo discreto su curve ellittiche, di cui non ne esistono algoritmi inferiori al puro esponenziale.\
Rimane il problema degli attacchi \textit{man-in-the-middle}, perciò è sempre bene usare i certificati.\

\section{Scambio di messaggi cifrati}

La prima cosa da fare è trasformare $m$ in un punto $P_m$ della curva ellittica $E_p(a, b)$, il quale verrà trasformato in un altro punto della curva ellittica che sarà il crittogramma.\

Vediamo come eseguire $m \rightarrow P_m$:\ considerando l'equazione della curva ellittica
\[y^2 \equiv x^3 + ax + b\ \mathit{mod}\ p\]
potremmo essere tentati di sostituire il messaggio $m$ al posto di $x$, ma non abbiamo la sicurezza che $m$ sia un residuo quadratico nel campo.\
In particolare, sappiamo con certezza che $\mathit{prob}(m^3 + am + b\ \mathrm{sia\ un\ R.Q.}) \simeq \frac{1}{2}$ e non possiamo accettare un metodo corretto per la metà delle volte (se $m$ non è un residuo quadratico, $P_m$ non è un punto della curva).\

\subsubsection{Algoritmo di Koblitz}

È un algoritmo polinomiale randomizzato che permette di eseguire $m \rightarrow P_m$.\

Dato $m < p \rightarrow P_m \in E_p(a, b)$, si sceglie un intero $h$ tale che $(m + 1)h < p$ e successivamente si calcola come ascissa $x = mh + i$ con $0 \leq i < h$.\
L'idea è di riuscire a fare $h$ tentativi.\

\begin{flushleft}
    \ttfamily
    Koblitz(m, h, a, b, p) //$(m + 1)h < p$

    \quad for(i = 0; i < h; i++)\{

    \qquad $x = mh + i$;

    \qquad $z = (x^3 + ax + b)\ \mathit{mod}\ p$;

    \qquad if($z$ è un residuo quadratico)\{ //costo polinomiale

    \quad\qquad $y = \sqrt z$

    \quad\qquad return $P_m = (x, y)$;

    \qquad \}

    \quad \}

    \quad return failure;
\end{flushleft}

\noindent L'algoritmo di Koblitz ha una probabilità di fallimento di circa $\left(\frac{1}{2}\right)^h$ e di successo di circa $1 - \left(\frac{1}{2}\right)^h$.\

Per risalire a $m$ da $x$, il destinatario può eseguire l'operazione
\[\left\lfloor \frac{x}{h}\right\rfloor = \left\lfloor \frac{mh + i}{h}\right\rfloor = \left\lfloor m +\frac{i}{h}\right\rfloor= m\]
poiché $0 \leq i < h$.

\subsubsection{Scambio di messaggi}

Dopo aver trasformato il messaggio in un punto dobbiamo capire come mandarlo.\
Supponendo di avere una curva $E_p(a, b)$, $B$ di ordine elevato ($n$) con $B \in E_p(a, b)$, e $h$ per algoritmo di Koblitz, ogni utente genera:
\[k_{U}[\mathit{priv}] = n_U < n\]
\[k_{U}[\mathit{pub}] = n_U B\]
Alice mappa $m$ su $P_m \in E_p(a,b)$ (per esempio usando l'algoritmo di Koblitz), dopodiché sceglie un \textbf{intero casuale} $r$ e calcola $V = rB$.\
A questo punto Alice calcola $W = P_m + r \cdot k_{\mathrm{Bob}}[\mathit{pub}]$ (poiché $r$ è un numero casuale, $r \cdot k_{\mathrm{Bob}}[\mathit{pub}]$ è un punto ``scelto a caso'' su $E_p(a,b)$) e invia a Bob $\langle V, W\rangle$.\

Bob riceve $\langle V, W\rangle$ da Alice e decifra $P_m$:
\[W - k_{\mathrm{Bob}}[\mathit{priv}]\cdot V = (P_m + r\cdot k_{\mathrm{Bob}}[\mathit{priv}]) - k_{\mathrm{Bob}}[\mathit{priv}]\cdot V = \]
\[ = P_m + r\cdot n_{\mathrm{Bob}}\cdot B - r\cdot n_{\mathrm{Bob}}\cdot B = P_m \]
Adesso può ricavare $m$
\[m = \left\lfloor \frac{x}{h}\right\rfloor\]

\subsubsection{Crittoanalista}

La sicurezza si basa sulla difficoltà del logaritmo discreto su curve ellittiche.\
Se trova $r$, decifra:
\[W - k_{\mathrm{Bob}}[\mathit{pub}] = (P_m + r\cdot k_{\mathrm{Bob}}[\mathit{pub}]) - r\cdot k_{\mathrm{Bob}}[\mathit{pub}] \]
Tuttavia, $r$ viaggia ben protetto all'interno di $V = rB$, quindi deve risolvere il logaritmo discreto che ha un costo almeno esponenziale.\

Se trova $k_{\mathrm{Bob}}[\mathit{priv}]$ decifra come farebbe Bob.\
Tuttavia $k_{\mathrm{Bob}}[\mathit{priv}] = n_{\mathrm{Bob}}$ viaggia protetta in $k_{\mathrm{Bob}}[\mathit{pub}] = n_{\mathrm{Bob}} \cdot B$:\ deve risolvere comunque il logaritmo discreto.\

\section{Motivazioni}

Il problema matematico che garantisce la sicurezza della crittografia su curve ellittiche è molto più difficile della fattorizzazione e del calcolo del logaritmo discreto:\ non esiste alcun limite inferiore esponenziale per i problemi sulle curve ellittiche, però, ad oggi non esistono algoritmi inferiori agli esponenziali.\

I \textbf{cifrari in algebra modulare} soffrono gli attacchi \textbf{index calculus} poiché sfruttano il fatto che il campo $\mathbb{Z}_p$ è sia un gruppo abeliano che un campo (questo permette di dare un concetto di moltiplicazione di punti e facilitare l'index calculus).\
I cifrari su curve ellittiche non hanno il concetto di moltiplicazione e non è possibile estendervi l'applicazione degli attacchi index calculus.\

\begin{table}[H]
    \centering
    \begin{tabular}{|c|c|}
        \hline
        Algebra modulare                   & Curve ellittiche                \\
        (bit del modulo)                   & (bit dell'ordine)               \\\hline
        $O\left(2^{\sqrt{b\log b}}\right)$ & $O\left(2^{\frac{b}{2}}\right)$ \\\hline
    \end{tabular}
\end{table}

\noindent Questi vantaggi si ripercuotono in termini molto pratici sulla dimensione delle chiavi.\


\begin{table}[H]
    \centering
    \begin{tabular}{|c|c|c|}
        \hline
        TDEA, AES          & RSA e DH         & ECC               \\
        (bit della chiave) & (bit del modulo) & (bit dell'ordine) \\\hline
        80                 & 1024             & 160               \\
        112                & 2048             & 224               \\
        128                & 3072             & 256               \\
        192                & 7680             & 384               \\
        256                & 15360            & 512               \\\hline
    \end{tabular}
\end{table}

\noindent La tabella riporta la dimensione in bit delle chiavi che garantiscono livelli di sicurezza equivalenti nei tre diversi sistemi, dove due sistemi si considerano di sicurezza equivalente se è richiesto lo stesso costo computazionale per forzarli.\

\chapter{K-NN, apprendimento senza supervisione e altri approcci}

\section{K-Nearest Neighbors}

Si tratta di un modello supervisionato.\
Memorizza semplicemente i dati di allenamento $\langle x_p, y_p\rangle\ p = 1\dots l$.\
Dato un input $x$ (con dimensione $n$) trova l'esempio di training più vicino $x_i$, cioè trova $i$ t.c.\ abbiamo \[\min d (x, x_i) \rightarrow i(x) = \mathrm{arg}\min_p\ d (x, x_p)\]
Per esempio la distanza euclidea:
\[d(x,x_p) = \sqrt{\sum_{t=1}^n (x_t -x_{p,t})^2} = ||x-x_p||\]
Quindi ritorna $y_i$.\

Un modo naturale per classificare un nuovo punto è dare un'occhiata ai suoi vicini e fare una media:
\[\mathit{avg}_k (x) = \frac{1}{k} \sum_{x_i \in N_k (x)} y_i\]
dove $N_k (x)$ è un intorno di $x$ che contiene esattamente $k$ vicini (pattern più vicini secondo $d$):\ k-nearest neighborhood (\textbf{K-nn}).

Se c'è una chiara dominanza di una delle classi nelle vicinanze di un'osservazione $x$, allora è probabile che anche l'osservazione stessa appartenga a quella classe.\
Quindi la regola di classificazione è il \textbf{voto a maggioranza} tra i membri di $N_k (x)$.\
Come prima,
\[h(x) = \left\{\begin{array}{ll}
		1 & \mathrm{se}\ \mathit{avg}_k(x) > 0.5 \\
		0 & \mathrm{altrimenti}                  \\
	\end{array}\right.\quad \mathrm{per}\ y_i = \{0,1\}\ (\mathit{targets})\]
Per l'attività di regressione utilizzare direttamente $\mathit{avg}$:\ media su K-nn.

\vspace{12pt}

\noindent Ancora una volta c'è un compromesso tra underfitting e overfitting dai possibili valori di K.\
Classica curva a forma di U dell'errore di test che si sposta tra a
\begin{itemize}
	\item caso estremamente flessibile ($K = 1$)
	\item fino a un modello molto rigido ($K = l$):\ 1 media per tutti i dati
\end{itemize}

\noindent Nearest Neighbor non fa nessuna ipotesi globale (per tutte le istanze), non c'è nessun modello che va a fare un fitting:\ non c'è un modello parametrico, è in contrasto al modello lineare che era parametrico con un insieme fissato di parametri $w$.\
Fa delle stime locali (mediante funzioni costanti locali), rispetto a un'approssimazione/stima lineare globale della funzione target (nello spazio dell'istanza).
\begin{flushleft}
	È un metodo pigro, basato sulla memoria sull'istanza e sulla \textbf{distanza}.
\end{flushleft}

\subsection{Limitazioni}

\subsubsection{Costo computazionale}

Si noti che K-NN effettua l'approssimazione locale alla funzione target per ogni nuovo esempio da prevedere:\ il costo computazionale è rimandato alla \textbf{fase di previsione}!
Inoltre c'è un alto costo di recupero:
\begin{itemize}
	\item Computazionalmente intensivo per ogni nuovo input:\ calcolo delle distanze dal campione di prova a tutti i vettori memorizzati e il tempo è proporzionale al numero di modelli memorizzati, anche se possono essere usati algoritmi di ricerca di prossimità ``ad-hoc'' da ottimizzare.
	\item Costo in spazio (tutti i dati vengono memorizzati).
\end{itemize}

\subsubsection{Curse of dimensionality}

Quando abbiamo molte variabili di input ($n$ grande), i metodi K-NN spesso falliscono a causa della \textbf{\textit{curse of dimensionality}}:\ quando la dimensionalità $d$ aumenta, il volume dello spazio aumenta così velocemente che i dati disponibili diventano scarsi, ovvero la quantità di dati necessari a supportare il risultato spesso cresce \textit{esponenzialmente} con la dimensionalità.

\textit{Features irrilevanti} (\textbf{\textit{Curse of Noisy}}):\ se il target dipende solo da poche delle molte features in $x$, si potrebbe recuperare un ``modello simile'' con la somiglianza dominata dal gran numero di features irrilevanti.\

\subsubsection{Metodi basati sulla distanza: considerazione e bias induttivo}

Approcci basati sulla \textbf{\textit{distanza}} o sulla \textbf{\textit{metrica}}:\ linea comune tra ad es.\ K-means, K-NN e approcci basati sul kernel; cosa c'è di \textbf{\textit{simile}}?\ Misura quando un paio di pattern sono \textit{simili}.\

Dare una metrica (ad esempio la metrica euclidea è stata assunta fino ad ora) pone un \textbf{\textit{bias}} rilevante sulla soluzione (molto più dei dettagli dell'algoritmo).\
La metrica dipende in genere dal dominio (per esempio due stringhe in una lingua, in biologia,\ \dots), è basata sulla conoscenza preliminare dell'esperto del dominio e in qualche modo è la stessa cosa dell'\textit{ingegnerizzazione delle features} per rappresentare il problema.

\section{Unsupervisioned learning}

\subsubsection{Clustering}

Partizione dei dati in cluster (sottoinsiemi di dati ``simili''):\ i modelli all'interno di un cluster valido sono più simili tra loro di quanto non lo siano a un modello appartenente a un cluster diverso.\
Inoltre è possibile individuare dei punti particolari chiamati centroidi (o prototipi), ossia i ``\textit{rappresentanti}''.

\subsubsection{Spazio delle ipotesi per il clustering}

Obiettivo:\ partizionamento ottimale della distribuzione sconosciuta nello spazio $x$ in regioni (cluster) approssimate da un centro o \textit{prototipo} del cluster.\
$H: x \rightarrow c(x)$, dove $x$ è un insieme di vettori di quantizzazione e $c(x)$ è il vettore che rappresenta il cluster.
\[\mathit{spazio\ continuo} \rightarrow \mathit{spazio\ discreto}\]
Un esempio di funzione di loss è una funzione che misura l'ottimalità del vettore di quantizzazione:\ una \textbf{funzione di loss} \textit{comune sarebbe la \textbf{distorsione dell'errore al quadrato}}:
\[ L(h(x_p)) = ||x_p - c(x_p)||^2\]
Il valore medio sulla distribuzione degli input è l'errore medio di distorsione o ricostruzione o quantizzazione.

\subsection{K-means}

Il $k$-means è l'algoritmo più semplice e più comunemente usato che impiega un \textbf{\textit{criterio di errore quadrato}}:\ è popolare perché è facile da implementare e, in generale, efficiente; tuttavia, ha diversi inconvenienti e \textbf{limitazioni} che vedremo dopo la presentazione dell'algoritmo.

\begin{enumerate}
	\item Sceglie $k$ centri del cluster in modo che coincidano con $k$ pattern scelti a caso o $k$ punti definiti in modo casuale all'interno dell'ipervolume contenente l'insieme di pattern.
	\item Assegna ogni modello al centro del cluster più vicino (il vincitore).
	\item Ricalcola i centri del cluster (centroide geometrico, cioè la media) utilizzando le attuali appartenenze al cluster.
	\item Se un criterio di convergenza non è soddisfatto, si ritorna al punto 2.
\end{enumerate}

\noindent Dati i centri del cluster $c_1,\dots,c_k$, per ogni $x$ il vincitore è (il prototipo più vicino):
\[i^*(x) = \mathrm{arg}\min_i ||x-c_i||^2\]
Adesso $x$ appartiene al cluster $i^*$.\
Per ogni cluster $i$ la nuova media (centroide) è:
\[c_i = \frac{1}{|\mathit{cluster}_i|}\sum_{j:x_j\in\mathit{cluster}_i} x_j\]

\subsubsection{Limitazioni}

\begin{itemize}
	\item Il numero dei cluster deve essere fornito.
	\item I minimi locali della loss $L(x)$ rendono il metodo dipendente dall'inizializzazione:\ è necessario eseguire più volte da diverse inizializzazioni casuali oppure inizializzare con un'euristica.\
	\item Può funzionare bene per cluster compatti e ipersferici.
	\item Nessuna proprietà di visualizzazione:\ $k$-means non consente di proiettare i dati in uno spazio più debole.
\end{itemize}

\section{Altri approcci per il preprocessing}

Tra la moltitudine di approcci che meritano di essere menzionati in questo corso:
\begin{itemize}
	\item \textbf{Riduzione della dimensionalità} (nel learning non supervisionato) \[\langle x_1,x_2,\dots,x_n\rangle\rightarrow\langle x'_1,x'_2,\dots,x'_n\rangle\quad n>n'\] dove le nuove features possono essere una combinazione di quelle originali.
	\item \textbf{Selezione delle features}:\ scegliamo un sottoinsieme di tutte le funzionalità in base alla conoscenza del dominio (\textit{features engineering}) oppure \textit{automaticamente} in base alla loro ridondanza o rilevanza nell'attività (la più informativa).\ Molti approcci:\ il problema è difficile come quello dell'apprendimento.
	\item \textbf{Rilevamento dei valori anomali}:\ trova valori di dati insoliti che non sono coerenti con la maggior parte delle osservazioni (ad esempio a causa di errori di misurazione anormali).
\end{itemize}

\section{Altri task}
\begin{itemize}
	\item \textbf{Reinforcement Learning} (apprendimento con critica positiva{\slash}nega\-tiva), viene usato nell'adattamento dei sistemi autonomi (soprattutto nella robotica):\ ``l'algoritmo apprende una politica di come agire data un'osservazione del mondo; ogni \textbf{azione} ha un certo impatto sull'ambiente e l'ambiente fornisce un feedback che guida l'algoritmo di apprendimento''.\ Invece della supervisione per ogni passaggio, abbiamo informazioni su vittorie/sconfitte (premi o punizioni) per lo stato finale; le azioni devono massimizzare la quantità di ricompense ricevute.\ L'apprendimento decide quali azioni sono state maggiormente responsabili di vittorie/sconfitte
	\item \textbf{Apprendimento semi-supervisionato}:\ combina esempi etichettati e non etichettati (in genere in numero maggiore) per generare una funzione o un classificatore appropriato.
	\item \textbf{Learn to rank}:\ (ad es.\ per i motori di ricerca) quando l'input è un insieme di oggetti e l'output desiderato è un ranking di quegli oggetti.
	\item \textbf{On-line learning}:\ nuovi esempi vengono appresi nel tempo.
	\item \textbf{Apprendimento del dominio strutturato e apprendimento relazionale}:\ il dominio di input e/o output può essere strutturato sotto forma di sequenze (segnali, serie temporali,\ \dots) o anche qualcosa di più complesso come alberi, grafi e reti.
\end{itemize}

\section{Altri modelli}
\textbf{Reti neurali}:\ per l'apprendimento sia supervisionato che non supervisionato.\
Sono vicine alla nostra visione di Linear Threshold Unit (LTU), in effetti la LTU è vicina all'unità di base di una rete neurale artificiale, il \textit{perceptron} e una rete neurale è una \textit{rete} di tali unità \textit{non lineari} con capacità di approssimazione universale.\

Approccio alla discesa in gradiente per l'apprendimento (backprogation).
\vspace{12pt}

.\noindent Lo strato interno (nascosto) con unità non lineari fornisce a NN la capacità di estrarre apprendendo una nuova rappresentazione dei dati (learning abstract features); le nuove rappresentazioni semplificano l'attività di classificazione all'ultimo strato.\
È una \textbf{\textit{basis expansion}} non lineare in $w$ adattativa in cui \textbf{\textit{vengono appresi}} i $\phi$ (dipendono da $w$), ma sfortunatamente porta anche a un problema di ottimizzazione non lineare.\
Usa una rappresentazione distribuita (contro simbolica):\ la somiglianza può essere trattata più facilmente da vettori di valori reali; approccio molto flessibile per attività non lineari, sia per attività di classificazione che di regressione (capacità di approssimazione universale).\
\vspace{12pt}

\noindent Aspetti comuni del \textbf{Deep Learning} (DL) tra diversi approcci:
\begin{itemize}
	\item \textit{Più strati} di unità di elaborazione non lineari
	\item Apprendimento supervisionato o non supervisionato delle \textit{rappresentazioni delle features} in ogni livello, con i livelli che formano una gerarchia dalle caratteristiche/rappresentazioni di livello basso a quelle di alto livello (\textit{diversi livelli di \textbf{astrazione}}).
\end{itemize}



\end{document}

