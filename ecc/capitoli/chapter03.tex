\section{Linguaggi \textit{\small FOR} e \textit{\small WHILE}}

Introduciamo adesso un altro formalismo per esprimere funzioni, certamente più familiare:\ un linguaggio di programmazione, o meglio il suo nucleo, in forma essenziale.\
Si tratta di un linguaggio imperativo semplicissimo, che opera solo sui numeri naturali e sui valori di verità, con strutture di controllo che sono presenti in ogni linguaggio di programmazione.

\subsection*{Sintassi (astratta)}

\begin{table}[H]
    \centering
    \begin{tabular}{c l l}
        $E\rightarrow$  & $n \mid x \mid  E_1 + E_2 \mid E_1 \times E_2 \mid E_1 - E_2$                                            & \small \footnotesize Espr.\ Aritmetiche \\
        $B\rightarrow$  & $b \mid E_1 < E_2 \mid \lnot B \mid B_1 \lor B_2$                                                        & \footnotesize Espr.\ Booleane           \\
        $C \rightarrow$ & $\texttt{skip} \mid x:=E \mid C_1;C_2 \mid \mathtt{if}\ B\ \mathtt{then}\ C_1\ \mathtt{else}\ C_2 \mid $ & \footnotesize Comandi                   \\
                        & $\mathtt{for}\ x=E_1\ \mathtt{to}\ E_2\ \mathtt{do}\ C \mid \mathtt{while}\ B\ \mathtt{do}\ C$           &                                         \\
    \end{tabular}
\end{table}

dove $n \in \mathbb{N}$ (i numeri naturali), $x \in \mathit{Var}$ (insieme numerabile di variabili) e $b\in \mathit{Bool} = \{\mathit{tt}, \mathit{ff}\}$\footnote{Poiché non vi sono comandi di lettura né di scrittura su periferiche, assumeremo nel seguito che i programmi abbiano già la memoria inizializzata con i dati di ingresso (cioè che abbiamo una $\sigma$ iniziale, vedi sotto) e che i risultati finali siano leggibili in memoria.} (valore di verità).\

\vspace{12 pt}

\noindent Chiameremo \textit{\footnotesize WHILE} il linguaggio definito dalla grammatica BNF definita sopra\footnote{Di solito si indica con \textit{\scriptsize WHILE} il linguaggio definito dalla grammatica di cui sopra, privata della produzione relativa al comando $\mathtt{for}\ x = E_1\ \mathtt{to}\ E_2\ \mathtt{do}\ C$.\ Avere o non avere comandi di questo tipo non crea alcun problema, in quanto essi sono ``esprimibili'' mediante un oppotuno comando di tipo $\mathtt{while}\ B\ \mathtt{do}\ C$.}; chiameremo invece \textit{\footnotesize FOR} il linguaggio risultante dall'omissione del comando $\mathtt{while}\ B\ \mathtt{do}\ C$ nella definizione della sintassi.\

\subsection*{Semantica (ovvero calcolare è dimostrare)}

Prima di definire la dinamica del nostro linguaggio abbiamo bisogno di un paio di nozioni ausiliarie.\
Rappresentiamo la memoria tramite una funzione (da sottoinsiemi \textit{finiti} di \textit{Var} nei numeri naturali, tanto in un programma ci possono essere solo un numero finito di variabili)
\[\sigma : \mathit{Var} \rightarrow \mathbb{N}\]
e definiamo l'operazione di \textit{aggiornamento} tramite la funzione, o meglio il funzionale a tre argomenti
\[ -[-/-]:(\mathit{Var}\rightarrow \mathbb{N}) \times \mathbb{N} \times \mathit{Var} \rightarrow (\mathit{Var} \rightarrow \mathbb{N}) \]
definito come \qquad
$\sigma[n/x](y) = \left\{\begin{array}{l l}
        n         & \mathrm{se}\ y=x    \\
        \sigma(y) & \mathrm{altrimenti} \\
    \end{array} \right.$

\vspace{12pt}

\noindent Il significato, o \textit{semantica} di un'espressione aritmetica è data dalla seguente funzione, definita ovunque.\
Seguendo la tradizione scriveremo il suo argomento principale, l'espressione aritmetica, tra le parentesi $[\![$ e $]\!]$, cui viene giustapposto il secondo argomento, cioè la memoria in cui l'espressione va valutata.\
\[\mathcal{E} [\![-]\!]- : E \times (\mathit{Var} \rightarrow \mathbb{N})\rightarrow\mathbb{N}\]

\begin{table}[H]
    \centering
    \begin{tabular}{l c l}
        $\mathcal{E} [\![n]\!]\sigma$             & = & $n$                                                                                  \\
        $\mathcal{E} [\![x]\!]\sigma$             & = & $\sigma(x)$                                                                          \\
        $\mathcal{E} [\![E_1+E_2]\!]\sigma$       & = & $\mathcal{E} [\![E_1]\!]\sigma\ \mathit{pi\acute{u}}\ \mathcal{E} [\![E_2]\!]\sigma$ \\
        $\mathcal{E} [\![E_1\times E_2]\!]\sigma$ & = & $\mathcal{E} [\![E_1]\!]\sigma$ \textit{per} $\mathcal{E} [\![E_2]\!]\sigma$         \\
        $\mathcal{E} [\![E_1 - E_2]\!]\sigma$     & = & $\mathcal{E} [\![E_1]\!]\sigma$ \textit{meno} $\mathcal{E} [\![E_2]\!]\sigma$        \\
    \end{tabular}
\end{table}

\noindent dove \textit{più, meno, per} sono ``vere'' funzioni da coppie di naturali a numeri naturali, a differenza di $+,-,\times$ che sono solo simboli (\textit{tokens}) del nostro linguaggio di programmazione.\
Poiché i nostri dati sono solo numeri naturali, la funzione \textit{meno} che impieghiamo è la funzione, di solito chiamata \textit{meno limitato}, che si comporta come aspettato quando il minuendo è maggiore o uguale al sottraendo e restituisce 0 altrimenti.\

Invece, nella prima equazione, non abbiamo fatto distinzione tra i \textit{numerali} e i \textit{numeri naturali} e li abbiamo rappresentati con gli stessi simboli; pur non essendo del tutto corretto, ciò viene comunemente fatto e non dovrebbe creare problemi perché di solito è chiaro quando $n$ rappresenta un numero e quando un simbolo del linguaggio.

\medskip

\noindent La semantica di un'espressione booleana è data dalla seguente funzione ovunque definita, che usa la $\mathcal{E}$ appena introdotta.
\[\mathcal{B} [\![-]\!]- : B \times (\mathit{Var} \rightarrow \mathbb{N})\rightarrow\mathit{Bool}\]

\begin{table}[H]
    \centering
    \begin{tabular}{l c l}
        $\mathcal{B} [\![t]\!]\sigma$            & = & $\mathit{tt}$                                                                   \\
        $\mathcal{B} [\![f]\!]\sigma$            & = & $\mathit{ff}$                                                                   \\
        $\mathcal{B} [\![E_1 < E_2]\!]\sigma$    & = & $\mathcal{E} [\![E_1]\!]\sigma\ \mathit{minore}\ \mathcal{E} [\![E_2]\!]\sigma$ \\
        $\mathcal{B} [\![\lnot B]\!]\sigma$      & = & $\mathit{not}\ \mathcal{B} [\![B]\!]\sigma$                                     \\
        $\mathcal{B} [\![B_1 \lor B_2]\!]\sigma$ & = & $\mathcal{B} [\![B_1]\!]\sigma\ \mathit{or}\ \mathcal{B} [\![B_2]\!]\sigma$     \\
    \end{tabular}
\end{table}

\noindent dove \textit{minore}, \textit{not}, \textit{or} sono ``vere'' funzioni.\

\medskip

\noindent Lo stile di definizione seguito per dar semantica alle espressioni va sotto il nome di stile \textit{denotazionale} e si propone di associare una funzione a ciascun operatore, o più in generale, a ciascun costrutto del linguaggio.\
La definizione di un costrutto composto viene data in termini dei suoi componenti, e per questa proprietà a volte questo stile viene anche detto \textit{composizionale}.\
Adesso daremo la semantica dei comandi usando un approccio \textit{operazionale}, guidato dalla sintassi.\
In effetti, andremo a definire una sorta di macchina astratta, che procede per passi discreti, apportando modifiche discrete sulle sue configurazioni, o stati, che possiamo immaginare come coppie $\langle \mathit{comando}, \sigma\rangle$.\
Questa macchina astratta, che definisce in modo intensionale la semantica di un linguaggio di programmazione o di un sistema, va sotto il nome di \textit{sistema di transizioni} e formalmente è una coppia $(\Gamma, \rightarrow)$ dove

\begin{itemize}
    \item $\Gamma$ è la classe delle \textit{configurazioni} (nel nostro caso la classe delle coppie $\langle c \in C, \sigma\rangle$, dove $\sigma$ è definita per ogni variabile che appaia in \textit{comando}; la coppia con il comando $c$ vuoto ($\neq \mathtt{skip}$) è abbreviata da $\sigma$).
    \item $\rightarrow\ \subseteq \Gamma \times \Gamma$ è una funzione, detta di \textit{transizione}.\
\end{itemize}

\noindent Noi seguiremo un approccio detto SOS (Structural Operational Semantics) (o \textit{small-step}), in cui ciascuna transizione \[\langle c, \sigma\rangle \rightarrow \langle c', \sigma' \rangle\]
rappresenta un singolo passo di computazione, ovvero il fatto che la macchina ha attraversato lo stato $\langle c, \sigma\rangle$ e transisce nello stato $\langle c', \sigma'\rangle$ a causa di una sua certa attività.
\footnote{Un approccio alternativo consiste nel definire una semantica \textit{naturale} o \textit{big-step}, in cui le transizioni hanno tutte la forma \[\langle c \in C, \sigma\rangle \Rightarrow \sigma'\] cioè $\sigma'$ memorizza le modifiche prodotte dall'intera esecuzione del comando $c$ nella memoria iniziale $\sigma$.\ Un'intera computazione viene allora vista come la deduzione della transizione in questione.\ Potremmo però trovarci nell'impossibilità di dimostare l'esistenza di una transizione per uno stato:\ in questo caso avremmo o una situazione di stallo, o una situazione di non-terminazione della computazione, quando la deduzione non sia finita.}

In maniera del tutto simile a quanto fatto con le macchine di Turing, si
può allora definire una \textit{computazione} come la chiusura riflessiva e transitiva della funzione di transizione, formalmente, una \textit{computazione} è \[\langle c, \sigma\rangle \rightarrow^* \langle c', \sigma' \rangle\]
Una computazione termina con successo, o converge, se $\langle c, \sigma\rangle \rightarrow^* \sigma'$.

Si noti che la semantica operazionale di un programma è allora data dal grafo $(\Gamma, \rightarrow$), mentre le computazioni danno una descrizione dei passi necessari al calcolo con granularità fine quanto si voglia.\

In questo approccio, l'attività svolta dalla macchina quando transisce da una configurazione nella successiva viene a volte resa esplicita, etichettando la funzione di transizione e ottenendo così i \textit{sistemi di transizioni etichettati}.\
\footnote{Ovviamente questo può venir fatto anche nell'approccio naturale, componendo le etichette e rappresentando così le computazioni attraverso una sequenza di etichette.}
Un'ulteriore estensione piuttosto comune riguarda l'arricchimento della coppia ($\Gamma, \rightarrow$) con un insieme di configurazioni terminali, ovvero di stati finali (a volte si aggiunge alla coppia anche una configurazione iniziale, da cui si assume partano tutte le computazioni).\
Nel nostro caso, il sistema di transizioni che useremo avrà come configurazioni finali le coppie formate da un comando vuoto e una memoria $\sigma$, che abbiamo già abbreviato con $\sigma$.

È evidente come gli stili di definizione denotazionale e operazionale siano differenti, in quanto non c'è modo immediato di comporre il ``risultato'' di un'esecuzione con quello di un'altra esecuzione.\
Tuttavia, anche l'approccio operazionale è in larga misura composizionale, per il modo che seguiamo nel definire la classe delle transizioni.\
Infatti, di seguito introduciamo un insieme di assiomi e regole di inferenza, che inducendo sulla sintassi dei comandi, ci permettono di dedurre tutte e sole le transizioni (ovvero prendiamo la minima classe che contiene tutte le instanze degli assiomi ed è chiusa rispetto alle regole di inferenza).

\[\frac{-}{\langle \mathtt{skip}, \sigma\rangle \rightarrow \sigma}\quad \qquad \frac{-}{\langle x:=E, \sigma\rangle \rightarrow \sigma[n/x]},\ \mathrm{se}\ \mathcal{E}[\![E]\!]\sigma = n\]
\[\frac{\langle C_1, \sigma\rangle \rightarrow \langle C'_1, \sigma'\rangle}{\langle C_1; C_2, \sigma\rangle \rightarrow \langle C'_1; C_2,\sigma'\rangle}\quad \qquad \frac{\langle C_1, \sigma\rangle \rightarrow \sigma'}{\langle C_1; C_2, \sigma\rangle \rightarrow \langle C_2, \sigma'\rangle}\]
\[\frac{-}{\langle \mathtt{if}\ B\ \mathtt{then}\ C_1\ \mathtt{else}\ C_2, \sigma\rangle \rightarrow \langle C_1, \sigma\rangle},\ \mathrm{se}\ \mathcal{B}[\![B]\!]\sigma = \mathit{tt}\]
\[\frac{-}{\langle \mathtt{if}\ B\ \mathtt{then}\ C_1\ \mathtt{else}\ C_2, \sigma\rangle \rightarrow \langle C_2, \sigma\rangle},\ \mathrm{se}\ \mathcal{B}[\![B]\!]\sigma = \mathit{ff}\]
\[\frac{-}{\langle \mathtt{for}\ i=E_1\ \mathtt{to}\ E_2\ \mathtt{do}\ C, \sigma\rangle \rightarrow \langle i:= n_1;\ C; \mathtt{for}\ i = n_1 + 1\ \mathtt{to}\ n_2\ \mathtt{do}\ C, \sigma \rangle}\]
\[\mathrm{se}\ \mathcal{B}[\![E_2 < E_1]\!]\sigma = \mathit{ff} \land \mathcal{E}[\![E_1]\!]\sigma = n_1 \land \mathcal{E}[\![E_2]\!]\sigma = n_2\]
\[\frac{-}{\langle \mathtt{for}\ i=E_1\ \mathtt{to}\ E_2\ \mathtt{do}\ C, \sigma\rangle \rightarrow \sigma},\ \mathrm{se}\ \mathcal{B}[\![E_2 < E_1]\!]\sigma = \mathit{tt}\]
\[\frac{-}{\langle \mathtt{while}\ B\ \mathtt{do}\ C, \sigma\rangle \rightarrow \langle \mathtt{if}\ B\ \mathtt{then}\ C;\ \mathtt{while}\ B\ \mathtt{do}\ C\ \mathtt{else\ skip}, \sigma \rangle}\]
