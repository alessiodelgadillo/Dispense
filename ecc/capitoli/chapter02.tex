\section{Macchine di Turing}

Introdurremo di seguito uno dei formalismi più importanti e più diffusi per esprimere algoritmi:\ le Macchine di Turing, che ricordano con straordinaria verosomiglianza i comuni elaboratori di Von Neumann (o a programma memorizzabile) cui siamo abituati.\
Ve ne sono moltissime definizioni, che differiscono per varianti spesso irrilevanti; la definizione originale prevede un esecutore umano e fu introdotta da Alan Turing nel 1936, ben prima che ci fossero dei computer funzionanti.\

\begin{definition}[Macchina di Turing]
    \label{Macchina di Turing}
    Si dice Macchina di Turing (in breve MdT) una quadrupla \[M=(Q, \Sigma, \delta, q_0)\] dove:

    \begin{itemize}
        \item $Q (\ni q_i)$ è l'insieme \textit{finito} degli \textit{stati}, con $h \not\in Q$ (con lo stato speciale $h$ indicheremo il caso di arresto ``corretto'' di un calcolo di $M$).\ Attenzione che \textit{computazione terminata} $\neq$ \textit{la macchina si arresta} poiché la macchina si può arrestare senza aver raggiunto la configurazione finale (ad esempio se la macchina entra in uno stato $q_k$ e la funzione $\delta$ non è definita per quello stato).
        \item $\Sigma (\ni \sigma, \sigma', \dots)$ è l'insieme \textit{finito} dei simboli (\textit{alfabeto}) con
              \begin{itemize}
                  \item $\# \in \Sigma$ rappresenta il \textit{carattere bianco};
                  \item $\triangleright \in \Sigma$ (\textit{respingente}) rappresenta l'inizio della stringa;
                  \item $\{L, R, -\} \notin \Sigma$ indicano rispettivamente ``scorri a sinistra'', ``scorri a destra'' e ``resta fermo'' sul nastro.
              \end{itemize}
        \item $q_0 \in Q$ è lo \textit{stato iniziale}.
        \item $\delta \subseteq (Q \times \Sigma) \times (Q \cup \{h\}) \times \Sigma \times \{L,R,-\}$ è la \textit{relazione di transizione}, tale che
              \[\mathrm{se}\ (q, \triangleright, q', \sigma, D) \in \delta\ \mathrm{allora}\  \sigma = \triangleright, D = R\]
              Questa condizione dice che il carattere corrente, ovvero il cursore, non può mai trovarsi a sinistra della \textit{marca di inizio stringa}, $\triangleright$.\
              Più intuitivamente se $\delta (q, \triangleright) = (q', \sigma, D)$ allora $\sigma = \triangleright, D = R$, secondo quanto stipulato sotto.

              In questa parte del corso restringeremo la relazione $\delta$ in modo che sia una funzione rispetto ai suoi primi due argomenti, imponendo cioè che, per ogni coppia di quintuple:
              \[(q,\sigma,q',\sigma',D'),(q,\sigma,q'',\sigma'',D'')\in\delta \Rightarrow q'=q'', \sigma'=\sigma'', D'=D''\]
              Grazie a questa condizione possiamo scrivere $\delta(q,\sigma)=(q', \sigma', D')$ al posto della quintupla $(q,\sigma,q',\sigma',D')$.\
              Se volessimo essere più precisi, una relazione è una corrispondenza tra gli elementi di due insiemi; una funzione invece è una particolare relazione che associa ad ogni elemento del dominio uno e un solo elemento del codominio.
    \end{itemize}
\end{definition}

\noindent È facile verificare che la condizione (i) dell'idea intuitiva di algoritmo elencata dianzi è soddisfatta, così come lo è anche la prima parte della condizione (ii).\
Esse richiedono che i programmi siano finiti e che le possibili istruzioni, operanti su dati discreti, siano in numero finito.\
Infatti, ogni macchina ha un numero finito di possibili istruzioni, poiché gli insiemi $Q$ e $\Sigma$ sono finiti; di conseguenza la sua funzione di transizione $\delta$ contiene un numero finito di elementi.\
Inoltre, i dati su cui opera una MdT sono stringhe $w$ di caratteri appartenenti a $\Sigma$, in simboli $w \in \Sigma^*$.\
Più precisamente $\Sigma^*$ contiene la stringa vuota $\epsilon$ e per tutte le stringhe $w \in \Sigma^*$ e tutti i caratteri $a \in \Sigma^*$ contiene la stringa $a\cdot w$, dove ``$\cdot$'' è l'operazione associativa (quasi sempre omessa) di giustapposizione tra caratteri esteso alle stringhe; brevemente, $\Sigma^*$ è il monoide libero generato da $\Sigma$ avente come unica operazione interna associativa ``$\cdot$'' ($\forall w, w', w'' \in \Sigma^*\ \mathrm{vale}\ w \cdot (w' \cdot w'') = (w \cdot w') \cdot w''$) e con identità destra e sinistra $\epsilon$ ($\forall w \in \Sigma^*\ \mathrm{vale}\ \epsilon \cdot w = w = w \cdot \epsilon$).

Esiste una variante delle MdT detta \textit{non-deterministica} in cui non si richiede che $\delta$ sia una funzione.\
Nella seconda parte del corso ne daremo una definizione formale, e ricorderemo che ha la ``stessa potenza espressiva'' della MdT definita sopra.\
Più avanti useremo, introducendola in modo intuitivo, anche la variante in cui si hanno più nastri, la cui definizione si trova nella definizione \ref{MdT_k-nastri}.\
Il suo potere espressivo non cambia, così come non cambia quello di tutte le altre varianti, per esempio quelle macchine con un nastro infinito anziché semi-infinito, oppure con più di un nastro, oppure ancora quelle che possono o scrivere o spostarsi, o hanno altre diavolerie ``ragionevoli''.\
Questa robustezza rafforza la sensazione che il modello sia azzeccato.\

\begin{example}[Una macchina che qualcosa fa]{\label{Simple MdT}}

    La tabella riportata sotto rappresenta la funzione di transizione della MdT $M = (Q, \Sigma, \delta, q_0)$, con $Q=\{q_0,q_1\}$ e $\Sigma=\{\#,\triangleright,a\}$.\

    \begin{table}[H]
        \centering
        \begin{tabular}{ |c|c|c|c| }
            \hline
            $q$   & $\sigma$         & $\delta(q,\sigma)$     \\\hline\hline
            $q_0$ & $\triangleright$ & $q_0,\triangleright,R$ \\
            $q_0$ & $\#$             & $h,\#,-$               \\
            $q_0$ & $a$              & $q_1,\#,L$             \\
            $q_1$ & $\#$             & $q_0,\#,L$             \\
            $q_1$ & $a$              & $q_0,a,-$              \\
            \hline
        \end{tabular}
    \end{table}

    \noindent In seguito, nel definire una MdT, ometteremo per brevità la definizione degli insiemi $Q$ e $\Sigma$ degli stati e dei simboli, i cui elementi possono essere dedotti guardando la tabella che definisce la funzione di transizione $\delta$.\
\end{example}

\noindent Passiamo ora a descrivere la dinamica delle macchine di Turing, cioè il loro comportamento quando operano su una stringa di caratteri memorizzati sul nastro.\
In altre parole, vogliamo definire cos'è una computazione di una macchina.\
Intuitivamente, una computazione di una MdT è una successione di configurazioni, ognuna ottenuta dalla precedente in accordo con la definizione della funzione di transizione $\delta$.\
Ci manca la definizione di configurazione, che verrà data tra poche righe.\
Per il momento, e in modo del tutto informale, indichiamo la coppia \textit{Stato}/\textit{Nastro} una \textit{configurazione istantanea} della macchina; dove \textit{Stato} è lo stato in cui si trova la macchina e \textit{Nastro} è una porzione ``abbastanza lunga'' e \textit{finita} del nastro che contiene almeno la sua parte non bianca.\

La seguente tabella rappresenta informalmente una computazione della MdT introdotta nell'esempio \ref{Simple MdT}, che si arresta con successo in sette passi.\
Nelle configurazioni, il carattere sottolineato nella parte \textit{Nastro} sta a indicare la posizione corrente del cursore, o \textit{testina}.\

\begin{table}[H]
    \centering
    \begin{tabular}{|l|l|}
        \hline
        Configurazione                                                & Azione effettuata                              \\\hline\hline
        $q_0$/$\triangleright \#\#a\#a\underline{a}\# \rightarrow$    & ``cancella'' $a$ cambia stato e vai a sinistra \\
        $q_1$/$\triangleright \#\#a\#\underline{a}\#\# \rightarrow$   & non scrive, non si sposta, cambia stato        \\
        $q_0$/$\triangleright \#\#a\#\underline{a}\#\# \rightarrow$   & ``cancella'' $a$ cambia stato e vai a sinistra \\
        $q_1$/$\triangleright \#\#a\underline{\#}\#\#\# \rightarrow$  & lascia \#, cambia stato, va a sinistra         \\
        $q_0$/$\triangleright \#\#\underline{a}\#\#\#\# \rightarrow$  & ``cancella'' $a$ cambia stato e vai a sinistra \\
        $q_1$/$\triangleright \#\underline{\#}\#\#\#\#\# \rightarrow$ & lascia \#, cambia stato, va a sinistra         \\
        $q_0$/$\triangleright \underline{\#}\#\#\#\#\#\# \rightarrow$ & si arresta                                     \\
        $h$/$\triangleright \underline{\#}\#\#\#\#\#\# \rightarrow$   &                                                \\\hline
    \end{tabular}
\end{table}

\noindent Più precisamente, una \textit{configurazione} $C$ di una MdT è una quadrupla
\[(q,u,\sigma,v) \in (Q \cup \{h\}) \times \Sigma^* \times \Sigma \times \Sigma^F\]
dove $q$ è lo stato corrente, $u$ è la stringa di caratteri a sinistra del simbolo corrente $\sigma$ e $v$ è quella dei caratteri alla sua destra.\
Per semplicità scriveremo $(q, u\underline{\sigma}v)$ al posto di $(q,u,\sigma,v)$, così come abbiamo già fatto sopra.\
Poiché la stringa $w\sigma v$ deve essere finita (e poiché ora)\footnote{Nella seconda parte del corso terremo traccia di tutte le caselle visitate dalla MdT, ottenendo così una stima dello spazio necessario per quella computazione.} non ci interessa considerare i simboli \# a destra del simbolo $\sigma_n \neq \#$ più a destra nel nastro (ovvero della porzione $v$), indichiamo con $\epsilon$ la stringa vuota e conveniamo che $\# \cdot \epsilon = \epsilon$ e $\sigma \cdot \epsilon = \sigma$, se $\sigma \neq \#$.\
Definiamo allora $\Sigma^F = \Sigma^* \cdot (\Sigma \backslash \{\#\}) \cup \{\epsilon\}$.\
Quindi scriviamo $v = \sigma_0\sigma_1\dots\sigma_n$, con $\sigma_n \neq \#$, al posto della stringa infinita $\sigma_0\sigma_1\dots\sigma_n\#\#\dots\#\dots$.\
Si noti tuttavia che può benissimo darsi che $\sigma_i = \#$ per qualche $i<n$ e anche che il carattere corrente $\sigma$ può essere \#; inoltre la stringa $u$ può essere vuota solo quando il carattere corrente $\sigma$ è la marca di inizio stringa $\triangleright$ per la seconda condizione sulla funzione di transizione $\delta$.\
Con questa convenzione ogni stringa è finita.

\vspace{12 pt}

\noindent Il frammento iniziale della computazione di prima viene rappresentato come:
\begin{table}[H]
    \begin{tabular}{l}
        $(q_0, \triangleright\#\#a\#a, a, \epsilon) \rightarrow (q_1, \triangleright\#\#a\#, a,\epsilon) \rightarrow$      \\
        $(q_0, \triangleright, \#\#a\#, a,\epsilon) \rightarrow (q_1, \triangleright\#\#a, \#,\epsilon) \rightarrow \dots$ \\
    \end{tabular}
\end{table}

\noindent Un altro esempio di computazione della macchina definita nell'es.\ \ref{Simple MdT} è:
\begin{table}[H]
    \begin{tabular}{l}
        $(q_0, \triangleright\#, a, a) \rightarrow (q_1, \triangleright, \#, \#a) \rightarrow (q_0, \epsilon, \triangleright, \#\#a) \rightarrow$ \\
        $(q_0, \triangleright, \#, \#a) \rightarrow (h, \triangleright, \#, \#a)$                                                                 \\
    \end{tabular}
\end{table}

\noindent Per non eccedere nella pignoleria, non sempre utilizzeremo la versione di configurazione nella forma definita sopra, ma ci riterremo liberi di scrivere $(q,w)$ quando non interessi sapere dove si trova il cursore (v.\ la definizione di computazione più sotto), o di usare altre convenzioni quando il loro significato sia chiaro dal contesto.

\vspace{12pt}

\noindent Formalmente un \textit{passo di computazione} è definito per casi nel modo seguente, intendendo che le quadruple che vi appaiono siano configurazioni (ricordandosi che $q \neq h$ e usando $q'$ per indicare, oltre a uno stato in $Q$, anche $h$; indicando con $a,b,c$ elementi generici di $\Sigma$; e ricordando inoltre, soprattutto nel caso (ii), che se $b = \#$ e $ v = \epsilon$ allora $bv = \epsilon$ e infine che nel caso (ii) se $a = \triangleright$ allora anche $b= \triangleright$).\

\begin{table}[H]
    \centering
    \begin{tabular}{l c l l}
        i)   &     & $(q, uav) \rightarrow (q', ubv)$    & se $\delta(q, a)=(q', b, -)$ \\
        ii)  &     & $(q, ucav) \rightarrow (q', ucbv)$  & se $\delta(q, a)=(q', b, L)$ \\
        iii) & (a) & $(q, uacv) \rightarrow (q' , ubcv)$ & se $\delta(q, a)=(q', b, R)$ \\
             & (b) & $(q, ua) \rightarrow (q', ub\#)$    & se $\delta(q, a)=(q', b, R)$ \\
    \end{tabular}
\end{table}

\noindent Ciascun passo ha un effetto limitato sulle configurazioni, come richiesto dalla seconda parte del punto (ii) nella idea intuitiva di algoritmo.\
Inoltre, un singolo passo dipende \textit{unicamente da un solo simbolo}, quello corrente,\footnote{Si noti che l'unica maniera per ispezionare una parte non finita del nastro sarebbe quella di accedere a tutto il nastro a destra del cursore.} e da \textit{un solo stato}, quello corrente (cf.\ i punti (iii) e (iv) richiesti dall'intuizione).\
Si noti che sia per determinare la regola da applicare che per applicarla diamo per primitiva la capacità dell'esecutore di ricorrere al ``pattern matching''.

\medskip

\noindent Una \textit{computazione} è una successione finita di passi \[(q_0,\triangleright w)\rightarrow^*(q', w')\] dove $\rightarrow^*$ è la chiusura riflessiva e transitiva di $\rightarrow$.\
Come d'abitudine, se vi sono $n$ passi, la computazione è lunga $n$ e la scriviamo così
\[(q_0,\triangleright w)\rightarrow^n(q', w')\]
Diremo invece che la computazione $(q_0, w) \rightarrow^* (q', w')$ \textit{termina} (converge, $\downarrow$) su $w$ sse $q' = h$, e che \textit{non termina} (diverge, $\uparrow$) sse per ogni $q', w'$ tali che $(q_0, w) \rightarrow^* (q', w')$ esistono $q'', w''$ tali che $(q', w') \rightarrow (q'', w'')$.\
Se fosse necessario precisare che la computazione in esame è una di quelle di una particolare macchina $M$, scriveremo $\rightarrow^*_M$; con $M(w)$ esprimeremo che la macchina $M$ inizia la sua computazione dalla configurazione $(q_0, \underline{\triangleright} w)$, cioè avendo come dato iniziale la stringa $w$, ovvero che \textit{applichiamo} $M$ a $w$.

\medskip

\noindent C'è un limite ai passi di computazione o allo spazio necessario a contenerne i dati (punto (v) della definizione intuitiva di algoritmo)?\
\textbf{NO}, come si vede dal seguente esempio.\

\begin{example}[Una macchina che non converge per nessun ingresso]

    \begin{table}[H]
        \centering
        \begin{tabular}{|l|l|l|}
            \hline
            $q$   & $\sigma$         & $\delta$                 \\\hline\hline
            $q_0$ & $\triangleright$ & $q_0, \triangleright, R$ \\
            $q_0$ & $a$              & $q_0, a, R$              \\
            $q_0$ & $\#$             & $q_0, \#, R$             \\\hline
        \end{tabular}
    \end{table}

    Un esempio di computazione non terminante della macchina di sotto è
    \[(q_0, \underline{\triangleright} a\#a\#) \rightarrow (q_0, \triangleright a\#\underline{a}\#) \rightarrow^* (q_0, \triangleright a\#a\#\dots\underline{\#}\#) \rightarrow \dots\]

\end{example}
