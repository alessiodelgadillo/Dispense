\chapter{Il Livello Hardware:\ Reti Logiche}

\section{Reti combinatorie}

Una rete combinatoria è una rete logica che ha $n$ ingressi binari $x_1, \dots, x_n$ ed $m$ uscite binarie $z_1, \dots, z_n$.\
A ciascuna combinazione dei valori degli ingressi corrisponde una e una sola combinazione dei valori delle uscite:\ la corrispondenza definisce la funzione implementata dalla rete combinatoria.

$x_1, \dots, x_n$ e $z_1, \dots, z_n$ sono dette \textbf{\textit{variabili logiche}}, di ingresso e di uscita rispettivamente.\
Tutte le combinazioni possibili delle variabili logiche sono dette \textbf{\textit{stati}}, di ingresso (in numero di $2_n$) e di uscita (in numero di $2_m$) rispettivamente.

\subsection{Elementi di algebra booleana della commutazione}

L'algebra della commutazione è un sistema algebrico in cui ogni variabile può assumere uno solo tra due valori, 0 e 1, nel quale sono applicate alle variabili le operazioni binarie di \textit{moltiplicazione logica} e \textit{somma logica} e l'operazione unaria di \textit{complementazione} o \textit{negazione}.

\begin{figure}[H]
    \centering
    \includegraphics[width=\textwidth]{immagini/Proprietà.png}
    \caption*{Proprietà notevoli}
\end{figure}

\noindent Una funzione logica può essere rappresentata sia da una \textbf{tabella di verità}, sia un'\textbf{espressione algebrica}.\
Tuttavia, \textit{mentre c'è una sola tabella per ogni funzione, vi sono molte espressioni che possono rappresentare una stessa funzione}.\
Due espressioni che rappresentano la stessa funzione sono chiamate \textit{equivalenti}.

Chiamiamo \textit{lettera} una variabile affermata o complementata e \textit{termine} un prodotto di lettere (termine prodotto) una somma di lettere (termine somma) che compare in un'espressione.\
Chiamiamo \textit{mintermine} un termine prodotto e \textit{maxtermine} un termine somma che contengono un numero di lettere di variabili distinte uguale al numero \textit{n} delle variabili della funzione rappresentata dall'espressione.

Indichiamo con SP e PS un'espressione in Somma di termini Prodotto (o Somma di Prodotti) e Prodotti di termini Somma (o Prodotti di Somma), rispettivamente.\
Chiamiamo infine \textit{forma normale} la forma di un'espressione SP.

Chiameremo \textbf{forma canonica} di una funzione in n variabili l'espressione in SP o Ps in cui ogni termine è un mintermine o un maxtermine.

\subsection{Definizione di reti combinatorie, loro formalizzazione e comportamento}

Una volta ricavata l'espressione logica dalla tabella di verità, è immediato realizzare lo \textit{schema logico} utilizzando \textit{i componenti hardware elementari}, detti anche \textbf{\textit{porte logiche}}, \texttt{AND}, \texttt{OR}, \texttt{NOT}.

Partendo da un'espressione logica in forma SP, la rete combinatoria ottenuta è ``\textbf{\textit{a due livelli di logica}}'', con questo intendendo che esiste un primo livello di porte AND operanti in parallelo, seguite da una porta OR finale.

\subsection{Specifica di reti combinatorie}

Le seguenti sono le specifiche di alcune reti combinatorie che supporremo standard, o primitive, cioè componenti che è possibile usare (al pari delle porte AND, OR, NOT) come blocchi basici nella progettazione di strutture più complesse:

\begin{itemize}
    \item \textbf{\textit{confrontatore}} ($\oplus$) a due ingressi $x$, $y$ ed una uscita $z$ ($z$ è vero se $x$, $y$ sono diversi):\ $z= x \oplus y =\ \mathbf{not}\ (x=y) \rightarrow$ \textit{rete a due livelli di logica}.
    \item \textbf{\textit{commutatore}} (K) a due ingressi primari $x$, $y$, un ingresso secondario di controllo $\alpha$, ed una uscita $z$ ($z$ assume il valore di $x$ o $y$ a seconda che $\alpha$ sia falso o vero rispettivamente):\ $z = \mathbf{if\ not}\ \alpha\ \mathbf{then}\ x\ \mathbf{else}\ y \rightarrow$ \textit{rete a due livelli di logica}
    \item \textbf{\textit{selezionatore}}, o selettore (S), con un ingresso primario $x$, un ingresso secondario o di controllo $\alpha$, e due uscite $z_1, z_2$ (se $\alpha$ è falso $z_1$ assume il valore di $x$ e $z_2$ vale falso (0), se $\alpha$ è vero $z_1$ vale falso (0) e $z_2$ assume il valore di $x$):\ $\mathbf{if\ not}\ \alpha\ \mathbf{then}\ (z_1 = x, z_2 = 0)\ \mathbf{else}\ (z_1 = 0, z_2 = x)  \rightarrow$ \textit{rete a un solo livello di logica}.
\end{itemize}

\subsection{Procedimento di sintesi}

Una volta data la specifica di una rete (di una funzione logica), nel \textit{caso più generale} il procedimento di sintesi della rete stessa è il seguente:

\begin{enumerate}
    \item traduzione della specifica nella tabella di verità, per enumerazione di tutti i casi ($2_n$, per $n$ variabili di ingresso);
    \item per ogni variabile di uscita, scrittura della espressione logica in forma canonica SP (i termini \texttt{AND} corrispondono agli stati d'ingresso per i quali la variabile di uscita assume valore vero);
    \item eventuale riduzione delle espressioni logiche;
    \item traduzione di ogni espressione logica in uno schema di rete a due livelli di logica.
\end{enumerate}

\noindent Questo procedimento ha complessità $O(2^n)$, ed è quindi praticamente applicabile solo nei casi in cui il numero degli ingressi $n$ assuma un valore ``contenuto''.

\subsection{Comportamento temporale delle reti combinatorie}

Ogni rete reale è caratterizzata da un \textit{ritardo} $t_r$, necessario affinché, in seguito ad una variazione dello stato d'ingresso, si produca la corrispondente variazione dello stato di uscita.\
Solo dopo questo tempo si dice che la rete si è \textbf{\textit{stabilizzata}}:\ durante l'intervallo di durata $t_r$, detto appunto \textbf{\textit{ritardo di stabilizzazione}}, il comportamento della rete, in ogni suo punto, è impredicibile e, in particolare, il valore delle uscite non è significativo.

Nel seguito supporremo che

\begin{enumerate}
    \item Le variazioni dello stato d'ingresso siano sincronizzate, cioè avvengano ad istanti discreti ed equidistanti, di una sequenza temporale;
    \item gli istanti di tale sequenza siano distanziati di un periodo $\geq t_r$.
\end{enumerate}

\noindent Per una porta logica, indichiamo con $t_p$ il ritardo di stabilizzazione.\
Nel seguito supporremo che le porte \texttt{NOT} abbiano ritardo nullo, o meglio inglobato nel ritardo delle porte \texttt{AND}/\texttt{OR} connesse alle uscite delle porte \texttt{NOT}.

È importante ricordare che $t_p$ è il massimo valore che può assumere il ritardo di stabilizzazione di una porta.

Dalle assunzioni 1), 2) deriva che il massimo ritardo di stabilizzazione di una rete a $L$ \textit{livelli di logica} è dato da $t_r = Lt_p$.\
\textit{Il valore di $t_p$ dipende dal numero d'ingressi $n$ della porta}.\
La funzione $tp(n)$ ha un andamento monotono crescente; la crescita è relativamente lenta per $n\leq n_0$, tale da poter considerare $t_p$ costante, mentre è molto più rapida per $n>n_0$.\
Valori tipici di $n_0$ allo stato attuale variano da 4 a 8.\
Il valore $t_p$ che si associa a una porta logica è quello per $n\leq n_0$.

La conseguenza di questa caratteristica è che, ove l'espressione logica di una variabile di uscita contenga un termine \texttt{AND} con più di $n_0$ variabili di ingresso, la porta \texttt{AND} della rete deve essere decomposta in più porte secondo una struttura ad albero, ognuna con al più $n_0$ ingressi.\
Lo stesso vale per una porta \texttt{OR} finale.

\subsection{Reti combinatorie operanti su parola}

Capita spesso che la funzione che definisce la rete è applicata a parole di $N$ bit (ad esempio, 32) invece che a semplici variabili booleane.\
In questi casi, la complessità del procedimento di sintesi delle reti combinatorie impedisce di ricorre al metodo generale basato sulla tabella di verità.\
Si cerca invece di \textit{comporre reti a 1 bit, del tipo corrispondente, per ottenere la rete a N bit}; ad esempio, realizzare un commutatore ad $N$ bit utilizzando commutatori ad 1 bit.

\begin{itemize}
    \item Per alcune reti la composizione è \textbf{\textit{parallela}}:\ la rete a $N$ bit è composta da $N$ reti a 1 bit tutte indipendenti, cioè nessuna utilizza in ingresso le uscite di altre.\ Il ritardo della rete a $N$ bit è dunque uguale a quello della rete a 1 bit.\ È il caso del \textit{commutatore}, \textit{selezionatore}, \textit{confrontatore}.
    \item Per altre reti, la composizione può essere \textbf{\textit{in cascata}}:\ la rete a $N$ bit è composta da $N$ reti ad 1 bit tali che le uscite dell'$i$-esima costituiscono ingressi della $(i+1)$-esima.\ Il ritardo della rete a $N$ bit è ora $N$ volte quello della rete a 1 bit.\ È il caso dell'\textit{addizionatore}, a causa della propagazione del riporto.
    \item Un altro caso molto significativo è quello della composizione \textbf{\textit{ad albero}}:\ la rete a $N$ bit è composta da $\log_k N$ reti a 1 bit disposte ad albero di arietà $k$.
\end{itemize}

\section{Reti sequenziali}

\subsection{Definizione di reti sequenziali, loro formalizzazione e comportamento}

Un \textbf{\textit{automa a stati finiti}} è una macchina caratterizzata da
\begin{itemize}
    \item $n$ variabili logiche di ingresso, e corrispondentemente $h = 2^n$ \textbf{\textit{stati d'ingresso}} $X_1, \dots, X_h$;
    \item $m$ variabili logiche di uscita, e corrispondentemente $k = 2^m$ \textbf{\textit{stati d'uscita}} $Z_1, \dots, Z_k$;
    \item $r$ variabili logiche dello stato interno, e corrispondentemente $p = 2^r$ \textbf{\textit{stati interni}} $S_1, \dots, S_p$;
    \item una \textbf{\textit{funzione di transizione dello stato interno}}:
          \[\sigma:\ X \times S \rightarrow S\]
          Tale funzione definisce la trasformazione dello stato interno dal valore \textbf{\textit{presente}} al valore \textbf{\textit{successivo}} in corrispondenza del valore dello stato d'ingresso;
    \item una \textbf{\textit{funzione delle uscite}}:
          \[\omega:\ X \times S \rightarrow Z \]
          Tale funzione definisce la trasformazione dello stato di uscita in corrispondenza del valore dello stato d'ingresso e dello stato interno \textit{presente}.
\end{itemize}

Una rete logica sequenziale implementa, a livello hardware, un automa a stati finiti.

In un \textit{modello strutturale ideale} la rete sequenziale di tipo \textbf{\textit{sincrono}}, le variazione degli stati avvengono in corrispondenza degli istanti di una sequenza temporale discreta $t_0, t_1, \dots, t_n, \dots$ di intervallo (periodo) costante $\Delta = t_{i+1}-t_i$.

Si possono definire due distinti \textbf{\textit{modelli matematici di automa}}, e quindi di rete sequenziale:\ il modello di \textit{Mealy} e il modello di \textit{Moore}.

\begin{itemize}
    \item In entrambi i modelli, considerando il comportamento dell'automa al tempo \textit{t}, lo stato interno successivo \textit{S(t+1)} dipende tanto dallo stato d'ingresso al tempo \textit{t}, \textit{X(t)}, quanto dallo stato interno presente, \textit{S(t)}:
          \[S(t+1) = \sigma(X(t), S(t))\]
    \item Nel modello di Mealy, lo stato di uscita al tempo \textit{t}, \textit{Z(t)}, dipende tanto dallo stato d'ingresso al tempo \textit{t}, \textit{X(t)}, quando dallo stato interno presente, \textit{S(t)}:
          \[Z(t) = \omega(X(t), S(t))\]
    \item Nel modello di Moore, \textit{Z(t)} dipende solo da \textit{S(t)}:
          \[Z(t) = \omega(S(t))\]
\end{itemize}

\noindent Nel modello di Moore la dipendenza tra stati di uscite e stati d'ingresso è quindi espressa da:
\[Z(t) = \omega(S(t)) = \omega(\sigma(X(t-1), S(t-1))) = \omega_1((X(t-1), S(t-1)))\]
Ne discende che, dati due automi di Mealy e di Moore equivalenti, con stati interni iniziali equivalenti e per la stessa sequenza d'ingresso si ha che la sequenza di uscite dell'automa di Moore è ritardata, rispetto a quella dell'automa di Mealy, di un intervallo $\Delta$ della sequenza temporale.

In generale, il numero di stati interni dell'automa di Moore è maggiore o uguale del numero di stati interni dell'automa di Mealy equivalente.

\subsection{Reti sequenziali sincrone LLC}

Il modello strutturale di rete sequenziale sincrona reale comprende le due reti combinatorie reali $\omega$ (funzione delle uscite) e $\sigma$ (funzione di transizione dello stato interno), e le richiusure dello stato interno sono realizzate con $k$ registri in parallelo impulsati dallo stesso segnale di clock (registro di $k$ bit).

Il modello è detto LLC (\textbf{L}evel input, \textbf{L}evel output, \textbf{C}locked), a significare che i segnali su cui si applicano le funzioni $\omega$ e $\sigma$ sono ancora livelli, mentre l'unico segnale impulsivo è quello per la sincronizzazione dei (per provocare la scrittura dei) registri.

Il periodo $\tau$ dell'impulso è detto \textbf{ciclo di clock} della rete sequenziale.\
Per tenere conto della durata $\delta$ dell'impulso, e quindi per evitare il fenomeno dello stato metastabile dei registri, si ha che il \textit{ciclo di clock della rete va determinato come}
\[\tau \geq t_r+\delta\]
La durata $\delta$ dell'impulso di clock può essere assunta uguale a $t_p$.

\subsection{Reti sequenziali realizzate con componenti standard}

Nella parte dedicata al firmware saremo interessati a sintetizzare, per ogni unità di elaborazione, due specifiche reti sequenziali:\ la Parte Controllo e la Parte Operativa, per ognuna delle quali adotteremo due diverse metodologie.

La \textit{Parte Controllo} ha spesso un numero relativamente basso di stati (interno, d'ingresso, di uscita); inoltre, dal microprogramma è possibile ricavare formalmente una sua descrizione tipo grafo di stato.\
Di conseguenza, la sua sintesi è effettuata con il metodo classico visto finora; le reti combinatorie $\omega$ e $\sigma$ sono realizzate mediante porte \texttt{AND}, \texttt{OR}, \texttt{NOT}.

La \textit{Parte Operativa} presenta, anche per le unità più semplici, un numero di stati (interni, d'ingresso, d'uscita) relativamente molto grande; basti pensare che i dati su cui si opera sono tipicamente a parola a $N$ bit:\ per $N=32$ il numero di stati è sull'ordine di molti miliardi.\
In questo caso l'approccio è completamente diverso e si basa sulla composizione di componenti standard a partire dalla specifica delle operazioni elementari delegate alla Parte Operativa, specifica ricavata formalmente dal microprogramma.

\section{Componente logico memoria}

Nella realizzazione di unità di elaborazione è spesso conveniente, o necessario, fare uso di un ulteriore componente logico ``standard'':\ il componente memoria.

\subsection{Realizzazione logica}

In versione \textbf{\textit{RAM}} (lettura-scrittura) un componente logico memoria è definito come un \textit{array unidimensionale M di registri} su cui sono definite le seguenti operazioni fondamentali:

\begin{itemize}
    \item lettura sull'uscita \textit{out} del contenuto della cella d'indirizzo $i$:\ $\mathit{out} = \mathtt{M[i]}$;
    \item scrittura del valore presente sull'ingresso \textit{in} nella locazione di indirizzo i:\ $\mathtt{M[i]} = \mathit{in}$.
\end{itemize}

\noindent L'operazione di lettura è realizzata dal \textit{commutatore} K di uscita, i cui ingressi primari sono le uscite dei registri e i cui ingressi secondari sono i bit dell'indirizzo.

L'operazione di scrittura, invece, è realizzata dal \textit{selezionatore} S d'ingresso, avente anch'esso come ingressi secondari i bit dell'indirizzo.\
Come ingresso primario ha il segnale di controllo $\beta$ per l'abilitazione alla scrittura.\
Inoltre, l'ingresso \textit{in} viene collegato a tutti gli ingressi dei registri; di conseguenza, se $\beta = 1$, il valore di \textit{in} viene scritto solo nel registro indirizzato.

Nel caso di una memoria \textbf{\textit{ROM}}, è ovviamente presente solo il commutatore di uscita.

Si possono avere RAM \textit{a più ingressi e più uscite} utilizzando il numero corrispondenti di selezionatori e commutatori.

\subsection{Realizzazione fisica e ritardi}

Oltre alle modalità in cui verranno realizzati il commutatore e il selezionatore per indirizzare le celle, il tempo di accesso è influenzato in modo significativo dal \textit{tipo} di tecnologia elettronica utilizzato:

\begin{itemize}
    \item le memorie RAM più economiche sono quelle così dette \textbf{\textit{dinamiche}} (DRAM), in quanto, per ragioni di potenza del segnale relativo al contenuto delle celle, ogni cella necessita di un ``rinfresco'' periodico (la cella va letta e quindi riscritta con lo stesso valore letto);
    \item le memorie \textbf{\textit{statiche}} (SRAM), che non necessitano di rinfresco, presentano tempi di accesso decisamente più bassi, ma a parità di generazione tecnologica, sono anche assai meno ``dense'' delle dinamiche e quindi caratterizzate da una minore capacità per chip.
\end{itemize}

\subsection{Numero massimi d'ingressi per porta nelle reti combinatorie}

Ricordiamo che, in generale, per una porta logica \texttt{AND}/\texttt{OR} è fissato un \textit{numero massimo d'ingressi}, $N_0$, ad esempio $N_0 = 8$.\
Il valore del massimo ritardo di stabilizzazione della porta, $t_p$, vale per un numero di ingressi $N \leq N_0$, in quanto per $ N > N_0$ il ritardo di stabilizzazione aumenterebbe linearmente con forte pendenza.\
In pratica, non esistono porte con $N > N_0$.\
Di questa caratteristica tecnologica occorre tenere conto nella realizzazione di una \textit{qualunque rete combinatoria}, in particolare, nella realizzazione di commutatori (selezionatori) con un elevato numero d'ingressi, come si ha anche (ma non solo) nell'implementazione dei componenti logici memoria.

Dal punto di vista tecnologico, occorre tener conto del vincolo che, per ogni porta, deve essere $N \leq N_0$.\
Se questa condizione non si verifica, \textbf{\textit{occorre sostituire la singola porta con una struttura ad albero di arietà}} $N_0$ che, di tutte le possibili realizzazioni, è quella che garantisce il minimo ritardo di stabilizzazione.

In generale, quindi, \textit{il ritardo di stabilizzazione comportato dalla realizzazione di un termine} \texttt{AND}/\texttt{OR} \textit{vari in modo logaritmico con il numero delle sue variabili d'ingresso}.

\subsection{Memorie realizzate a partire da memorie preesistenti}

Di regola, le memorie di grande capacità sono realizzate a partire da componenti logici base dati, che vanno combinati opportunamente in quella che prende il nome di \textbf{memoria modulare}.

\subsection{Organizzazione di memoria modulare}

La memoria modulare \textbf{sequenziale} è caratterizzata dalla distribuzione degli indirizzi sequenzialmente all'interno di un modulo:\ quando si esaurisce la capacità di un modulo, si passa a distribuire gli indirizzi sequenzialmente nel modulo successivo.

\begin{center}
    \textit{identificatore del modulo} = \textit{indirizzo{\slash}C} \\
    \textit{indirizzo interno del modulo} = \textit{indirizzo\%C}
\end{center}

\noindent (dove $C$ è la capacità del singolo modulo).

\noindent L'organizzazione alternativa di memoria modulare è detta organizzazione \textbf{interallacciata}, per la quale si ha

\begin{center}
    \textit{identificatore del modulo} = \textit{indirizzo{\slash}m} \\
    \textit{indirizzo interno del modulo} = \textit{indirizzo\%m}
\end{center}

\noindent Dove $m$ è il numero dei moduli.\
Ciò significa che, nella memoria modulare interallacciata, $m$ parole aventi indirizzi consecutivi sono allocate in altrettanti moduli di memoria distinti e consecutivi.

