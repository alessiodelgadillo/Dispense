\chapter{Fondamenti di Strutturazione di Sistemi di Elaborazione}

\section{Strutturazione a livelli di interpretazione}

\subsection{Macchine virtuali e supporto a tempo di esecuzione}

Le funzionalità di un sistema di elaborazione nel suo complesso possono essere viste come ripartite, secondo un certo numero di livelli, o \textbf{macchine virtuali} $\{\mathrm{MV}_0,\ \mathrm{MV}_1,\ \dots\ , \mathrm{MV}_n\}$.\
I livelli $\{\mathrm{MV}_0,\ \mathrm{MV}_1,\ \dots\ , \mathrm{MV}_n\}$ sono caratterizzati da un proprio \textbf{linguaggio} e sono ordinati secondo una \textbf{relazione gerarchica}.\
Ciò significa che ogni istruzione primitiva $\mathrm{I}_k^i$, o \textbf{\textit{meccanismo}} del linguaggio $\mathrm{L}_i$ ($i>0$) è implementata da un programma $\mathrm{P}_k$, o \textbf{\textit{politica}}, scritta nel linguaggio $\mathrm{L}_j$.

Il concetto che sta alla base della strutturazione a livelli è il seguente:

\begin{enumerate}
    \item per ogni livello $\mathrm{MV}_i$ ($i>0$) l'insieme dei livelli $\{\mathrm{MV}_j\ mid  0 \leq j < i\}$ permette di implementare complessivamente il \textbf{supporto a tempo di esecuzione} di $\mathbf{L}_i$, $\mathbf{STE(L_i)}$, cioè una collezione di algoritmi e strutture dati che provvedono all'interpretazione dei meccanismi di $\mathrm{L}_i$;
    \item \textit{tale supporto è a sua volta strutturato in modo gerarchico} riconoscendo un livello più alto $\mathrm{MV}_{i-1}$ (per $i>1$) il cui supporto a tempo di esecuzione è implementato dai livelli $\{\mathrm{MV}_h \mid 0 < h < i-1\}$ e così via.
\end{enumerate}

\noindent Questi principi sono alla base di una strutturazione dei sistemi con elevate potenzialità di \textit{modularità}, \textit{modificabilità}, \textit{portabilità}, \textit{manutenibilità} e \textit{testabilità}.

Il livello $\mathrm{MV}_0$ è tipicamente quello dei componenti fisici (elettronici) a partire dai quali viene costruita un'astrazione, o virtualizzazione, sempre maggiore delle funzionalità di interesse per il sistema.\
Il livello $\mathrm{MV}_n$ è quello che possiamo chiamare delle ``applicazioni''.

Notiamo che per rendere possibile l'implementazione di un meccanismo $\mathrm{I}_k^i$, di $\mathrm{MV}_i$ occorre definire con chiarezza un'\textbf{interfaccia} nei confronti del livello $\mathrm{MV}_j$ che lo implementa.\
La natura di questa interfaccia sarà di volta in volta dipendente dallo specifico livello considerato.

\subsection{Compilazione e interpretazione}

La traduzione del programma \textit{sorgente} scritto in linguaggio ad alto livello, nel programma oggetto o \textit{eseguibile}, può essere effettuata secondo una delle due ben note tecniche, la \textit{compilazione} e l'\textit{interpretazione}, o loro combinazioni.\
In entrambi i casi, il punto di partenza è la sequenza di comandi, costituenti il programma scritto nel linguaggio ad alto livello, operanti su determinati insiemi di dati.\
Un compilatore o un interprete sono essi stessi dei programmi, già disponibili in forma eseguibile, che accettano come dato d'ingresso il programma sorgente e producono come dato di uscita il programma eseguibile.

In generale nella traduzione da parte di un interprete, poiché questa avviene passo per passo, non viene considerato il contesto in cui si trova il comando da interpretare a ogni passo, applicando sempre \textit{la stessa regola di traduzione} per uno stesso comando.\
Invece, la traduzione da parte di un compilatore prende in considerazione una sequenza, più o meno ampia, in cui si trova inserito un comando e, per uno stesso comando, sceglie tra \textit{regole di traduzione diverse} allo scopo di ottimizzare le prestazioni.

\section{Strutturazione a moduli}

In un sistema a qualunque livello, \textit{una funzionalità complessa è ottenuta mediante composizione di funzionalità più semplici}: ognuna di queste, affidata a un modulo (autonomo e sequenziale), è specializzata verso determinati obiettivi, ma la cooperazione tra moduli rende possibile l'implementazione della funzionalità complessiva del sistema.

\subsection{Modello di calcolatore general - purpose a programma memorizzato}

\textbf{Modello Von Neumann}: il programma assembler da eseguire risiede in una memoria accessibile al Processore, detta \textit{Memoria Principale} del calcolatore. Questo comporta che le \textit{istruzioni del programma assembler}, rappresentate in binario, siano \textit{memorizzate in locazioni di memoria}, che possono essere lette dal Processore nel giusto ordine: \textit{è il Processore stesso (il suo microprogramma) che reperisce le istruzioni dalla memoria e provvede alla loro esecuzione (interpretazione)}.
