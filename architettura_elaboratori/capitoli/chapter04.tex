\chapter{Livello Firmware}

\section{Caratteristiche di un sistema a livello firm\-ware}

Un \textit{sistema di elaborazione} a livello firmware è costituito da un certo numero di \textit{unità di elaborazione} tra loro interagenti, $\mathrm{U}_1, \dots, \mathrm{U}_n$.

A ogni unità di elaborazione è affidato un certo sottoinsieme delle funzionalità dell'intero sistema.\
Attraverso l'interazione tra unità è realizzato il compito che globalmente s'intende affidare all'intero sistema.\
Ogni unità è specializzata verso uno specifico compito, ma attraverso l'interazione tra esse si realizza un sistema avente la caratteristica di \textit{essere generale nei confronti del supporto fornito ai livelli superiori del sistema}.\
In particolare, il funzionamento di ogni unità e l'interazione tra le varie unità realizzano l'\textit{interpretazione} di qualunque programma espresso con il formalismo che definisce il livello assembler del sistema.

Essendo un modulo di elaborazione, un'unità svolge il proprio specifico compito in modo:

\begin{itemize}
    \item autonomo:\ \textit{l'unità è capace di controllare la propria elaborazione in modo del tutto indipendente, pur ricavando informazioni sulle funzionalità da eseguire e valori dei dati attraverso l'interazione con altre unità};
    \item sequenziale:\ \textit{il fondamento dell'unità è descritto da un programma sequenziale.\ Chiameremo microprogramma la descrizione del funzionamento dell'unità, e microlinguaggio il linguaggio di programmazione sequenziale con cui esprimere il microprogramma}.
\end{itemize}

\subsection{Modello Parte Controllo - Parte Operativa}

Il microprogramma di un'unità di elaborazione è a sua volta implementato dal \textit{livello hardware} sottostante.\
Questa interpretazione è realizzata da due reti sequenziali LLC tra loro interagenti, dette \textbf{Parte Controllo} (PC) e \textbf{Parte Operativa} (PO).

PO provvede all'esecuzione delle operazioni dai comandi del microlinguaggio, o \textbf{microistruzioni}, disponendo delle tipiche risorse strutturali messe a disposizioni dal livello hardware:\ \textit{commutatori}, \textit{selezionatori}, \textit{rete di calcolo} mono-funzione o multi-funzione (ALU), \textit{registri}.

PC provvede:
\begin{itemize}
    \item al \textit{controllo della sequenzializzazione} delle microistruzioni, allo scopo utilizzando i valori delle \textbf{variabili di condizionamento} relative allo stato di PO;
    \item a \textit{ordinare a PO l'esecuzione} delle operazioni di ogni microistruzione mediante i valori delle \textbf{variabili di controllo} $\alpha$ e $\beta$.
\end{itemize}

\noindent PC e PO sono \textit{reti sequenziali LLC} impulsate dallo stesso segnale di clock, e quindi aventi lo stesso ciclo di clock:

\begin{itemize}
    \item questo, detto \textbf{\textit{ciclo di clock}} dell'unità è determinato in modo tale da permettere la stabilizzazione di entrambe le reti per l'esecuzione di una qualsiasi microistruzione.
\end{itemize}

\noindent Poiché la struttura PC-PO rappresenta l'interprete hardware del microlinguaggio del livello firmware, \textit{il modello di programmazione del livello firmware è dunque \textbf{sincrono}}:

\begin{itemize}
    \item \textit{ogni microistruzione è eseguita in tempo costante uguale al ciclo di clock}.
\end{itemize}

\subsection{Il procedimento di progettazione delle unità}

I passi essenziali nel procedimento formale di progettazione di un'unità sono i seguenti:

\begin{enumerate}
    \item Specifica delle operazioni esterne affidate all'unità
    \item Scrittura del microprogramma che interpreta le operazioni esterne
    \item Progetto PO
    \item Progetto PC
    \item Valutazione del ciclo di clock dell'unità
    \item Valutazione tempo medio di elaborazione
\end{enumerate}

\section{Formalizzazione del procedimento}

A partire dalla specifica delle operazioni esterne, il procedimento di progettazione ha lo scopo di realizzare la struttura dell'unità di elaborazione secondo il modello PC-PO, e di valutarne le prestazioni.

\subsection{Microlinguaggio}

La formalizzazione del procedimento è basata sulla scrittura del \textit{microprogramma eseguibile}, ogni passo del quale (\textit{microistruzione}) è eseguito esattamente in un ciclo di clock.

\subsection{Ottimizzazione delle micro operazioni}

Esistono diversi modi per scrivere microprogrammi efficienti:

\begin{itemize}
    \item Eliminazione delle \textit{nop}:\ ogni volta che si esegue una nop, agli effetti del tempo di elaborazione viene perduto un ciclo di clock;
    \item Parallelismo delle micro operazioni:\ trasformare una computazione sequenziale nella computazione parallela equivalente.
\end{itemize}

\noindent Formalmente valgono le seguenti \textbf{condizioni di Bernstein}:

\begin{itemize}
    \item $W(\mu-op_1) \cap R(\mu-op_2) = \emptyset $
    \item $R(\mu-op_1) \cap W(\mu-op_2) = \emptyset $
    \item $W(\mu-op_1) \cap W(\mu-op_2) = \emptyset $
\end{itemize}

La terza condizione vale solo in un modello asincrono di computazione, nel quale non si fanno ipotesi sulla durata delle operazioni e sui loro istanti di inizio.

\subsection{Struttura di PO}

La struttura di PO, vista come rete sequenziale di Moore, è ricavata formalmente dal microprogramma.

\begin{enumerate}
    \item Si individuano tante classi di operazioni elementari per quanti sono i registri destinazione (che compaiono a destra dell'operatore $\rightarrow$).\ Ciò permette di individuare i possibili ingressi di ogni registro.
    \item Si ricava una sottostruttura indipendente per ogni registro, con la propria variabile di controllo $\beta$; nel caso in cui gli ingressi possibili IN siano in numero $r>1$, sull'ingresso del registro viene inserito un commutatore, comandato da $\lceil \log_2r\rceil$ variabili di controllo $\alpha$ e i cui ingressi principali sono quelli di IN.
    \item Si individuano tante classi di operazioni elementari per quante sono le reti logiche dagli operatori che compaiono nelle operazioni elementari.
    \item Si ricava una struttura indipendente per ogni rete logica:\ se si tratta di una rete multifunzione, come una ALU, con $r$ funzioni, occorrono $\lceil \log_2r\rceil$ variabili di controllo $\alpha$ per comandare la scelta della funzione; per ogni ingresso principale, se è prevista più di una sorgente, si inserisce un commutatore pilotato da opportune variabili di controllo $\alpha$.
    \item Si implementano le variabili di condizionamento, come funzioni di uscite di registri o uscite di reti combinatorie
\end{enumerate}

\subsection{Struttura di PC}

La struttura di PC, vista come rete sequenziale di Mealy, è ricavata formalmente dal microprogramma mediante il seguente procedimento.

\begin{enumerate}
    \item Gli stati interni corrispondono biunivocamente alle etichette delle microistruzioni del microprogramma.\ Se \textit{m} è il numero degli stati interni, il registro di stato del controllo è di \textit{s} bit con $s= \lceil \log_2m\rceil$ .
    \item Gli stati d'ingresso corrispondono biunivocamente alle possibili combinazioni di variabili di condizionamento per formare le condizioni logiche del microprogramma
    \item Gli stati di uscita corrispondono biunivocamente alle possibili combinazioni di segnali di controllo $\alpha$ e $\beta$ necessari a eseguire tutte le micro-operazioni del microprogramma.
    \item La tabella di verità della funzione delle uscite di $\omega_{PC}$ e della funzione di transizione dello stato interno $\sigma_{PC}$ viene ricavata in base alle tre corrispondenze suddette e alla struttura del microprogramma.\ Da questa tabella è possibile sintetizzare PC con parte combinatoria a due o più livelli di logica.
\end{enumerate}

\noindent Si osservi che \textit{tutti i segnali $\beta$ devono sempre essere specificati per ogni possibile combinazione}, mentre i segnali $\alpha$ potranno eventualmente essere non specificati per qualche combinazione.

\subsection{Ciclo di clock}

A differenza di una rete sequenziale isolata, per un'unità di elaborazione PC-PO occorre ricavare la \textit{condizione di stabilizzazione di entrambe le reti sequenziali PC e PO}.\
Il ciclo di clock non può dunque concludersi prima che siano stabili le funzioni di transizione di PC e di PO, cioè prima che siano stabili, rispettivamente, gli ingressi del registro di stato del controllo e di tutti i registri di PO.

Poiché PO è una rete di Moore, dopo un tempo $T_\omega\mathrm{PO}$, sono stabili le variabili di condizionamento.\
Solo a questo punto, poiché PC è una rete di Mealy, può iniziare, per l'ultima volta all'interno di questo ciclo di clock, la stabilizzazione delle funzioni $\omega_{\mathrm{PC}}$ e $\sigma_{\mathrm{PC}}$ in parallelo.\
Una volta che $\omega_{\mathrm{PC}}$ si è stabilizzata definitivamente, dopo un intervallo $T_{\omega}\mathrm{PC}$, può quindi iniziare, per l'ultima volta all'interno di questo ciclo di clock, la stabilizzazione della funzione $\sigma_{\mathrm{PO}}$; il ciclo di clock si può concludere quando si sono stabilizzate definitivamente tanto la $\sigma_{\mathrm{PC}}$ quanto la $\sigma_{\mathrm{PO}}$.

Di conseguenza, la lunghezza del ciclo di clock è data da:
\[\tau = T_{{\omega}_{PO}} + \max\{ T_{{\omega}_{PC}} +  T_{{\sigma}_{PO}},  T_{{\sigma}_{PC}}\} + \delta\]
Mentre il tempo medio di elaborazione di una unità è calcolato come
\[T = k\tau\]
Dove $k$ è il \textit{numero medio di cicli di clock necessari a eseguire la generica operazione esterna}.

In generale, in un microprogramma sono presenti più sottosequenze di microistruzioni eseguite con una certa probabilità $p_i$ dove:
\[\sum^{n-1}_{i =0}p_i = 1\]
Di ogni sottosequenza si calcola il numero di cicli di clock $k$, necessario a eseguirla e quindi si ricava il valore di $T$ come media pesata:
\[T = \tau \times \sum^{n-1}_{i =0}p_ik_i\]
Quando non siano note le $p_i$ si assume che tutte le sequenze siano equiprobabili.

\textit{La \textbf{banda di elaborazione} è definita come il numero medio di operazioni esterne che l'unità può eseguire nell'unità di tempo}:\ $B = \frac{1}{T}$

\section{Comunicazioni a livello firmware}

\subsection{Comunicazioni tra unità}

Nel modello a scambio di messaggi a livello firmware i canali di comunicazione sono implementati da supporti fisici (strutture d'interconnessione) costituiti da interfacce presenti nella Parte Operativa delle unità e da collegamenti tra le unità stesse.

Il canale di comunicazione tra due unità è implementato da tre componenti:\ interfaccia di uscita, collegamento fisico e interfaccia d'ingresso.

Le forme di comunicazioni possibili sono
\begin{itemize}
    \item Simmetrica vs asimmetrica:\ nel caso simmetrico il canale di comunicazione ha un singolo mittente e un singolo destinatario, nel caso asimmetrico in ingresso il canale di comunicazione ha più mittenti e un singolo destinatario, nel caso asimmetrico in uscita il canale di comunicazione ha un singolo mittente un più destinatari.
    \item Asincrona vs sincrona:\ si dice che un canale (simmetrico) ha grado di asincronia $k$, con $k \geq 0$, se il mittente può inviare fino a $k$ messaggi senza attendere che il destinatario ne abbia ricevuto uno o più.\ Nel caso che il mittente intenda inviare un $(k+1)$-esimo messaggio senza che il destinatario ne abbia ricevuto alcuno dei $k$ precedenti, occorre che il mittente stesso attenda che il destinatario ne abbia ricevuto almeno uno.\ Il caso $k = 0$ è quello della comunicazione sincrona:\ per ogni messaggio, il mittente, prima di poter proseguire, deve attendere che il destinatario lo abbia ricevuto.
\end{itemize}

\subsection{Protocollo mediante segnali di RDY e ACK}

La trasmissione del messaggio (\texttt{MSG}) avviene attraverso un registro di uscita di $\mathrm{U}_1$ (\texttt{OUT}) e un registro d'ingresso di $\mathrm{U}_2$ (\texttt{IN}) connessi da un collegamento fisico.\
Si tratta dei registri d'ingresso e di uscita con le caratteristiche tipiche di un'unità firmware.\
Una volta che $\mathrm{U}_1$ ha scritto il valore di \texttt{MSG} in \texttt{OUT}, questo valore si propaga all'ingresso di \texttt{IN} di $\mathrm{U}_2$, e sarà quindi scritto in tale registro al primo impulso di clock di $\mathrm{U}_2$.\
Il valore di \texttt{MSG} è ora disponibile a $\mathrm{U}_2$ come stato interno di PO.

La \textit{sincronizzazione} è implementata facendo uso di due linee addizionali, di un bit ciascuna, detta \textbf{\textit{linea di Ready}} (\texttt{RDY}) e \textbf{\textit{linea di Acknowledgement}} (\texttt{ACK}):

\begin{itemize}
    \item mediante la linea \texttt{RDY} $\mathrm{U}_1$ fa sapere a $\mathrm{U}_2$ che ha inviato un nuovo messaggio; essa ha dunque il significato di segnalare a $\mathrm{U}_2$ la presenza di un nuovo messaggio;
    \item mediante la line \texttt{ACK} $\mathrm{U}_2$ fa sapere $\mathrm{U}_1$ che ha ricevuto il nuovo messaggio; essa ha dunque il significato di segnalare a $\mathrm{U}_1$ la ricezione di un messaggio.
\end{itemize}

\noindent Poiché intendiamo implementare un protocollo per la comunicazione asincrona di grado $k = 1$, il significato della linea \texttt{ACK} è di far sapere a $\mathrm{U}_1$ che è possibile inviare un nuovo messaggio.

\subsection{Interfaccia a transizione di livello}

I quattro elementi di memoria, \texttt{RDYOUT}, \texttt{ACKOUT}, \texttt{RDYIN}, \texttt{ACKIN}, sono detti \textbf{\textit{indicatori d'interfaccia}} e sono così caratterizzati:

\begin{itemize}
    \item sugli \textit{indicatori d'interfaccia di uscita} (\texttt{RDYOUT} e \texttt{ACKIN}) è definita l'unica operazione chiamata \textbf{\textit{set}}:\ l'esecuzione di \texttt{set RDYOUT} da parte di $\mathrm{U}_1$ ha come effetto finale il fatto che l'indicatore d'interfaccia d'ingresso di $\mathrm{U}_2$ \texttt{RDYIN} assume l'uscita uguale a uno; analogamente, l'esecuzione di \texttt{set ACKIN} da parte di $\mathrm{U}_2$ ha come effetto finale il fatto che l'uscita di \texttt{ACKOUT} in $\mathrm{U}_1$ assume il valore uno;
    \item sugli \textit{indicatori d'interfaccia d'ingresso} (\texttt{RDYIN} e \texttt{ACKOUT}) sono definite due operazioni.\ Una è il test del valore di uscita di tali indicatori.\ L'altra, chiamata \textbf{\textit{reset}}, ha come effetto di portare l'uscita degli indicatori stessi a zero:\ \texttt{reset RDYIN} porta a zero l'uscita di \texttt{RDYIN}, \texttt{reset ACKOUT} porta a zero l'uscita di \texttt{ACKOUT}.
\end{itemize}

\noindent Per quanto riguarda l'impatto del protocollo sulla struttura di PC e PO:

\begin{itemize}
    \item tutti gli indicatori d'interfaccia fanno parte dello stato interno di PO;
    \item \texttt{RDYIN} e \texttt{ACKOUT} fanno parte delle variabili di condizionamento.
\end{itemize}

\subsection{Altre forme di comunicazione}

\textbf{Comunicazione domanda e risposta}:\ si consideri una unità $\mathrm{U}_1$ che invia un messaggio a $\mathrm{U}_2$ dopodiché attende un messaggio da $\mathrm{U}_2$ stessa; tipicamente, si tratta di una situazione cliente-servente, in cui il cliente $\mathrm{U}_1$ prima invia un messaggio al servente $\mathrm{U}_2$ per chiedere un servizio, e quindi attende dal servente un messaggio di risposta.\
Anche se le interfacce tra $\mathrm{U}_1$ e $\mathrm{U}_2$ contengono gli indicatori di \texttt{ACK}, questi non servono nella comunicazione a domanda e risposta:\ $\mathrm{U}_1$ non invierà un nuovo messaggio prima di aver ricevuto la risposta da $\mathrm{U}_2$, quindi l'\texttt{ACK} è implicito nella risposta.

\textbf{Comunicazione asincrona con grado di asincronia \textit{k} $>$ 1}:\ in questo caso non esiste una soluzione basata su semplici interfacce nelle due unità comunicanti, ma è necessario interporre una terza unità (Unità Buffer) che implementi una coda FIFO a \textit{k} posizioni.

\textbf{Collegamenti a bus}:\ il bus risulta composto da $n +2$ linee, per dati di $n$ bits.\
Tutti i bit di \texttt{RDY}, sia ingresso sia in uscita, di tutte le unità sono infatti connessi a una singola linea, così come tutti i bit di \texttt{ACK}, e tutti i bit facenti parte del messaggio.\
L'uscita del registro \texttt{OUT} è messa in \texttt{AND} con il segnale \texttt{RDYOUT} allo scopo di ``ripulire'' il bus dopo che il messaggio è stato ricevuto.

