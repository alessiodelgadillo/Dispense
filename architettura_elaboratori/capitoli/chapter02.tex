\chapter{Rappresentazione Binaria e Strutture di Calcolo}

\section{Rappresentazione binaria delle informazioni}

\subsection{Conversione decimale-binario e binario-decimale}

La conversione da binario a decimale si effettua, nel modo più semplice, applicando lo sviluppo
polinomiale.\
La conversione da decimale in binario può essere fatta in due modi:

\begin{enumerate}
    \item dal numero decimale viene sottratta la massima potenza di 2 minore del numero stesso e lo stesso processo è ripetuto sulla differenza risultante.\ Una volta decomposto il numero in potenze di due, si ottiene il numero binario ponendo 1 nelle posizioni di bit corrispondenti alla potenza di due usata nella decomposizione e 0 nelle restanti posizioni;
    \item dividere ripetutamente il numero decimale per 2, considerando i resti, finché si ottiene come quoziente il numero 0.\ Il numero binario è dato dalla stringa dei resti in ordine inverso rispetto a come sono stati ottenuti.
\end{enumerate}

\subsection{Numeri binari con segno}

Per la rappresentazione dei numeri con segno sono usati principalmente tre metodi.\
In tutti, il bit più significativo è (anche) il bit del segno:\ $0 = +$, $1 = -$.

\begin{itemize}
    \item \textbf{Modulo e segno}:\ i bit restanti, tolto il bit del segno, rappresentano il valore assoluto (modulo) del numero.
    \item \textbf{Complemento a uno}:\ se il numero è positivo, il numero in complemento a uno è uguale a quello modulo e segno, mentre, se il numero è negativo, si ottiene complementando bit a bit la rappresentazione modulo e segno.
    \item \textbf{Complemento a due}:\ se il numero è positivo, il numero in complemento a due è uguale a quello del modulo e segno, mentre, se il numero è negativo, si ottiene sommando 1 alla rappresentazione in complemento a uno.
\end{itemize}

\noindent Sia il sistema modulo e segno che quello del complemento a uno hanno due rappresentazioni possibili per il valore zero, cioè $+0$ e $-0$.\
Questa situazione è indesiderata.\
\textit{Il sistema del complemento a due non ha questo problema} perché il complemento a due di $+0$ è ancora $+$.\
Anche questo sistema presenta una singolarità:\ il numero costituito da un 1 seguito da tutti 0 è il complemento di sé stesso e questo fatto rende \textit{l'intervallo dei numeri positivi asimmetrico rispetto a quelli dei negativi}; esiste cioè un numero negativo senza il corrispondente controvale positivo.

Nell'aritmetica in complemento a due il riporto generato dall'addizione sul bit più a sinistra è semplicemente scartato.\
Se due addenti sono di segno opposto, non può capitare un errore di traboccamento (overflow), mentre se sono di ugual segno e il risultato è di segno opposto significa che si è avuto un errore di overflow e il risultato non è corretto.\
L'errore di overflow può capitare se e solo se il riporto che arriva al bit di segno differisce dal riporto che esce dal bit del segno.\
Normalmente in uscita da un addizionatore si ha sia il riporto sul bit del segno che uno speciale bit di overflow.

\section{Sull'utilizzazione della rappresentazione}

\subsection{Proprietà fondamentali}

Una proprietà fondamentale dei numeri binari è la seguente:

\begin{itemize}
    \item Con \textit{k} bit si possono rappresentare tutti i numeri naturali che vanno da 0 a $2^k - 1$, estremi inclusi.
\end{itemize}

\noindent Altrettanto fondamentale è la seguente proprietà:

\begin{itemize}
    \item Sia un insieme di \textit{m} oggetti dello stesso tipo.\ Vogliamo denotare ogni oggetto con un numero naturale distinto, cioè un \textbf{\textit{identificatore unico}} dell'oggetto, rappresentato in binario.\ Il minimo numero di bit necessario per rappresentare un qualunque identificatore è $n = \lceil lg_2m \rceil $.
\end{itemize}

\subsection{Alcune proprietà utili}

\subsubsection{\textit{a) Quoziente e resto della divisione per potenze di due}}

Siano A e B due numeri binari corrispondenti a numeri naturali, e sia $B = 2^h$.\
In questa ipotesi, è molto facile il calcolo del quoziente della divisione intera e del resto (modulo):

\begin{itemize}
    \item il valore di A\%B (\textit{A mod B}) è dato dagli \textit{h} bit meno significativi di A,
    \item il valore di A/B (\textit{A div B}) è dato dai rimanenti bit più significativi di A.
\end{itemize}

\noindent Come caso particolare notevole:\ un numero binario \textit{pari} (\textit{dispari}) ha il bit meno significativo uguale a 0 (1).

\subsubsection{\textit{b) Operazioni di traslazione(shift)}}

Dato un numero binario naturale A, la sua traslazione destra (\textit{shift destro}) di 1 bit $sh_R^1(A)$ è il numero naturale che si ottiene spostando tutti i suoi bit a destra di una posizione e inserendo il valore 0 nel bit più significativo; il vecchio bit meno significativo viene perso.

Dato un numero binario naturale A, la sua traslazione sinistra (\textit{shift sinistro}) di 1 bit $sh_L^1(A)$ è il numero naturale che si ottiene spostando tutti i suoi bit a sinistra di una posizione e inserendo il valore 0 nel bit meno significativo; il vecchio bit più significativo viene perso.

Valgono le seguenti proprietà, facilmente dimostrabili:

\begin{enumerate}
    \item $sh_R^k(A) = A/2^k$
    \item $sh_L^k(A) = A \cdot 2^k$ mod $2^n$
\end{enumerate}

\section{Strutture di calcolo, tipi di dato e memorie}

\subsection{Parole d'informazione}

Le informazioni a livello di macchina firmware e di macchina assembler sono rappresentate come \textit{stringhe di bit}, che possono avere il significato più vario a seconda dell'uso che intendiamo farne.\
In generale, useremo il termine \textbf{\textit{parola}} per denotare una stringa di bit significativa.

\subsection{Registri}

Fondamentale per la strutturazione a livello firmware e a livello assembler è il registro, che funge da \textit{componente con memoria} (\textit{con stato}).\

Un registro di un bit ha il compito di memorizzare, per tutto il tempo che si ritiene necessario, l'informazione presente sull'ingresso \textit{in} e di rendere disponibile tale informazione (\textit{contenuto}) sull'uscita \textit{out}.\
Per scrivere (memorizzare) il valore \textit{in}, è necessario che sia presente un \textit{segnale di controllo}:\ la scrittura avviene se $\mathit{enable} = 1$.\
Finché $\mathit{enable} = 0$, successivi valori che si presentino su \textit{in} vengono perduti (non memorizzati), mentre appena $\mathit{enable} = 1$ il valore che in quel momento è presente sull'ingresso \textit{in} viene memorizzato (diviene il \textit{contenuto} del registro) e comparirà sull'uscita \textit{out}.\
Il valore su \textit{out} può sempre essere letto ed elaborato (un valore è \textit{sempre} presente, non occorre abilitazione per utilizzarlo in lettura).

Nella trattazione delle reti logiche e delle unità di elaborazione, faremo uso di \textit{registri impulsati} nei quali il segnale \textit{enable} è messo in \texttt{AND} con un segnale, detto \textit{clock}, impulsivo e periodico.\
Il periodo, cioè l'intervallo tra l'inizio di due impulsi di clock consecutivi, è detto \textbf{\textit{ciclo di clock}}.\
La scrittura nel registro avviene in corrispondenza di un impulso di clock a condizione che contemporaneamente $\mathit{enable} = 1$.\
Questa caratteristica permetterà un funzionamento sincrono delle reti (Parte Operativa e Parte Controllo) costituenti ogni unità di elaborazione.

\subsection{Funzioni e reti combinatorie}

A livello hardware implementeremo funzioni ``pure'' (computazioni senza stato:\ per ogni valore dell'ingresso sia ha sempre e un solo valore dell'uscita) mediante opportune strutture dette \textbf{\textit{reti combinatorie}}, ottenute a loro volta come composizioni di componenti più elementari.\
Ad esempio una funzione molto utilizzata è la \textbf{ALU}:\ si tratta di \textbf{reti di calcolo multifunzione}.\
A seconda del valore di alcune variabili d'ingresso secondarie (\textit{variabili di controllo}), il valore dell'uscita è dato dall'applicazione agli ingressi primari di una ben specifica funzione:\ addizione, sottrazione, confronto, shift, \texttt{AND} bit a bit, \texttt{OR} bit a bit, ecc.

