\chapter{Cifratura a blocchi e a chiave pubblica}

\section{Cifratura a blocchi}

Cifrari come l'AES e il DES lavorano a blocchi:\ se si ha un messaggio di $n$ bit, questo \textbf{viene diviso e cifrato in vari blocchi} sempre con la stessa chiave.\
Si noti che un messaggio ha la seguente forma:\ \[m = m_1 m_2 m_3 \dots m_l\ \mathrm{t.c.}\ |m_i| = 128\ \mathrm{bit}\]
Qualora non valga che $|m_l| < 128$ bit allora si aggiunge una concatenazione t.c.\ $|m_l 100\dots 0| = 128\ \mathrm{bit}$.\
Se invece $|m_l| = 128$ bit allora si aggiunge un blocco terminatore $|10\dots 0| = 128$.\
\textit{Per evitare che messaggi diversi diano crittogrammi uguali}, al messaggio viene concatenata una stringa iniziale $c_0$ che potrebbe rappresentare anche il timestamp in cui viene cifrato il messaggio.\

La \textbf{cifratura} è semplice:\
\[c_i = C(c_{i-1} \oplus m_i, k)\]
si prende l'$i$-esimo blocco del messaggio da cifrare, si esegue lo $\oplus$ col crittogramma $c_{i-1}$ calcolato per il blocco precedente e infine si cifra il risultato ottenuto con la chiave $k$.\
Sostanzialmente per cifrare il blocco di messaggio $(i + 1)$-esimo è necessario aver già cifrato il blocco di messaggio $i$-esimo.\

La \textbf{decifrazione} è anch'essa semplice:\ dato che la doppia applicazione di una stringa con lo $\oplus$ ha l'effetto di annullare l'effetto stesso dello $\oplus$, allora per decifrare $c_i$ si esegue
\[m_i = D(c_i, k) \oplus c_{i-1}\]

\noindent Ci sono proprietà molto interessanti nella cifratura a blocchi:\
\begin{itemize}
    \item La decifrazione può essere fatta \textit{in parallelo} perché il blocco $c_{i-1}$ è già noto nel momento in cui bisogna decifrare il blocco $c_i$.
    \item La cifratura a blocchi prevede che il crittogramma $c_i$ influenzi solo se stesso e $c_{i+1}$.\ Questo significa che se alcuni bit di $c_i$ sono trasmessi male, la decifratura stessa di $c_i$ non influenzerà tutto il messaggio $m$ ma solo i blocchi $m_i$ e $m_{i+1}$.\
\end{itemize}

\section{Crittografia a chiave pubblica}

Per tanto tempo, il problema principale della crittografia è stato lo \textbf{scambio della chiave segreta}.\
Questo problema è rimasto tale fino al 1976, quando Diffie e Hellman hanno proposto due metodi per generare e scambiare una chiave segreta su un canale insicuro senza che le due parti si debbano incontrare precedentemente.\
Il primo metodo è un algoritmo chiamato \textbf{protocollo DH} ed è ancora usato nella crittografia su Internet; il secondo è la \textbf{crittografia a chiave pubblica} che non prevede affatto alcuno scambio di chiave.\
In quest'ultimo metodo sostanzialmente le due parti cifrano il messaggio con una chiave pubblica e lo decifrano rispettivamente con una chiave privata.\

\vspace{12pt}
\noindent Nei cifrari a chiave privata (o simmetrici) si ha una chiave segreta per ogni coppia di utenti che vuole parlare, facendo crescere il numero di chiavi con un fattore esponenziale.
\vspace{12pt}

\noindent Nei cifrari a chiave pubblica o cifrari asimmetrici, invece, tutti possono inviare messaggi cifrati e solo il ricevente può decifrarli.\
Le operazioni di cifratura e decifrazione sono pubbliche e utilizzano due chiavi diverse:
\begin{itemize}
    \item  $k_{\mathit{pub}}$ per cifrare:\ è pubblica, nota a tutti.
    \item  $k_{\mathit{priv}}$ per decifrare:\ è privata, nota solo al destinatario.
\end{itemize}

\noindent In questo caso, ogni utente possiede una coppia $\langle k_{\mathit{pub}}, k_{\mathit{priv}}\rangle$:\ la chiave pubblica è nota a chiunque e qualsiasi persona interessata a inviare un messaggio all'utente che possiede $k_{\mathit{pub}}$ come chiave pubblica, può e deve cifrare il proprio messaggio esattamente con $k_{\mathit{pub}}$.\
Il destinatario provvederà a decifrare con la sua chiave privata $k_{\mathit{priv}}$ nota solo a lui.\
In questo modo il numero di chiavi è $2n$ se si hanno $n$ utenti.

\subsubsection{Cifratura e decifrazione in un cifrario a chiave pubblica}

\begin{flushleft}
    Cifratura:\ $c = C(m, k_{\mathit{pub}})$.\

    Decifrazione:\ $m = D(c, k_{\mathit{priv}})$.\
\end{flushleft}

\noindent I requisiti perché un cifrario a chiave pubblica funzioni sono:

\begin{enumerate}
    \item Correttezza:\ per ogni possibile messaggio $m = D(C(m, k_{\mathit{pub}}), k_{\mathit{priv}})$.\
    \item Efficiente e sicuro:\ tutto ciò che è lecito, legale, deve richiedere tempo polinomiale, mentre tutto ciò che è ``illegale'' deve richiedere tempo esponenziale.\ Quindi:
          \begin{itemize}
              \item la \textbf{coppia di chiavi è facile da generare} e deve risultare pressoché impossibile che due utenti scelgano la stessa chiave (\textit{generazione casuale delle chiavi});
              \item dati $m$ e $k_{\mathit{pub}}$, \textbf{è facile calcolare il crittogramma} $c = C(m, k_{\mathit{pub}})$ (\textit{adottabilità del sistema});
              \item dati $c$ e $k_{\mathit{priv}}$, \textbf{è facile calcolare il messaggio in chiaro} $m = D(c, k_{\mathit{priv}})$ (\textit{adottabilità del sistema});
              \item pur conoscendo il crittogramma $c,\ k_{\mathit{pub}},\ C\ \mathrm{e}\ D$ la \textbf{decifrazione} di $c$ deve essere \textbf{difficile per il crittoanalista} (\textit{sicurezza del cifrario}).
          \end{itemize}
\end{enumerate}

\noindent Al fine di poter soddisfare questi requisiti, per la funzione di cifratura $C$ si deve ricorrere a una funzione \textbf{one-way trapdoor}, cioè calcolare $c = C(m, k_{\mathit{pub}})$ è \textit{computazionalmente facile}, ma calcolare $m = D(c, k_{\mathit{priv}})$ è \textit{computazionalmente difficile} se non si conosce la trapdoor ($k_{\mathit{priv}}$).\
Tuttavia, Diffie e Hellman non riuscirono mai a trovare una funzione one-way trapdoor.\

\subsection{RSA}

Rivest, Adleman e Shamir proposero un sistema a chiave pubblica riuscendo a trovare una funzione one-way trapdoor.\
RSA si basa sulla moltiplicazione di due numeri primi $p$ e $q$:

\begin{itemize}
    \item Calcolare $n = p\cdot q$ è facile.\ Una moltiplicazione richiede sempre tempo polinomiale.
    \item Calcolare $p$ e $q$ conoscendo $n$ è difficile a meno che non si conosca uno dei due fattori.\ In pratica, fattorizzare $n$ senza conoscere $p$ o $q$ richiede tempo esponenziale.
\end{itemize}

\noindent RSA \textbf{utilizza algebra modulare} ($\mathit{mod}\ n$).

\subsection{Algebra modulare}

Usata in molti algoritmi crittografici per
\begin{itemize}
    \item ridurre lo spazio dei numeri su cui si opera e quindi aumentare la velocità di calcolo;
    \item rendere difficili problemi computazionali che sono semplici (o anche banali) nell'algebra non modulare.
\end{itemize}

\noindent In algebra modulare le funzioni tendono a comportarsi in modo ``\textit{imprevedibile}''.\
Si prenda, per esempio, la funzione $2^x$.\
Il suo comportamento nell'algebra ordinaria è monotono crescente:

\begin{table}[H]
    \centering
    \begin{tabular}{|c|c|c|c|c|c|c|c|c|c|c|c|c|c|}
        \hline
        x     & 1 & 2 & 3 & 4  & 5  & 6  & 7   & 8   & 9   & 10   & 11   & 12   \\\hline
        $2^x$ & 2 & 4 & 8 & 16 & 32 & 64 & 128 & 256 & 512 & 1024 & 2048 & 4096 \\\hline
    \end{tabular}
\end{table}

\noindent Se invece si prende $2^x\ \mathit{mod}\ 13$, il comportamento è:
\begin{table}[H]
    \centering
    \begin{tabular}{|c|c|c|c|c|c|c|c|c|c|c|c|c|c|}
        \hline
        x     & 1 & 2 & 3 & 4 & 5 & 6  & 7  & 8 & 9 & 10 & 11 & 12 \\\hline
        $2^x$ & 2 & 4 & 8 & 3 & 6 & 12 & 11 & 9 & 5 & 10 & 7  & 1  \\\hline
    \end{tabular}
\end{table}

\noindent Sale, scende, sale\dots ha un andamento caotico:\ perde struttura rispetto alla precedente e non dà alcun suggerimento.\

\subsubsection{Caratteristiche}

Preso $n \in \mathbb{N}$, intero positivo
\[\mathbb{Z}_n = \{0,1,2.\dots, n-1\}\qquad \mathbb{Z}_n^* \subseteq \mathbb{Z}_n\]
è l'insieme degli elementi di $\mathbb{Z}_n$ co-primi con $n$.\
Se $n$ è primo, $\mathbb{Z}_n^* = \mathbb{Z}_n$.\
Se $n$ non è primo, calcolare $\mathbb{Z}_n^*$ è computazionalmente difficile:\ richiede tempo proporzionale al valore di $n$ (per confrontare $n$ con gli elementi di $\mathbb{Z}_n$) quindi esponenziale nella lunghezza della sua rappresentazione.\

\vspace{12pt}
\noindent Dati due interi $a,b \geq 0$ e $n\geq0$, $a$ è congruo a $b$ modulo $n$
\[a \equiv b\ \mathit{mod}\ n\]
se e solo se esiste $k$ intero per cui
\[a = b + kn\]
Nelle relazioni di congruenza la notazione $\mathit{mod}\ n$ si riferisce all'intera relazione, nelle relazioni di uguaglianza la stessa notazione si riferisce solo al membro dove appare:\ $5\equiv 8\ \mathit{mod}\ 3$, ma $5\neq 8\ \mathit{mod}\ 3$

\subsubsection{Proprietà}

\[(a+b)\ \mathit{mod}\ m = (a\ \mathit{mod}\ m + b\ \mathit{mod}\ m)\ \mathit{mod}\ m\]
\[(a-b)\ \mathit{mod}\ m = (a\ \mathit{mod}\ m - b\ \mathit{mod}\ m)\ \mathit{mod}\ m\]
\[(a\times b)\ \mathit{mod}\ m = (a\ \mathit{mod}\ m \times b\ \mathit{mod}\ m)\ \mathit{mod}\ m\]
\[a^{r \times s}\ \mathit{mod}\ m = (a^r\ \mathit{mod}\ m)^s\ \mathit{mod}\ m\quad (r, s\ \mathrm{interi\ positivi})\]

\subsubsection{Funzione di Eulero}

Il numero di interi minori di $n$ e co-primi con esso
\[\phi(n) = |\mathbb{Z}_n^*|\]
Se $n$ è primo $\Rightarrow \phi (n) = n-1$.\

\begin{theorem}

    \[n\ \mathrm{composto} \Rightarrow \phi(n) = n\left(1-\frac{1}{p_1}\right)\dots \left(1-\frac{1}{p_k}\right)\]
    $p_1, \dots, p_k$ fattori primi di $n$, presi senza molteplicità.\
\end{theorem}

\begin{theorem}
    $n$ prodotto di due primi (semi-primo)
    \[n = p\cdot q \Rightarrow \phi(n) = (p-1)(q-1)\]
\end{theorem}

\begin{theorem}[Eulero]
    Per $n>1$ e per ogni $a$ primo con $n$
    \[a^{\phi(n)} \equiv 1\ \mathit{mod}\ n\]
\end{theorem}

\begin{theorem}[Fermat]
    Per $n$ primo e per ogni $a \in \mathbb{Z}_n^*$
    \[a^{n-1} \equiv 1\ \mathit{mod}\ n\]
\end{theorem}

\subsubsection{Conseguenze}

Per qualunque $a$ primo con $n$
\[a \times a^{\phi(n)-1}\equiv 1\ \mathit{mod}\ n \quad (\mathrm{Teorema\ di\ Eulero})\]
\[a \times a^{-1}\equiv 1\ \mathit{mod}\ n \quad (\mathrm{definizione\ di\ inverso})\]
quindi,
\[a^{-1} = a^{\phi(n)-1}\ \mathit{mod}\ n \]
L'inverso $a^{-1}$ di $a$ modulo $n$ si può dunque calcolare per esponenziazione di $a$ se si conosce $\phi(n)$.\
In generale, nell'algebra modulare l'esistenza dell'inverso non è garantita perché $a^{-1}$ deve essere intero.\

\begin{theorem}
    L'equazione $ax \equiv b\ \mathit{mod}\ n$ ammette soluzione se e solo se $\mathrm{MCD}(a,n)$ divide $b$.\
    In questo caso si hanno esattamente $\mathrm{MCD}(a,n)$ soluzioni distinte.\
\end{theorem}

\begin{corollario}
    L'equazione $ax \equiv b\ \mathit{mod}\ n$ ammette un'unica soluzione se e solo se $a$ e $n$ sono co-primi ($\mathrm{MCD}(a,n) = 1 $) $\Leftrightarrow$ esiste l'inverso $a^{-1}$ di $a$.\
\end{corollario}

\noindent Se nel corollario si sostituisce 1 con $b$, si ottiene $ax \equiv 1\ \mathit{mod}\ n$.\
Ne consegue che ammette esattamente una soluzione (l'inverso di $a$) se e solo se $a$ e $n$ sono primi tra loro.\
L'inverso si può calcolare come
\[a^{-1} = a^{\phi(n)-1}\ \mathit{mod}\ n \]
ma occorre conoscere $\phi(n)$, cioè fattorizzare $n$:\ è un problema ``difficile''.\

\subsubsection{Algoritmo di Euclide Esteso}

L'algoritmo di Euclide per il calcolo del MCD si può estendere per risolvere l'equazione in due incognite $ax+by=\mathrm{MCD}(a,b)$.\

\begin{flushleft}
    \ttfamily
    Function Extended\_Euclid(a,b)

    \quad if ($b=0$) then return $\langle a, 1, 0\rangle$

    \quad else

    \qquad $\langle d', x', y' \rangle = $ Extended\_Euclid($b, a\ \mathit{mod}\ b$);

    \qquad $\langle d, x, y \rangle = \langle d', y', x' - \left\lfloor \frac{a}{b}\right\rfloor y' \rangle$

    \qquad return $\langle d, x, y\rangle$
\end{flushleft}

\noindent \textbf{Osservazioni}:\

\begin{itemize}
    \item La funzione \texttt{Extended\_Euclid} restituisce una delle triple di valori \[\langle\mathrm{MCD} (a, b), x, y\rangle\] con $x,y$ tali che $ax + by = \mathrm{MCD}(a,b)$.\ Quindi $d = \mathrm{MCD}(a,b)$.\
    \item complessità logaritmica nel valore di $a$ e $b$, quindi \textit{polinomiale} nella dimensione dell'input.\
\end{itemize}

\noindent L'algoritmo di Euclide esteso si può applicare al \textbf{calcolo dell'inverso}.\

$ax \equiv 1\ \mathit{mod}\ b \Leftrightarrow ax = bz + 1$ per un opportuno valore di $z$ se e solo se $ax +by = \mathrm{MCD}(a,b)$, dove $y =-z$ e $\mathrm{MCD}(a,b) =1$.\

\subsubsection{Generatori}

$a \in \mathbb{Z}_n^*$ è un \textbf{generatore} di $\mathbb{Z}_n^*$ se la funzione
\[a^k\ \mathit{mod}\ n\qquad 1 \leq k \leq \phi(n)\]
\textbf{genera tutti e soli} gli elementi di $\mathbb{Z}_n^*$.\
Produce come risultati tutti gli elementi di $\mathbb{Z}_n^*$, ma in un ordine difficile da prevedere.\

\begin{theorem}[Eulero]
    $a^{\phi(n)} \equiv 1\ \mathit{mod}\ n \Rightarrow 1 \in \mathbb{Z}_n^*$ è generato per $k = \phi(n)$.\
    Per ogni generatore \[a^k \not\equiv 1\ \mathit{mod}\ n\qquad 1\leq k < \phi(n)\]
\end{theorem}

\begin{theorem}
    Se $n$ è un numero primo, $\mathbb{Z}_n^*$ ha almeno un generatore.\
\end{theorem}

\begin{itemize}
    \item Per $n$ primo, non tutti gli elementi di $\mathbb{Z}_n^*$ sono suoi generatori (1 non è mai generatore e altri elementi non possono esserlo).
    \item Per $n$ primo, i generatori di $\mathbb{Z}_n^*$ sono in totale $\phi(n-1)$.
\end{itemize}

\subsubsection{Problemi sui generatori rilevanti in crittografia}

Risolvere nell'incognita $x$ l'equazione $a^x = b\ \mathit{mod}\ n$, con $n$ primo.

L'equazione ammette una soluzione per ogni valore di $b$ se e solo se $a$ è un generatore di $\mathbb{Z}_n^*$.\
Tuttavia non è noto a priori in che ordine sono generati gli elementi di $\mathbb{Z}_n^*$, quindi non è noto per quale valore di $x$ si genera $b\ \mathit{mod}\ n$.\
Un esame diretto della successione richiede tempo esponenziale nella dimensione di $n$:\ non è noto un algoritmo polinomiale di soluzione.\

\subsection{Funzioni one-way trap-door}

Esistono funzioni matematiche che sembrano possedere i requisiti richiesti:\ il loro calcolo risulta incondizionatamente semplice e la loro inversione semplice se si dispone di un'informazione aggiuntiva sui dati (cioè una chiave privata).\
Senza questa informazione, l'inversione richiede la soluzione di un problema NP-hard, o comunque di un problema noto per cui non si conosce un algoritmo polinomiale.\

\subsubsection{Fattorizzazione}

Calcolare $n = p \times q$ è facile e richiede tempo quadratico nella lunghezza della loro rappresentazione.\
Invertire la funzione per trovare $p$ e $q$ a partire da $n$ (univocamente possibile solo se $p$ e $q$ sono primi) richiede tempo (sub)esponenziale.\
Per quanto noto fino a oggi, non è mai stato dimostrato che il problema è NP hard ma non è nemmeno mai stato dimostrato che il problema ammette un algoritmo risolutivo polinomiale (cosa da non escludere).\
La trap door in questo caso è uno qualsiasi dei due fattori $p$ e $q$.\

\subsubsection{Calcolo della radice in modulo}

Calcolare $y = x^z\ \mathit{mod}\ s$ con $x, z, s$ interi, richiede tempo polinomiale e si può fare con l'algoritmo delle esponenziazioni successive eseguendo $\Theta(\log_2 z)$ moltiplicazioni.\
Se $s$ non è primo, invertire la funzione e calcolare $x = y^{\frac{1}{z}}\ \mathit{mod}\ s$ richiede tempo esponenziale per quanto noto ad oggi.\
La trap door la vedremo successivamente.\

\subsubsection{Calcolo del logaritmo discreto}

Calcolare la potenza $y = x^z\ \mathit{mod}\ s$ è facile.\
Invertire rispetto a $z$, cioè trovare $z$ t.c.\ $y = x^z\ \mathit{mod}\ s$ dati $x$, $y$ e $s$ è computazionalmente difficile.\
Tutti gli algoritmi noti hanno la stessa complessità della fattorizzazione.\

\subsection{Vantaggi e svantaggi}

Vantaggi
\begin{itemize}
    \item Se ho $n$ utenti, il numero di chiavi del sistema sono $2n$ anziché $\frac{n(n - 1)}{2}$.
    \item Non è richiesto alcuno scambio di chiavi.
\end{itemize}
Svantaggi
\begin{itemize}
    \item Molto più lenti dei cifrari simmetrici.
    \item Sono esposti per natura ad attacchi di tipo \textit{chosen plain-text}.\
\end{itemize}

\subsubsection{Attacchi chosen plain-text}

Un crittoanalista può crearsi svariati crittogrammi $c = C(m, k_{\mathit{pub}})$ di un determinato destinatario e successivamente può mettersi in ascolto sul canale verso il quale sono diretti i crittogrammi per quel determinato destinatario.\
A questo punto il crittoanalista può confrontare i crittogrammi che passano sul canale con quelli che si era preparato in precedenza e, conoscendone il testo in chiaro, se ne trova due uguali ne conosce automaticamente la decifrazione.\
Qualora non trovasse alcuni dei suoi crittogrammi uguali a quelli passati sul canale, saprebbe comunque che il messaggio in chiaro è diverso da quelli da lui preparati.

\section{Cifrari ibridi}

Si usa un cifrario a chiave segreta (AES) per le comunicazioni di massa e un cifrario a chiave pubblica per scambiare le chiavi segrete relative al primo, senza incontri fisici tra gli utenti.\

La trasmissione dei messaggi lunghi avviene ad alta velocità, mentre lo scambio delle chiavi segrete è lento (sono composte al massimo da qualche decina di byte).\
L'attacco \textit{chosen plain-text} è risolto se l'informazione cifrata con la chiave pubblica (chiave segreta dell'AES) è scelta in modo da risultare imprevedibile al crittoanalista.\

La chiave pubblica deve essere estratta da un certificato digitale valido, per evitare attacchi \textit{man-in-the-middle}.\
