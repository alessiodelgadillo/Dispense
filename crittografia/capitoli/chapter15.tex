\chapter{Protocolli di identificazione, autenticazione e firma digitale}

\textbf{Identificazione}\quad Un sistema di elaborazione, isolato o in rete, deve essere in grado di \textit{accertare l'identità di un utente} che richiede accesso ai suoi servizi.\

\vspace{12pt}
\noindent\textbf{Autenticazione}\quad Il destinatario di un messaggio deve essere in grado di accertare \textit{l'identità del mittente} e \textit{l'integrità del crittogramma ricevuto}.\

\vspace{12pt}
\noindent\textbf{Firma digitale}
\begin{itemize}
    \item Il mittente non deve poter negare di aver inviato un messaggio $m$.
    \item Il destinatario deve essere in grado di autenticare il messaggio.
    \item Il destinatario non deve poter sostenere che $m' \neq m$ è il messaggio inviato dal mittente.
\end{itemize}
Tutto ciò deve essere verificabile da terzi.

\vspace{12pt}
\noindent Nessuna delle tre funzionalità esclude l'altra, ciascuna estende le precedenti:\ l'autenticazione del messaggio garantisce l'identificazione del mittente e la firma digitale garantisce l'autenticazione del messaggio.\
Sono usate per contrastare gli attacchi attivi ed esistono sia sui cifrari simmetrici che asimmetrici.

\section{Funzioni Hash}

Una \textbf{funzione hash} $f: X \rightarrow Y$ è una funzione tale che
\[n = |X| >> m = |Y|\]
$\exists\ X_1, X_2,\dots, X_m \subseteq X$ \textbf{disgiunti} tale che
\[ X = X_1 \cup X_2 \cup \dots \cup X_m\]
\[\forall\ i,\ \forall\ x \in X_i,\ f(x) = y\]

\noindent Sostanzialmente, una funzione hash non è iniettiva, quindi più input corrispondono allo stesso output (più elementi del dominio hanno la stessa immagine).\

Una buona funzione hash deve assicurare che \textbf{i sottoinsiemi $X_1, \dots, X_m$ abbiano circa la stessa cardinalità}:\ due elementi estratti a caso da $X$ hanno probabilità $\sim \frac{1}{m}$ di avere la stessa immagine in $Y$.\
Un'altra proprietà che viene richiesta è che \textbf{elementi di $X$ molto simili tra loro appartengano a due sottoinsiemi diversi}:\ se $X$ è un insieme di interi, due elementi con valori prossimi devono avere immagini diverse.\
Inoltre deve poter \textbf{gestire le collisioni}:\ l'algoritmo che impiega la funzione hash dovrà affrontare la situazione in cui più di un elemento di $X$ ha lo stesso valore hash $Y$.\

\subsection{Funzioni hash one-way}

Se la funzione è applicata in crittografia, deve soddisfare le seguenti proprietà:
\begin{enumerate}
    \item $\forall\ x \in X$ è \textbf{computazionalmente facile} calcolare
          \[y = f(x)\]
    \item Proprietà one-way:\ per la maggior parte degli $y \in Y$ è \textbf{computazionalmente difficile} determinare $x \in X$ tale che
          \[f(x) = y,\ \mathrm{cio\grave{e}}\ x = f^{-1}(y)\]
    \item Proprietà claw-free:\ è difficile trovare due elementi $x_1, x_2 \in X$ tali che
          \[f(x_1) = f(x_2)\]
\end{enumerate}

\subsection{Funzioni hash usate in crittografia}

\subsubsection{MD5 (Message Digest, versione 5)}

È una famiglia di algoritmi di cui l'originale non venne mai pubblicato:\ furono pubblicati MD2 seguito da MD4.\
Tuttavia, in MD2 e MD4 furono trovate debolezze e, nel 1992, Ron Rivest formulò MD5.\

Riceve una sequenza $S$ di 512 bit e produce un'immagine di 128 bit:\ la sequenza è \textbf{digerita} riducendone la lunghezza ad un quarto.\
In seguito è stato dimostrato che MD5 non resiste alle collisioni e nel 2004 sono state trovate debolezze severe:\ oggi la sua sicurezza si considera severamente compromessa.\
Lo stesso Rivest ha affermato (2005) che era da considerarsi chiaramente forzata dal punto di vista della resistenza alle collisioni.\

\subsubsection{RIPEMD-160}

Versione ``matura'' delle funzioni MD.\
Nasce nel 1995 nell'ambito di un progetto dell'Unione Europea.\
Produce immagini da 160 bit ed è esente dai difetti di MD5.

\subsubsection{SHA (Secure Hash Algorithm)}

Progettata da NIST e NSA nel 1993, si adotta quando la proprietà di claw-free è cruciale per la sicurezza del sistema.\

Opera su sequenze lunghe fino a $2^{64}$ bit e produce immagini di 160 bit.\
È una funzione crittograficamente sicura:\ soddisfa i requisiti delle funzioni hash one-way e genera immagini molto diverse per sequenze molto simili.\
La prima versione pubblicata (SHA-0) conteneva una debolezza, scoperta in seguito da NSA, che portò alla revisione dello standard.\

\begin{itemize}
    \item 1995:\ SHA-1, progettato da NSA e raccomandato da NIST.
    \item 2001:\ SHA-2, produceva digest più lunghi.
    \item 2007:\ a causa degli attacchi su MD5 e SHA-0 e di attacchi teorici su SHA-1, il NIST ha sollecitato proposte per un nuovo algoritmo hash.\
    \item 2012:\ si è conclusa la valutazione ed è stato selezionata una funzione hash di progettazione non governativa (SHA-3, team di analisti italiani e belgi, rilasciata ufficialmente nel 2015).\
\end{itemize}

\subsubsection{SHA-1}

Opera su sequenze lunghe fino a $2^{64} - 1$ bit e produce immagini di 160 bit.\
È molto usata nei protocolli crittografici anche se non è più certificata come standard.\

Opera su blocchi di 160 bit, contenuti in un buffer di 5 registri da 32 bit ciascuno, in cui sono caricati inizialmente dei valori pubblici.\
Il messaggio $m$ viene concatenato con una sequenza di padding che ne rende la lunghezza un multiplo di 512 bit.\
Il contenuto dei registri varia nel corso dei cicli successivi in cui questi valori si combinano tra loro e con blocchi da 32 bit provenienti da $m$.\
Alla fine del procedimento, i registri contengono SHA-1($m$).\

\section{Identificazione}

\subsection{Identificazione su canali sicuri}

\textbf{Esempio}:\ accesso di un utente alla propria casella di posta elettronica, o a file personali memorizzati su un calcolatore ad accesso riservato ai membri della sua organizzazione.\
\begin{itemize}
    \item L'utente inizia il collegamento inviando in chiaro login e password.\
    \item Se il canale è protetto in lettura e scrittura, un attacco può essere sferrato solo da un utente locale al sistema (admin o hacker).\
\end{itemize}

\noindent Il meccanismo di identificazione prevede una cifratura delle password realizzata con funzioni hash one-way.\

Ad esempio, in Unix, quando un utente $U$ fornisce per la prima volta la sua password $P$, il sistema associa a $U$ due sequenze binarie:\
\begin{itemize}
    \item $S$ (seme) prodotta da un generatore pseudocasuale.
    \item $Q = h(PS)$ con $h$ funzione hash one-way.
\end{itemize}

\noindent Nel file delle password si memorizza $\langle S, Q\rangle$.\
Quando l'utente si riconnetterà e fornirà nuovamente la password, verrà recuperata $S$ che viene concatenata con la password fornita e viene calcolata $h(PS)$; se risulterà uguale a $Q$ l'identificazione sarà corretta altrimenti no.\

Un accesso illecito al file delle password non fornisce informazioni interessanti:\ è computazionalmente difficile ricavare la password originale dalla sua immagine one-way.\

L'aggiunta del seme è molto utile perché, nel caso in cui due utenti usino la stessa password, $Q$ resta comunque diverso.\

\subsection{Identificazione su canali non sicuri}

Se il canale è insicuro, la password può essere intercettata durante la sua trasmissione in chiaro.\
Il sistema non dovrebbe mai maneggiare direttamente la password, ma una sua immagine inattaccabile.\

\begin{flushleft}
    Si adottano strategie che si basano sulla cifratura a chiave pubblica.\
\end{flushleft}

\noindent Supponiamo che $\langle e, n \rangle$ e $\langle d \rangle$ siano rispettivamente la chiave pubblica e privata dell'utente $U$ che richiede l'accesso ai servizi offerti dal sistema $S$.\

\begin{enumerate}
    \item $S$ genera un numero casuale $r < n$ e lo invia in chiaro a $U$.
    \item $U$ calcola
          \[f = r^d\ \mathit{mod}\ n\quad (\mathrm{firma\ di}\ U\ \mathrm{su}\ r)\]
          con la sua chiave privata e lo spedisce a $S$.\
    \item $S$ verifica la correttezza del valore ricevuto calcolando e verificando se
          \[f^e\ \mathit{mod}\ n = r \]
          se cio avviene, l'identificazione ha successo.
\end{enumerate}

\noindent Le operazioni di cifratura e decifrazione sono invertite rispetto all'impiego standard nell'RSA:\ è possibile perché le due operazioni sono commutative
\[(x^e\ \mathit{mod}\ n)^d\ \mathit{mod}\ n = (x^d\ \mathit{mod}\ n)^e\ \mathit{mod}\ n\]
Inoltre, $f$ può essere generata solo da $U$ che possiede $\langle d \rangle$, quindi, se il passo 3 va a buon fine, il sistema ha la garanzia che l'utente che ha richiesto l'identificazione sia effettivamente $U$, anche se il canale è insicuro.\

Fintanto che il sistema è ``onesto'' e $r$ è davvero un numero casuale non ci sono problemi; l'utente deve fidarsi del sistema:\ quest'ultimo potrebbe opportunamente scegliere $r$ in modo da carpire informazioni sulla chiave privata $\langle d \rangle$.\

\section{Autenticazione}

Tipicamente per garantire l'autenticazione si usa il MAC (Message Authentication Code):\ un breve codice associato al messaggio.\
Per poter calcolare tale codice, il mittente e il destinatario devono concordare una chiave segreta simmetrica $k$.\

Il mittente, quindi, allega al messaggio un MAC $A(m, k)$, allo scopo di garantire la provenienza e l'integrità del messaggio, e spedisce in chiaro la coppia $\langle m, A (m, k)\rangle$, oppure cifra $m$ e spedisce $\langle C(m, k'), A (m, k)\rangle$ ($C$ è una funzione di cifratura, mentre $k$ è la chiave pubblica o segreta del cifrario scelto).\

Il destinatario entra in possesso di $m$ (dopo averlo eventualmente decifrato) e, essendo a conoscenza di $A$ e $k$, calcola $A(m,k)$ confrontando il valore ottenuto con quello inviato dal mittente (per verificare che il MAC ricevuto corrisponda al messaggio a cui risulta allegato).\
Se la verifica ha successo il messaggio è autenticato, altrimenti scarta il messaggio.\

\subsection{MAC}

Il MAC è un'immagine breve del messaggio che può essere generata solo da un mittente conosciuto dal destinatario, previ opportuni accordi.\
Ne sono state proposte varie realizzazioni, basate su cifrari asimmetrici, simmetrici e sulle funzioni hash one-way.\

\subsubsection{MAC con funzioni hash one-way}

$A(m,k) = h(mk)$, con $h$ funzione hash one-way.\
Risulta computazionalmente difficile per un crittoanalista scoprire la chiave segreta $k$:\ $h$ è nota a tutti e $m$ può viaggiare in chiaro o essere scoperto per altra via, ma $k$ viaggia all'interno del MAC e quindi per poterlo recuperare si dovrebbe invertire $h$.\

Il crittoanalista non può sostituire facilmente il messaggio $m$ con un altro messaggio $m'$ perché dovrebbe allegare alla comunicazione di $m'$ il MAC $A(m', k)$ che può essere prodotto solo conoscendo $k$.\

\subsubsection{CBC e MAC}

Usando un cifrario a blocchi in modalità CBC, si può usare il blocco finale del crittogramma come MAC:\ l'ultimo blocco è infatti una funzione dell'intero messaggio.\

\section{Firma digitale}

I requisiti fondamentali di una \textbf{firma manuale} sono

\begin{enumerate}
    \item è autentica e non falsificabile, prova che chi l'ha prodotta è chi ha sottoscritto il documento;
    \item non è riutilizzabile, è legata strettamente al documento su cui è stata apposta;
    \item il documento firmato non è alterabile, chi ha prodotto la firma è sicuro che questa si riferirà solo al documento sottoscritto nella sua forma originale;
    \item non può essere ripudiata da chi l'ha apposta, costituisce prova legale di un accordo o dichiarazione.
\end{enumerate}

\noindent La \textbf{firma digitale} non può consistere semplicemente di una digitalizzazione del documento originale firmato manualmente:\ un crittoanalista potrebbe ``tagliare'' dal documento digitale la parte contenente la firma e ``copiarla'' su un altro documento.\
Deve avere una forma che dipenda dal documento su cui viene apposta, per essere inscindibile da questo.\
Per progettare firme digitali si possono usare cifrari sia i simmetrici che quelli asimmetrici.\

\subsection{Protocollo:\ messaggio in chiaro e firmato}

Si consideri un utente $U$ proprietario della coppia di chiavi $k_U[\mathit{priv}]$ e $k_U[\mathit{pub}]$.\
$C$ e $D$ sono le funzioni di cifratura e decifrazione di un cifrario asimmetrico.\

$U$ genera la firma $f = D(m, k_U[\mathit{priv}])$ per $m$ e spedisce all'utente $V$ la tripla $\langle U,m,f \rangle$.\
$V$ riceve $\langle U,m,f \rangle$ e verifica l'autenticità della firma $f$ calcolando e controllando che
\[C(f, k_U[\mathit{pub}]) = m\]
L'indicazione del mittente $U$ consente a $V$ di selezionare la chiave pubblica $k_U[\mathit{pub}]$ da utilizzare nel calcolo.\

I processi di firma e verifica impiegano le funzioni $C$ e $D$ in ordine inverso a quello standard, $C$ e $D$ devono essere commutative:
\[C(D(m)) = D(C(m)) = m\]
\textbf{Osservazione}:\ sostituire $m$ col suo crittogramma non serve a nulla perché il messaggio in chiaro può essere ottenuto conoscendo la chiave pubblica e la firma, che sono note a tutti.\
\vspace{12pt}

\noindent Il protocollo soddisfa i requisiti della firma manuale:

\begin{itemize}
    \item $f$ è autentica e non falsificabile:\ per falsificare la firma occorre conoscere $k_U[\mathit{priv}]$, che è nota solo a $U$, e inoltre $D$ è one-way.\
    \item Il documento firmato $\langle U,m,f \rangle$ non può essere alterato se non da $U$, pena la non consistenza tra $m$ e $f$.
    \item Poiché solo $U$ può aver prodotto $f$, $U$ non può ripudiare la firma.\
    \item La firma $f$ non è riutilizzabile su un altro documento $m' \neq m$ poiché è immagine di $m$.
\end{itemize}


\subsubsection{Altre caratteristiche}

\begin{itemize}
    \item È definito per un particolare utente $U$, ma non per un particolare destinatario.\
    \item Chiunque può convincersi dell'autenticità della firma facendo uso solo della chiave pubblica di $U$.\
    \item È solo uno schema di principio:\ comporta lo scambio di un messaggio di lunghezza doppia dell'originale poiché la dimensione della firma è paragonabile alla dimensione del messaggio; il messaggio non può essere cifrato perché è ricavabile pubblicamente dalla firma attraverso la verifica di questo.\
\end{itemize}


\subsection{Protocollo:\ messaggio cifrato e firmato}

Si consideri un utente $U$ proprietario della coppia di chiavi $k_U[\mathit{priv}]$ e $k_U[\mathit{pub}]$.\
$C$ e $D$ sono le funzioni di cifratura e decifrazione di un cifrario asimmetrico.\

$U$ genera la firma $f = D(m, k_U[\mathit{priv}])$ per $m$ e calcola il crittogramma firmato $c = C(f, k_V[\mathit{pub}])$ con la chiave pubblica del destinatario $V$ (si incapsula la firma nel documento cifrato).\
A questo punto spedisce la coppia $\langle U,c \rangle$ all'utente $V$.\
$V$ riceve $\langle U,c \rangle$, decifra il crittogramma $D(c, k_V[\mathit{priv}]) = f$ e cifra tale valore con la chiave pubblica di $U$ ottenendo
\[C(f, k_U[\mathit{pub}]) = C (D(m, k_U[\mathit{priv}]), k_U[\mathit{pub}]) = m \]
Se $m$ è significativo attesta l'identità di $U$:\ un utente diverso da $U$ ha probabilità praticamente nulla di generare un crittogramma di significato accettabile se ``cifrato'' con la chiave pubblica di $U$.\

\subsubsection{Algoritmo}

Immaginiamo che $U$ e $V$ utilizzino l'RSA sia per produrre la firma che per proteggere la loro comunicazione.\
In questo caso si avranno due coppie di chiavi.

\begin{itemize}
    \item $\langle d_U \rangle, \langle e_U, n_U\rangle$ chiavi di $U$.
    \item $\langle d_V \rangle, \langle e_V, n_V\rangle$ chiavi di $V$.
\end{itemize}

\noindent $U$ genera la firma del messaggio $m$
\[f=m^{d_U}\ \mathit{mod}\ n_U\] e la cifra con la chiave pubblica di $V$
\[c = f^{e_V}\ \mathit{mod}\ n_V\]
e spedisce a $V$ la coppia $\langle U, c \rangle$.

\vspace{12pt}
\noindent L'utente $V$ riceve la coppia $\langle U, c \rangle$ e decifra $c$
\[c^{d_v}\ \mathit{mod}\ n_V = f\]
``decifra'' $f$ con la chiave pubblica di $U$
\[f^{e_U}\ \mathit{mod}\ n_U = m\]
Se $m$ è significativo, conclude che è autentico.\
\vspace{12pt}

\noindent Affinché il procedimento sia corretto è necessario che $n_U \leq n_V$ perché risulti $f < n_V$ e $f$ possa essere cifrata correttamente e spedita a $V$.\
Questo impedirebbe a $V$ di inviare messaggi firmati e cifrati a $U$; per ovviare a questo tipo di problema, ogni utente stabilisce chiavi distinte per la firma e per la cifratura:\ si fissa pubblicamente $H$ molto grande, ad esempio $H = 2^{1024}$, e si prendono le chiavi di firma $< H$ e le chiavi di cifratura $> H$.\

Il valore elevato di $H$ assicura che tutte le chiavi possano essere scelte sufficientemente grandi e, quindi inattaccabili con brute force.\
Tuttavia, il meccanismo si presta ad altri tipi di attacchi basati sulla possibilità che il crittoanalista si procuri la firma di un utente su messaggi apparentemente privi di senso.\

\subsubsection{Attacco}

Supponiamo che un utente $U$ invii una risposta automatica (ack) al mittente ogni volta che riceve un messaggio $m$ e supponiamo che questo segnale di ack sia il crittogramma della firma di $U$ su $m$.\
Un crittoanalista attivo $X$ può decifrare i messaggi inviati a $U$:\

\begin{itemize}
    \item intercetta il crittogramma $c$ firmato inviato da $V$ ad $U$, lo rimuove dal canale e lo rispedisce a $U$, facendogli credere che $c$ sia stato inviato da lui.
    \item $U$ spedisce automaticamente a $X$ un ack.
    \item $X$ usa l'ack per risalire al messaggio originale applicando le funzioni del cifrario con le chiavi pubbliche di $V$ e $U$.
\end{itemize}

\noindent Vediamo il procedimento:
\begin{enumerate}
    \item $V$ invia il crittogramma $c$ a $U$
          \[c = C(f, k_U[\mathit{pub}])\]
          \[f = D(m, k_V[\mathit{priv}])\]
    \item Il crittoanalista $X$ intercetta $c$, lo rimuove dal canale e lo rispedisce a $U$ ($U$ crede che $c$ arrivi da $X$).
    \item $U$ decifra $c$
          \[f = D(c, k_U[\mathit{priv}])\]
          e verifica la firma con la chiave pubblica di $X$ ottenendo un messaggio
          \[m' = C(f, k_X[\mathit{pub}])\]
    \item $m' \neq m$ è un messaggio privo di senso, ma il sistema manda l'ack $c'$ a $X$
          \[f' = D(m', k_U[\mathit{priv}])\]
          \[c' = C(f', k_X[\mathit{pub}])\]
\end{enumerate}

\noindent $X$ usa $c'$ per risalire al messaggio $m$
\begin{enumerate}
    \item decifra $c'$ e trova $f'$
          \[D(c', k_X[\mathit{priv}]) = D(C(f', k_X[\mathit{pub}]), k_X[\mathit{priv}]) = f'\]
    \item verifica $f'$ e trova $m'$
          \[C(f', k_U[\mathit{pub}]) = C(D(m', k_U[\mathit{priv}]), k_U[\mathit{pub}]) = m'\]
    \item da $m'$ ricava $f$
          \[D(m', k_X[\mathit{priv}]) = D(C(f, k_X[\mathit{pub}]), k_X[\mathit{priv}]) = f\]
    \item verifica $f$ con la chiave pubblica di $V$ e trova $m$
          \[C(f, k_V[\mathit{pub}]) = C(D(m, k_V[\mathit{priv}]), k_V[\mathit{pub}]) = m\]
\end{enumerate}

\noindent Per evitare l'attacco occorre che $U$ blocchi l'ack automatico:\ l'ack deve essere inviato solo dopo un esame preventivo di $m$ e scartando tutti i messaggi che non si desidera firmare.\

\subsection{Protocollo resistente agli attacchi}

Si evita la firma del messaggio:\ la firma digitale si appone su un'immagine del messaggio ottenuta con una funzione hash one-way e pubblica.\
La firma non è più soggetta a giochi algebrici.

\vspace{12pt}

\noindent Il mittente $U$ calcola $h(m)$ e genera la firma
\[f = D(h(m), k_U[\mathit{priv}])\]
Calcola il crittogramma $c = C(m, k_V[\mathit{pub}])$ e spedisce a $V$ la tripla $\langle U, c, f\rangle$.

\vspace{12pt}

\noindent Il destinatario $V$ attua la fase di decifrazione e verifica:
$V$ riceve $\langle U, c, f\rangle$ e decifra $c$ \[m = D(c, k_V[\mathit{priv}])\]
Calcola separatamente $h(m)$ e $C(f, k_U[\mathit{pub}])$ e se risultano uguali allora il messaggio è autentico.\

\vspace{12pt}

\noindent Il protocollo consente una firma più veloce e lo scambio di una maggiore, ma trascurabile, quantità di dati.

\subsection{Attacchi man-in-the-middle}

Poiché le chiavi di cifratura sono pubbliche e non richiedono un incontro diretto tra gli utenti per il loro scambio, un crittoanalista attivo può intromettersi proprio in questa fase iniziale del protocollo, pregiudicando il suo corretto svolgimento.\

Attacco attivo chiamato \textbf{man-in-the-middle}:\ il crittoanalista si intromette nella comunicazione tra $U$ e $V$ e si comporta agli occhi di $U$ come se fosse $V$ (e viceversa), intercettando e bloccando le comunicazioni di $U$ e $V$.\

Il crittoanalista $X$ devia le comunicazioni che provengono da $U$ e $V$ e le dirige verso se stesso:
\begin{itemize}
    \item $U$ richiede a $V$ la sua chiave pubblica.
    \item $X$ intercetta la risposta contenente $k_V[\mathit{pub}]$ e la sostituisce con la sua $k_X[\mathit{pub}]$.
    \item $X$ si pone in ascolto in attesa dei crittogrammi da $U$ a $V$ cifrati con la sua chiave pubblica $k_X[\mathit{pub}]$.
    \item $X$ rimuove tutti questi crittogrammi, li decifra con $k_X[\mathit{priv}]$, se li legge per bene, li cifra con $k_V[\mathit{pub}]$ e li spedisce a $V$.
    \item $U$ e $V$ non si accorgono della presenza di $X$ se egli è sufficientemente veloce nel rimuovere, decifrare e reinserire sul canale i crittogrammi.
\end{itemize}

\subsection{Certification Authority}

Per evitare attacchi \textit{man-in-the-middle} si ricorre alle cosiddette \textbf{certification authority}:\ sono delle infrastrutture, degli enti che garantiscono la validità delle chiavi pubbliche e ne regolano l'uso, gestendo la distribuzione sicura delle chiavi alle due entità che vogliono comunicare.\

La certification authority autentica l'associazione $\langle$\texttt{utente,chiave pub\-blica}$\rangle$ emettendo un certificato digitale.\
Il certificato digitale consiste della chiave pubblica e di una lista di informazioni relative al suo proprietario, opportunamente firmate dalla CA.\
Per svolgere correttamente il suo ruolo mantiene un archivio di chiavi pubbliche sicuro, accessibile a tutti e protetto da attacchi di scrittura non autorizzati.

La chiave pubblica della CA è nota agli utenti che la mantengono protetta da attacchi esterni e la utilizzano per verificarne la firma sui certificati.\

\subsubsection{Certificato digitale}

Un certificato digitale contiene:

\begin{itemize}
    \item un'indicazione del suo formato (numero di versione);
    \item il nome della CA che lo ha rilasciato;
    \item un numero seriale che lo individua univocamente all'interno della CA;
    \item la specifica dell'algoritmo usato dalla CA per creare la firma elettronica;
    \item il periodo di validità del certificato (data di inizio e data di fine);
    \item un'indicazione del protocollo a chiave pubblica adottato da questo utente per la cifratura e la firma:\ nome dell'algoritmo, suoi parametri e chiave pubblica dell'utente;
    \item firma della CA eseguita su tutte le informazioni precedenti.
\end{itemize}

\noindent Se $U$ vuole comunicare con $V$ può richiedere $k_V[\mathit{pub}]$ alla CA che risponde inviando a $U$ il certificato digitale di $V$ e, poiché $U$ conosce $k_{CA}[\mathit{pub}]$, può controllare l'autenticità del certificato verificandone il periodo di validità e la firma prodotta dalla CA su di esso.\
Se tutti i controlli vanno a buon fine, $k_V[\mathit{pub}]$ nel certificato è corretta e $U$ avvia la comunicazione con $V$.\
Il crittoanalista potrebbe falsificare la certificazione ma si assume che la CA abbia un archivio di chiavi inattaccabile.

\subsection{Protocollo:\ messaggio cifrato, firmato in hash e certificato}

Il mittente $U$ si procura $\mathit{cert}_V$ di $V$, calcola $h(m)$ e genera la firma
\[f = D(h(m), k_U[\mathit{priv}])\]
Calcola il crittogramma $c = C(m, k_V[\mathit{pub}])$ e spedisce a $V$ la tripla
\[\langle \mathit{cert}_U, c, f\rangle\]
$\mathit{cert}_U$ contiene la chiave $k_U[\mathit{pub}]$ e la specificazione della funzione $h$ utilizzata.

Il destinatario $V$ riceve $\langle \mathit{cert}_U, c, f\rangle$ e verifica l'autenticità di $\mathit{cert}_U$ (e dunque di $k_U[\mathit{pub}]$) utilizzando la propria copia di $k_{CA}[\mathit{pub}]$.\
$V$ decifra il crittogramma $c$ mediante la sua chiave privata e ottiene
\[m = D(c, k_V[\mathit{priv}])\]
$V$ verifica l'autenticità della firma apposta da $U$ su $m$ controllando che risulti
\[C(f, k_U[\mathit{pub}]) = h(m)\]

\subsubsection{Conclusioni}

Il punto debole di questo meccanismo è rappresentato dai certificati revocati, cioè non più validi:\ le CA mettono a disposizione degli utenti un archivio pubblico contenente i certificati non più validi.\
La frequenza di controllo di questi certificati e le modalità della loro comunicazione agli utenti sono cruciali per la sicurezza.\

Attacchi \textit{man-in-the-middle} possono essere evitati se si stabilisce un incontro diretto tra utente e CA nel momento in cui si istanza il sistema asimmetrico dell'utente.
