\chapter{Generatore di sequenze pseudocasuali}

Una \textbf{sorgente binaria casuale} genera una sequenza di bit con le seguenti caratteristiche:

\begin{enumerate}
    \item $P(0) = P(1) = \frac{1}{2}$ (si può indebolire richiedendo $P(0), P(1) > 0$ e immutabili durante il processo di generazione);
    \item la generazione di un bit è indipendente da quella degli altri bit:\ non si può prevedere il valore di un bit osservando quelli già generati.\
\end{enumerate}

\noindent Si supponga di avere $P(0)>P(1)$, è sempre possibile bilanciare la sequenza:\ si generi una sequenza come
\[0\ 0\ 1\ 1\ 0\ 0\ 1\ 1\ 1\ 0\ 0\ 0\ 0\ 1\ 0\ 0\]
Si dividono i bit a coppie e si scartano le coppie con bit uguali:
\[1\ 0\ 0\ 1\]
e associamo alle coppie discordi un valore:\ $01 \rightarrow 0,\ 10\rightarrow 1$

Nella pratica è impossibile garantire la perfetta casualità di una sorgente e, soprattutto, l'indipendenza; non si può garantire che la generazione di un bit avvenga in modo assolutamente indipendente dalla generazione di altri bit:\ ogni esperimento modifica l'ambiente, quindi si va a influenzare l'esperimento successivo.\
Perciò, si accettano compromessi.\

Ci sono varie tecniche per generare sequenze casuali brevi.\
È possibile sfruttare alcuni \textbf{fenomeni casuali} presenti in natura (decadimento di particelle, \dots), ma il problema di questo approccio è che nessuno deve avere accesso ai dispositivi utilizzati per sfruttare queste sorgenti; inoltre più il fenomeno è lungo, più tende a ripetersi periodicamente e quindi è accettabile solo per sequenze casuali molto brevi.\
Può essere sfruttata anche la casualità presente in alcuni \textbf{processi software}, come la posizione della testina sull'hard disk, lo stato dell'orologio interno del calcolatore, \dots

\section{Generatori di numeri pseudocasuali}

Si genera la casualità mediante un algoritmo, cercandola all'interno di processi matematici:\ si parla di \textbf{generatori pseudo-casuali} perché sono programmi brevi e, quindi, per Kolmogorov non sono casuali.\
I generatori di numeri pseudo-casuali partono da una sequenza piccola chiamata \textit{seme} (che può essere generata con processi come i due citati sopra) e generano una sequenza molto più lunga.\

Un \textbf{generatore} prende in input un \textit{seme} e restituisce in output un flusso di bit arbitrariamente lungo e periodico:\ al suo interno contiene una sottosequenza che si ripete (\textit{periodo}) ed è quella che viene usata come sequenza casuale; un generatore è tanto migliore tanto è più lungo il periodo.\
Si tratta di un ``\textit{amplificatore di casualità}'':\ prende la casualità all'interno del seme iniziale e la ``estende'', la ``amplifica''.\

\subsubsection{Limitazioni}

Si supponga che $S$ sia il numero di bit del seme e che si generi una sequenza di lunghezza $n$; tipicamente $n>>S$.\
Una volta che il seme di $S$ bit è stato scelto, la sequenza lunga $n$ che ne deriva è già determinata implicitamente dal seme; questo significa che due semi uguali forniscono due sequenze uguali.\
Inoltre, il numero di sequenze diverse che possono essere generate con un seme da $S$ bit sono $2^S$ (meno di tutte quelle che si potrebbero costruire:\ $2^n$), proprio perché questo è il numero di semi che si possono avere con $S$ bit:\ ecco perché si parla di generatore \textit{pseudocasuale}.\

\subsubsection{Esempio di generatore}

Generatore lineare:\ $x_i = (ax_{i-1} + b)\ \mathit{mod}\ m$, dove $a, b$ e $m$ sono interi positivi.\
Il seme è un valore intero iniziale $x_0$ (che potremmo pensare di aver estratto casualmente).\
Si osservi che, poiché lavora col modulo, questo generatore può generare i numeri da 0 a $m - 1$:
\[x_0,\ x_1 = (ax_0 + b)\ \mathit{mod}\ m,\ x_2 = (ax_1 +b)\ \mathit{mod}\ m,\ \dots\]

\noindent È chiaro che nel momento in cui $x_i = x_0$ la sequenza si ripeterà perché ricomincerà a generare gli stessi numeri.\
Quindi il periodo massimo è proprio $m$, ma non è garantito che lo sia sempre; per far ciò bisogna imporre certe condizioni sui parametri:
\begin{itemize}
    \item $\mathrm{MCD}(b,m) = 1$;
    \item ($a-1$) deve essere divisibile per ogni fattore primo di $m$;
    \item ($a-1$) deve essere un multiplo di 4 se anche $m$ lo è.
\end{itemize}

\noindent Se si rispettano queste condizioni il periodo sarà $m$ e quindi verranno generate permutazioni degli interi da $0$ a $m-1$.\
Esempio:
\[a = 7,\ b = 7,\ m = 9,\ x_0 = 3\]
\[x_i = (7x_{i-1} + 7)\ \mathit{mod}\ 9\]
\[x_0 = 3,\ x_1 = 1,\ x_2 = 5,\ x_3 = 7,\ x_4 = 2,\ \dots,\ x_i=3 \]
\textbf{Nota bene}:\ per trasformarlo in generatore binario si fa $\frac{x_i}{m}$, si guarda la prima cifra decimale e se è pari allora si ottiene uno 0, altrimenti si ottiene un 1.

\subsubsection{Valutare un generatore}

Le sequenze generate da un generatore si valutano tramite dei \textbf{test statistici}, in particolare valutano se la sequenze presenta proprietà tipiche di una sequenza casuale.\

\begin{itemize}
    \item \textbf{Test di frequenza}:\ controlla che i vari simboli diversi abbiano la stessa frequenza.
    \item \textbf{Poker test}:\ controlla che le sottosequenze di lunghezza fissa occorrano lo stesso numero di volte nella sequenza totale.
    \item \textbf{Test di autocorrelazione}:\ verifica che non ci siano irregolarità nella sequenza, per esempio, guardando quali sono gli elementi che si ripetono dopo una certa lunghezza (ad esempio:\ ogni cinque posti c'è uno 0).
    \item \textbf{Run test}:\ verifica che le sottosequenze massimali di elementi tutti ripetuti abbiano una decrescenza esponenziale di apparizione nella sequenza in base a quanto sono lunghe (11 probabile, 111 poco probabile, 1111 quasi impossibile, 11111 impossibile, \dots).
\end{itemize}

\noindent Per le applicazioni crittografiche si richiede anche il \textbf{test di prossimo bit} (implica anche i test precedenti):\ controlla che sia impossibile fare previsioni in tempo polinomiale sugli elementi della sequenza prima che siano generati.\
In particolare, un generatore binario supera il test di prossimo bit se non esiste un algoritmo polinomiale in grado di prevedere l'($i+1$)-esimo bit della sequenza a partire dalla conoscenza degli $i$-esimi bit precedenti con probabilità maggiore di $\frac{1}{2}$.\
I generatori che superano il test di prossimo bit sono detti \textbf{crittograficamente sicuri}.\

\subsubsection{Generatore polinomiale}

Non è crittograficamente sicuro.\

\[x_i = (a_1 x_{i-1}^t + a_2 x_{i-1}^{t-1} +\ \dots + a_t x_{i-1} + a_{t+1})\ \mathit{mod}\ m\]

\subsection{Generatori crittograficamente sicuri}

Si fa ricorso alle \textbf{funzioni one-way}

\begin{itemize}
    \item compututazionalmente facile da calcolare:\ $x\rightarrow y=f(x)$ richiede tempo polinomiale.
    \item compututazionalmente difficile da invertire:\ $y\rightarrow x=f^{-1}(y)$ richiede tempo esponenziale.
\end{itemize}

\noindent Idea:\ si supponga di avere un seme $x_0$ dal quale si ottiene un valore $x$; si calcola $f(x)$ considerando $f$ come funzione one-way e si trova $x_1$.\
In seguito si trova $x_2$ con $f(x_1)=f(f(x))=f^2(x)$.\
In generale, $x_i = f^i(x) = f(x_{i-1})$:\ si itera l'applicazione della funzione one-way un numero arbitrario di volte.\

Ogni elemento della sequenza $S$ si può calcolare facilmente dal precedente, ma non dai valori successivi perché $f$ è one-way; di conseguenza, si calcola la sequenza per un certo numero di passi, senza svelarne il risultato, e si comunicano/utilizzano gli elementi in ordine inverso:\ ogni elemento non è prevedibile in tempo polinomiale, pur conoscendo quelli comunicati in precedenza.\

Per costruire generatori binari crittograficamente sicuri si usano i predicati ``\textit{hard core}'' delle funzioni one-way.\
$b(x)$ è un predicato hard core di una funzione one-way $f(x)$ se:\
\begin{itemize}
    \item $b(x)$ è facile da calcolare conoscendo $x$;
    \item $b(x)$ è difficile da prevedere con probabilità $> \frac{1}{2}$ conoscendo solo $f(x)$.
\end{itemize}

\subsubsection{Esempio di funzione one-way}

Funzione one-way:\ $f(x) = x^2\ \mathit{mod}\ n$, $n$ non primo.\
Supponiamo $n = 77$ e $x = 10$:
\begin{itemize}
    \item  $f(x) = 10^2\ \mathit{mod}\ 77 = 23 \rightarrow$ molto facile.
    \item  $x =\ ? \rightarrow$ molto difficile:\ tipicamente si fa un forza bruta (costoso) dove si provano tutti i numeri da 0 a $n - 1$ e quando troviamo quello che corrisponde a $f(x)$ (23 in questo caso) ci fermiamo e guardiamo a quale $x$ corrisponde.
\end{itemize}
\textbf{Nota bene}:\ è importante ricordare che nell'algebra modulare, la funzione di elevamento a quadrato ($x^2$) è one-way.\
In questo caso il predicato hard core $b(x)$ è ``$x$ è dispari''.\

\subsubsection{Generatore BBS (1986)}

Si prende un numero $n = p \cdot q$, con $p,q$ primi e molto grandi; inoltre
\[p\ \mathit{mod}\ 4 = 3\ \mathrm{e}\ q\ \mathit{mod}\ 4 =3 \qquad 2 \left\lfloor\frac{p}{4}\right\rfloor + 1\ \mathrm{e}\ 2 \left\lfloor\frac{q}{4}\right\rfloor + 1\ \mathrm{primi\ tra\ loro} \]
e $y$ coprimo con $n$.\

Si calcola il seme come $x_0 = y^2\ \mathit{mod}\ n$ e una successione di $m \leq n$ interi
\[x_i = (x_{i-1})^2\ \mathit{mod}\ n\]
Successivamente si pone $b_i = 1 \Leftrightarrow x_{n-1}$ è dispari.\

Il problema del generatore BBS è che per mantenersi sicuro deve usare numeri molto grandi ed effettuare potenze e moduli, perciò ha computazioni molto lente.\

\subsubsection{Generatori basati su cifrari simmetrici}

Si prende un cifrario simmetrico e la chiave e si sostituisce il messaggio con un valore iniziale legato al generatore.\
\vspace{12pt}

\noindent Esempio approvato dal FIPS:\ usa il DES.\

\begin{itemize}
    \item $r$ = numero di bit delle parole prodotte ($r=64$ in DES)
    \item $s$ = seme casuale di $r$ bit
    \item $m$ = numero di parole da produrre
    \item $k$ = chiave segreta del cifrario
\end{itemize}

\begin{verbatim}
Generatore(s, m){ //flusso di output di m*r bit
    d = rappresentazione su r bit di data e ora
    y = C(d, k);
    z = s;
    for(i = 1; i <= m; i++){
        x_i = C(y XOR z, k);
        z = C(y XOR x_i, k);
        comunica x_i a chi di dovere;
    }
}
\end{verbatim}
