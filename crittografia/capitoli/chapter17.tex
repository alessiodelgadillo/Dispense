\chapter{Quantum Distribution Key}

Il \textbf{Quantum Distribution Key} (\textbf{QDK}) è un protocollo per la distribuzione quantistica delle chiavi e si pone come alternativa al protocollo Diffie-Hellman.\
I protocolli per la distribuzione quantistica delle chiavi sono già in uso, non hanno bisogno del computer quantistico per funzionare; si usano insieme al One-Time Pad e permettono di creare e scambiare lunghissime chiavi binarie in massima sicurezza.\

La meccanica quantistica non può essere utilizzata per lo scambio dei messaggi perché utilizza dei principi che permettono di verificare se c'è stata un'intrusione sul canale:\ una volta che Alice e Bob hanno costruito una chiave, prima di usarla eseguono una verifica per controllare se durante lo scambio di chiavi qualcuno ha provato ad intercettarle; in caso affermativo la chiave viene scartata prima che venga usata.\
Siccome la verifica dell'intrusione avviene a comunicazione conclusa, è chiaro che non sarebbe efficace usarlo per lo scambio di messaggi.\

Inoltre, esiste la crittografia \textit{post quantistica} che riguarda la cifratura a chiave pubblica e l'obiettivo di questo tipo di crittografia è andare ad individuare dei cifrari a chiave pubblica che rimangono difficili anche per le macchine quantistiche.\

\subsubsection{Principi della meccanica quantistica}

\begin{itemize}
    \item \textbf{Sovrapposizione di stati}:\ proprietà di un sistema quantistico di trovarsi in più stati contemporaneamente.
    \item \textbf{Decoerenza}:\ l'osservazione o la misurazione di un sistema quantistico disturba il sistema stesso che va a perdere la proprietà della sovrapposizione di stati collassando in uno stato singolo.
    \item \textbf{No-cloning}:\ impossibilità di duplicare un sistema quantistico conservando nella copia lo stato quantico dell'originale; sostanzialmente è impossibile copiare uno stato quantistico non noto.
    \item \textbf{Entanglement}:\ possibilità che due o più elementi si trovino in stato quantico correlato fra loro in modo che, pur se portati a grande distanza, mantengano la loro correlazione.
\end{itemize}

\section{Protocollo BB84}

È un protocollo per scambiare le chiavi mediante invio di \textbf{fotoni polarizzati}.\
I fotoni sono privi di massa e si propagano alla velocità della luce.\
Un fotone può avere diversi stati di polarizzazione (o piani di oscillazione del campo elettrico), a noi ne interessano quattro:\ verticale, orizzontale, a $+45^{\circ}$ e a $-45^{\circ}$.\
Questi stati vengono divisi in due coppie, la \textbf{base di polarizzazione ortogonale} e la \textbf{basi di polarizzazione diagonale}.\
In ciascuno di questi stati viene codificato un bit di informazione, per esempio:
\begin{itemize}
    \item polarizzazione verticale $\rightarrow$ 0
    \item polarizzazione orizzontale $\rightarrow$ 1
    \item polarizzazione a $+45^{\circ}$ $\rightarrow$ 0
    \item polarizzazione a $-45^{\circ}$ $\rightarrow$ 1
\end{itemize}

\noindent Perché usare due basi e non una sola?\
Scegliendo una base sola, non saremmo in grado di misurare con precisione lo stato di polarizzazione del fotone.\
Se Bob conosce la base di preparazione, la misura del fotone sarà accurata e conoscerà con certezza il bit di informazione; se invece la base è sconosciuta, è possibile distinguere solo due stati di polarizzazione nell'ambito della stessa base:\ se Bob sceglie una base diversa da quella usata da Alice, il risultato della misura è probabilistico.\

\vspace{12pt}
\noindent\textbf{Osservazione}:\ il protocollo BB84 non dipende in alcun modo da difficoltà di particolari problemi matematici e nemmeno dal fatto che un giorno potrebbe essere dimostrato che $\mathit{P} = \mathit{NP}$.

\subsection{Apparecchiatura}

Il protocollo richiede due canali:\ la prima comunicazione avviene sul canale quantistico, dove viaggiano i fotoni che incapsulano il bit casuale scelto da Alice.\
La seconda fase di comunicazione avviene sul canale standard e serve a comunicare la base utilizzata per misurare e per inviare.\

L'\textbf{OPG} è un'apparecchiatura che permette di emettere un fotone alla volta.\
Un'altra componente, la \textbf{PC}, consente di imporre al fotone una certa polarizzazione.\
Alice utilizzerà un \textit{random number generator} per scegliere (casualmente) se mandare 0 o 1 e in quale base mandarlo.\

Bob sceglie casualmente una base tramite un \textit{random number generator} e usa un \textbf{deviatore} per scegliere il \textbf{PBS} (\textit{beam splitter polarizzante}) corretto:\ serve a misurare il fotone per determinarne la polarizzazione; si noti che risponde correttamente soltanto se la base alla quale fa riferimento è la stessa che Alice aveva usato al momento dell'invio.\

\subsubsection{PBS}

Sia $S$ l'asse di polarizzazione del PBS e $F$ la direzione di polarizzazione del fotone.\
Dopo che il fotone è entrato nel beam splitter viene deviato in una delle due uscite possibili:\ $A$ (\textit{assorbimento}) e $R$ (\textit{riflessione}).\

\begin{itemize}
    \item Il fotone viene inviato all'uscita $A$ con probabilità $\cos^2\Theta$ e assume polarizzazione $S$.
    \item Il fotone viene inviato all'uscita $R$ con probabilità $\sin^2\Theta$ e assume polarizzazione $\perp$ a $S$.
\end{itemize}

\noindent $\Theta$ è l'angolo tra le due polarizzazioni.\

\begin{itemize}
    \item $\Theta = 0^{\circ}$ ($F \equiv S$):\ il fotone esce da $A$, con polarizzazione $S=F$.\
    \item $\Theta = 90^{\circ}$ ($F \perp S$):\ il fotone esce da $R$, con polarizzazione $S^{\perp}=F$.\
    \item $\Theta = \pm 45^{\circ} \Rightarrow \cos^2\Theta = \sin^2\Theta = \frac{1}{2}$:\ il fotone esce con pari probabilità da $A$ o $R$ e la sua polarizzazione cambia ($S, S^{\perp}$).\
\end{itemize}

\noindent \textbf{Fenomeno quantistico}:\ la lettura attraverso il PBS \textit{distrugge} lo stato quantistico precedente.\

Tutti quei bit ottenuti tramite una base diversa da quella usata per generarlo, essendo il risultato completamente casuale, verranno scartati e non saranno inclusi nella chiave.\
Poiché la base scelta per interpretare il fotone è casuale, si ha il 50\% di probabilità di trovare la base corretta, quindi solo metà dei bit inviati sotto forma di fotoni da Alice verranno effettivamente ``percepiti'' come parte della chiave da Bob.\

Alla fine di questo procedimento, Alice e Bob si salvano quali basi sono state scelte per polarizzare e quali per interpretare e se le comunicheranno sul canale standard.\

\subsubsection{Fasi del BB84}

\begin{enumerate}
    \item Alice invia una sequenza $S_A$ di bit codificati nelle polarizzazioni dei fotoni sul canale quantistico.
    \item Bob interpreta $S_A$ con le sue basi che ha scelto casualmente e ottiene $S_B$ (dove le basi coincidono, le misure sono uguali).
    \item Sul canale standard, Bob comunica ad Alice le basi scelte e Alice indica quali sono comuni alle sue.
    \item Alice e Bob sacrificano una porzione $S_A'', S_B''$ di $S_A'$ e $S_B'$ in posizioni prestabilite, comunicandole sul canale standard (si passano alcuni bit di $S_A'$ e $S_B'$).\ Se $S_A'' \neq S_B''$ interrompono la comunicazione, altrimenti usano $S_A' \backslash S_A'' = S_A' \backslash S_A''$ come chiave.
\end{enumerate}

\noindent In assenza di interferenze di Eve, Alice e Bob possiedono una sottosequenza identica ($S_A' = S_B'$), formata dai bit codificati e decodificati con basi comuni.\
\[|S_A'| = |S_B'| \simeq \frac{|S_A|}{2}\]

\subsubsection{Quantum Bit Error Rate}

Il \textbf{Quantum Bit Error Rate} (QBER) è la percentuale di bit errati (dovuti ad errori dell'apparato sperimentale e non da intrusioni).\
Nel confronto tra le due sequenze che Alice e Bob sacrificano per verificare se c'è stato un crittoanalista, se la percentuale di errori è maggiore di QBER allora concludono che Eve è stata sul canale, altrimenti l'errore viene attribuito all'apparato sperimentale:\ usano un codice a correzione di errori per ricostruire una chiave corretta.\

\vspace{12pt}
\noindent\textbf{Nota bene}:\ Eve può comunque non far notare il suo passaggio sul canale quantistico intercettando solo pochi fotoni così da far attribuire la colpa degli errori all'apparato.\
\vspace{12pt}

\noindent Per evitare che Eve entri in possesso di qualche bit della chiave senza che Alice e Bob lo sappiano, le sequenze ottenute vengono inserite in funzioni hash e  come chiavi vengono usate le immagini hash.\

\newpage

\subsubsection{Protocollo}

\textit{Canale quantistico}
\vspace{12pt}

\noindent $S_A[1,n] \rightarrow$ sequenza iniziale di bit da cui verrà estratta la chiave (rappresentata con un codice a correzione di errori).\

\begin{verbatim}
for i=1 to n
    Alice: sceglie una base a caso, codifica S_A[i],
            invia il fotone a Bob

    Eve intercetta il fotone, lo misura con una sua base,
            lo invia a Bob e calcola S_E[i]

    Bob: sceglie una base a caso, interpreta il fotone
            ricevuto e costruisce S_B[i]
\end{verbatim}

\noindent\textit{Canale standard}
\vspace{12pt}

\noindent QBER è la \% di errore dovuti al rumore, mentre $h$ è una funzione hash crittografica.

\begin{flushleft}
    Bob comunica ad Alice la sequenza di basi scelte.

    Alice comunica a Bob le basi comuni.
\end{flushleft}

\begin{enumerate}
    \item estraggono $S_A'$ e $S_B'$ corrispondenti alle basi comuni e da queste estraggono due sottosequenze in posizioni concordate $\rightarrow S_A'', S_B''$.
    \item Si scambiano $S_A''$ e $S_B''$.\ Se la \% di bit $\neq$ è $>$ QBER $\Rightarrow$ Stop.
    \item Altrimenti calcolano $S_A' \backslash S_A''$ e $S_B' \backslash S_B''$, le decodificano con il codice a correzione di errori $\Rightarrow$ ottengono una sequenza comune $S_C$.
          \begin{center}
              \textbf{Attenzione}:\ Eve potrebbe conoscere alcuni bit di $S_C$.\
          \end{center}
    \item Alice e Bob calcolano $k = h(S_C)$ e usano $k$ come chiave.\
\end{enumerate}


