\chapter{Bitcoin e Cryptovalute}

Nasce nel 2008 dopo l'uscita dell'articolo \textit{``Bitcoin'':\ a peer to peer Electronic Cash System} di Satoshi Nakamoto (pseudonimo).\
Nel 2009 viene rilasciato il primo software che permette di utilizzare il suddetto protocollo di pagamento e il 3 Gennaio dello stesso anno nasce il ``\textit{blocco genesi}'' della \textbf{block chain} che porta alla nascita dei primi 50 BTC (ogni volta che viene aggiunto un blocco alla block chain viene creata nuova moneta).

La prima transazione eseguita è del 12 Gennaio 2009:\ Satoshi manda 10 Bitcoin a Hal Finney; mentre la prima transazione commerciale è avvenuta nel 2010.\

\subsubsection{Idea}

Ogni utente $A$ possiede una coppia di chiavi $\langle k_A[\mathit{pub}], k_A[\mathit{priv}]\rangle$ appartenenti a un sistema crittografico basato sulle curve ellittiche e per ogni transazione viene generata una nuova coppia di chiavi.\
A partire dalla chiave pubblica si calcola l'\textbf{indirizzo}, ossia l'identificatore di $A$, che serve per inviare pagamenti all'utente e verificarne la firma.\
La chiave privata serve invece per firmare le transazioni.\
Il \textbf{wallet} è l'insieme delle credenziali che attestano la proprietà in BTC di un utente ($\langle \rm indirizzo, chiave\ privata\rangle$ e software di gestione).\

\section{Transazione}

Una transazione è un movimento di Bitcoin da un utente ad un altro.\
Supponendo che Alice voglia mandare $x$ BTC a Bob, la transazione è un messaggio $m$
\[ m = \mathrm{add}_A - x - \mathrm{add}_B\]
firmato tramite $h = \mathrm{SHA\textrm{-}256}(m)$
\[ f = D(h, k_A[\mathit{priv}]) \]
Dopodiché $A$ diffonde in rete (\textbf{broadcast}) la coppia $\langle m, f\rangle$.\
Tuttavia, essendo in un sistema distribuito senza un'unità centrale di controllo, Bob deve aspettare ($\sim$ 10 minuti) che la rete convalidi la transazione.\

Inoltre, non è nemmeno sicuro che la transazione sia parte della block chain giusta dopo 10 minuti; esiste il fenomeno dei \textbf{blocchi orfani}:\ due utenti potrebbero aggiungere un blocco alla block chain contemporaneamente e a questo punto solo il blocco che sviluppa una catena più consistente viene considerata ``la vera'' block chain.\
Tutti i blocchi che facevano parte del blocco orfano sono da considerarsi annullati.\
Tipicamente, dopo l'aggiunta di sei blocchi dopo quello della transazione in questione, essa può essere considerata valida (un'ora di tempo).

\subsubsection{Struttura di una transazione}

Una transazione si usa come banconota:\ se qualcuno riceve un input di 0.8 BTC è come se ricevesse una banconota da 0.8 BTC; quindi se deve pagare a qualcuno 1.1 BTC deve possedere una ``banconota'' da 0.8 BTC e una da 0.3 BTC.\
Alternativamente può usare una banconota da 2 BTC mandando 1.1 BTC al destinatario e 0.9 BTC a un wallet secondario che usa come ``banca resti''.

Tipicamente nelle transazioni deve valere che $\sum \mathit{input} \geq \sum \mathit{output}$ e nel caso in cui sia strettamente maggiore, la differenza andrà a colui che confermerà il blocco della block chain che la contiene:\ l'utente dovrebbe lasciare una piccola differenza per incentivare i miner a confermare proprio il suo blocco.\

\subsubsection{Validazione di una transazione}

I nodi controllano un vario numero di transazioni e controllano che siano valide (controllano in particolare se esiste la transazione di input per verificare se è disponibile la liquidità in BTC).\
Quando si sono raccolte un discreto numero di transazioni valide (centinaia o migliaia), esse diventano un blocco di transazioni e gli utenti cercano di aggiungerlo alla block chain.

Aggiungere il blocco alla block chain è un processo molto laborioso poiché i vari nodi della catena si mettono d'accordo sul fatto che le transazioni siano valide o meno e questo richiede un po' di tempo; questo fa nascere il problema del double spending:\ un utente furbetto potrebbe approfittarsi di questi 10 minuti di accordi per effettuare due transazioni e ``pagarne solo uno''.\
Supponiamo che egli abbia 2 BTC ed esegua due pagamenti da 2 BTC uno subito dopo l'altro.\
Al momento in cui devono essere validate le suddette transazioni in un blocco, nessuno ha rimosso i 2 BTC dal suo wallet perché il blocco dove esse sono contenute non è stato ancora confermato e quindi entrambe le richieste di pagamento risultano valide sulla carta.\
Tuttavia, l'architettura della block chain mitiga bene questo problema.

\section{Architettura della block chain}

Nel blocco $i$-esimo della block chain sono contenuti vari componenti:
\begin{itemize}
    \item Le transazioni da confermare (a cui viene aggiunta la transazione ``premio'' destinata a chi conferma il blocco).
    \item La funzione hash del blocco $i-1$.\ È importante perché l'$i$-esimo blocco deve essere collegato al blocco $(i-1)$-esimo.
    \item Il numero intero \textbf{nonce} che è sconosciuto e trovarlo è compito dell'utente che vuole confermarlo.
\end{itemize}

\noindent Queste tre componenti entrano all'interno di una funzione SHA-256 che andrà a produrre un certo valore hash $h$ che è richiesto essere inferiore ad una certa soglia e che abbia una sequenza iniziale di $t$ zeri.\
L'utente che conferma il blocco, deve trovare il valore \textit{nonce} tale che $h <$ soglia.\
Il \textit{nonce} può essere trovato solo con una ricerca enumerativa (computazionalmente difficile), ma per verificarlo lo si può fare in maniera molto semplice (computazionalmente facile).\

Il parametro $t$ è calibrato dal sistema in base a quanto sono potenti e a quante sono le macchine che stanno cercando il \textit{nonce} di modo che il tempo medio per trovarlo sia di 10 minuti.\

\section{Miner}

È il nome degli utenti o nodi che validano le transazioni e aggiungono i blocchi alla block chain (\textbf{validazione tramite mining}).\
I miner provano continuamente a raggruppare transazioni per poi calcolarne il \textit{nonce}, ovviamente sono tutti ``in competizione'' fra loro, quindi vince chi lo trova per primo.
Chi riesce a trovarlo, lo diffonde in broadcast a tutti gli altri nodi.\

A questo punto essi dovranno verificare se il \textit{nonce} è valido
\[\mathit{nonce} + h_{i-1} + \mathrm{transazioni} = 0_1 0_2 \dots 0_t \{0,1\}^{n-t}\]
e in caso affermativo dovranno esprimere il loro consenso, col quale ``ammettono'' che è stato aggiunto un nuovo blocco valido alla block chain, cercando di creare nuovi blocchi da attaccare.\

Dato che il \textit{nonce} non è semplicissimo da trovare, è difficile che due miner ne trovino due contemporaneamente.\
Tuttavia, qualora succeda, tipicamente i blocchi nati successivamente alla biforcazione, vengono attaccati al blocco che ha più transazioni e quella diventerà la block chain ufficiale.\
Anche questo processo si ripete sempre uguale per le poche volte che succede.\

\textbf{Nota bene}:\ le uniche transazioni a cui è concesso di non specificare la transazione di input da cui prendere i BTC sono quelle aggiunte dai miner alla fine un blocco di transazioni.\
